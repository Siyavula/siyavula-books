         \chapter{Irrational numbers and rounding off}
    \setcounter{figure}{1}
    \setcounter{subfigure}{1}
    \label{m38349}
    \subsection{ Introduction}
            \nopagebreak
            \label{m38349*cid2} $ \hspace{-5pt}\begin{array}{cccccccccccc}   \end{array} $ \hspace{2 pt}\raisebox{-0.2em}{\includegraphics[height=1em]{../icons/www.eps}} {(subsection shortcode: MG10055 )} \par 
      \label{m38349*id324260}You have seen that repeating decimals may take a lot of paper and ink to write out. Not only is that impossible, but writing numbers out to many decimal places or \textsl{a high accuracy} is very inconvenient and rarely gives practical answers. For this reason we often estimate the number to a certain number of decimal places or to a given number of \textsl{significant figures}, which is even better.\par 
    \subsection{ Irrational Numbers}
            \nopagebreak
            \label{m38349*cid3} $ \hspace{-5pt}\begin{array}{cccccccccccc}   \end{array} $ \hspace{2 pt}\raisebox{-0.2em}{\includegraphics[height=1em]{../icons/www.eps}} {(subsection shortcode: MG10056 )} \par \label{m38349*id324624}Irrational numbers are numbers that cannot be written as a fraction with the numerator and denominator as integers. This means that any number that is \textsl{not} a terminating decimal number or a repeating decimal number is irrational. Examples of irrational numbers are:\par 
      \label{m38349*id324635}\nopagebreak\noindent{}
        \settowidth{\mymathboxwidth}{\begin{equation}
    \begin{array}{cc}\hfill \sqrt{2},\sqrt{3},\sqrt[3]{4},\pi ,\\ \hfill \frac{1+\sqrt{5}}{2}\approx 1,618\phantom{\rule{0.166667em}{0ex}}033\phantom{\rule{0.166667em}{0ex}}989\phantom{\rule{0.166667em}{0ex}}\end{array}\tag{7.1}
      \end{equation}
    }
    \typeout{Columnwidth = \the\columnwidth}\typeout{math as usual width = \the\mymathboxwidth}
    \ifthenelse{\lengthtest{\mymathboxwidth < \columnwidth}}{% if the math fits, do it again, for real
    \begin{equation}
    \begin{array}{cc}\hfill \sqrt{2},\sqrt{3},\sqrt[3]{4},\pi ,\\ \hfill \frac{1+\sqrt{5}}{2}\approx 1,618\phantom{\rule{0.166667em}{0ex}}033\phantom{\rule{0.166667em}{0ex}}989\phantom{\rule{0.166667em}{0ex}}\end{array}\tag{7.1}
      \end{equation}
    }{% else, if it doesn't fit
    \setlength{\mymathboxwidth}{\columnwidth}
      \addtolength{\mymathboxwidth}{-48pt}
    \par\vspace{12pt}\noindent\begin{minipage}{\columnwidth}
    \parbox[t]{\mymathboxwidth}{\large$
    \sqrt{2},\sqrt{3},\sqrt[3]{4},\pi ,\frac{1+\sqrt{5}}{2}\approx 1,618\phantosubm{\rule{0.166667em}{0ex}}033\phantom{\rule{0.166667em}{0ex}}989\phantom{\rule{0.166667em}{0ex}}$}\hfill
    \parbox[t]{48pt}{\raggedleft 
    (7.1)}
    \end{minipage}\vspace{12pt}\par
    }% end of conditional for this bit of math
    \typeout{math as usual width = \the\mymathboxwidth}
\label{m38349*notfhsst!!!underscore!!!id128}
\begin{tabular}{cc}
	   \hspace*{-50pt}\raisebox{-8 mm}{ \includegraphics[width=0.5in]{col11306.imgs/pstip2.png}  }& 
	\begin{minipage}{0.85\textwidth}
	\begin{note}
      {tip: }When irrational numbers are written in decimal form, they go on forever and
there is no repeated pattern of digits.
	\end{note}
	\end{minipage}
	\end{tabular}
	\par
      \label{m38349*id324739}If you are asked to identify whether a number is rational or irrational, first write the number in decimal form. If the number is terminated then it is rational. If it goes on forever, then look for a repeated pattern of digits. If there is no repeated pattern, then the number is irrational.\par 
      \label{m38349*id324745}When you write irrational numbers in decimal form, you may (if you have a lot of time and paper!) continue writing them for many, many decimal places. However, this is not convenient and it is often necessary to round off.\par 
\label{m38349*secfhsst!!!underscore!!!id133}
            \subsubsection{  Investigation : Irrational Numbers }
            \nopagebreak
      \label{m38349*id324757}Which of the following cannot be
written as a rational number?\par 
      \label{m38349*id324763}\textbf{Remember}: A rational number is a fraction with numerator and denominator as integers. Terminating decimal numbers or repeating decimal numbers are rational.\par 
      \label{m38349*id324775}\begin{enumerate}[noitemsep, label=\textbf{\arabic*}. ] 
            \label{m38349*uid1}\item 
          $\pi =3,14159265358979323846264338327950288419716939937510...$
        \label{m38349*uid2}\item $1,4$
\label{m38349*uid3}\item 
          $1,618\phantom{\rule{0.166667em}{0ex}}033\phantom{\rule{0.166667em}{0ex}}989\phantom{\rule{0.166667em}{0ex}}...$
        \label{m38349*uid4}\item $100$
\end{enumerate}
    \subsection{ Rounding Off}
            \nopagebreak
            \label{m38349*cid4} $ \hspace{-5pt}\begin{array}{cccccccccccc}   \includegraphics[width=0.75cm]{col11306.imgs/summary_fullmarks.png} &   \end{array} $ \hspace{2 pt}\raisebox{-5 pt}{} {(subsection shortcode: MG10057 )} \par 
      \label{m38349*id324198}Rounding off or approximating a decimal number to a given number of decimal places is the quickest way to approximate a number. For example, if you wanted to round-off $2,6525272$ to three decimal places then you would first count three places after the decimal and place a $|$ between the third and fourth number after the decimal.\par 
      \label{m38349*id325085}\nopagebreak\noindent{}
        \settowidth{\mymathboxwidth}{\begin{equation}
    2,652|5272\tag{7.2}
      \end{equation}
    }
    \typeout{Columnwidth = \the\columnwidth}\typeout{math as usual width = \the\mymathboxwidth}
    \ifthenelse{\lengthtest{\mymathboxwidth < \columnwidth}}{% if the math fits, do it again, for real
    \begin{equation}
    2,652|5272\tag{7.2}
      \end{equation}
    }{% else, if it doesn't fit
    \setlength{\mymathboxwidth}{\columnwidth}
      \addtolength{\mymathboxwidth}{-48pt}
    \par\vspace{12pt}\noindent\begin{minipage}{\columnwidth}
    \parbox[t]{\mymathboxwidth}{\large$
    2,652|5272$}\hfill
    \parbox[t]{48pt}{\raggedleft 
    (7.2)}
    \end{minipage}\vspace{12pt}\par
    }% end of conditional for this bit of math
    \typeout{math as usual width = \the\mymathboxwidth}
      \label{m38349*id325105}All numbers to the right of the $|$ are ignored after you determine whether the number in the third decimal place must be rounded up or rounded down. You \textsl{round up} the final digit if the first digit after the $|$ was greater than or equal to 5 and \textsl{round down} (leave the digit alone) otherwise. In the case that the first digit before the $|$ is 9 \textsl{and} you need to round up, then the 9 becomes a 0 and the second digit before the $|$ is rounded up.\par 
      \label{m38349*id325160}So, since the first digit after the $|$ is a 5, we must round up the digit in the third decimal place to a 3 and the final answer of $2,6525272$ rounded to three decimal places is\par 
      \label{m38349*id325186}\nopagebreak\noindent{}
        \settowidth{\mymathboxwidth}{\begin{equation}
    2,653\tag{7.3}
      \end{equation}
    }
    \typeout{Columnwidth = \the\columnwidth}\typeout{math as usual width = \the\mymathboxwidth}
    \ifthenelse{\lengthtest{\mymathboxwidth < \columnwidth}}{% if the math fits, do it again, for real
    \begin{equation}
    2,653\tag{7.3}
      \end{equation}
    }{% else, if it doesn't fit
    \setlength{\mymathboxwidth}{\columnwidth}
      \addtolength{\mymathboxwidth}{-48pt}
    \par\vspace{12pt}\noindent\begin{minipage}{\columnwidth}
    \parbox[t]{\mymathboxwidth}{\large$
    2,653$}\hfill
    \parbox[t]{48pt}{\raggedleft 
    (7.3)}
    \end{minipage}\vspace{12pt}\par
    }% end of conditional for this bit of math
    \typeout{math as usual width = \the\mymathboxwidth}
\par
            \label{m38349*secfhsst!!!underscore!!!id199}\vspace{.5cm} 
      \noindent
      \hspace*{-30pt}\includegraphics[width=0.5in]{col11306.imgs/pspencil2.png}   \raisebox{25mm}{   
      \begin{mdframed}[linewidth=4, leftmargin=40, rightmargin=40]  
      \begin{exercise}
    \noindent\textbf{Exercise 7.1:  Rounding-Off }
      \label{m38349*probfhsst!!!underscore!!!id200}
      \label{m38349*id325213}Round-off the following numbers to the indicated number of decimal places:\par 
      \label{m38349*id325219}\begin{enumerate}[noitemsep, label=\textbf{\arabic*}. ] 
            \leftskip=20pt\rightskip=\leftskip\label{m38349*uid5}\item $\frac{120}{99}=1,212121212\dot{1}\dot{2}$ to 3 decimal places
\label{m38349*uid6}\item $\pi =3,141592654...$ to 4 decimal places
\label{m38349*uid7}\item $\sqrt{3}=1,7320508...$ to 4 decimal places
\label{m38349*uid789}\item $2,78974526...$ to 3 decimal places\end{enumerate}
      \vspace{5pt}
      \label{m38349*solfhsst!!!underscore!!!id212}\noindent\textbf{Solution to Exercise } \label{m38349*listfhsst!!!underscore!!!id212}\begin{enumerate}[noitemsep, label=\textbf{Step} \textbf{\arabic*}. ] 
            \leftskip=20pt\rightskip=\leftskip\item  
      \label{m38349*id325360}\begin{enumerate}[noitemsep, label=\textbf{\alph*}. ] 
            \leftskip=20pt\rightskip=\leftskip\label{m38349*uid8}\item 
          $\frac{120}{99}=1,212|121212\dot{1}\dot{2}$
        \label{m38349*uid9}\item 
          $\pi =3,1415|92654...$
        \label{m38349*uid10}\item 
          $\sqrt{3}=1,7320|508...$
        \item $2,789|74526...$\end{enumerate}
      \item  
      \label{m38349*id325490}\begin{enumerate}[noitemsep, label=\textbf{\alph*}. ] 
            \leftskip=20pt\rightskip=\leftskip\label{m38349*uid11}\item The last digit of $\frac{120}{99}=1,212|121212\dot{1}\dot{2}$  must be rounded down.
\label{m38349*uid12}\item The last digit of $\pi =3,1415|92654...$ must be rounded up.
\label{m38349*uid13}\item The last digit of $\sqrt{3}=1,7320|508...$ must be rounded up.
\item  The last digit of $2,789|74526...$ must be rounded up. Since this is a 9, we replace it with a 0 and round up the second last digit.\end{enumerate}
      \item  
      \label{m38349*id325626}\begin{enumerate}[noitemsep, label=\textbf{\alph*}. ] 
            \leftskip=20pt\rightskip=\leftskip\label{m38349*uid14}\item $\frac{120}{99}=1,212$ rounded to 3 decimal places
\label{m38349*uid15}\item $\pi =3,1416$  rounded to 4 decimal places
\label{m38349*uid16}\item $\sqrt{3}=1,7321$ rounded to 4 decimal places
\item $2,790$\end{enumerate}
      \end{enumerate}
    \end{exercise}
    \end{mdframed}
    }
    \noindent
    \subsection{ Summary}
            \nopagebreak
            \label{m38349*eip-361} $ \hspace{-5pt}\begin{array}{cccccccccccc}   \end{array} $ \hspace{2 pt}\raisebox{-0.2em}{\includegraphics[height=1em]{../icons/www.eps}} {(subsection shortcode: MG10058 )} \par \label{m38349*uid0821}\begin{itemize}[noitemsep]
            \item Irrational numbers are numbers that cannot be written as a fraction with the numerator and denominator as integers.\item For convenience irrational numbers are often rounded off to a specified number of decimal places\end{itemize}
        \subsection{ End of Chapter Exercises}
            \nopagebreak
            \label{m38349*cid5} $ \hspace{-5pt}\begin{array}{cccccccccccc}   \end{array} $ \hspace{2 pt}\raisebox{-0.2em}{\includegraphics[height=1em]{../icons/www.eps}} {(subsection shortcode: MG10059 )} \par \label{m38349*id325742}\begin{enumerate}[noitemsep, label=\textbf{\arabic*}. ] 
            \label{m38349*uid17}\item Write the following rational numbers to 2 decimal places:
\label{m38349*id325757}\begin{enumerate}[noitemsep, label=\textbf{\alph*}. ] 
            \label{m38349*uid18}\item $\frac{1}{2}$\label{m38349*uid19}\item 1
\label{m38349*uid20}\item $0,11111\overline{1}$\label{m38349*uid21}\item $0,99999\overline{1}$\end{enumerate}
        \label{m38349*uid22}\item Write the following irrational numbers to 2 decimal places:
\label{m38349*id325863}\begin{enumerate}[noitemsep, label=\textbf{\alph*}. ] 
            \label{m38349*uid23}\item $3,141592654...$\label{m38349*uid24}\item $1,618\phantom{\rule{0.166667em}{0ex}}033\phantom{\rule{0.166667em}{0ex}}989\phantom{\rule{0.166667em}{0ex}}...$\label{m38349*uid25}\item $1,41421356...$\label{m38349*uid26}\item $2,71828182845904523536...$\end{enumerate}
        \label{m38349*uid27}\item Use your calculator and write the following irrational numbers to 3 decimal places:
\label{m38349*id325991}\begin{enumerate}[noitemsep, label=\textbf{\alph*}. ] 
            \label{m38349*uid28}\item $\sqrt{2}$\label{m38349*uid29}\item $\sqrt{3}$\label{m38349*uid30}\item $\sqrt{5}$\label{m38349*uid31}\item $\sqrt{6}$\end{enumerate}
        \label{m38349*uid32}\item Use your calculator (where necessary) and write the following numbers to 5 decimal places. State whether the numbers are irrational or rational.
\label{m38349*id326080}\begin{enumerate}[noitemsep, label=\textbf{\alph*}. ] 
            \label{m38349*uid33}\item $\sqrt{8}$\label{m38349*uid34}\item $\sqrt{768}$\label{m38349*uid35}\item $\sqrt{100}$\label{m38349*uid36}\item $\sqrt{0,49}$\label{m38349*uid37}\item $\sqrt{0,0016}$\label{m38349*uid38}\item $\sqrt{0,25}$\label{m38349*uid39}\item $\sqrt{36}$\label{m38349*uid40}\item $\sqrt{1960}$\label{m38349*uid41}\item $\sqrt{0,0036}$\label{m38349*uid42}\item $-8\sqrt{0,04}$\label{m38349*uid43}\item $5\sqrt{80}$\end{enumerate}
        \label{m38349*uid44}\item Write the following irrational numbers to 3 decimal places and then write them as a rational number to get an approximation to the irrational number. For example, $\sqrt{3}=1,73205...$. To 3 decimal places, $\sqrt{3}=1,732$. $1,732=1\frac{732}{1000}=1\frac{183}{250}$. Therefore, $\sqrt{3}$ is approximately $1\frac{183}{250}$.
\label{m38349*id326443}\begin{enumerate}[noitemsep, label=\textbf{\alph*}. ] 
            \label{m38349*uid45}\item $3,141592654...$\label{m38349*uid46}\item $1,618\phantom{\rule{0.166667em}{0ex}}033\phantom{\rule{0.166667em}{0ex}}989\phantom{\rule{0.166667em}{0ex}}...$\label{m38349*uid47}\item $1,41421356...$\label{m38349*uid48}\item $2,71828182845904523536...$\end{enumerate}
        \end{enumerate}
  \label{m38349**end}
\practiceinfo
\par 
 \par \begin{tabular}[h]{ccccc}
 (1.) 00hd&  (2.) 00he&  (3.) 00hf&  (4.) 00hg&  (5.) 00hh& \end{tabular}
