\chapter{Analytical geometry}
Analytical geometry is the study of geometric properties, relationships and measurement of points, lines and angles in the Cartesian plane. Geometrical shapes are defined using a coordinate system and algebraic principles. Some consider the introduction of analytical geometry, also called coordinate or Cartesian geometry, to be the beginning of modern mathematics.\par 

\chapterstartvideo{VMbkm}

\section{Drawing figures on the Cartesian plane}
If we are given the coordinates of the vertices of a figure, we can draw the figure on the Cartesian plane. For example, quadrilateral $ABCD$ with coordinates $A(1;1)$, $B(3;1)$, $C(3;3)$ and $D(1;3)$.  

\setcounter{subfigure}{0}
\begin{figure}[H] % horizontal\label{m39107*id63458}
\begin{center}
\scalebox{1}{
\begin{pspicture}(-5,-5)(5.5,5.5)
% \psaxes{<->}(0,0)(5,5)
% \psgrid[gridcolor=lightgray,linecolor=lightgray,subgriddiv=1](0,0)(0,0)(4,4)
\psaxes[linewidth=1pt,labels=all,ticks=all]{<->}(0,0)(-1,-1)(5,5)
\pspolygon[linewidth=1pt](1,1)(1,3)(3,3)(3,1)(1,1)
\uput[dl](1,1){{$A$}}
\uput[dr](3,1){{$B$}}
\uput[ur](3,3){{$C$}}
\uput[ul](1,3){{$D$}}
\uput[l](5.7,0){{$x$}}
\uput[d](0,5.7){{$y$}}
\uput[d](-0.2,0){{$0$}}
\end{pspicture}
}
% \caption{Drawing a figure on the Cartesian plane}
\end{center}
\label{fig:cartesianplane}
\end{figure} 

The order of the letters for naming a figure is important. It indicates the order in which points must be joined: $A$ to $B$, $B$ to $C$, $C$ to $D$ and $D$ back to $A$. It would also be correct to write quadrilateral $CBAD$ or $BADC$ but it is better to follow the convention of writing letters in alphabetical order.     

\section{Distance between two points}
A point is a simple geometric object having location as its only property. 
\Definition{Point}{A point is an ordered pair of numbers written as $(x;y)$.}
\Definition{Distance}{Distance is a measure of the length between two points.}

\begin{Investigation}{Distance between two points}
Points $P(2;1)$, $Q(-2;-2)$ and $R(2,-2)$ are given. 
\begin{itemize}
 \item Can we assume that $\hat{R}=90^{\circ}$? If so, why?
\item Apply the Theorem of Pythagoras in $\triangle PQR$ to find the length of $PQ$.
\end{itemize}

\vspace*{-16.5pt}
\setcounter{subfigure}{0}
\begin{figure}[H] % horizontal\label{m39107*id63458}
\begin{center}
\scalebox{1}{
\begin{pspicture}(-5,-5)(5.5,5.5)
% \psaxes{<->}(0,0)(5,5)
% \psgrid[gridcolor=lightgray,linecolor=lightgray,subgriddiv=1](0,0)(-3,-3)(3,3)
\psaxes[inewidth=1pt,labels=all,ticks=all]{<->}(0,0)(-3,-3)(3,3)
\psline[linestyle=dashed,linewidth=1pt](-2,-2)(2,-2)(2,1)
\psline[linestyle=dashed,linewidth=1pt](-2,-2)(2,1)
\psdots(-2,-2)(2,1)(2,-2)
% \uput[l](-2,-1.8){\Large{$Q$}}
\uput[dl](-1.5,-2){{$Q(-2;-2)$}}
\uput[dr](2,-2){{$R(2;-2)$}}
% \uput[u](2.3,0.5){\Large{$P$}}
\uput[ur](2,1){{$P(2;1)$}}
\uput[l](3.5,0){{$x$}}
\uput[d](0,3.5){{$y$}}
% \uput[dr](2.5,-2){\Large{$(2;-2)$}}
\uput[d](-0.2,0){{$0$}}
\end{pspicture}
}
% \caption{Triangle $PQR$}
\end{center}
\label{fig:trianglePQR}
\end{figure} 
\vspace*{-16.5pt}
\end{Investigation}  
    
% SOLUTION FOR TEACHERS' GUIDE BELOW:
% In figure~\ref{fig:trianglePQR}, it can be seen that the length of the line $PR$ is 3 units and the length of the line $QR$ is four units. However, $\triangle PQR$, has a right angle at $R$. Therefore, the length of the side $PQ$ can be obtained by using the Theorem of Pythagoras:\par 
%       
% \begin{eqnarray*}
% P{Q}^{2} & = & P{R}^{2}+Q{R}^{2} \\
% \therefore P{Q}^{2} & = & {3}^{2}+{4}^{2} \\ 
% \therefore PQ & = & \sqrt{{3}^{2}+{4}^{2}}=5  
% \end{eqnarray*}
% The length of $PQ$ is the distance between the points $P$ and $Q$.\par 
To derive a general formula for the distance between two points $A({x}_{1};{y}_{1})$ and $B({x}_{2};{y}_{2})$, we use the Theorem of Pythagoras:\par 

\setcounter{subfigure}{0}
\begin{figure}[H] % horizontal\label{m39107*id63458}
\begin{center}
\scalebox{1}{
\begin{pspicture}(-5,-5)(5.5,5.5)
% \psaxes{<->}(0,0)(5,5)
% \psgrid[gridcolor=lightgray,linecolor=lightgray,subgriddiv=1,gridlabels=0.0cm](0,0)(-3,-3)(3,3)
\psaxes[linewidth=1pt,labels=all,ticks=all]{<->}(0,0)(-3,-3)(3,3)
\psline[linestyle=dashed, linewidth=1pt](-2,-2)(2,-2)(2,1)
\psline[linestyle=dashed,linewidth=1pt](-2,-2)(2,1)
\psdots(-2,-2)(2,1)
% \uput[l](-2,-1.8){\Large{$A$}}
\uput[dl](-1,-2){{$A(x_{1};y_{1})$}}
\uput[dr](2,-1.8){{$C$}}
\uput[ur](2.5,-2.5){{$(x_{2};y_{1})$}}
% \uput[u](2.3,0.5){\Large{$B$}}
\uput[ur](1.8,1){{$B(x_{2};y_{2})$}}
\uput[l](3.5,0){{$x$}}
\uput[d](0,3.5){{$y$}}
\uput[d](-0.2,-0.05){{$0$}}
\end{pspicture}
}
\end{center}
\end{figure}       

\begin{align*}
AB^2&=AC^{2}+BC^{2}\\
\therefore AB&=\sqrt{AC^{2}+BC^{2}}
\end{align*}

And
\begin{eqnarray*}
AC & = & {x}_{2}-{x}_{1}\\
BC & = & {y}_{2}-{y}_{1} 
\end{eqnarray*}
Therefore,
\begin{eqnarray*} 
AB & = & \sqrt{A{C}^{2}+B{C}^{2}} \\ 
& =& \sqrt{(x_{2}-x_{1})^{2}+(y_{2}-y_{1})^{2}} 
\end{eqnarray*}
Therefore to calculate the distance between any two points, $({x}_{1};{y}_{1})$ and $({x}_{2};{y}_{2})$, we use:\par 
\Identity{
\vspace*{-3em}
\begin{center}$\mbox{Distance  } (d)=\sqrt{{({x}_{1}-{x}_{2})}^{2}+{({y}_{1}-{y}_{2})}^{2}}$\end{center}}

Notice that $(x_1 - x_2)^2 = (x_2 - x_1)^2$.
\par
\mindsetvid{the distance formula}{VMbls}
\clearpage
\begin{wex}{Using the distance formula}{Find the distance between $S(-2;-5)$ and $Q(7;-2)$.}{
\westep{Draw a sketch}
% \begin{figure}
 \begin{center}
\scalebox{1} % Change this value to rescale the drawing.
{
\begin{pspicture}(0,-2.5467188)(6.5735936,2.5867188)
\rput(2.26,0.45328125){\psaxes[linewidth=0.028222222,arrowsize=0.05291667cm 2.0,arrowlength=1.4,arrowinset=0.4,ticksize=0.10583333cm,dx=0.75cm,dy=0.75cm,Dx=2,Dy=2]{<->}(0,0)(-2,-3)(3,2)}
\psline[linewidth=0.028222222cm](1.52,-1.3932812)(4.9,-0.26671866)
\psline[linewidth=0.028222222,linestyle=dashed,dash=0.16cm 0.16cm](1.54,-1.4332813)(4.92,-1.4142337)(4.92,-0.23328125)
\psline[linewidth=0.028222222](4.68,-1.3867188)(4.68,-1.1332811)(4.92,-1.1332811)
\usefont{T1}{ppl}{m}{n}
\rput(5.8,-0.33671865){$T(7,-2)$}
\usefont{T1}{ppl}{m}{n}
\rput(0.4,-1.4267187){$S(-2,-5)$}
\psdots[dotsize=0.127](4.92,-0.26671866)
\usefont{T1}{ppl}{m}{n}
\rput(2.0790625,0.25328124){$0$}
\usefont{T1}{ppl}{m}{n}
\rput(2.5290625,2.45){$y$}
\usefont{T1}{ppl}{m}{n}
\rput(5.4,0.5){$x$}
\psdots[dotsize=0.127](1.48,-1.4267187)
\end{pspicture} 
}
 \end{center}
% \end{figure}

\westep{Assign values to $(x_1;y_1)$ and $(x_2;y_2)$}
Let the coordinates of $S$ be $(x_1;y_1)$ and the coordinates of $T$ be $(x_2;y_2)$.
\begin{equation*}
x_1 = -2 \hskip2em y_1 = -5 \hskip2em x_2 = 7 \hskip2em y_2 = -2
\end{equation*}
\westep{Write down the distance formula}
\begin{equation*}
d = \sqrt{(x_1 - x_2)^2 + (y_1 - y_2)^2}
\end{equation*}
\westep{Substitute values}
\begin{equation*}
\begin{array}{rl}
d_{ST} &= \sqrt{(-2 - 7)^2 + (-5- (-2))^2}\\
& = \sqrt{(-9)^2 + (-3)^2}\\
&= \sqrt{81 + 9}\\
&= \sqrt{90}\\
&= 9,5
\end{array}
\end{equation*}
\westep{Write the final answer}
The distance between $S$ and $T$ is $9,5$ units.

}
\end{wex}
\vspace*{-20pt}
\begin{wex}{Using the distance formula}{Given $RS = 13$, $R(3;9)$ and $S(8;y)$, find $y$.}{
\westep{Draw a sketch}

% \usepackage{pst-plot} % For axes
 \begin{center}
\vspace*{-16.5pt}
\scalebox{1} % Change this value to rescale the drawing.
{
\begin{pspicture}(0,-4.146719)(7.8990626,4.186719)
\rput(2.0,-2.1467187){\psaxes[linewidth=0.028222222,arrowsize=0.05291667cm 2.0,arrowlength=1.4,arrowinset=0.4,ticksize=0.10583333cm,dx=0.5cm,dy=0.5cm,Dx=2,Dy=2]{<->}(0,0)(-2,-2)(5,6)}
\psline[linewidth=0.028222222,linestyle=dashed,dash=0.16cm 0.16cm](3.96,3.2067187)(2.64,0.12671874)(4.0,-2.8932812)
\usefont{T1}{ppl}{m}{n}
% \rput(1.86,-2.33){$0$}
\usefont{T1}{ppl}{m}{n}
\rput(2.3,3.9832811){$y$}
\usefont{T1}{ppl}{m}{n}
\rput(7.2,-1.9867188){$x$}
\usefont{T1}{ppl}{m}{n}
\rput(4.9,3.1832812){$S_2(8;y_2)$}
\usefont{T1}{ppl}{m}{n}
\rput(4.9,-3.0067186){$S_1(8;y_1)$}
\psdots[dotsize=0.12](2.66,0.1)
\psdots[dotsize=0.12](3.96,3.1667187)
\psdots[dotsize=0.12](4.0,-2.8932812)
\usefont{T1}{ppl}{m}{n}
\rput(3.3,0.10328125){$R(3;9)$}
\psline[linewidth=0.04cm,linestyle=dotted,dotsep=0.16cm](3.96,4.0067186)(3.96,-3.7532814)
\end{pspicture} 
}

\end{center}
\westep{Assign values to $(x_1;y_1)$ and $(x_2;y_2)$}
Let the coordinates of $R$ be $(x_1;y_1)$ and the coordinates of $S$ be $(x_2;y_2)$.
\begin{equation*}
x_1 = 3 \hskip2em y_1 = 9 \hskip2em x_2 = 8 \hskip2em y_2 = y
\end{equation*}
\westep{Write down the distance formula}
\begin{equation*}
d = \sqrt{(x_1 - x_2)^2 + (y_1 - y_2)^2}
\end{equation*}
\westep{Substitute values and solve for $y$}
\begin{equation*}
\begin{array}{rl}
13 &= \sqrt{(3 - 8)^2 + (9 - y)^2}\\
13^2 & = (-5)^2 + (81 - 18y + y^2)\\
0 &= y^2 - 18y - 63\\
&= (y+3) (y-21)\\
\therefore y &= -3 \mbox{ or } y = 21
\end{array}

\end{equation*}
\westep{Write the final answer}
$S$ is $(8;-3)$ or $(8;21)$
\vspace{2pt}
\vspace{.1in}
}
\end{wex}
\textbf{Important:} always draw a sketch. It helps with your calculation and makes it easier to check if your answer is correct.\\

\begin{exercises}{}{
\begin{enumerate}[label=\textbf{\arabic*}.]
\item Find the length of $AB$ if:
 \begin{enumerate}[noitemsep, label=\textbf{(\alph*)} ] 
\item $A(2;7)$ and $B(-3;5)$
\item $A(-3;5)$ and $B(-9;1)$
\item $A(x;y)$ and $B(x+4;y-1)$
\end{enumerate}

\item The length of $CD=5$. Find the missing coordinate if:
 \begin{enumerate}[noitemsep, label=\textbf{(\alph*)} ] 
\item $C(6;-2)$ and $D(x;2)$
\item $C(4;y)$ and $D(1;-1)$
\end{enumerate}
\end{enumerate}
\practiceinfo
\par 
\par \begin{tabular}[h]{ccccc}
(1.) 00cy&  (2.) 00cz\end{tabular}

}
\end{exercises}

%          \section{ Calculation of the gradient line}
%     \nopagebreak
%%%             \label{m39108} $ \hspace{-5pt}\begin{array}{cccccccccccc}   \includegraphics[width=0.75cm]{col11306.imgs/summary_video.png} &   \end{array} $ \hspace{2 pt}\raisebox{-5 pt}{} {(section shortcode: MG10109 )} \par 
%     
%     
%     

\section{Gradient of a line}
\Definition{Gradient}{The gradient of a line is determined by the ratio of vertical change to horizontal change.}
Gradient ($m$) describes the slope or steepness of the line joining two points. In the figure below, line $OQ$ is the least steep and line $OT$ is the steepest.

\setcounter{subfigure}{0}
\begin{figure}[H] % horizontal\label{m39107*id63458}
\begin{center}
\scalebox{1}{
\begin{pspicture}(-5,-5)(5.5,5.5)
% \psaxes{<->}(0,0)(5,5)
% \psgrid[gridcolor=lightgray,linecolor=lightgray,subgriddiv=1,gridlabels=0.0cm](0,0)(-1,-1)(4,4)
\psaxes[linewidth=1pt,labels=all,ticks=all]{<->}(0,0)(-1,-1)(4,4)
\psline[linewidth=1pt](0,0)(1,3.5)
\psline[linewidth=1pt](0,0)(3,3.5)
\psline[linewidth=1pt](0,0)(3,2)
\psline[linewidth=1pt](0,0)(3,0.5)
\uput[ur](.9,3.5){$T$}
% \uput[dl](-1,-2){\Large{$(x_{2},y_{2})$}}
\uput[ur](3,3.5){$S$}
\uput[ur](3,2){$R$}
\uput[ur](3,0.4){$Q$}
\uput[dl](0,0){$0$}
% \uput[ur](1.8,1){\Large{$(x_{1},y_{1})$}}
\uput[l](4.5,0){$x$}
\uput[d](0,4.5){$y$}
\end{pspicture}
}
\end{center}
\end{figure}        
To derive the formula for gradient, we consider any right-angled triangle formed from $A~(x_1;y_1)$ and $B~(x_2;y_2)$ with hypotenuse $AB$, as shown in the diagram below. The gradient is determined by the ratio of the length of the vertical side of the triangle to the length of the horizontal side of the triangle. The length of the vertical side of the triangle is the difference in $y$-values of points $A$ and $B$. The length of the horizontal side of the triangle is the difference in $x$-values of points $A$ and $B$. 
\setcounter{subfigure}{0}
\begin{figure}[H] % horizontal\label{m39107*id63458}
\begin{center}
\scalebox{1}{
\begin{pspicture}(-5,-5)(5.5,5.5)
% \psaxes{<->}(0,0)(5,5)
% \psgrid[gridcolor=lightgray,linecolor=lightgray,subgriddiv=1,gridlabels=0.0cm](0,0)(-3,-3)(3,3)
\psaxes[linewidth=1pt,labels=all,ticks=all]{<->}(0,0)(-3,-3)(3,3)
\psline(-2,-2)(2,-2)(2,1)
\psline[linestyle=solid,linewidth=1pt](-2,-2)(2,1)
% \uput[l](-2,-1.8){\Large{$A$}}
\psdots(-2,-2)(2,1)
\uput[dl](-1,-2){$A(x_{1},y_{1})$}
\uput[dr](2,-2){$C$}
% \uput[u](2.3,0.5){\Large{$B$}}
\uput[ur](1.8,1){$B(x_{2},y_{2})$}
\uput[l](3.5,0){$x$}
\uput[d](0,3.5){$y$}
\uput[dl](0,0){$0$}
\end{pspicture}
}
\end{center}
\end{figure}  
Therefore, gradient is determined using the following formula:

\Identity{
\vspace*{-3em}
\begin{center}Gradient $(m)= \dfrac{y_{2} - y_{1}}{x_{2} - x_{1}}$ or $\dfrac{y_{1} - y_{2}}{x_{1} - x_{2}}$\end{center}}
\textbf{Important:} remember to be consistent: $m \neq \dfrac{y_{1} - y_{2}}{x_{2} - x_{1}}$

\par
\mindsetvid{The gradient line}{VMbmr}


\begin{wex}{Gradient between two points}{Find the gradient of the line between points E$(2;5)$ and F$(-3;9)$.}{
\westep{Draw a sketch}
\begin{center}
\scalebox{1} % Change this value to rescale the drawing.
{
\begin{pspicture}(0,-3.1267188)(6.6735935,3.1667187)

\rput(3.0,-2.1267188){\psaxes[linewidth=1pt,arrowsize=0.05291667cm 2.0,arrowlength=1.4,arrowinset=0.4,ticksize=0.10583333cm,dx=0.5cm,dy=0.5cm]{<->}(0,0)(-3,-1)(3,5)}
\psdots[dotsize=0.12](3.96,0.3532813)
\psdots[dotsize=0.12](1.48,2.3732812)
\psline[linewidth=1pt](4.0,0.30671874)(1.48,2.3867188)
\usefont{T1}{ppl}{m}{n}
\rput(6.1,-1.8767188){$x$}
\usefont{T1}{ppl}{m}{n}
\rput(3.2790625,2.9632812){$y$}
\usefont{T1}{ppl}{m}{n}
\rput(4.0590625,0.06328135){$E(2;5)$}
\usefont{T1}{ppl}{m}{n}
\rput(1.2390624,2.6032813){$F(-3;9)$}
\usefont{T1}{ppl}{m}{n}
\rput(2.85,-2.3){$0$}
\end{pspicture} 
}
\end{center}
\westep{Assign values to $(x_1;y_1)$ and $(x_2;y_2)$}
Let the coordinates of $E$ be $(x_1;y_1)$ and the coordinates of $F$ be $(x_2;y_2)$.
\begin{equation*}
x_1 = 2 \hskip2em y_1 = 5 \hskip2em x_2 = -3 \hskip2em y_2 = 9
\end{equation*}
\westep{Write down the formula for gradient}
\begin{equation*}
m = \dfrac{y_2 - y_1}{x_2 - x_1}
\end{equation*}
\westep{Substitute known values}
\begin{equation*}
\begin{array}{cl}
m_{EF} &= \dfrac{9 - 5}{-3 - 2}\\[5pt]
&= \dfrac{4}{-5}
\end{array}
\end{equation*}
\westep{Write the final answer}
The gradient of $EF = -\dfrac{4}{5}$

}
\end{wex}


\begin{wex}{Gradient between two points}{Given $G(7;-9)$ and $H(x;0)$, with $m_{GH}= 3$, Find $x$.}{
\westep{Draw a sketch}
\begin{center}
\scalebox{1} % Change this value to rescale the drawing.
{
\begin{pspicture}(0,-2.6867187)(7.1790624,2.7267187)
\psset{xunit=1.2,yunit=1.2}
\rput(3.0,0.31328124){\psaxes[linewidth=1pt,arrowsize=0.05291667cm 2.0,arrowlength=1.4,arrowinset=0.4,ticksize=0.10583333cm,dx=0.6cm,dy=0.6cm,Dx=2,Dy=2]{<->}(0,0)(-3,-3)(3,2)}
\psdots[dotsize=0.12](5.48,0.35328126)
\psdots[dotsize=0.12](4.62,-1.8467188)
\psline[linewidth=1pt](4.64,-1.8267188)(5.46,0.35328126)
\usefont{T1}{ppl}{m}{n}
\rput(6.2,0.4){$x$}
\usefont{T1}{ppl}{m}{n}
\rput(3.1445312,2.5232813){$y$}
\usefont{T1}{ppl}{m}{n}
\rput(5.5245314,0.6432812){$H~(x;0)$}
\usefont{T1}{ppl}{m}{n}
\rput(4.574531,-2.0767188){$G~(7;-9)$}
\usefont{T1}{ppl}{m}{n}
\rput(2.85,0.12328125){$0$}
\usefont{T1}{ppl}{m}{n}
\rput(5.85,-0.71015626){$m_{GH} = 3$}
\end{pspicture} 
}
\end{center}
\westep{Assign values to $(x_1;y_1)$ and $(x_2;y_2)$}
Let the coordinates of $G$ be $(x_1;y_1)$ and the coordinates of $H$ be $(x_2;y_2)$.
\begin{equation*}
x_1 = 7 \hskip2em y_1 = -9 \hskip2em x_2 = x \hskip2em y_2 = 0
\end{equation*}
\westep{Write down the formula for gradient}
\begin{equation*}
m = \dfrac{y_2 - y_1}{x_2 - x_1}
\end{equation*}
\westep{Substitute values and solve for $x$}
\begin{equation*}
\begin{array}{rl}
3 &= \dfrac{0 - (-9)}{x - 7}\\[5pt]
3(x-7)&= 9\\
x-7 &= \dfrac{9}{3}\\
x-7 &= 3\\
x &= 3 + 7\\
&= 10 \\
\end{array}
\end{equation*}
\westep{Write the final answer}
The coordinates of $H$ are $(10;0)$.
\vspace{2pt}
\vspace{.1in}
}
\end{wex}


\begin{exercises}{}{
\begin{enumerate}[noitemsep, label=\textbf{\arabic*}. ]
\item Find the gradient of $AB$ if
 \begin{enumerate}[noitemsep, label=\textbf{(\alph*)} ] 
\item $A(7;10)$ and $B(-4;1)$
\item $A(-5;-9)$ and $B(3;2)$
\item $A(x-3;y)$ and $B(x;y+4)$
\end{enumerate}

\item If the gradient of $CD=\frac{2}{3}$, find $p$ given
\begin{enumerate}[noitemsep, label=\textbf{(\alph*)} ] 
\item $C(16;2)$ and $D(8;p)$
\item $C(3;2p)$ and $D(9;14)$
\end{enumerate}
\end{enumerate}

\practiceinfo
\par 
\par \begin{tabular}[h]{ccccc}
(1.) 00d0&  (2.) 00d1\end{tabular}
}
\end{exercises}
\subsection*{Straight lines}
\Definition{Straight line}{A straight line is a set of points with a constant gradient between any two of the points.}
Consider the diagram below with points $A(x;y)$, $B(x_2;y_2)$ and $C(x_1;y_1)$. 
\begin{center}
 \scalebox{1} % Change this value to rescale the drawing.
{
\begin{pspicture}(0,-3.0667188)(7.5290623,3.1067188)
\rput(3.97,-0.06671875){\psaxes[linewidth=1pt,arrowsize=0.05291667cm 2.0,arrowlength=1.4,arrowinset=0.4,labels=all,ticks=all,ticksize=0.10583333cm]{<->}(0,0)(-3,-3)(3,3)}
\psdots[dotsize=0.12](3.39,0.47328126)
\psdots[dotsize=0.12](4.57,2.2132812)
\psline[linewidth=1pt](2.17,-1.3467188)(4.69,2.3932812)
\usefont{T1}{ptm}{m}{n}
\rput(7.1545315,0.14328125){$x$}
\usefont{T1}{ptm}{m}{n}
\rput(4.2545314,2.9032812){$y$}
\usefont{T1}{ptm}{m}{n}
\rput(5.3,2.1832812){$A(x;y)$}
\usefont{T1}{ptm}{m}{n}
\rput(3.6745312,-0.33671874){$0$}
\psdots[dotsize=0.12](2.51,-0.86671877)
\usefont{T1}{ptm}{m}{n}
\rput(2.5845313,0.5832813){$B(x_2;y_2)$}
\usefont{T1}{ptm}{m}{n}
\rput(1.7,-0.7967188){$C(x_1;y_1)$}
\end{pspicture} 
}
\end{center}
We have $m_{AB} = m_{BC}=m_{AC}$ and $m = \dfrac{y_2-y_1}{x_2-x_1} = \dfrac{y_1-y_2}{x_1-x_2}$\par

The general formula for finding the equation of a straight line is $\dfrac{y-y_1}{x-x_1} = \dfrac{y_2-y_1}{x_2-x_1}$ where $(x;y)$ is any point on the line.\par This formula can also be written as $y-y_1 = m(x-x_1)$.\par

The standard form of the straight line equation is $y=mx+c$ where $m$ is the gradient and $c$ is the $y$-intercept.

\begin{wex}{Finding the equation of a straight line}
 {Find the equation of the straight line through $P(-1;-5)$ and $Q(5;4)$.}
{
\westep{Draw a sketch}
\begin{center}
\scalebox{1} % Change this value to rescale the drawing.
{
\begin{pspicture}(0,-3.0667188)(6.508125,3.1067188)
\psset{xunit=1.2, yunit=1.2}
\rput(3.0,-0.06671871){\psaxes[linewidth=0.04,arrowsize=0.05291667cm 2.0,arrowlength=1.4,arrowinset=0.4,ticksize=0.10583333cm,dx=0.6cm,dy=0.6cm]{<->}(0,0)(-3,-3)(3,3)}
\psdots[dotsize=0.12,dotangle=-5.9493704](5.4890275,1.926018)
\psline[linewidth=0.04cm](2.5,-2.5332813)(5.48,1.9467187)
\usefont{T1}{ptm}{m}{n}
\rput(6.1335936,0.1432813){$x$}
\usefont{T1}{ptm}{m}{n}
\rput(3.233594,2.9032812){$y$}
\usefont{T1}{ptm}{m}{n}
\rput(5.563594,2.203281){$Q(5;4)$}
\usefont{T1}{ptm}{m}{n}
% \rput(2.8245313,-0.265){$0$}
\psdots[dotsize=0.12,dotangle=-5.9493704](2.4808824,-2.5638745)
\usefont{T1}{ptm}{m}{n}
\rput(1.4735936,-2.5167186){$P(-1;-5)$}
\end{pspicture} 
}
\end{center}
\westep{Assign values}
Let the coordinates of $P$ be $(x_1;y_1)$ and $Q(x_2;y_2)$ \\
\begin{equation*}
x_1 = -1 \hskip2em y_1 = -5 \hskip2em x_2 = 5 \hskip2em y_2 = 4
\end{equation*}


\westep{Write down the general formula of the line}
\begin{align*}
\dfrac{y-y_1}{x-x_1} &= \dfrac{y_2-y_1}{x_2-x_1}
\end{align*}
\westep{Substitute values and make $y$ the subject of the equation}
\begin{align*}
 \dfrac{y-(-5)}{x-(-1)} &= \dfrac{4-(-5)}{5-(-1)} \\[5pt]

 \dfrac{y+5}{x+1} &=\dfrac{3}{2}\\[5pt]
2(y+5) &=3(x+1)\\
2y +10&=3x+3\\
2y&=3x-7\\
y&=\frac{3}{2}x - \frac{7}{2}
\end{align*}
\westep{Write the final answer}
The equation of the straight line is $y&=\frac{3}{2}x - \frac{7}{2}$.
}


\end{wex}

\subsection*{Parallel and perpendicular lines}    
%         \label{m39108*eip-332}We can use the gradient of a line to determine if two lines are parallel or perpendicular. If the lines are parallel (Figure~\ref{fig:parallelperpendicular}a) then they will have the same gradient, i.e. ${m}_{\mbox{AB}}={m}_{\mbox{CD}}$. If the lines are perpendicular (Figure~\ref{fig:parallelperpendicular}b) than we have: 
%     \setcounter{subfigure}{0}
%  	\begin{figure}[H] % horizontal\label{m39107*id63458}
%     \begin{center}
% \scalebox{.8}{
% \begin{pspicture}(-5,-5)(5.5,5.5)
% % \psaxes{<->}(0,0)(5,5)
% \rput(-2,-2){
% \psline[linewidth=.05cm](0,0)(0,3)
% \psline[linewidth=.05cm](1,0)(1,3)
% \uput[ur](-1,2.8){\Large{$a)$}}
% \uput[d](0,0){\Large{$A$}}
% \uput[u](0,3){\Large{$B$}}
% \uput[d](1,0){\Large{$C$}}
% \uput[u](1,3){\Large{$D$}}}
% 
% \rput(3.4,2){
% \psline[linewidth=.05cm](1,-4)(-2,-1)
% \psline[linewidth=.05cm](-2,-4)(1,-1)
% \uput[ur](-3.2,-1.2){\Large{$b)$}}
% \uput[dr](1,-4){\Large{$A$}}
% \uput[ul](-2,-1){\Large{$B$}}
% \uput[dl](-2,-4){\Large{$C$}}
% \uput[ur](1,-1){\Large{$D$}}}
% \end{pspicture}
% }
%     \end{center}
% \caption{a) Parallel and b) perpendicular lines}
% \label{fig:parallelperpendicular}
%  \end{figure}       
Two lines that run parallel to each other are always the same distance apart and have equal gradients. \\
% In other words: $\mbox{gradient}_{AB}=\mbox{gradient}_{CD}$. \par
If two lines intersect perpendicularly, then the product of their gradients is equal to $-1$. \\

If line $WX \perp $ line $YZ$, then $m_{WX} \times m_{YZ} = -1$. Perpendicular lines have gradients that are the negative inverse of each other.
\par
\mindsetvid{Parallel and perpendicular lines}{VMbon}
\pagebreak
\begin{wex}{Parallel lines}{Prove that  line $AB$ with $A(0;2)$ and $B(2,6)$ is parallel to  line $2x-y = 2$.}{
\westep{Draw a sketch}

\begin{center}
\scalebox{1} % Change this value to rescale the drawing.
{

\begin{pspicture}(0,-3.6067188)(6.5590625,3.6467187)
\psset{xunit=1.2, yunit=1.2}
\rput(3.0,-0.6067188){\psaxes[linewidth=1pt,arrowsize=0.05291667cm 2.0,arrowlength=1.4,arrowinset=0.4,ticksize=0.10583333cm,dx=0.6cm,dy=0.6cm]{<->}(0,0)(-3,-3)(3,4)}
\psdots[dotsize=0.12,dotangle=-5.9493704](4.0290275,2.4060183)
\psline[linewidth=1pt](1.9,-1.7867187)(4.52,3.3332813)
\usefont{T1}{ppl}{m}{n}
\rput(6.184531,-0.39671874){$x$}
\usefont{T1}{ppl}{m}{n}
\rput(3.4045312,3.4432812){$y$}
\usefont{T1}{ppl}{m}{n}
\rput(4.6745315,2.3832812){$B(2;6)$}
\usefont{T1}{ppl}{m}{n}
\rput(2.8245313,-0.7967188){$0$}
\usefont{T1}{ppl}{m}{n}
\rput(3.52,0.32328126){$A(0;2)$}
\psline[linewidth=1pt](2.34,-2.8067188)(5.04,2.2532814)
\psdots[dotsize=0.12](3.0,0.39328125)
\usefont{T1}{ppl}{m}{n}
\rput(5.7,1.6432812){$y=2x-2$}
\end{pspicture} 
}

\end{center}
(Be careful - some lines may look parallel but are not!)

\westep{Write down the formula for gradient}
\begin{equation*}
m = \dfrac{y_2-y_1}{x_2-x_1}
\end{equation*}
\westep{Substitute values to find the gradient for line $AB$}
\begin{equation*}
\begin{array}{rl}
m_{AB} &= \dfrac{6 - 2}{2 - 0}\\[5pt]
&= \dfrac{4}{2}\\
&= 2
\end{array}
\end{equation*}
\westep{Check that the equation of $CD$ is in the standard form $y=mx+c$}
\begin{equation*}
\begin{array}{cl}
2x-y&=2\\
y&=2x-2\\
\therefore m_{CD}&= 2
\end{array}
\end{equation*}
\westep{Write the final answer}
\begin{equation*}
\begin{array}{cl}
m_{AB} &= m_{CD}\\

\end{array}
\end{equation*}
therefore line $AB$ is parallel to $cd$, $y=2x-2$.
}
\end{wex}



\begin{wex}{Perpendicular lines}{Line $AB$ is perpendicular to line $CD$. Find $y$ given $A(2;-3)$, $B(-2;6)$, $C(4;3)$ and $D(7;y)$.}{
\westep{Draw a sketch}
\begin{center}
\scalebox{1} % Change this value to rescale the drawing.
{

\begin{pspicture}(0,-3.6267188)(7.9990625,3.6667187)
\psset{xunit=1.2,yunit=1.2}
\rput(3.0,-0.62671876){\psaxes[linewidth=0.04,arrowsize=0.05291667cm 2.0,arrowlength=1.4,arrowinset=0.4,ticksize=0.10583333cm,dx=0.6cm,dy=0.6cm]{<->}(0,0)(-3,-3)(4,4)}
\psdots[dotsize=0.12,dotangle=-5.9493704](4.9890275,0.8660181)
\psline[linewidth=0.04cm](0.98,-0.9267188)(6.58,1.5932814)
\usefont{T1}{ppl}{m}{n}
\rput(7.2745314,-0.53671867){$x$}
\usefont{T1}{ppl}{m}{n}
\rput(3.1745312,3.4632812){$y$}
\usefont{T1}{ppl}{m}{n}
\rput(1.5790626,2.6032813){$B(-2;6)$}
\usefont{T1}{ppl}{m}{n}
% \rput(2.8,-0.81671876){$0$}
\usefont{T1}{ppl}{m}{n}
\rput(4.8,-2.0967188){$A(2;-3)$}
\psline[linewidth=0.04cm](2.0,2.3132813)(4.02,-2.1067188)
\psdots[dotsize=0.12](2.0,2.3532813)
\psdots[dotsize=0.12](4.02,-2.1067188)
\usefont{T1}{ppl}{m}{n}
\rput(4.3745313,1.0){$C(4;3)$}
\psline[linewidth=0.04cm,linestyle=dashed,dash=0.16cm 0.16cm](6.52,3.3932812)(6.52,-3.4867187)
\usefont{T1}{ppl}{m}{n}
\rput(7.25,1.5){$D(7;y)$}
\psdots[dotsize=0.12](6.52,1.5732813)
\rput{23.284023}(0.24130535,-1.2977501){\psframe[linewidth=0.04,dimen=outer](3.44,0.10671873)(3.1,-0.23328127)}
\end{pspicture}
}
\end{center}


\westep{Write down the relationship between the gradients of the perpendicular lines $AB \perp CD $}
\begin{align*}

m_{AB} \times m_{CD} &= -1\\
\dfrac{y_B-y_A}{x_B-x_A} \times \dfrac{y_D-y_C}{x_D-x_C} &=-1
\end{align*}
\westep{Substitute values and solve for $y$}
\begin{equation*}
\begin{array}{rl}
\dfrac{6 - (-3)}{-2 -2} \times \dfrac{y - 3}{7 - 4} &= -1\\[7pt]
\dfrac{9}{-4} \times \dfrac{y-3}{3} &= -1\\[7pt]
\dfrac{y-3}{3} &= -1 \times \dfrac{-4}{9}\\[7pt]
\dfrac{y-3}{3} &= \dfrac{4}{9}\\[7pt]
y-3 &= \dfrac{4}{9} \times 3\\[7pt]
y-3 &= \dfrac{4}{3}\\[7pt]
\end{array}
\end{equation*}
\begin{equation*}
\begin{array}{rl}
y &= \dfrac{4}{3} + 3\\[7pt]
&= \dfrac{4 + 9}{3}\\[7pt]
&= \dfrac{13}{3}\\[7pt]
&= 4 \dfrac{1}{3}
\end{array}
\end{equation*}
\westep{Write the final answer}
Therefore the coordinates of $D$ are $(7; 4\frac{1}{3})$.
}
\end{wex}

\subsection*{Horizontal and vertical lines}

A line that runs parallel to the $x$-axis is called a horizontal line and has a gradient of zero. This is
because there is no vertical change:\par
\begin{equation*}m = \dfrac{\mbox{change in }y}{\mbox{change in }x} = \dfrac{0}{\mbox{change in }x} =0\end{equation*}

A line that runs parallel to the $y$-axis is called a vertical line and its gradient is undefined. This is because there is no horizontal change:\par
\begin{equation*}
  m = \dfrac{\mbox{change in }y}{\mbox{change in }x} = \dfrac{\mbox{change in }y}{0}=\mbox{ undefined}
\end{equation*}

\subsection*{Points on a line}
A straight line is a set of points with a constant gradient between any of the two points.
There are two methods to prove that points lie on the same line; the gradient method and a
longer method using the distance formula.
\pagebreak
% If three points are collinear then the line joining them will have the same gradient at any point along the line.
% Therefore to prove collinearity, we must prove that two of the gradients between any of the three points are
% equal.

\begin{wex}{Points on a line}{Prove that $A(-3;3)$, $B(0;5)$ and $C(3;7)$ are on a straight line.}{
\westep{Draw a sketch}
\begin{center}
\scalebox{1} % Change this value to rescale the drawing.
{
\begin{pspicture}(0,-2.6667187)(6.4090624,2.7067187)
\psset{xunit=1.2,yunit=1.2}
\rput(3.01,-1.6667187){\psaxes[linewidth=1pt,arrowsize=0.05291667cm 2.0,arrowlength=1.4,arrowinset=0.4,ticksize=0.10583333cm,dx=0.6cm,dy=0.6cm]{<->}(0,0)(-3,-1)(3,4)}
\psdots[dotsize=0.12,dotangle=-5.9493704](4.4790277,1.8660182)
\usefont{T1}{ppl}{m}{n}
\rput(6.034531,-1.4167187){$x$}
\usefont{T1}{ppl}{m}{n}
\rput(3.2945313,2.5032814){$y$}
\usefont{T1}{ppl}{m}{n}
\rput(3.8445313,0.8232812){$B(0;5)$}
\usefont{T1}{ppl}{m}{n}
% \rput(2.8545313,-1.8767188){$0$}
\usefont{T1}{ppl}{m}{n}
\rput(0.77453125,-0.13671875){$A(-3;3)$}
\psdots[dotsize=0.12](1.53,-0.16671875)
\psdots[dotsize=0.12](3.01,0.8332813)
\usefont{T1}{ppl}{m}{n}
\rput(5.3745313,1.8832812){$C(3;7)$}
\end{pspicture} 
}

\end{center}

\westep{Calculate two gradients between any of the three points}
\begin{equation*}
 \begin{array}{rll}

m&=\dfrac{y_2-y_1}{x_2-x_1}&\\[6pt]
m_{AB} &= \dfrac{5-3}{0-(-3)} &= \dfrac{2}{3}
\end{array}
\end{equation*}
and
\begin{equation*}
 \begin{array}{rll}
m_{BC} &= \dfrac{7-5}{3-0} &= \dfrac{2}{3}
\end{array}
\end{equation*}
OR
\begin{equation*}
 \begin{array}{rlll}
m_{AC} &= \dfrac{3-7}{3-3} &= \dfrac{-4}{-6}&=\dfrac{2}{3}
\end{array}
\end{equation*}
and
\begin{equation*}
 \begin{array}{rll}
m_{BC} &= \dfrac{7-5}{3-0} &= \dfrac{2}{3}
\end{array}
\end{equation*}
\westep{Explain your answer}
\begin{equation*}
 \begin{array}{r[l}
m_{AB} &= m_{BC}&= m_{AC}
\end{array}
\end{equation*}
Therefore the points $A$, $B$ and $C$ are on a straight line.
}
\end{wex}

To prove that three points are on a straight line using the distance formula, we must calculate the
distances between each pair of points and then prove that the sum of the two smaller distances
equals the longest distance.


\begin{wex}{Points on a straight line}{Prove that $A(-3;3)$, $B(0;5)$ and $C(3;7)$ are on a straight line.}{
\westep{Draw a sketch}

\begin{center}
\scalebox{1} % Change this value to rescale the drawing.
{
\begin{pspicture}(0,-2.6667187)(6.4090624,2.7067187)
\psset{xunit=1.2,yunit=1.2}
\rput(3.01,-1.6667187){\psaxes[linewidth=1pt,arrowsize=0.05291667cm 2.0,arrowlength=1.4,arrowinset=0.4,ticksize=0.10583333cm,dx=0.6cm,dy=0.6cm]{<->}(0,0)(-3,-1)(3,4)}
\psdots[dotsize=0.12,dotangle=-5.9493704](4.4790277,1.8660182)
\usefont{T1}{ppl}{m}{n}
\rput(6.034531,-1.4167187){$x$}
\usefont{T1}{ppl}{m}{n}
\rput(3.2945313,2.5032814){$y$}
\usefont{T1}{ppl}{m}{n}
\rput(3.8445313,0.6){$B(0;5)$}
\usefont{T1}{ppl}{m}{n}
% \rput(2.8545313,-1.8767188){$0$}
\usefont{T1}{ppl}{m}{n}
\rput(0.77453125,-0.13671875){$A(-3;3)$}
\psdots[dotsize=0.12](1.53,-0.16671875)
\psdots[dotsize=0.12](3.01,0.8332813)
\usefont{T1}{ppl}{m}{n}
\rput(5.1,1.8832812){$C(3;7)$}
\psline[linewidth=1pt,linestyle=dashed,dash=0.16cm 0.16cm,arrowsize=0.05291667cm 2.0,arrowlength=1.4,arrowinset=0.4]{<->}(1.65,-0.12671874)(2.93,0.7732813)
\psline[linewidth=1pt,linestyle=dashed,dash=0.16cm 0.16cm,arrowsize=0.05291667cm 2.0,arrowlength=1.4,arrowinset=0.4]{<->}(3.09,0.91328126)(4.39,1.8132813)
\psline[linewidth=1pt,linestyle=dashed,dash=0.16cm 0.16cm,arrowsize=0.05291667cm 2.0,arrowlength=1.4,arrowinset=0.4]{<->}(1.79,-0.42671874)(4.49,1.5332812)
\end{pspicture} 
}

\end{center}
\westep{Calculate the three distances $AB$, $BC$ and $AC$}
\begin{equation*}
\begin{array}{rl}
&d_{AB} = \sqrt{(-3 - 0)^2 + (3 - 5)^2} = \sqrt{(-3)^2 + (-2)^2} = \sqrt{9 + 4} = \sqrt{13}\\

&d_{BC} = \sqrt{(0 - 3)^2 + (5 - 7)^2} = \sqrt{(-3)^2 + (-2)^2}= \sqrt{9 + 4} = \sqrt{13}\\

&d_{AC} = \sqrt{((-3) - 3)^2 + (3 - 7)^2} = \sqrt{(-6)^2 + (-4)^2} = \sqrt{36 + 16} = \sqrt{52}
\end{array}
\end{equation*}
\westep{Find the sum of the two shorter distances}
\begin{equation*}
d_{AB} + d_{BC} = \sqrt{13} + \sqrt{13} = 2\sqrt{13} = \sqrt{4 \times 13} = \sqrt{52}
\end{equation*}
\westep{Explain your answer}
\begin{equation*}
d_{AB} + d_{BC} = d_{AC}
\end{equation*}
therefore points $A$, $B$ and $C$ lie on the same straight line.
}
\end{wex}

\begin{exercises}{}{
\begin{enumerate}[itemsep=5pt, label=\textbf{\arabic*}. ]
\item Determine whether $AB$ and $CD$ are parallel, perpendicular or neither if:
  \begin{enumerate}[noitemsep, label=\textbf{(\alph*)} ]
  \item $A(3;-4)$, $B(5;2)$, $C(-1;-1)$, $D(7;23)$
  \item $A(3;-4)$, $B(5;2)$, $C(-1;-1)$, $D(0;-4)$
  \item $A(3;-4)$, $B(5;2)$, $C(-1;-1)$, $D(1;2)$
  \end{enumerate}
\item Determine whether the following points lie on the same straight line:
  \begin{enumerate}[noitemsep, label=\textbf{(\alph*)} ]
  \item $E(0;3)$, $F(-2;5)$, $G(2;1)$
  \item $H(-3;-5)$, $I(-0;0)$, $J(6;10)$
  \item $K(-6;2)$, $L(-3;1)$, $M(1;-1)$
  \end{enumerate}
\item Points $P(-6;2)$, $Q(2;-2)$ and $R(-3;r)$ lie on a straight line. Find the value of $r$.
\item Line $PQ$ with $P(-1;-7)$ and $Q(q;0)$ has a gradient of $1$. Find $q$.
\end{enumerate}
\practiceinfo
\par 
\par \begin{tabular}[h]{ccccc}
(1.) 00d2&  (2.) 00d3&  (3.) 00d4&  (4.) 00d5\end{tabular}
}
\end{exercises}
\par
% The following video provides a summary of the gradient of a line:
% \setcounter{subfigure}{0}
% \begin{figure}[H] % horizontal\label{m39108*uid993}
% \textnormal{Gradient of a line}\vspace{.1in} \nopagebreak
% \label{m39108*yt-media1}\label{m39108*yt-video1}
% \raisebox{-5 pt}{ \includegraphics[width=0.5cm]{col11306.imgs/summary_www.png}} { (Video:  MG10110 )}
% \vspace{2pt}
% \vspace{.1in}
% \end{figure}      
%          \section{ Midpoint of a line}
%     \nopagebreak
%%%             \label{m39119} $ \hspace{-5pt}\begin{array}{cccccccccccc}   \includegraphics[width=0.75cm]{col11306.imgs/summary_video.png} &   \end{array} $ \hspace{2 pt}\raisebox{-5 pt}{} {(section shortcode: MG10111 )} \par 
%     
%     
%     
\pagebreak
\section{Mid-point of a line}

\begin{Investigation}{Finding the mid-point of a line}
On graph paper, accurately plot the points $P(2;1)$ and $Q(-2;2)$ and draw the line $PQ$.
\begin{itemize}
\item Fold the piece of paper so that point $P$ is exactly on top of point $Q$.
\item Where the folded line intersects with line $PQ$, label point $S$.
\item Count the blocks and find the exact position of $S$.
\item Write down the coordinates of $S$.
\end{itemize}
\end{Investigation}


To calculate the coordinates of the mid-point  $M(x;y)$ of any line between two points $A(x_1;y_1)$ and $B(x_2;y_2)$ we use the following formulae:

\setcounter{subfigure}{0}
\begin{figure}[H] % horizontal\label{m39107*id63458}
\begin{center}
\scalebox{1}{
\begin{pspicture}(-5,-5)(5.5,5.5)
% \psgrid[gridcolor=lightgray,linecolor=lightgray,subgriddiv=1,gridlabels=0.0cm](0,0)(0,0)(5,5)
\psaxes[linewidth=1pt,labels=all,ticks=all]{<->}(0,0)(-2,-2)(5.5,5.5)
\psline[linewidth=1pt](-2,-1)(5,5)
\uput[l](-0.5,0.45){{$A (x_{1};y_{1})$}}
\uput[r](-0.5,0.45){\qdisk(0,0){2pt}}
\uput[r](2.5,2.6){{$M (x;y)$}}
\uput[u](2.5,2.7){\qdisk(0,0){2pt}}
\uput[ur](5,5){{$B (x_{2};y_{2})$}}
\uput[u](5,4.8){\qdisk(0,0){2pt}}
\uput[l](6.1,0){{$x$}}
\uput[d](0,6.2){{$y$}}
\uput[d](-0.2,0.1){{$0$}}
\end{pspicture}
}
\end{center}
\end{figure}      

\begin{eqnarray*}
x & = & \frac{{x}_{1} + {x}_{2}}{2} \\ 
y & = & \frac{{y}_{1} + {y}_{2}}{2} \\  
\end{eqnarray*}
From this we obtain the mid-point of a line:
\Identity{
\vspace*{-3em}
\begin{center}\mbox{Midpoint } $M(x;y)=(\dfrac{{x}_{1} + {x}_{2}}{2};\dfrac{{y}_{1}+{y}_{2}}{2}) $\end{center}}

\mindsetvid{the midpoint of a line segment}{VMbpr}

\begin{wex}{Calculating the mid-point}
 {Calculate the coordinates of the mid-point $F(x;y)$ of the line segment between point $E(2;1)$ and point $G(-2;-2)$.}
{
\westep{Draw a sketch}
\setcounter{subfigure}{0}
\begin{figure}[H] % horizontal\label{m39107*id63458}
\begin{center}
\scalebox{1}{
\begin{pspicture}(-5,-5)(5.5,5.5)
% \psaxes{<->}(0,0)(5,5)
% \psgrid[gridcolor=lightgray,linecolor=lightgray,subgriddiv=1,gridlabels=0.0cm](0,0)(-3,-3)(3,3)
\psaxes[linewidth=1pt,labels=all,ticks=all]{<->}(0,0)(-3,-3)(3,3)
\psline[linewidth=1pt](-2,-2)(2,1)
% \psline[linewidth=.05cm](0,0)(3,3.5)
% \psline[linewidth=.05cm](0,0)(3,2)
% \psline[linewidth=.05cm](0,0)(3,0.5)
% \uput[ur](.9,3.5){\Large{$T$}}
\uput[dl](-1,-2){$G(-2,-2)$}
\uput[u](-2,-2.2){\qdisk(0,0){2pt}}
\uput[r](0,-.7){$F(x;y)$}
\uput[r](0,3.2){$y$}
\uput[r](3,0){$x$}
\uput[r](-0.5,-0.3){$0$}
% \uput[r](0,-1.2){\Large{mid-point}}
\uput[u](0,-.7){\qdisk(0,0){2pt}}
\uput[ur](2,1){$E (2;1)$}
\uput[u](2,.8){\qdisk(0,0){2pt}}
% \uput[l](4.8,0){\Large{$x$}}
% \uput[d](0,4.8){\Large{$y$}}
\end{pspicture}
}
\end{center}
\end{figure} 
\westep{Assign values to $(x_1;y_1)$ and $(x_2;y_2)$}
\begin{equation*}
x_1 = -2 \hskip2em y_1 = -2 \hskip2em x_1 = 2 \hskip2em y_2 = 1
\end{equation*}
\westep{Write down the mid-point formula}
\begin{equation*}
F(x;y)=(\dfrac{{x}_{1} + {x}_{2}}{2};\dfrac{{y}_{1}+{y}_{2}}{2})
\end{equation*}

\westep{Substitute values into the mid-point formula}
\begin{eqnarray*}
x & = & \frac{{x}_{1} + {x}_{2}}{2} \\ [5pt]
& = & \frac{-2 + 2}{2} \\ [5pt]
& = & 0 \\ 
y & = & \frac{{y}_{1} + {y}_{2}}{2} \\ [5pt]
& = & \frac{-2 + 1}{2} \\ [5pt]
& = & -\frac{1}{2} 
\end{eqnarray*}
\westep{Write the answer}
The mid-point is at $F(0;-\frac{1}{2})$
\westep{Confirm answer using distance formula}
Using the distance formula, we can confirm that the distances from the mid-point to each end point are equal: 
\begin{eqnarray*}
PS & = & \sqrt{{({x}_{1} - {x}_{2})}^{2} + {({y}_{1} - {y}_{2})}^{2}} \\ 
& = & \sqrt{{(0 - 2)}^{2} + {(-0,5 - 1)}^{2}} \\ 
& = & \sqrt{{(-2)}^{2} + {(-1,5)}^{2}} \\ 
& = & \sqrt{4 + 2,25} \\ 
& = & \sqrt{6,25}
\end{eqnarray*}
and
\begin{eqnarray*}
QS & = & \sqrt{{({x}_{1} - {x}_{2})}^{2} + {({y}_{1} - {y}_{2})}^{2}} \\ 
& = & \sqrt{{(0 - (-2))}^{2} + {(-0,5 - (-2))}^{2}} \\ 
& = & \sqrt{{(0 + 2)}^{2}{+(-0,5 + 2)}^{2}} \\ 
& = & \sqrt{{(2)}^{2}{+(-1,5)}^{2}} \\ 
& = & \sqrt{4 + 2,25} \\ 
& = & \sqrt{6,25}
\end{eqnarray*}
As expected, $PS=QS$, therefore $F$ is the mid-point. 
}
\end{wex}
\vspace*{-20pt}
\begin{wex}{Calculating the mid-point}{Find the mid-point of line $AB$, given $A(6;2)$ and $B(-5;-1)$.}{
\westep{Draw a sketch}
\begin{center}
\vspace*{-16.5pt}
\scalebox{1} % Change this value to rescale the drawing.
{
\begin{pspicture}(0,-3.0667188)(7.8490624,3.1067188)
\psset{xunit=1.2,yunit=1.2}
\rput(3.23,-0.06671875){\psaxes[linewidth=1pt,arrowsize=0.05291667cm 2.0,arrowlength=1.4,arrowinset=0.4,ticksize=0.10583333cm,dx=0.6cm,dy=0.6cm]{<->}(0,0)(-3,-3)(4,3)}
\psdots[dotsize=0.12](3.47,0.17328125)
\psdots[dotsize=0.12](6.23,0.91328126)
\psline[linewidth=1pt](0.79,-0.5467188)(6.25,0.91328126)
\usefont{T1}{ppl}{m}{n}
\rput(7.2,0.14328125){$x$}
\usefont{T1}{ppl}{m}{n}
\rput(3.5145311,2.9032812){$y$}
\usefont{T1}{ppl}{m}{n}
\rput(6.3745313,1.1232812){$A~(6;2)$}
\usefont{T1}{ppl}{m}{n}
\rput(3.1,-0.25){$0$}
\psdots[dotsize=0.12](0.75,-0.5467188)
\usefont{T1}{ppl}{m}{n}
\rput(4.0445313,0.65){$M(x;y)$}
\usefont{T1}{ppl}{m}{n}
\rput(0.84453124,-0.83671874){$B(-5;-1)$}
\end{pspicture} 
}


\end{center}

From the sketch, we can estimate that $M$ will lie in quadrant I, with positive $x$- and $y$-coordinates.
\westep{Assign values to $(x_1;y_1)$ and $(x_2;y_2)$}
Let the mid-point be $M(x;y)$
\begin{equation*}
x_1= 6 \hskip2em y_1=2 \hskip2em x_2=-5 \hskip2em y_2=-1
\end{equation*}
\westep{Write down the mid-point formula}
\begin{equation*}
M(x;y) = \left(\frac{x_1+x_2}{2};\frac{y_1+y_2}{2}\right)
\end{equation*}
\westep{Substitute values and simplify}
\begin{equation*}
M(x;y) = \left(\frac{6-5}{2};\frac{2-1}{2}\right) = \left(\frac{1}{2};\frac{1}{2}\right)
\end{equation*}
\westep{Write the final answer}
$M(\frac{1}{2};\frac{1}{2})$ is the mid-point of line $AB$.
}
\end{wex}

\begin{wex}{Using the mid-point formula}{The line joining $C(-2;4)$ and $D(x;y)$ has the mid-point $M(1;-3)$. Find point $D$.}{
\westep{Draw a sketch}
\begin{center}
\scalebox{1} % Change this value to rescale the drawing.
{
\begin{pspicture}(0,-4.224219)(7.5390625,4.224219)
\psset{xunit=1.2,yunit=1.2}
\rput(3.0,0.93078125){\psaxes[linewidth=1pt,arrowsize=0.05291667cm 2.0,arrowlength=1.4,arrowinset=0.4,ticksize=0.10583333cm,dx=0.6cm,dy=0.6cm]{<->}(0,0)(-3,-5.5)(4,3)}
\psdots[dotsize=0.12](3.5,-0.58921874)
\psdots[dotsize=0.12](5.0,-4.0492187)
\psline[linewidth=1pt](1.96,2.9107811)(5.0,-4.0292187)
\usefont{T1}{ppl}{m}{n}
\rput(7,1.2207812){$x$}
\usefont{T1}{ppl}{m}{n}
\rput(3.2,4.020781){$y$}
\usefont{T1}{ppl}{m}{n}
\rput(5.65,-3.9992187){$D(x;y)$}
\usefont{T1}{ppl}{m}{n}
\rput(3.1245313,0.72078127){$0$}
\psdots[dotsize=0.12](1.96,2.9107811)
\usefont{T1}{ppl}{m}{n}
\rput(4.4,-0.5992187){$M(1;-3)$}
\usefont{T1}{ppl}{m}{n}
\rput(1.7145313,3.1207812){$C(-2;4)$}
\psline[linewidth=1pt](2.638401,1.7359167)(2.3761587,1.5784051)
\psline[linewidth=1pt](2.687574,1.6269872)(2.4253316,1.4694756)
\psline[linewidth=1pt](4.478401,-2.4440832)(4.216159,-2.601595)
\psline[linewidth=1pt](4.547574,-2.5530128)(4.2853317,-2.7105243)
\end{pspicture} 
}
\end{center}
From the sketch, we can estimate that $D$ will lie in Quadrant IV, with a positive $x$- and negative $y$-coordinate.
\westep{Assign values to $(x_1;y_1)$ and $(x_2;y_2)$}
Let the coordinates of $C$ be $(x_1;y_1)$ and the coordinates of $D$ be $(x_2;y_2)$.
\begin{equation*}
x_1=-2 \hskip2em y_1=4 \hskip2em x_2=x \hskip2em y_2=y
\end{equation*}
\westep{Write down the mid-point formula}
\begin{equation*}
M(x;y) = \left(\frac{x_1+x_2}{2}; \frac{y_1+y_2}{2}\right)
\end{equation*}
\westep{Substitute values and solve for $x_2$ and $y_2$}
\begin{equation*}
\begin{array}{rllrl}
1&=\dfrac{-2+x_2}{2}&\hskip10em&-3&=\dfrac{4+y_2}{2}\\[5pt]
1\times 2&=-2+x_2&&-3\times 2&=4+y_2\\
2&=-2+x_2&&-6&=4+y_2\\
x_2&=2+2&&y_2&=-6-4
\end{array}
\end{equation*}
\begin{equation*}
\begin{array}{rllrl}
x_2&=4&&y_2&=-10\\
\end{array}
\end{equation*}
\westep{Write the final answer}
The coordinates of point $D$ are $(4;-10)$.
}
\end{wex}
\vspace*{-30pt}
\begin{wex}{Using the mid-point formula}{Points $E(-1;0)$, $F(0;3)$, $G(8;11)$ and $H(x;y)$ are points on the Cartesian plane. Find $H(x;y)$ if $EFGH$ is a parallelogram.}{
\westep{Draw a sketch}
\begin{center}
\vspace*{-20pt}
\scalebox{1} % Change this value to rescale the drawing.
{
\begin{pspicture}(0,-3.6667187)(8.549063,3.7067187)
\psset{xunit=1.2,yunit=1.2}
\rput(1.87,-2.6667187){\psaxes[linewidth=0.04,arrowsize=0.05291667cm 2.0,arrowlength=1.4,arrowinset=0.4,ticksize=0.10583333cm,dx=0.6cm,dy=0.6cm]{<->}(0,0)(-1,-1)(5,6)}
\psdots[dotsize=0.12](5.87,2.8332813)
\psline[linewidth=0.04cm](1.89,-1.1667187)(5.85,2.8132813)
\psdots[dotsize=0.12](1.87,-1.1667187)
\usefont{T1}{ppl}{m}{n}
\rput(7,-2.4567187){$x$}
\usefont{T1}{ppl}{m}{n}
\rput(2.1545312,3.5032814){$y$}
\usefont{T1}{ppl}{m}{n}
\rput(6.534531,2.9832811){$G(8;11)$}
\usefont{T1}{ppl}{m}{n}
\rput(2,-2.8567188){$0$}
\usefont{T1}{ppl}{m}{n}
\rput(6.304531,1.4032812){$H(x;y)$}
\usefont{T1}{ppl}{m}{n}
\rput(0.76453125,-2.3967187){$E(-1;0)$}
\psline[linewidth=0.04cm](1.87,-1.1467187)(1.35,-2.7267187)
\psline[linewidth=0.04cm,linestyle=dashed,dash=0.16cm 0.16cm](1.43,-2.6267188)(5.55,1.3932812)
\psdots[dotsize=0.12](1.37,-2.6667187)
\usefont{T1}{ppl}{m}{n}
\rput(0.9,-1.1567187){$F(0;3)$}
\psline[linewidth=0.02cm](1.51,-2.4067187)(5.87,2.7932813)
\psline[linewidth=0.02cm](1.89,-1.1867187)(5.55,1.4532813)
\usefont{T1}{ppl}{m}{n}
\rput(3.5445313,-0.27671874){$M$}
\psline[linewidth=0.04cm,linestyle=dashed,dash=0.16cm 0.16cm](5.87,2.7732813)(5.57,1.4932812)
\end{pspicture} 
}
\end{center}

Method: the diagonals of a parallelogram bisect each other, therefore the mid-point of $EG$
will be the same as the mid-point of $FH$. We must first find the mid-point of $EG$. We can then use it to determine the coordinates of point $H$.
\westep{Assign values to $(x_1;y_1)$ and $(x_2;y_2)$}
Let the mid-point of $EG$ be $M(x;y)$
\begin{equation*}
x_1=-1 \hskip2em y_1=0 \hskip2em x_2=8 \hskip2em y_2=11
\end{equation*}
\westep{Write down the mid-point formula}
\begin{equation*}
 M(x;y) =\left(\frac{x_1+x_2}{2}; \frac{y_1+y_2}{2}\right)
\end{equation*}
\westep{Substitute values calculate the coordinates of $M$}
\begin{equation*}
M(x;y) =\left(\frac{-1+8}{2}; \frac{0+11}{2}\right) = \left(\frac{7}{2};\frac{11}{2}\right)
\end{equation*}

\westep{Use the coordinates of $M$ to determine $H$}
$M$ is also the mid-point of $FH$ so we use 
$M(\frac{7}{2};\frac{11}{2})$ and $F(0;3)$ to solve for $H(x;y)$
\westep{Substitute values and solve for $x$ and $y$}

\begin{equation*}
\begin{array}{rllrl}
 M\left(\dfrac{7}{2};\dfrac{11}{2}\right) &=~\left(\dfrac{x_1+x_2}{2}; \dfrac{y_1+y_2}{2}\right)\\[8pt]
\dfrac{7}{2}&=~\dfrac{0+x}{2}&\hskip10em&\dfrac{11}{2}&=~\dfrac{3+y}{2}\\
7&=~x+0&&11&=~3+y\\
x&=~7&&y&=~8\\
\end{array}
\end{equation*}
\westep{Write the final answer}

The coordinates of $H$ are $(7;8)$.
}
\end{wex}
%\Tip{Remember to draw sketches!}
\begin{exercises}{}{
\begin{enumerate}[itemsep=5pt, label=\textbf{\arabic*}. ]
\item Find the mid-points of the following lines:
  \begin{enumerate}[noitemsep, label=\textbf{(\alph*)} ]
\item $A(2;5)$, $B(-4;7)$
\item $C(5;9)$, $D(23;55)$
\item $E(x+2;y-1)$, $F(x-5;y-4)$
  \end{enumerate}
    \item The mid-point $M$ of $PQ$ is $(3;9)$. Find $P$ if $Q$ is $(-2;5)$.
    \item $PQRS$ is a parallelogram with the points $P(5;3)$, $Q(2;1)$ and $R(7;-3)$. Find point $S$.

    \end{enumerate}
\practiceinfo
\par 
\par \begin{tabular}[h]{ccccc}
(1.) 00d6&  (2.) 00d7&  (3.) 00d8\end{tabular}
}
\end{exercises}    

\summary{}
\begin{itemize}[noitemsep]
\item A point is an ordered pair of numbers written as $(x;y)$.
\item Distance is a measure of the length between two points.
\item The formula for finding the distance between  any two points is: 
\begin{equation*}
d=\sqrt{{({x}_{1}-{x}_{2})}^{2}+{({y}_{1}-{y}_{2})}^{2}}
\end{equation*}
\item The gradient between two points is determined by the ratio of vertical change to horizontal change.

\item The formula for finding the gradient of a line is: 
\begin{equation*}
m=\frac{{y}_{2}-{y}_{1}}{{x}_{2}-{x}_{1}}
\end{equation*}
\item A straight line is a set of points with a constant gradient between any two of the
points.
\item The standard form of the straight line equation is $y=mx+c$.
\item The equation of a straight line can also be written as  
\begin{equation*}
\dfrac{y-y_1}{x-x_1}=\dfrac{y_2-y_1}{x_2-x_1}
\end{equation*}
\item If two lines are parallel, their gradients are equal.
\item If two lines are perpendicular, the product of their gradients is equal to $-1$.
\item For horizontal lines the gradient is equal to $0$.
\item For vertical lines the gradient is undefined.

\item The formula for finding the mid-point between two points is: 
\begin{equation*}
M(x;y) = \left(\frac{{x}_{1}+{x}_{2}}{2};\frac{{y}_{1}+{y}_{2}}{2}\right)
\end{equation*}
\end{itemize}


\begin{eocexercises}{}
  \begin{enumerate}[noitemsep, label=\textbf{\arabic*}. ] 
  \item Represent the following figures in the Cartesian plane: 
    \begin{enumerate}[noitemsep, label=\textbf{(\alph*)} ]
    \item Triangle $DEF$ with $D(1;2)$, $E(3;2)$ and $F(2;4)$ 
    \item Quadrilateral $GHIJ$ with $G(2;-1)$, $H(0;2)$, $I(-2;-2)$ and $J(1;-3)$
    \item Quadrilateral $MNOP$ with $M(1;1)$, $N(-1;3)$, $O(-2;3)$ and $P(-4;1)$ 
    \item Quadrilateral $WXYZ$ with $W(1;-2)$, $X(-1;-3)$, $Y(2;-4)$ and $Z(3;-2)$
    \end{enumerate}
  \item In the diagram below, the vertices of the quadrilateral are $F(2;0)$, $G(1;5)$, $H(3;7)$ and $I(7;2)$.
    \setcounter{subfigure}{0}
    \begin{figure}[H] % horizontal\label{m39107*id63458}
      \begin{center}
        \scalebox{1}{
          \begin{pspicture}(-5,-5)(5.5,5.5)

            % \psaxes{<->}(0,0)(5,5)
            % \psgrid[gridcolor=lightgray,linecolor=lightgray,subgriddiv=1](0,0)(-1,-1)(7,7)
            \psaxes[linewidth=1pt,labels=all,ticks=all]{<->}(0,0)(-1,-1)(7.5,7.5)
            \pspolygon[linewidth=1pt](2,0)(1,5)(3,7)(7,2)(2,0)
            \psdots(2,0)(1,5)(3,7)(7,2)
            % \psline[linewidth=.05cm](0,0)(3,3.5)
            % \psline[linewidth=.05cm](0,0)(3,2)
            % \psline[linewidth=.05cm](0,0)(3,0.5)
            % \uput[ur](.9,3.5){\Large{$T$}}
            \uput[d](1.3,.6){\normalsize{$F (2;0)$}}
            \uput[u](0.8,5.1){\normalsize{$G (1;5)$}}
            \uput[u](3,7){\normalsize{$H (3;7)$}}
            \uput[r](7,2){\normalsize{$I (7;2)$}}
            \uput[l](8,0){{$x$}}
            \uput[d](0,8){{$y$}}
            \uput[d](-0.2,0){{$0$}}
          \end{pspicture}
        }
      \end{center}
    \end{figure}  

    \begin{enumerate}[noitemsep, label=\textbf{(\alph*)} ]
    \item Calculate the lengths of the sides of $FGHI$.
    \item Are the opposite sides of $FGHI$ parallel?
    \item Do the diagonals of $FGHI$ bisect each other?
    \item Can you state what type of quadrilateral $FGHI$ is? Give reasons for your answer.
    \end{enumerate}
  \item Consider a quadrilateral $ABCD$ with vertices $A(3;2)$, $B(4;5)$, $C(1;7)$ and $D(1;3)$.
    \begin{enumerate}[noitemsep, label=\textbf{(\alph*)} ]
    \item  Draw the quadrilateral.
    \item  Find the lengths of the sides of the quadrilateral.
    \end{enumerate}
  \item $ABCD$ is a quadrilateral with vertices $A(0;3)$, $B(4;3)$, $C(5;-1)$ and $D(-1;-1)$.
    \begin{enumerate}[noitemsep, label=\textbf{(\alph*)} ]
    \item Show by calculation that:
      \begin{enumerate}[noitemsep, label=\textbf{\roman*}. ] 
      \item $AD = BC$
      \item $AB \parallel DC$
      \end{enumerate}
    \item What type of quadrilateral is $ABCD$?
    \item Show that the diagonals $AC$ and $BD$ do not bisect each other.
    \end{enumerate}
  \item $P$, $Q$, $R$ and $S$ are the points $(-2;0)$, $(2;3)$, $(5;3)$, $(-3;-3)$ respectively.
    \begin{enumerate}[noitemsep, label=\textbf{(\alph*)} ]
    \item Show that:
      \begin{enumerate}[noitemsep, label=\textbf{\roman*}. ] 
      \item $SR = 2PQ$
      \item $SR \parallel PQ$
      \end{enumerate}
    \item Calculate:
      \begin{enumerate}[noitemsep, label=\textbf{\roman*}. ] 
      \item $PS$
      \item $QR$
      \end{enumerate}
    \item What kind of quadrilateral is $PQRS$? Give reasons for your answer.
    \end{enumerate}
  \item $EFGH$ is a parallelogram with vertices $E(-1;2)$, $F(-2;-1)$ and $G(2;0)$. Find the coordinates of $H$ by using the fact that the diagonals of a parallelogram bisect each other.
  \item  $PQRS$ is a quadrilateral with points $P(0;-3)$, $Q(-2;5)$, $R(3;2)$ and $S(3;-2)$  in the Cartesian plane.
    \begin{enumerate}[noitemsep, label=\textbf{(\alph*)} ]
    \item Find the length of $QR$.
    \item Find the gradient of $PS$.
    \item Find the mid-point of $PR$.
    \item Is $PQRS$ a parallelogram?  Give reasons for your answer.
    \end{enumerate}
  \item $A(-2;3)$ and $B(2;6)$ are points in the Cartesian plane. $C(a;b)$ is the mid-point of $AB$. Find the values of $a$ and $b$.
  \item Consider triangle $ABC$ with vertices $A(1; 3)$, $B(4;1)$ and $C(6; 4)$.
    \begin{enumerate}[noitemsep, label=\textbf{(\alph*)} ]
    \item Sketch triangle $ABC$ on the Cartesian plane. 
    \item Show that $ABC$ is an isosceles triangle.
    \item Determine the coordinates of $M$, the mid-point of $AC$.
    \item Determine the gradient of $AB$.
    \item Show that $D(7;-1)$ lies on the line that goes through $A$ and $B$.
    \end{enumerate}
\clearpage
  \item In the diagram, $A$ is the point $(-6;1)$ and $B$ is the point $(0;3)$
    \setcounter{subfigure}{0}
    \begin{figure}[H] % horizontal\label{m39107*id63458}
      \begin{center}
        \scalebox{1}{
          \begin{pspicture}(-5,-5)(5.5,5.5)

            % \psaxes{<->}(0,0)(5,5)
            % \psgrid[gridcolor=lightgray,linecolor=lightgray,subgriddiv=1](0,0)(-10,-1)(1,7)
            \psaxes[linewidth=1pt,labels=all,ticks=all]{<->}(0,0)(-10,-1)(1,7)
            \psline[linewidth=.05cm](-6,1)(0,3)
            % \psline[linewidth=.05cm](0,0)(3,3.5)
            % \psline[linewidth=.05cm](0,0)(3,2)
            % \psline[linewidth=.05cm](0,0)(3,0.5)
            % \uput[ur](.9,3.5){\Large{$T$}}
            \uput[d](-6,1){{$A (-6;1)$}}
            \uput[d](-5.9,1.2){\qdisk(0,0){2pt}}
            \uput[r](0,3){{$B (0;3)$}}
            \uput[d](0,3.2){\qdisk(0,0){2pt}}
            \uput[l](1.5,0){{$x$}}
            \uput[d](0,7.5){{$y$}}
            \uput[d](-0.2,0.1){{$0$}}
          \end{pspicture}
        }
      \end{center}
    \end{figure} 
    \begin{enumerate}[noitemsep, label=\textbf{(\alph*)} ]
    \item Find the equation of line $AB$.
    \item Calculate the length of $AB$.
      % \item  $A'$ is the image of $A$ and $B'$ is the image of $B$. Both these images are obtain by applying the transformation: $(x;y)\to(x - 4;y - 1)$. Give the coordinates of both $A'$ and $B'$
      % \item Find the equation of $A'B'$
      % \item Calculate the length of $A'B'$
      % \item Can you state with certainty that $AA'B'B$ is a parallelogram? Justify your answer.
    \end{enumerate}

\item $\triangle PQR$ has vertices $P(1;8)$, $Q(8;7)$ and $R(7;0)$. Show through calculation that $\triangle PQR$ is a right angled isosceles triangle.
\item $\triangle ABC$ has vertices $A(-3;4)$, $B(3;-2)$ and $R(-5;-2)$. $M$ is the midpoint of $AC$  and $N$ the midpoint of $BC$. Use $\triangle ABC$ to prove the midpoint theorem using analytical geometrical methods. 

  \end{enumerate}
\practiceinfo
\par 
\par \begin{tabular}[h]{ccccc}
(1.) 00d9&  (2.) 00da&  (3.) 00db&  (4.) 00dc&  (5.) 00dd&  (6.) 00de&  (7.) 00df&  (8.) 00dg&  (9.) 00dh&  (10.) 00di & (11.) 022r & (12.) 022s & \end{tabular}
\end{eocexercises}
