         \chapter{Number patterns}
    \setcounter{figure}{1}
    \setcounter{subfigure}{1}
            
In earlier grades you saw patterns in the form of pictures and numbers. In this chapter, we learn more about the mathematics of patterns. Patterns are repetitive sequences and can be found in nature, shapes, events, sets of numbers and almost everywhere you care to look. For example, seeds in a sunflower, snowflakes, geometric designs on quilts or tiles, the number sequence $0;4;8;12;16;\ldots$.\par 
See if you can spot any patterns in the following sequences: 
\begin{enumerate}[noitemsep, label=\textbf{\arabic*}. ] 
    \item $2;~4;~6;~8;~10;\ldots$
    \item $1;~2;~4;~7;~11;\ldots$
    \item $1;~4;~9;~16;~25;\ldots$
    \item $5;~10;~20;~40;~80;\ldots$
\end{enumerate}

\chapterstartvideo{VMcyv}   

% \section{Number pattern examples}
Numbers can have interesting patterns. Here we examine some types of patterns and how they are formed.
\par
Examples:
\begin{enumerate}[noitemsep, label=\textbf{\arabic*}. ] 
    \item $1;~4;~7;~10;~13;~16;~19;~22;~25;~\ldots$\\
    There is difference of $3$ between successive terms.\\
    The pattern is continued by adding $3$ to the last term.
    \item $3;~8;~13;~18;~23;~28;~33;~38;~\ldots$\\
    There is a difference of $5$ between successive terms.\\
    The pattern is continued by adding $5$ to the last term.
    \item $2;~4;~8;~16;~32;~64;~128;~256;~\ldots$\\
    This sequence has a factor of $2$ between successive terms.\\
    The pattern is continued by multiplying the last term by $2$.
    \item $3;~9;~27;~81;~243;~729;~2187;~\ldots$\\
    This sequence has a factor of $3$ between successive terms.\\
    The pattern is continued by multiplying the last term by $3$.
\end{enumerate}
     

% \subsection*{Special sequences}
% 
% \subsubsection*{Triangular numbers}
% 
% % \psset{xunit=1.0cm,yunit=1.0cm,algebraic=true,dotstyle=o,dotsize=3pt 0,linewidth=0.8pt,arrowsize=3pt 2,arrowinset=0.25}
% \begin{pspicture*}(-5.3,-3.74)(7.32,6.48)
% \rput[tl](-5.1,0.54){$1$}
% \rput[tl](-2.6,0.54){$3$}
% \rput[tl](0.94,0.5){$6$}
% \rput[tl](5.42,0.54){$10$}
% \psdots[dotstyle=*](0,1)
% \psdots[dotstyle=*](1,1)
% \psdots[dotstyle=*](2,1)
% \psdots[dotstyle=*](1.52,2)
% \psdots[dotstyle=*](0.6,2)
% \psdots[dotstyle=*](1.14,2.98)
% \psdots[dotstyle=*](-5,1)
% \psdots[dotstyle=*](-3,1)
% \psdots[dotstyle=*](-2,1)
% \psdots[dotstyle=*](-2.46,2)
% \psdots[dotstyle=*](7,1)
% \psdots[dotstyle=*](4,1)
% \psdots[dotstyle=*](5,1)
% \psdots[dotstyle=*](6,1)
% \psdots[dotstyle=*](5.46,2)
% \psdots[dotstyle=*](4.48,2)
% \psdots[dotstyle=*](6.32,2)
% \psdots[dotstyle=*](6,3)
% \psdots[dotstyle=*](5,3)
% \psdots[dotstyle=*](5.54,3.96)
% \end{pspicture*}
% $\ldots$\\
% \\
% $1;~3;~6;~10;~15;~21;~28;~36;~45;\ldots$\\
% \\
% This sequence is generated from a pattern of dots which form a triangle.
% By adding another row of dots that has one more dot than the previous row and then counting all the dots, we can find the next term in the sequence.\par 
% 
% \subsubsection*{Square numbers}
% $1;~4;~9;~16;~25;~36;~49;~64;~81;\ldots$\\
% The next term is made by squaring the number of the position in the pattern.\\
% The $2^{\mathrm{nd}}$ term is ${2}^{2} = 4$.
% The $7^{\mathrm{th}}$ term is ${7}^{2} = 49$. What would the $10^{\mathrm{th}}$ term be?
%             
% \subsubsection*{Cube numbers}
% $1;~8;~27;~64;~125;~216;~343;~512;~729;\ldots$\\
% The next term is made by cubing the number of the position in the pattern.\\
% The $2^{\mathrm{nd}}$ term is $2^{3}=8$.
% The $7^{\mathrm{th}}$ term is $7^{3}=343$. What would the $10^{\mathrm{th}}$ term be?
% 
% \subsubsection*{Fibonacci numbers}
% $0;~1;~1;~2;~3;~5;~8;~13;~21;~34;\ldots$\\
% The next term is made by adding together the two terms before it.\\
% The $4^{\mathrm{th}}$ term is found by adding together the two terms before it $1+1=2$.\\
% The $9^{\mathrm{th}}$ term is found by adding together the two terms before it $8+13=21$.\\
% The next term in the sequence above would be $21+34=55$.
% Can you work out the next few terms?
% 
% %\setcounter{subfigure}{0}
% %\begin{figure}[H] 
% %\textnormal{Khan Academy video on Number Patterns - 1}\vspace{.1in} \nopagebreak
% %\raisebox{-5 pt}{ \includegraphics[width=0.5cm]{col11306.imgs/summary_www.png}} { (Video:  MG10023 )}
% %\end{figure}       
% \mindsetvid{Number Patterns}{MG10023}

\begin{wex}{Study table}{You and $3$ friends decide to study for Maths and are sitting together at a square table. A few minutes later, $2$ other friends arrive and would like to sit at your table. You move another table next to yours so that $6$ people can sit at the table. Another $2$ friends also want to join your group, so you take a third table and add it to the existing tables. Now $8$ people can sit together. \\
\\
Examine how the number of people sitting is related to the number of tables. Is there a pattern?
\begin{figure}[H]
\begin{center}
\footnotesize\begin{pspicture}(0,0)(9.6,1.8)
%\psgrid[gridcolor=lightgray]
\psframe(0.4,0.4)(1.4,1.4)
\psframe(0,0.6)(0.2,1.2)
\psframe(0.6,0)(1.2,0.2)
\psframe(0.6,1.6)(1.2,1.8)
\psframe(1.6,0.6)(1.8,1.2)
\rput(2.4,0){\psframe(0.4,0.4)(1.4,1.4)
\psframe(0,0.6)(0.2,1.2)
\psframe(0.6,0)(1.2,0.2)
\psframe(0.6,1.6)(1.2,1.8)}
\rput(3.4,0){\psframe(0.4,0.4)(1.4,1.4)
\psframe(0.6,0)(1.2,0.2)
\psframe(0.6,1.6)(1.2,1.8)
\psframe(1.6,0.6)(1.8,1.2)}
\rput(5.8,0){\psframe(0.4,0.4)(1.4,1.4)
\psframe(0,0.6)(0.2,1.2)
\psframe(0.6,0)(1.2,0.2)
\psframe(0.6,1.6)(1.2,1.8)}
\rput(6.8,0){\psframe(0.4,0.4)(1.4,1.4)
\psframe(0.6,0)(1.2,0.2)
\psframe(0.6,1.6)(1.2,1.8)}
\rput(7.8,0){\psframe(0.4,0.4)(1.4,1.4)
\psframe(0.6,0)(1.2,0.2)
\psframe(0.6,1.6)(1.2,1.8)
\psframe(1.6,0.6)(1.8,1.2)}
\end{pspicture}\normalsize
\\
\begin{caption*}Two more people can be seated for each table added.\end{caption*}
\label{fig:mp:s:arithmetictables}
\end{center}
\end{figure}
}
{
\westep{Make a table to see if a pattern forms}

\begin{center} 
\begin{tabular}{|c|l|}
\hline \textbf{Number of Tables}, $n$ & \textbf{Number of people seated}\\\hline
 $1$ & $4 = 4$\\
\hline $2$ & $4 + 2 = 6$\\
\hline $3$ & $4 + 2 + 2 = 8$\\
\hline $4$ & $4 + 2 + 2 + 2 = 10$ \\
\hline \vdots & \qquad \qquad \quad \vdots\\
\hline $n$ & $4 + 2 + 2 + 2 + \cdots + 2 $\\
\hline
\end{tabular}
\end{center}
\westep{Describe the pattern}
We can see that for $3$ tables we can seat $ 8$ people, for $4$ tables we can seat $10 $ people and so on. We started out with $4$ people and added two each time. 
So for each table added, the number of people increased by $2$.
}
\end{wex}

\section{Describing sequences}


%\setcounter{subfigure}{0}
%\begin{figure}[H]
%\textnormal{Khan Academy video on Number Patterns}\vspace{.1in} \nopagebreak
% \label{m39362*yt-media3}\label{m39362*yt-video3}
%\raisebox{-5 pt}{ \includegraphics[width=0.5cm]{col11306.imgs/summary_www.png}} { (Video:  MG10025 )}
%\end{figure}       

 
To describe terms in a number pattern we use the following notation:\\
\\
The $1^{\mathrm{st}}$ term of a sequence is $T_{1}$.\\
The $4^{\mathrm{th}}$ term of a sequence is $T_{4}$.\\
The $10^{\mathrm{th}}$ term of a sequence is $T_{10}$.\\
\\
The general term is often expressed as the ${n}^{\mathrm{th}}$ term and is written as ${T}_{n}$. \\
\\A sequence does not have to follow a pattern but, when it does, we can write down the general formula to calculate any term.
For example, consider the following linear sequence:
     
\begin{equation*}
  1;~3;~5;~7;~9;~\ldots
\end{equation*}
The ${n}^{\mathrm{th}}$ term is given by the general formula
\begin{equation*}
  T_n = 2n-1
\end{equation*}
You can check this by substituting values into the formula:
\begin{equation*}
    \begin{array}{ccc}\hfill {T}_{1}& =& 2(1)-1 = 1 \hfill \\ 
    \hfill {T}_{2}& =& 2(2)-1 = 3\hfill \\
    \hfill {T}_{3}& =& 2(3)-1 = 5\hfill \\
    \hfill {T}_{4}& =& 2(4)-1 = 7\hfill \\
    \hfill {T}_{5}& =& 2(5)-1 = 9\hfill \\
\end{array}
\end{equation*}

If we find the relationship between the position of a term and its value, we can describe the pattern and find any term in the sequence.
\par
\mindsetvid{Number Patterns}{MG10025}

\subsection*{Common difference}
In some sequences, there is a constant difference between any two
successive terms. This is called a \textit{common difference}.

\Definition{Common difference} {The common difference is the difference between any term and the term before it and is denoted by $d$. } 

For example, consider the sequence $10;~7;~4;~1;~\ldots$ 
\\To calculate
the common difference, we find the difference between any term and the
previous term:

\begin{equation*}
    \begin{array}{ccl}d &=& T_{2} - T_{1}\\
      & =& 7-10\\& =&-3\end{array}
\end{equation*}

Let us check another two terms:
\begin{equation*}
    \begin{array}{ccl}d &=& T_{4} - T_{3}\\
    & =& 1-4\\& =&-3\end{array}
\end{equation*}
We see that $d$ is constant.
\par
\textbf{Important:} $d=T_{2}-T_{1}$, ~not $T_{1} -T_{2}$.
       
\begin{wex}
{Study table, continued}
{
\begin{minipage}{\textwidth}
As before, you and $3$ friends are studying for Maths and are sitting together at a square table. 
A few minutes later $2$ other friends arrive so you move another table next to yours. Now $6$ people can sit at the table. 
Another $2$ friends also join your group, so you take a third table and add it to the existing tables. Now $8$ people can sit together as shown below.\\
\begin{enumerate}[noitemsep, label=\textbf{\arabic*}.]
\item Find the expression for the number of people seated at $n$ tables. 
\item Use the general formula to determine how many people can sit around $12$ tables.
\item How many tables are needed to seat $20$ people?
\end{enumerate}

\begin{figure}[H]
\begin{center}
\footnotesize\begin{pspicture}(0,0)(9.6,1.8)
%\psgrid[gridcolor=lightgray]
\psframe(0.4,0.4)(1.4,1.4)
\psframe(0,0.6)(0.2,1.2)
\psframe(0.6,0)(1.2,0.2)
\psframe(0.6,1.6)(1.2,1.8)
\psframe(1.6,0.6)(1.8,1.2)
\rput(2.4,0){\psframe(0.4,0.4)(1.4,1.4)
\psframe(0,0.6)(0.2,1.2)
\psframe(0.6,0)(1.2,0.2)
\psframe(0.6,1.6)(1.2,1.8)}
\rput(3.4,0){\psframe(0.4,0.4)(1.4,1.4)
\psframe(0.6,0)(1.2,0.2)
\psframe(0.6,1.6)(1.2,1.8)
\psframe(1.6,0.6)(1.8,1.2)}
\rput(5.8,0){\psframe(0.4,0.4)(1.4,1.4)
\psframe(0,0.6)(0.2,1.2)
\psframe(0.6,0)(1.2,0.2)
\psframe(0.6,1.6)(1.2,1.8)}
\rput(6.8,0){\psframe(0.4,0.4)(1.4,1.4)
\psframe(0.6,0)(1.2,0.2)
\psframe(0.6,1.6)(1.2,1.8)}
\rput(7.8,0){\psframe(0.4,0.4)(1.4,1.4)
\psframe(0.6,0)(1.2,0.2)
\psframe(0.6,1.6)(1.2,1.8)
\psframe(1.6,0.6)(1.8,1.2)}
\end{pspicture}\normalsize
\\
\begin{caption*}Two more people can be seated for each table added.\end{caption*}
\label{fig:mp:s:arithmetictables2}
\end{center}
\end{figure}
\end{minipage}
}{
\westep{Make a table to see the pattern}
\begin{center}
\begin{tabular}{|c|l|c|}
\hline \textbf{Number of Tables}, $n$ & \textbf{Number of people seated} & \textbf{Pattern}\\
\hline 1 & $4 = 4$ & $= 4 + 2 (0)$ \\
\hline 2 & $4 + 2 = 6$ & $= 4 + 2 (1)$ \\
\hline 3 & $4 + 2 + 2 = 8$ & $= 4 + 2 (2)$ \\
\hline 4 & $4 + 2 + 2 + 2 = 10$ & $= 4 + 2(3)$ \\
\hline \vdots & \qquad \qquad \quad \vdots & \vdots \\
\hline $n$ & $4 + 2 + 2 + 2 + \cdots + 2 $ & \: \: \: $= 4 + 2 (n-1)$\\
\hline
\end{tabular}
\end{center}

\westep{Describe the pattern}
The number of people seated at $n$ tables is $T_n = 4 + 2(n-1)$.

\westep{Calculate the $12^{\mathrm{th}}$ term, in other words, find $T_n$ if $n=12$}

\begin{eqnarray*}
T_{12} &=& 4 + 2  (12 - 1) \\
&=& 4 + 2(11) \\
&=& 4 + 22 \\
&=& 26
\end{eqnarray*}
Therefore $26$ people can be seated at $12$ tables.

\westep{Calculate the number of tables needed to seat $20$ people, in other words, find $n$ if $T_n=20$}
\begin{eqnarray*}
T_n &=& 4 + 2  (n - 1) \\
20 &=& 4 + 2  (n - 1) \\
20 - 4 &=& 2  (n - 1) \\
\frac{16}{2} &=& n - 1 \\
8 + 1 &=& n \\
n &=& 9
\end{eqnarray*}
Therefore $9$ tables are needed to seat $20$ people.
}
\end{wex}

It is important to note the difference between $n$ and ${T}_{n}$. $n$ can be compared to a place holder indicating the position of the term in the sequence, 
while ${T}_{n}$ is the value of the place held by $n$. From the example above, the first table holds $4$ people. So for $n=1$, the value of ${T}_{1}=4$ and so on:\par 
    
\begin{center}
\begin{tabular}{|c|c|c|c|c|c|}
\hline $n$ & $1$ & $2$ & $3$ & $4$ & \ldots \\
\hline $T_n$ & $4$ & $6$ & $8$ & $10$ & \ldots \\
\hline
\end{tabular}
\end{center}


\begin{exercises}{}
{ 
\begin{enumerate}[noitemsep, label=\textbf{\arabic*}. ] 
\item Write down the next three terms in each of the following sequences:
  \begin{enumerate} [noitemsep, label=\textbf{(\alph*)} ]
  \item $5;~15;~25;~\ldots$
  \item $-8;-3;~2:`\ldots$
  \item $30;~27;~24;~\ldots$
  \end{enumerate}
 \item The general term is given for each sequence below. Calculate the missing terms.
  \begin{enumerate} [noitemsep, label=\textbf{(\alph*)} ]
  \item $0;3;~\ldots;~15;~24$\hspace{2.2cm}$T_{n}={n}^{2}-1$
  \item $3;~2;~1;~0;~\ldots;~-2$\hspace{2cm}$T_{n}=-n+4$
  \item $-11;~\ldots;~-7;~\ldots;~-3$\hspace{1.5cm}$T_{n}=-13+2n$
  \end{enumerate}
     
\item Find the general formula for the following sequences and then find ${T}_{10}$, ${T}_{50}$ and ${T}_{100}$
  \begin{enumerate}[noitemsep, label=\textbf{(\alph*)} ]
  \item $2;~5;~8;~11;~14;~\ldots$
  \item $0;~4;~8;~12;~16;~\ldots$
  \item $2;~-1;~-4;~-7;~-10;~\ldots$
  \end{enumerate}
\end{enumerate}
% Automatically inserted shortcodes - number to insert 3
\par \practiceinfo
\par \begin{tabular}[h]{ccccc}
% Question 1
(1.)	00i0	&
% Question 2
(2.)	00i1	&
% Question 3
(3.)	00i2	&
\end{tabular}
% Automatically inserted shortcodes - number inserted 3
}%End of exercise
\end{exercises}


\section{Patterns and conjecture}

%\setcounter{subfigure}{0}
%\begin{figure}[H] 
%\textnormal{Khan Academy video on Number Patterns - 2}\vspace{.1in} k
%\raisebox{-5 pt}{ \includegraphics[width=0.5cm]{col11306.imgs/summary_www.png}} { (Video:  MG10026 )}
%\end{figure} 

In mathematics, a conjecture is a mathematical statement which appears
to be true, but has not been formally proven.  A conjecture can be
thought of as the mathematicians way of saying ``I believe that this
is true, but I have no proof yet''. A conjecture is a good guess or an
idea about a pattern.
\par
For example, make a conjecture about the next number in the pattern
$2;~6;~11;~17;\ldots$ The terms increase by $4$, then $5$, and then $6$.
Conjecture: the next term will increase by $7$, so it will be $17+7=24$.
\par
\mindsetvid{Number Patterns}{MG10026}
\clearpage

\begin{wex}{Adding even and odd numbers}
{
\begin{minipage}{\textwidth}
\begin{enumerate}[noitemsep, label=\textbf{\arabic*}. ] 
\item Investigate the type of number you get if you find the sum of an odd number and an even number.
\item Express your answer in words as a conjecture.
\item Use algebra to prove this conjecture.
\end{enumerate}
\end{minipage}
}
{   
\westep{First try some examples}  
  \begin{equation*}
    \begin{array}{ccc}\hfill 23 + 12 &=& 35\\ 148 + 31 &=& 179\\ 11 +200 &=& 211 \end{array}
   \end{equation*}

\westep{Make a conjecture}  
The sum of any odd number and any even number is always odd.

\westep{Express algebraically}
Express the even number as $2x$.\\
Express the odd number as $2y-1$.

\begin{equation*}
    \begin{array}{ccc}\hfill \mbox{Sum} &=& 2x + (2y-1)\\  &=& 2x + 2y - 1\\ &=& (2x + 2y) - 1 \\ &=& 2(x+y) -1 \end{array}
\end{equation*}
From this we can see that $2(x+y)$ is an even number. So then
$2(x+y)-1$ is an odd number. Therefore our conjecture is true.
}
\end{wex}



\begin{wex}{Multiplying a two-digit number by $11$}
{
\begin{minipage}{\textwidth}
\begin{enumerate}[noitemsep, label=\textbf{\arabic*}. ] 
\item Consider the following examples of multiplying any two-digit number by $11$. What pattern can you see?
\begin{equation*}
    \begin{array}{ccc}\hfill 11 \times 42 &=& 462\\ 11 \times 71 &=& 781\\ 11 \times 45 &=& 495\end{array}
\end{equation*}

\item Express your answer as a conjecture.
\item Can you find an example that disproves your initial conjecture? If so, reconsider your conjecture.
\item Use algebra to prove your conjecture.
\end{enumerate}
\end{minipage}
}
{   
\westep{Find the pattern}
We notice in the answer that the middle digit is the sum of the two digits in the original two-digit number.

\westep{Make a conjecture}
The middle digit of the product is the sum of the two digits of the original number that is multiplied by $11$.
   
\westep{Reconsidering the conjecture}
We notice that 
\begin{equation*}
    \begin{array}{ccc}\hfill 11 \times 67 &=& 737\\ 11 \times 56 &=& 616\end{array}
\end{equation*}
Therefore our conjecture only holds true if the sum of the two digits is less than $10$.

\westep{Express algebraically}
Any two-digit number can be written as $10a+b$. For example, $34=
10(3)+4$. Any three-digit number can be written as $100a+10b+c$. For
example, $582=100(5)+ 10(8) +2$.
\begin{equation*}
    \begin{array}{ccl}\hfill 11 \times (10x+y) &=& 110x + 11y\\ &=&(100x + 10x) + 10y + y\\&=& 100x + (10x+ 10y) + y\\&=& 100x + 10(x+y) + y\end{array}
\end{equation*} 
From this equation we can see that the middle digit of the three-digit
number is equal to the sum of the two digits $x$ and $y$.
}
\end{wex}



\summary{VMcyz}
\begin{itemize}[noitemsep]
\item There are several special sequences of numbers.
    \begin{itemize}[noitemsep]
    \item Triangular numbers: $1;~3;~6;~10;~15;~21;~28;~36;~45;~\ldots$
    \item Square numbers: $1;~4;~9;~16;~25;~36;~49;~64;~81;~\ldots$
    \item Cube numbers: $1;~8;~27;~64;~125;~216;~343;~512;~729;~\ldots$

    \end{itemize}
\item The general term is expressed as the ${n}^{\mathrm{th}}$ term and written as ${T}_{n}$.
\item We define the common difference $d$ of a sequence as the difference between any two successive terms.
\item We can work out a general formula for each number pattern and use it to determine any term in the pattern.
\item A conjecture is something that you believe to be true but have not yet proven.
\end{itemize}
            

\begin{eocexercises}{}
\begin{enumerate}[noitemsep, label=\textbf{\arabic*}. ] 
\item Find the $6^{\mathrm{th}}$ term in each of the following sequences:
  \begin{enumerate}[noitemsep, label=\textbf{(\alph*)} ]
  \item $4;~13;~22;~31;~\ldots$
  \item $5;~2;~-1;~-4;~\ldots$
  \item $7,4;~9,7;~12;~14,3;~\ldots$
  \end{enumerate}


\item Find the general term of the following sequences:
  \begin{enumerate}[noitemsep, label=\textbf{(\alph*)} ]
  \item $3;~7;~11;~15;~\ldots$
  \item $-2;~1;~4;~7;~\ldots$
  \item $11;~15;~19;~23;~\ldots$
  \item $\frac{1}{3};~\frac{2}{3};~1;~1\frac{1}{3};~\ldots$
  \end{enumerate}

\item The seating of a sports stadium is arranged so that the first row has $15$ seats, the second row has $19$ seats, the third row has $23$ seats and so on. Calculate how many seats are in the twenty-fifth row.
\item A single square is made from $4$ matchsticks. Two squares in a row need $7$ matchsticks and three squares in a row need $10$ matchsticks. For this sequence determine:
  \begin{enumerate}[noitemsep, label=\textbf{(\alph*)} ]
  \item the first term;
  \item the common difference;
  \item the general formula;
  \item how many matchsticks there are in a row of twenty-five squares.
  \end{enumerate}
\setcounter{subfigure}{0}
\begin{figure}[H] 
\begin{center}
\footnotesize\begin{pspicture}(0,0)(8,2)
%\psgrid[gridcolor=gray]
\def\match{\psline(0,0)(2,0)\psellipse*(1.8,0)(0.2,0.1)}
\rput(0,0){\match}
\rput{90}(2,0){\match}
\rput{180}(2,2){\match}
\rput{270}(0,2){\match}
\rput(2,0){\rput(0,0){\match}
\rput{90}(2,0){\match}
\rput{180}(2,2){\match}}
\rput(4,0){\rput(0,0){\match}
\rput{90}(2,0){\match}
\rput{180}(2,2){\match}}
\rput(6,0){\rput(0,0){\match}
\rput{90}(2,0){\match}
\rput{180}(2,2){\match}}
\end{pspicture}\normalsize
\vspace{2pt}
\vspace{.1in}
\end{center}
\end{figure}       

\item You would like to start saving some money, but because you have never tried to save money before, you decide to start slowly. At the end of the first week you deposit R$~5$ into your bank account. Then at the end of the second week you deposit R$~10$ and at the end of the third week, R$~15$. After how many weeks will you deposit R$~50$ into your bank account?

\item A horizontal line intersects a piece of string at $4$ points and divides it into five parts, as shown below.
\setcounter{subfigure}{0}
\begin{figure}[H] 
\begin{center}

\footnotesize\begin{pspicture}(-1,-2)(6,2)
%\psgrid[gridcolor=gray]
\psplot[xunit=0.00556, linewidth=1pt]{90}{810}{x sin}
\psline[linestyle=dashed](-1,0)(6,0)
\psdots[dotsize=5pt](1,0)(2,0)(3,0)(4,0)
\rput(0.5,1.5){\psframebox{1}}
\rput(1.5,-1.5){\psframebox{2}}
\rput(2.5,1.5){\psframebox{3}}
\rput(3.5,-1.5){\psframebox{4}}
\rput(4.5,1.5){\psframebox{5}}
\end{pspicture}\normalsize
\end{center}
\end{figure}  
     
If the piece of string is intersected in this way by $19$ parallel lines, each of which intersects it at $4$ points, determine the number
of parts into which the string will be divided.
 
\item Consider what happens when you add $9$ to a two-digit number:
  \begin{equation*}
    \begin{array}{ccl}\hfill 9+16&=& 25\\ 9+28 &=& 37\\9+43&=& 52\end{array}
  \end{equation*} 
  \begin{enumerate}[noitemsep, label=\textbf{(\alph*)} ]
  \item What pattern do you see?
  \item Make a conjecture and express it in words.
  \item Generalise your conjecture algebraically.
  \item Prove that your conjecture is true.
  \end{enumerate}
\end{enumerate}


\par \practiceinfo

\begin{tabular}[h]{ccccc}
 (1a-c.) 00i3& (2a.) 00i4&  (2b.) 00i5& (2c.) 00i6& (2d.) 00i7&  (3.) 00i8&  (4a-d.) 00i9&  (5.) 00ia&  (6.) 00ib& (7.) 00ic
\end{tabular}

\end{eocexercises}
