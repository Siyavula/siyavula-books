\chapter{Analitiese meetkunde}
Analitiese meetkunde, ook bekend as koördinaatmeetkunde en vroëer bekend as Cartesiese meetkunde, is die studie van meetkunde op grond van die beginsels van algebra en die Cartesiese koördinaatstelsel. Dit is  op die definisie van meetkundige figure op ’n numeriese wyse gebaseer en onttrek numeriese inligting uit die voorstelling. Sommige beskou die ontwikkeling van analitiese meetkunde as die begin van moderne wiskunde.

\par 
\chapterstartvideo{VMbkm}
%english section heading below
\section{Figures op die Cartesiese vlak}
As ons die koördinate van die hoekpunte van ’n figuur het, dan kan ons die figuur op die Cartesiese vlak teken.
Byvoorbeeld, neem die vierhoek $ABCD$ met koördinate $A(1;1)$, $B(3;1)$, $C(3;3)$ en $D(1;3)$ en stel dit voor op die Cartesiese vlak. \par 

\setcounter{subfigure}{0}
\begin{figure}[H] % horizontal\label{m39107*id63458}
\begin{center}
\scalebox{.8}{
\begin{pspicture}(-5,-5)(5.5,5.5)
% \psaxes{<->}(0,0)(5,5)
% \psgrid[gridcolor=lightgray,linecolor=lightgray,subgriddiv=1](0,0)(0,0)(4,4)
\psaxes[linewidth=1pt,labels=all,ticks=all]{<->}(0,0)(-1,-1)(5,5)
\pspolygon[linewidth=1pt](1,1)(1,3)(3,3)(3,1)(1,1)
\uput[dl](1,1){\Large{$A$}}
\uput[dr](3,1){\Large{$B$}}
\uput[ur](3,3){\Large{$C$}}
\uput[ul](1,3){\Large{$D$}}
\uput[l](5.7,0){\Large{$x$}}
\uput[d](0,5.7){\Large{$y$}}
\uput[d](-0.2,0){\Large{$0$}}
\end{pspicture}
}
% \caption{Drawing a figure on the Cartesian plane}
\end{center}
\label{fig:cartesianplane}
\end{figure} 

Die volgorde van die letters waarmee ons die figuur benoem, is belangrik. Dit dui vir ons aan dat ons beweeg van punt $A$ na punt $B$, $B$ na punt $C$, $C$ na punt $D$ en dan weer terug na punt  $A$. Dit sou ook reg wees om te skryf vierkant $CBAD$ of $BADC$ maar dit is beter om die konvensie te volg om die letters in alfabetiese volgorde te skryf.     

\section{Afstand tussen twee punte}
'n Punt is 'n eenvoudige meetkundige voorwerp met ligging (posisie) as enigste eienskap. 
\Definition{Punt}{'n Punt is 'n geordende getallepaar wat geskryf word as $(x;y)$.}

\Definition{Afstand}{Afstand is 'n getal wat beskryf hoe ver twee punte van mekaar is.}

\begin{Investigation}{Afstand tussen twee punte}
Punte $P~(2;1)$, $Q~(-2;-2)$ en $R~(2,-2)$ word gegee. 
\begin{itemize}
 \item Kan ons aanvaar dat $\hat{R}=90^{\circ}$? Indien wel, hoekom?
\item Gebruik die stelling van Pythagoras in $\triangle PQR$ om die lengte van $PQ$ te vind.
\end{itemize}
\vspace*{-10pt}
\setcounter{subfigure}{0}
\begin{figure}[H] % horizontal\label{m39107*id63458}
\begin{center}
\scalebox{.8}{
\begin{pspicture}(-5,-5)(5.5,5.5)
% \psaxes{<->}(0,0)(5,5)
% \psgrid[gridcolor=lightgray,linecolor=lightgray,subgriddiv=1](0,0)(-3,-3)(3,3)
\psaxes[linewidth=1pt,labels=all,ticks=all]{<->}(0,0)(-3,-3)(3,3)
\psline[linestyle=dashed,linewidth=1pt](-2,-2)(2,-2)(2,1)
\psline[linestyle=dashed,linewidth=1pt](-2,-2)(2,1)
\psdots(-2,-2)(2,1)
% \uput[l](-2,-1.8){\Large{$Q$}}
\uput[dl](-1.5,-2){\Large{$Q(-2;-2)$}}
\uput[dr](2,-2){\Large{$R(2;-2)$}}
% \uput[u](2.3,0.5){\Large{$P$}}
\uput[ur](2,1){\Large{$P(2;1)$}}
\uput[l](3.5,0){\Large{$x$}}
\uput[d](0,3.5){\Large{$y$}}
% \uput[dr](2.5,-2){\Large{$(2;-2)$}}
\uput[d](-0.2,0){\Large{$0$}}
\end{pspicture}
}
% \caption{Triangle $PQR$}
\end{center}
\label{fig:trianglePQR}
\end{figure} 
\end{Investigation}       
% SOLUTION FOR TEACHERS' GUIDE BELOW:
% In figure~\ref{fig:trianglePQR}, it can be seen that the length of the line $PR$ is 3 units and the length of the line $QR$ is four units. However, $\Delta PQR$, has a right angle at $R$. Therefore, the length of the side $PQ$ can be obtained by using the Theorem of Pythagoras:\par 
%       
% \begin{eqnarray*}
% P{Q}^{2} & = & P{R}^{2}+Q{R}^{2} \\ \
% \therefore P{Q}^{2} & = & {3}^{2}+{4}^{2} \\ 
% \therefore PQ & = & \sqrt{{3}^{2}+{4}^{2}}=5  
% \end{eqnarray*}
% The length of $PQ$ is the distance between the points $P$ and $Q$.\par 
Die formule vir die berekening van die afstand tussen twee punte word as volg afgelei. Die afstand tussen twee
punte $A~({x}_{1};{y}_{1})$ en $B~({x}_{2};{y}_{2})$, word afgelei deur gebruik te maak van die stelling van Pythagoras:\par 

\setcounter{subfigure}{0}
\begin{figure}[H] % horizontal\label{m39107*id63458}
\begin{center}
\scalebox{1}{
\begin{pspicture}(-5,-5)(5.5,5.5)
% \psaxes{<->}(0,0)(5,5)
% \psgrid[gridcolor=lightgray,linecolor=lightgray,subgriddiv=1,gridlabels=0.0cm](0,0)(-3,-3)(3,3)
\psaxes[linewidth=1pt,labels=all,ticks=all]{<->}(0,0)(-3,-3)(3,3)
\psline[linestyle=dashed, linewidth=1pt](-2,-2)(2,-2)(2,1)
\psline[linestyle=dashed,linewidth=1pt](-2,-2)(2,1)
\psdots(-2,-2)(2,1)
% \uput[l](-2,-1.8){\Large{$A$}}
\uput[dl](-1,-2){{$A(x_{1};y_{1})$}}
\uput[dr](2,-1.8){{$C$}}
\uput[ur](2.5,-2.5){{$(x_{2};y_{1})$}}
% \uput[u](2.3,0.5){\Large{$B$}}
\uput[ur](1.8,1){{$B(x_{2};y_{2})$}}
\uput[l](3.5,0){{$x$}}
\uput[d](0,3.5){{$y$}}
\uput[d](-0.2,-0.05){{$0$}}
\end{pspicture}
}
\end{center}
\end{figure}

\begin{align*}
AB^2&=AC^{2}+BC^{2}\\
\therefore AB&=\sqrt{AC^{2}+BC^{2}}
\end{align*}
}Ons sien


\begin{eqnarray*}
AC & = & {x}_{2}-{x}_{1}\\
BC & = & {y}_{2}-{y}_{1} 
\end{eqnarray*}
Dan is
\begin{eqnarray*} 
AB & = & \sqrt{A{C}^{2}+B{C}^{2}} \\ 
& =& \sqrt{(x_{2}-x_{1})^{2}+(y_{2}-y_{1})^{2}} 
\end{eqnarray*}
Gevolglik, vir enige twee punte, $({x}_{1};{y}_{1})$ en $({x}_{2};{y}_{2})$, is die formule:\par 
\Identity{
\vspace*{-3em}
\begin{center}{$\mbox{Afstand tussen twee punte } d=\sqrt{{({x}_{1}-{x}_{2})}^{2}+{({y}_{1}-{y}_{2})}^{2}}$}\end{center}}
Let daarop dat $(x_1 - x_2)^2 = (x_2 - x_1)^2$.

\par
\mindsetvid{the distance formula}{VMbls}

\setcounter{subfigure}{0}
\begin{figure}[H] % horizontal\label{m39107*uid99}
% \textnormal{Khan academy video on distance formula}\vspace{.1in} \nopagebreak
% \label{m39107*yt-media}\label{m39107*yt-video}
% \raisebox{-5 pt}{ \includegraphics[width=0.5cm]{col11306.imgs/summary_www.png}} { (Video:  MG10108 )}
\vspace{2pt}
\vspace{.1in}
\end{figure}       

\begin{wex}{Gebruik van afstandformule}{Vind die afstand tussen $S(-2;-5)$ en $Q(7;-2)$.}{
\westep{Teken 'n skets}
% \begin{figure}
 \begin{center}
\scalebox{1} % Change this value to rescale the drawing.
{
\begin{pspicture}(0,-2.5467188)(6.5735936,2.5867188)
\rput(2.26,0.45328125){\psaxes[linewidth=0.028222222,arrowsize=0.05291667cm 2.0,arrowlength=1.4,arrowinset=0.4,ticksize=0.10583333cm,dx=0.75cm,dy=0.75cm,Dx=2,Dy=2]{<->}(0,0)(-2,-3)(3,2)}
\psline[linewidth=0.028222222cm](1.52,-1.3932812)(4.9,-0.26671866)
\psline[linewidth=0.028222222,linestyle=dashed,dash=0.16cm 0.16cm](1.54,-1.4332813)(4.92,-1.4142337)(4.92,-0.23328125)
\psline[linewidth=0.028222222](4.68,-1.3867188)(4.68,-1.1332811)(4.92,-1.1332811)
\usefont{T1}{ppl}{m}{n}
\rput(5.9,-0.33671865){$T(7;-2)$}
\usefont{T1}{ppl}{m}{n}
\rput(0.4,-1.4267187){$S(-2;-5)$}
\psdots[dotsize=0.127](4.92,-0.26671866)
\usefont{T1}{ppl}{m}{n}
\rput(2.0790625,0.25328124){$0$}
\usefont{T1}{ppl}{m}{n}
\rput(2.5290625,2.45){\small $y$}
\usefont{T1}{ppl}{m}{n}
\rput(5.4,0.5){\small $x$}
\psdots[dotsize=0.127](1.48,-1.4267187)
\end{pspicture} 
}
 \end{center}
% \end{figure}

\westep{Ken waardes toe aan $(x_1;y_1)$ en $(x_2;y_2)$}
Gestel die koördinate van punt $S$ is $(x_1;y_1)$ en die koördinate van punt $T$ is $(x_2;y_2)$.
\begin{equation*}
x_1 = -2 \hskip2em y_1 = -5 \hskip2em x_2 = 7 \hskip2em y_2 = -2
\end{equation*}
\westep{Skryf die afstandformule neer}
\begin{equation*}
d = \sqrt{(x_1 - x_2)^2 + (y_1 - y_2)^2}
\end{equation*}
\westep{Vervang waardes}
\begin{equation*}
\begin{array}{cl}
d_{ST} &= \sqrt{(-2 -7)^2 + (-5- (-2))^2}\\
& = \sqrt{(-9)^2 + (-3)^2}\\
&= \sqrt{81 + 9}\\
&= \sqrt{90}\\
&= 9,5
\end{array}
\end{equation*}
\westep{Skryf die finale antwoord neer}
Die afstand tussen $S$ en $T$ is $9,5$ eenhede.

}
\end{wex}
\vspace*{-30pt}
\begin{wex}{Gebruik die afstandformule}{Gegee $RS = 13$, $R(3;9)$ en $S(8;y)$, vind $y$.}{
\westep{Teken 'n skets}
% \usepackage{pst-plot} % For axes
 \begin{center}
\scalebox{1} % Change this value to rescale the drawing.
{
\begin{pspicture}(0,-4.146719)(7.8990626,4.186719)
\rput(2.0,-2.1467187){\psaxes[linewidth=0.028222222,arrowsize=0.05291667cm 2.0,arrowlength=1.4,arrowinset=0.4,ticksize=0.10583333cm,dx=0.5cm,dy=0.5cm,Dx=2,Dy=2]{<->}(0,0)(-2,-2)(5,6)}
\psline[linewidth=0.028222222,linestyle=dashed,dash=0.16cm 0.16cm](3.96,3.2067187)(2.64,0.12671874)(4.0,-2.8932812)
\usefont{T1}{ppl}{m}{n}
% \rput(1.86,-2.33){$0$}
\usefont{T1}{ppl}{m}{n}
\rput(2.3,3.9832811){$y$}
\usefont{T1}{ppl}{m}{n}
\rput(7.2,-1.9867188){$x$}
\usefont{T1}{ppl}{m}{n}
\rput(4.9,3.1832812){$S_2(8;y_2)$}
\usefont{T1}{ppl}{m}{n}
\rput(4.9,-3.0067186){$S_1(8;y_1)$}
\psdots[dotsize=0.12](2.66,0.08671875)
\psdots[dotsize=0.12](3.96,3.1667187)
\psdots[dotsize=0.12](4.0,-2.8932812)
\usefont{T1}{ppl}{m}{n}
\rput(3.4,0.10328125){$R(3;9)$}
\psline[linewidth=0.04cm,linestyle=dotted,dotsep=0.16cm](3.96,4.0067186)(3.96,-3.7532814)
\end{pspicture} 
}

\end{center}
\westep{Ken waardes toe aan $(x_1;y_1)$ en $(x_2;y_2)$}
Gestel die koördinate van $R$ is $(x_1;y_1)$ en die koördinate van punt $S$ is $(x_2;y_2)$.
\begin{equation*}
x_1 = 3 \hskip2em y_1 = 9 \hskip2em x_2 = 8 \hskip2em y_2 = y
\end{equation*}
\westep{Skryf die afstandformule neer}
\begin{equation*}
d = \sqrt{(x_1 - x_2)^2 + (y_1 - y_2)^2}
\end{equation*}
\westep{Vervang waardes en los op vir $y$}
\begin{equation*}
\begin{array}{cl}
13 &= \sqrt{(3 - 8)^2 + (9 - y)^2}\\
13^2 & = (-5)^2 + (81 - 18y + y^2)\\
0 &= y^2 - 18y - 63\\
&= (y+3) (y-21)\\
\therefore y &= -3 \mbox{ of } y = 21
\end{array}

\end{equation*}
\westep{Skryf die finale antwoord neer}
$S$ is $(8;-3)$ of $(8;21)$.
\vspace{2pt}
\vspace{.1in}
}
\end{wex}

 \textbf{Belangrik:} Teken altyd 'n skets - dit help met jou berekening en ook om te kontroleer of jou antwoord reg kan wees.

\begin{exercises}{}{
\begin{enumerate}[label=\textbf{\arabic*}.]
\item Vind die lengte van $AB$ as:
 \begin{enumerate}[noitemsep, label=\textbf{(\alph*)} ] 
\item $A(2;7)$ en $B(-3;5)$
\item $A(-3;5)$ en $B(-9;1)$
\item $A(x;y)$ en $B(x+4;y-1)$
\end{enumerate}

\item Die lengte van $CD=5$. Vind die ontbekende ko\"ordinaat as:
 \begin{enumerate}[noitemsep, label=\textbf{(\alph*)} ] 
\item $C(6;-2)$ en $D(x;2)$
\item $C(4;y)$ en $D(1;-1)$
\end{enumerate}
\end{enumerate}

% Automatically inserted shortcodes - number to insert 2
\par \practiceinfo
\par \begin{tabular}[h]{cccccc}
% Question 1
(1.)	02ip	&
% Question 2
(2.)	02iq	&
\end{tabular}
% Automatically inserted shortcodes - number inserted 2
\end{exercises}

%          \section{ Calculation of the gradient line}
%     \nopagebreak
%%%             \label{m39108} $ \hspace{-5pt}\begin{array}{cccccccccccc}   \includegraphics[width=0.75cm]{col11306.imgs/summary_video.png} &   \end{array} $ \hspace{2 pt}\raisebox{-5 pt}{} {(section shortcode: MG10109 )} \par 
%     
%     
%     

\section{Berekening van gradi\"ent}
\Definition{Die gradi\"ent van die lyn }{Gradi\"ent is die verhouding tussen die vertikale verandering in posisie en die
horisontale verandering in posisie.}
Die gradi\"ent $m$  van ’n reguitlyn beskryf hoe steil die lyn is, met ander woorde hoe groot die helling van die lyn is. In die figuur hieronder is lyn $OQ$ se helling die kleinste en die gradi\"ent van lyn $OT$ is die grootste want die lyn is die steilste.

\setcounter{subfigure}{0}
\begin{figure}[H] % horizontal\label{m39107*id63458}
\begin{center}
\scalebox{.8}{
\begin{pspicture}(-5,-5)(5.5,5.5)
% \psaxes{<->}(0,0)(5,5)
% \psgrid[gridcolor=lightgray,linecolor=lightgray,subgriddiv=1,gridlabels=0.0cm](0,0)(-1,-1)(4,4)
\psaxes[linewidth=1pt,labels=all,ticks=all]{<->}(0,0)(-1,-1)(4,4)
\psline[linewidth=1pt](0,0)(1,3.5)
\psline[linewidth=1pt](0,0)(3,3.5)
\psline[linewidth=1pt](0,0)(3,2)
\psline[linewidth=1pt](0,0)(3,0.5)
\uput[ur](.9,3.5){\Large{$T$}}
% \uput[dl](-1,-2){\Large{$(x_{2},y_{2})$}}
\uput[ur](3,3.5){\Large{$S$}}
\uput[ur](3,2){\Large{$R$}}
\uput[ur](3,0.4){\Large{$Q$}}
\uput[dl](0,0){\Large{$0$}}
% \uput[ur](1.8,1){\Large{$(x_{1},y_{1})$}}
\uput[l](4.5,0){{$x$}}
\uput[d](0,4.5){{$y$}}
\end{pspicture}
}
\end{center}
\end{figure}       
Om die formule vir gradi\"ent af te lei, beskou ons ’n lyn met koördinate $A~(x_1;y_1)$ en $B~(x_2;y_2)$ met skuinssy $AB$, soos in die diagram hieronder getoon.  
Die gradiënt is die verhouding van die lengte van die vertikale sy van die driehoek tot die horisontale sy. Die vertikale sylengte is die verskil tussen die $y$-waardes van punt $A$ en punt $B$. Die lengte van die horisontale sy van die driehoek is die verskil tussen die $x$-waardes van $A$ en $B$. 
\setcounter{subfigure}{0}
\begin{figure}[H] % horizontal\label{m39107*id63458}
\begin{center}
\scalebox{.8}{
\begin{pspicture}(-5,-5)(5.5,5.5)
% \psaxes{<->}(0,0)(5,5)
% \psgrid[gridcolor=lightgray,linecolor=lightgray,subgriddiv=1,gridlabels=0.0cm](0,0)(-3,-3)(3,3)
\psaxes[linewidth=1pt,labels=all,ticks=all]{<->}(0,0)(-3,-3)(3,3)
\psline(-2,-2)(2,-2)(2,1)
\psline[linestyle=solid,linewidth=1pt](-2,-2)(2,1)
% \uput[l](-2,-1.8){\Large{$A$}}
\psdots(-2,-2)(2,1)
\uput[dl](-1,-2){\Large{$A(x_{1},y_{1})$}}
\uput[dr](2,-2){\Large{$C$}}
% \uput[u](2.3,0.5){\Large{$B$}}
\uput[ur](1.8,1){\Large{$B(x_{2},y_{2})$}}
\uput[l](3.5,0){\Large{$x$}}
\uput[d](0,3.5){\Large{$y$}}
\uput[dl](0,0){\Large{$0$}}
\end{pspicture}
}
\end{center}
\end{figure}  
Dus word die gradi\"ent met die volgende formule bereken:

\Identity{
\vspace*{-3em}
\begin{center}{Gradiënt $(m) = \dfrac{y_{2} - y_{1}}{x_{2} - x_{1}}$ of $= \dfrac{y_{1} - y_{2}}{x_{1} - x_{2}}$}\end{center}}
\textbf{Belangrik:} Onthou om konsekwent te wees: $m \neq \dfrac{y_{1} - y_{2}}{x_{2} - x_{1}}$}
% \Tip{Volgens konvensie gebruik ons $m$ vir die gradi\"ent van 'n reguitlyn}

\par
\mindsetvid{The gradient line}{VMbmr}

\begin{wex}{Gradi\"ent tussen twee punte}{Vind die gradi\"ent van die lyn tussen twee punte E$(2;5)$ en F$(-3;9)$.}{
\westep{Teken 'n skets}
\begin{center}
\scalebox{1} % Change this value to rescale the drawing.
{
\begin{pspicture}(0,-3.1267188)(6.6735935,3.1667187)
\rput(3.0,-2.1267188){\psaxes[linewidth=1pt,arrowsize=0.05291667cm 2.0,arrowlength=1.4,arrowinset=0.4,ticksize=0.10583333cm,dx=0.5cm,dy=0.5cm]{<->}(0,0)(-3,-1)(3,5)}
\psdots[dotsize=0.12](3.96,0.3532813)
\psdots[dotsize=0.12](1.48,2.3732812)
\psline[linewidth=1pt](4.0,0.30671874)(1.48,2.3867188)
\usefont{T1}{ppl}{m}{n}
\rput(6.1,-1.8767188){$x$}
\usefont{T1}{ppl}{m}{n}
\rput(3.2790625,2.9632812){$y$}
\usefont{T1}{ppl}{m}{n}
\rput(4.0590625,0.06328135){$E(2;5)$}
\usefont{T1}{ppl}{m}{n}
\rput(1.2390624,2.6032813){$F(-3;9)$}
\usefont{T1}{ppl}{m}{n}
% \rput(2.85,-2.3){$0$}
\end{pspicture} 
}
\end{center}
\westep{Ken waardes toe aan $(x_1;y_1)$ en $(x_2;y_2)$}
Laat die ko\"ordinate van $E$ $(x_1;y_1)$ wees en die ko\"ordinate van $F$ $(x_2;y_2)$.
\begin{equation*}
x_1 = 2 \hskip2em y_1 = 5 \hskip2em x_2 = -3 \hskip2em y_2 = 9
\end{equation*}
\westep{Skryf die formule vir gradi\"ent  neer }
\begin{equation*}
m = \dfrac{y_2 - y_1}{x_2 - x_1}
\end{equation*}
\westep{Vervang bekende waardes}
\begin{equation*}
\begin{array}{cl}
m_{EF} &= \dfrac{9 - 5}{-3 - 2}\\[5pt]
&= \dfrac{4}{-5}
\end{array}
\end{equation*}
\westep{Skryf die finale antwoord}
Die gradi\"ent van $EF = -\dfrac{4}{5}$

}
\end{wex}


\begin{wex}{Gradi\"ent tussen twee punte}{Gegee $G(7;-9)$ en $H(x;0)$, met $m_{GH}= 3$. Vind $x$.}{
\westep{Teken 'n skets}
\begin{center}
\scalebox{1}{
\begin{pspicture}(0,-2.6867187)(7.1790624,2.7267187)
\psset{xunit=1.2,yunit=1.2}
\rput(3.0,0.31328124){\psaxes[linewidth=1pt,arrowsize=0.05291667cm 2.0,arrowlength=1.4,arrowinset=0.4,ticksize=0.10583333cm,dx=0.6cm,dy=0.6cm,Dx=2,Dy=2]{<->}(0,0)(-3,-3)(3,2)}
\psdots[dotsize=0.12](5.48,0.35328126)
\psdots[dotsize=0.12](4.62,-1.8467188)
\psline[linewidth=1pt](4.64,-1.8267188)(5.46,0.35328126)
\usefont{T1}{ppl}{m}{n}
\rput(6.2,0.4){$x$}
\usefont{T1}{ppl}{m}{n}
\rput(3.1445312,2.5232813){$y$}
\usefont{T1}{ppl}{m}{n}
\rput(5.5245314,0.6432812){$H~(x;0)$}
\usefont{T1}{ppl}{m}{n}
\rput(4.574531,-2.0767188){$G~(7;-9)$}
\usefont{T1}{ppl}{m}{n}
\rput(2.85,0.12328125){$0$}
\usefont{T1}{ppl}{m}{n}
\rput(5.85,-0.71015626){$m_{GH} = 3$}
\end{pspicture} 
}
\end{center}
\westep{Ken waardes toe aan $(x_1;y_1)$ en $(x_2;y_2)$}
Laat die koördinate van $G$ $(x_1;y_1)$ wees en die koördinate van $H$ $(x_2;y_2)$.
\begin{equation*}
x_1 = 7 \hskip2em y_1 = -9 \hskip2em x_2 = x \hskip2em y_2 = 0
\end{equation*}
\westep{Skryf die formule vir gradi\"ent neer}
\begin{equation*}
m = \dfrac{y_2 - y_1}{x_2 - x_1}
\end{equation*}
\westep{Vervang waardes en los op vir $x$}
\begin{equation*}
\begin{array}{cl}
3 &= \dfrac{0 - (-9)}{x - 7}\\[5pt]
3(x-7)&= 9\\
x-7 &= \frac{9}{3}\\
x-7 &= 3\\
x &= 3 + 7\\
&= 10 \\
\end{array}
\end{equation*}
\westep{Skryf die finale antwoord neer}
Die koördinate van $H$ is $(10;0)$.
\vspace{2pt}
\vspace{.1in}
}
\end{wex}


\begin{exercises}{}
\begin{enumerate}[label=\textbf{\arabic*}.]
\item Vind die gradi\"ent van $AB$ as:
 \begin{enumerate}[noitemsep, label=\textbf{(\alph*)} ] 
\item $A(7;10)$ en $B(-4;1)$
\item $A(-5;-9)$ en $B(3;2)$
\item $A(x-3;y)$ en $B(x;y+4)$
\end{enumerate}
\item Die gradi\"ent van  $CD=\frac{2}{3}$, vind $p$ as:
\begin{enumerate}[noitemsep, label=\textbf{(\alph*)} ] 
\item $C(16;2)$ en $D(8;p)$
\item $C(3;2p)$ en $D(9;14)$
\end{enumerate}
\end{enumerate}

% Automatically inserted shortcodes - number to insert 2
\par \practiceinfo
\par \begin{tabular}[h]{cccccc}
% Question 1
(1.)	02ir	&
% Question 2
(2.)	02is	&
\end{tabular}
% Automatically inserted shortcodes - number inserted 2
}
\end{exercises}
\subsection*{Reguitlyn}
\Definition{Reguitlyn}{'n Reguitlyn is 'n versameling punte met 'n konstante gradi\"ent tussen enige twee van die punte.}
Beskou die diagram hieronder met punte $A(x;y)$, $B(x_2;y_2)$ en $C(x_1;y_1)$ wat op 'n reguitlyn l\^e. 
\begin{center}
 \scalebox{1} % Change this value to rescale the drawing.
{
\begin{pspicture}(0,-3.0667188)(7.5290623,3.1067188)
\rput(3.97,-0.06671875){\psaxes[linewidth=1pt,arrowsize=0.05291667cm 2.0,arrowlength=1.4,arrowinset=0.4,labels=all,ticks=all,ticksize=0.10583333cm]{<->}(0,0)(-3,-3)(3,3)}
\psdots[dotsize=0.12](3.39,0.47328126)
\psdots[dotsize=0.12](4.57,2.2132812)
\psline[linewidth=1pt](2.17,-1.3467188)(4.69,2.3932812)
\usefont{T1}{ptm}{m}{n}
\rput(7.1545315,0.14328125){$x$}
\usefont{T1}{ptm}{m}{n}
\rput(4.2545314,2.9032812){$y$}
\usefont{T1}{ptm}{m}{n}
\rput(5.3,2.1832812){$A(x;y)$}
\usefont{T1}{ptm}{m}{n}
\rput(3.6745312,-0.33671874){$0$}
\psdots[dotsize=0.12](2.51,-0.86671877)
\usefont{T1}{ptm}{m}{n}
\rput(2.5845313,0.5832813){$B(x_2;y_2)$}
\usefont{T1}{ptm}{m}{n}
\rput(1.7,-0.7967188){$C(x_1;y_1)$}
\end{pspicture} 
}
\end{center}
Ons het $m_{AB} = m_{BC}=m_{AC}$ en $m = \dfrac{y_2-y_1}{x_2-x_1} = \dfrac{y_1-y_2}{x_1-x_2}$\par

Die algemene formule vir 'n reguitlyn is $\dfrac{y-y_1}{x-x_1} = \dfrac{y_2-y_1}{x_2-x_1}$ waar $(x;y)$ enige punt op die reguitlyn is.\par

Hierdie formule kan ook as $y-y_1 = m(x-x_1)$ beskryf word.\par
% This formula can also be written as 

Die standaardvorm van die reguitlynvergelyking is $y=mx+c$ waar $m$ die gradi\"ent is en $c$ die $y$-afsnit.

\begin{wex}{Vind die vergelyking van 'n reguitlyn}
 {Vind die vergelyking van 'n reguitlyn deur $P(-1;-5)$ en $Q(5;4)$.}
{
\westep{Teken 'n skets}
\begin{center}
\scalebox{1} % Change this value to rescale the drawing.
{
\begin{pspicture}(0,-3.0667188)(6.508125,3.1067188)
\psset{xunit=1.2, yunit=1.2}
\rput(3.0,-0.06671871){\psaxes[linewidth=0.04,arrowsize=0.05291667cm 2.0,arrowlength=1.4,arrowinset=0.4,ticksize=0.10583333cm,dx=0.6cm,dy=0.6cm]{<->}(0,0)(-3,-3)(3,3)}
\psdots[dotsize=0.12,dotangle=-5.9493704](5.4890275,1.926018)
\psline[linewidth=0.04cm](2.5,-2.5332813)(5.48,1.9467187)
\usefont{T1}{ptm}{m}{n}
\rput(6.1335936,0.1432813){$x$}
\usefont{T1}{ptm}{m}{n}
\rput(3.233594,2.9032812){$y$}
\usefont{T1}{ptm}{m}{n}
\rput(5.563594,2.203281){$Q(5;4)$}
\usefont{T1}{ptm}{m}{n}
% \rput(2.8245313,-0.265){$0$}
\psdots[dotsize=0.12,dotangle=-5.9493704](2.4808824,-2.5638745)
\usefont{T1}{ptm}{m}{n}
\rput(1.4735936,-2.5167186){$P(-1;-5)$}
\end{pspicture} 
}
\end{center}
\westep{Ken waardes toe}
Laat $(x;y)$ enige punt op die lyn wees. \\
$x_1 = -1 \hskip2em y_1 = -5 \hskip2em x_1 = 5 \hskip2em y_2 = 4$


\westep{Skryf die algemene formule neer}
\begin{align*}
\dfrac{y-y_1}{x-x_1} &= \dfrac{y_2-y_1}{x_2-x_1}
\end{align*}
\westep{Vervang waardes en maak $y$ die onderwerp van die vergelyking}
\begin{align*}
 \dfrac{y-(-5)}{x-(-1)} &= \dfrac{4-(-5)}{5-(-1)} \\

 \dfrac{y+5}{x+1)} &= \dfrac{3}{2}\\
2(y+5) &=3(x+1)\\
2y +10&=3x+3\\
2y&=3x-7\\
y&=\frac{3}{2}x - \frac{7}{2}
\end{align*}
\westep{Skryf die finale antwoord neer}
Die vergelyking van die reguitlyn is $y&=\frac{3}{2}x - \frac{7}{2}$.
}


\end{wex}
\vspace*{-30pt}
\subsection*{Ewewydige en loodregte lyne}    
%         \label{m39108*eip-332}We can use the gradient of a line to determine if two lines are parallel or perpendicular. If the lines are parallel (Figure~\ref{fig:parallelperpendicular}a) then they will have the same gradient, i.e. ${m}_{\mbox{AB}}={m}_{\mbox{CD}}$. If the lines are perpendicular (Figure~\ref{fig:parallelperpendicular}b) than we have: 
%     \setcounter{subfigure}{0}
%  	\begin{figure}[H] % horizontal\label{m39107*id63458}
%     \begin{center}
% \scalebox{.8}{
% \begin{pspicture}(-5,-5)(5.5,5.5)
% % \psaxes{<->}(0,0)(5,5)
% \rput(-2,-2){
% \psline[linewidth=.05cm](0,0)(0,3)
% \psline[linewidth=.05cm](1,0)(1,3)
% \uput[ur](-1,2.8){\Large{$a)$}}
% \uput[d](0,0){\Large{$A$}}
% \uput[u](0,3){\Large{$B$}}
% \uput[d](1,0){\Large{$C$}}
% \uput[u](1,3){\Large{$D$}}}
% 
% \rput(3.4,2){
% \psline[linewidth=.05cm](1,-4)(-2,-1)
% \psline[linewidth=.05cm](-2,-4)(1,-1)
% \uput[ur](-3.2,-1.2){\Large{$b)$}}
% \uput[dr](1,-4){\Large{$A$}}
% \uput[ul](-2,-1){\Large{$B$}}
% \uput[dl](-2,-4){\Large{$C$}}
% \uput[ur](1,-1){\Large{$D$}}}
% \end{pspicture}
% }
%     \end{center}
% \caption{a) Parallel and b) perpendicular lines}
% \label{fig:parallelperpendicular}
%  \end{figure}       
Twee lyne wat ewewydig is aan mekaar het gelyke gradi\"ente. As twee lyne mekaar loodreg sny, dan is die produk van hulle gradi\"ente $-1$. \\
% Met ander woorde: $\mbox{gradi\"ent}_{AB}=\mbox{gradi\"ent}_{CD}$. \par

As $WX \perp $ $ YZ$ sal $m_{WX} \times m_{YZ} = -1$. Loodregte lyne se gradi\"ente is die negatiewe inverses van mekaar.

\par
\mindsetvid{Parallel and perpendicular lines}{VMbon}

\begin{wex}{Ewewydige lyne}{Bewys lyn $AB$ met $A(0;2)$ en $B(2;6)$ is ewewydig aan die lyn $2x-y = 2$.}{
\westep{Teken 'n skets}

\begin{center}
\scalebox{1} % Change this value to rescale the drawing.
{

\begin{pspicture}(0,-3.6067188)(6.5590625,3.6467187)
\psset{xunit=1.2, yunit=1.2}
\rput(3.0,-0.6067188){\psaxes[linewidth=1pt,arrowsize=0.05291667cm 2.0,arrowlength=1.4,arrowinset=0.4,ticksize=0.10583333cm,dx=0.6cm,dy=0.6cm]{<->}(0,0)(-3,-3)(3,4)}
\psdots[dotsize=0.12,dotangle=-5.9493704](4.0290275,2.4060183)
\psline[linewidth=1pt](1.9,-1.7867187)(4.52,3.3332813)
\usefont{T1}{ppl}{m}{n}
\rput(6.184531,-0.39671874){$x$}
\usefont{T1}{ppl}{m}{n}
\rput(3.4045312,3.4432812){$y$}
\usefont{T1}{ppl}{m}{n}
\rput(4.6745315,2.3832812){$B(2;6)$}
\usefont{T1}{ppl}{m}{n}
\rput(2.8245313,-0.7967188){$0$}
\usefont{T1}{ppl}{m}{n}
\rput(3.52,0.32328126){$A(0;2)$}
\psline[linewidth=1pt](2.34,-2.8067188)(5.04,2.2532814)
\psdots[dotsize=0.12](3.0,0.39328125)
\usefont{T1}{ppl}{m}{n}
\rput(5.7,1.6432812){$y=2x-2$}
\end{pspicture} 
}

\end{center}
(Wees versigtig - soms mag lyne ewewydig lyk terwyl hulle nie is nie!)

\westep{Skryf die formule vir gradi\"ent neer}
\begin{equation*}
m = \dfrac{y_2-y_1}{x_2-x_1}
\end{equation*}
\westep{Vervang waardes om die gradi\"ent van $AB$ te vind}
\begin{equation*}
\begin{array}{cl}
m_{AB} &= \dfrac{6-2}{2-0}\\[5pt]
&= \dfrac{4}{2}\\
&= 2
\end{array}
\end{equation*}
\westep{Maak seker die vergelyking is in standaardvorm $y=mx+c$}
\begin{equation*}
\begin{array}{cl}
2x-y& = 2\\
y& = 2x-2\\
\therefore m_{CD}&= 2
\end{array}
\end{equation*}
\westep{Skryf die finale antwoord}
\vspace*{-12pt}
\begin{equation*}
\begin{array}{cl}
m_{AB} &= m_{CD}\\

\end{array}
\end{equation*}
dus is lyn $AB$ ewewydig aan $y=2x-2$. \vspace*{-12pt}
}
\end{wex}



\begin{wex}{Loodregte lyne}{Lyn $AB$ is loodreg op lyn $CD$. Vind $y$ as $A(2;-3)$, $B(-2;6)$, $C(4;3)$ en $D(7;y)$ gegee word.}{
\westep{Teken 'n skets}
\begin{center}
\scalebox{1} % Change this value to rescale the drawing.
{

\begin{pspicture}(0,-3.6267188)(7.9990625,3.6667187)
\psset{xunit=1.2,yunit=1.2}
\rput(3.0,-0.62671876){\psaxes[linewidth=0.04,arrowsize=0.05291667cm 2.0,arrowlength=1.4,arrowinset=0.4,ticksize=0.10583333cm,dx=0.6cm,dy=0.6cm]{<->}(0,0)(-3,-3)(4,4)}
\psdots[dotsize=0.12,dotangle=-5.9493704](4.9890275,0.8660181)
\psline[linewidth=0.04cm](0.98,-0.9267188)(6.58,1.5932814)
\usefont{T1}{ppl}{m}{n}
\rput(7.2745314,-0.53671867){$x$}
\usefont{T1}{ppl}{m}{n}
\rput(3.1745312,3.4632812){$y$}
\usefont{T1}{ppl}{m}{n}
\rput(1.5790626,2.6032813){$B(-2;6)$}
\usefont{T1}{ppl}{m}{n}
% \rput(2.8,-0.81671876){$0$}
\usefont{T1}{ppl}{m}{n}
\rput(4.8,-2.0967188){$A(2;-3)$}
\psline[linewidth=0.04cm](2.0,2.3132813)(4.02,-2.1067188)
\psdots[dotsize=0.12](2.0,2.3532813)
\psdots[dotsize=0.12](4.02,-2.1067188)
\usefont{T1}{ppl}{m}{n}
\rput(4.3745313,1.0){$C(4;3)$}
\psline[linewidth=0.04cm,linestyle=dashed,dash=0.16cm 0.16cm](6.52,3.3932812)(6.52,-3.4867187)
\usefont{T1}{ppl}{m}{n}
\rput(7.25,1.5){$D(7;y)$}
\psdots[dotsize=0.12](6.52,1.5732813)
\rput{23.284023}(0.24130535,-1.2977501){\psframe[linewidth=0.04,dimen=outer](3.44,0.10671873)(3.1,-0.23328127)}
\end{pspicture}
}
\end{center}


\westep{Skryf neer die verband tussen die gradi\"ente van loodregte lyne $AB$ en $CD $}
\begin{align*}
m_{AB} \times m_{CD} &= -1\\
\dfrac{y_B-y_A}{x_B-x_A} \times \dfrac{y_D-y_C}{x_D-x_C} &=-1
\end{align*}
\westep{Vervang waardes en los op vir $y$}
\begin{equation*}
\begin{array}{rl}
\dfrac{6 - (-3)}{-2 -2} \times \dfrac{y - 3}{7 - 4} &= -1\\[5pt]
\dfrac{9}{-4} \times \dfrac{y-3}{3} &= -1\\[5pt]
\dfrac{y-3}{3} &= -1 \times \dfrac{-4}{9}\\[5pt]
\dfrac{y-3}{3} &= \dfrac{4}{9}\\[5pt]
y-3 &= \dfrac{4}{9} \times 3\\[5pt]
y-3 &= \dfrac{4}{3}\\[5pt]
y &= \dfrac{4}{3} + 3\\[5pt]
&= \dfrac{4 + 9}{3}\\[5pt]
&= \dfrac{13}{3}\\[5pt]
&= 4 \dfrac{1}{3}
\end{array}
\end{equation*}
\westep{Skryf die finale antwoord}
Dus is die ko\"ordinate van $D$ $(7; 4\frac{1}{3})$.
}
\end{wex}
\clearpage
\subsection*{Horisontale en vertikale lyne}

'n Lyn wat ewewydig aan die $x$-as is, word 'n horisontale lyn genoem en het 'n gradi\"ent van zero, want daar is geen vertikale verandering in die helling van die lyn nie:\par
\begin{equation*}
m = \dfrac{\mbox{verandering in }y}{\mbox{verandering in }x} = \dfrac{0}{\mbox{verandering in }x} =0
\end{equation*}


'n Lyn wat ewewydig loop aan die $y$-as, word 'n vertikale lyn genoem en sy gradi\"ent is ongedefinieerd, want daar is geen horisontale verandering in die helling van die lyn nie:\par
\begin{equation*}
 m = \dfrac{\mbox{verandering in }y}{\mbox{verandering in }x} = \dfrac{\mbox{verandering in }y}{0}= \mbox{ongedefinieerd}
\end{equation*}


\subsection*{Punte op 'n lyn}

'n Reguitlyn is 'n versameling punte met 'n konstante gradi\"ent tussen enige twee punte. Daar is verskeie metodes om te bewys dat punte op dieselfde reguitlyn l\^e, bv. die gradi\"ent metode en 'n langer metode wat die afstandsformule gebruik. 

\begin{wex}{Punte op 'n lyn}{Bewys dat $A(-3;3)$, $B(0;5)$ en $C(3;7)$ op dieselfde reguitlyn l\^e. }{
\westep{Stip die punte}

\begin{center}
\scalebox{1} % Change this value to rescale the drawing.
{
\begin{pspicture}(0,-2.6667187)(6.4090624,2.7067187)
\psset{xunit=1.2,yunit=1.2}
\rput(3.01,-1.6667187){\psaxes[linewidth=1pt,arrowsize=0.05291667cm 2.0,arrowlength=1.4,arrowinset=0.4,ticksize=0.10583333cm,dx=0.6cm,dy=0.6cm]{<->}(0,0)(-3,-1)(3,4)}
\psdots[dotsize=0.12,dotangle=-5.9493704](4.4790277,1.8660182)
\usefont{T1}{ppl}{m}{n}
\rput(6.034531,-1.4167187){$x$}
\usefont{T1}{ppl}{m}{n}
\rput(3.2945313,2.5032814){$y$}
\usefont{T1}{ppl}{m}{n}
\rput(3.8445313,0.8232812){$B(0;5)$}
\usefont{T1}{ppl}{m}{n}
% \rput(2.8545313,-1.8767188){$0$}
\usefont{T1}{ppl}{m}{n}
\rput(0.77453125,-0.13671875){$A(-3;3)$}
\psdots[dotsize=0.12](1.53,-0.16671875)
\psdots[dotsize=0.12](3.01,0.8332813)
\usefont{T1}{ppl}{m}{n}
\rput(5.3745313,1.8832812){$C(3;7)$}
\end{pspicture} 
}

\end{center}

\westep{Bereken twee gradi\"ente tussen enige twee van die drie punte}
\begin{equation*}
 \begin{array}{rll}
m&=\dfrac{y_2-y_1}{x_2-x_1}&\\[6pt]
m_{AB} &= \dfrac{5-3}{0-(-3)} &= \dfrac{2}{3}
\end{array}
\end{equation*}
en
\begin{equation*}
m_{BC} = \frac{7-5}{3-0} = \frac{2}{3}
\end{equation*}
OF
\begin{equation*}
m_{AC} = \frac{3-7}{3-3} = \frac{-4}{-6}=\frac{2}{3}
\end{equation*}
en
\begin{equation*}
m_{BC} = \frac{7-5}{3-0} = \frac{2}{3}
\end{equation*}
\westep{Verduidelik jou antwoord}
\begin{equation*}
m_{AB} = m_{BC}= m_{AC}
\end{equation*}
Dus l\^e die punte $A$, $B$ en $C$  op dieselfde reguitlyn. 
}
\end{wex}

Om te bewys die punte reglynig is met behulp van die afstandsformule, moet ons die afstande tussen elke paar punte bereken en dan bewys dat die som van die twee korter afstande gelyk is aan die langste afstand.


\begin{wex}{Punte op 'n reguitlyn}{Bewys dat $A(-3;3)$, $B(0;5)$ en $C(3;7)$ op dieselfde reguitlyn l\^e.}{
\westep{Stip die punte}
\vspace*{-20pt}
\begin{center}
\scalebox{1} % Change this value to rescale the drawing.
{
\begin{pspicture}(0,-2.6667187)(6.4090624,2.7067187)
\psset{xunit=1.2,yunit=1.2}
\rput(3.01,-1.6667187){\psaxes[linewidth=1pt,arrowsize=0.05291667cm 2.0,arrowlength=1.4,arrowinset=0.4,ticksize=0.10583333cm,dx=0.6cm,dy=0.6cm]{<->}(0,0)(-3,-1)(3,4)}
\psdots[dotsize=0.12,dotangle=-5.9493704](4.4790277,1.8660182)
\usefont{T1}{ppl}{m}{n}
\rput(6.034531,-1.4167187){$x$}
\usefont{T1}{ppl}{m}{n}
\rput(3.2945313,2.5032814){$y$}
\usefont{T1}{ppl}{m}{n}
\rput(3.8445313,0.6){$B(0;5)$}
\usefont{T1}{ppl}{m}{n}
% \rput(2.8545313,-1.8767188){$0$}
\usefont{T1}{ppl}{m}{n}
\rput(0.77453125,-0.13671875){$A(-3;3)$}
\psdots[dotsize=0.12](1.53,-0.16671875)
\psdots[dotsize=0.12](3.01,0.8332813)
\usefont{T1}{ppl}{m}{n}
\rput(5.1,1.8832812){$C(3;7)$}
\psline[linewidth=1pt,linestyle=dashed,dash=0.16cm 0.16cm,arrowsize=0.05291667cm 2.0,arrowlength=1.4,arrowinset=0.4]{<->}(1.65,-0.12671874)(2.93,0.7732813)
\psline[linewidth=1pt,linestyle=dashed,dash=0.16cm 0.16cm,arrowsize=0.05291667cm 2.0,arrowlength=1.4,arrowinset=0.4]{<->}(3.09,0.91328126)(4.39,1.8132813)
\psline[linewidth=1pt,linestyle=dashed,dash=0.16cm 0.16cm,arrowsize=0.05291667cm 2.0,arrowlength=1.4,arrowinset=0.4]{<->}(1.79,-0.42671874)(4.49,1.5332812)
\end{pspicture} 
}

\end{center}
\westep{Bereken die drie afstande $AB$, $BC$ en $AC$}
\begin{equation*}
d_{AB} = \sqrt{(-3 - 0)^2 + (3 - 5)^2} = \sqrt{(-3)^2 + (-2)^2} = \sqrt{9 + 4} = \sqrt{13}
\end{equation*}
\begin{equation*}
d_{BC} = \sqrt{(0 - 3)^2 + (5 - 7)^2} = \sqrt{(-3)^2 + (-2)^2} = \sqrt{9 + 4} = \sqrt{13}
\end{equation*}
\begin{equation*}
d_{AC} = \sqrt{((-3) - 3)^2 + (3 - 7)^2} = \sqrt{(-6)^2 + (-4)^2} = \sqrt{36 + 16} = \sqrt{52}
\end{equation*}
\westep{Vind die som van die twee korter afstande}
\begin{equation*}
d_{AB} + d_{BC} = \sqrt{13} + \sqrt{13} = 2\sqrt{13} = \sqrt{4 \times 13} = \sqrt{52}
\end{equation*}
\westep{Verduidelik jou antwoord}
\begin{equation*}
d_{AB} + d_{BC} = d_{AC}
\end{equation*}
Dus l\^e punte $A$, $B$ en $C$ op dieselfde reguitlyn. \vspace*{-15pt}
}
\end{wex}

\begin{exercises}{}
\begin{enumerate}[itemsep=5pt, label=\textbf{\arabic*}. ]

\item Bepaal of $AB$ ewewydig aan of loodreg op $CD$ is, of nie een van die twee nie:
\begin{enumerate}[noitemsep, label=\textbf{(\alph*)} ]
\item $A(3;-4)$, $B(5;2)$, $C(-1;-1)$, $D(7;23)$
\item $A(3;-4)$, $B(5;2)$, $C(-1;-1)$, $D(0;-4)$
\item $A(3;-4)$, $B(5;2)$, $C(-1;-1)$, $D(1;2)$
\end{enumerate}

\item Bepaal of die volgende punte op dieselfde reguitlyne l\^e:
\begin{enumerate}[noitemsep, label=\textbf{(\alph*)} ]
\item $E(0;3)$, $F(-2;5)$, $G(2;1)$
\item $H(-3;-5)$, $I(-0;0)$, $J(6;10)$
\item $K(-6;2)$, $L(-3;1)$, $M(1;-1)$
\end{enumerate}
\item Punte $P(-6;2)$, $Q(2;-2)$ en $R(-3;r)$ l\^e op 'n reguitlyn. Vind die waarde van $r$.
\item Lyn $PQ$ met $P(-1;-7)$ en $Q(q;0)$ het 'n gradi\"ent van $1$. Vind $q$.
\end{enumerate}
\end{enumerate}

% Automatically inserted shortcodes - number to insert 4
\par \practiceinfo
\par \begin{tabular}[h]{cccccc}
% Question 1
(1.)	02it	&
% Question 2
(2.)	02iu	&
% Question 3
(3.)	02iv	&
% Question 4
(4.)	02iw	&
\end{tabular}
% Automatically inserted shortcodes - number inserted 4
\end{exercises}

% Die volgende video bied ’n opsomming van die gradiënt van ’n lyn.
% \setcounter{subfigure}{0}
% \begin{figure}[H] % horizontal\label{m39108*uid993}
% \textnormal{Gradient of a line}\vspace{.1in} \nopagebreak
% \label{m39108*yt-media1}\label{m39108*yt-video1}
% \raisebox{-5 pt}{ \includegraphics[width=0.5cm]{col11306.imgs/summary_www.png}} { (Video:  MG10110 )}
% \vspace{2pt}
% \vspace{.1in}
% \end{figure}      
%          \section{ Midpoint of a line}
%     \nopagebreak
%%%             \label{m39119} $ \hspace{-5pt}\begin{array}{cccccccccccc}   \includegraphics[width=0.75cm]{col11306.imgs/summary_video.png} &   \end{array} $ \hspace{2 pt}\raisebox{-5 pt}{} {(section shortcode: MG10111 )} \par 
%     
%     
%     
\clearpage
\section{Middelpunt van ’n lynstuk}
\begin{Investigation}{Vind die middelpunt van 'n lynstuk}
\item Stip punte $P(2;1)$ en $Q(-2;2)$ akkuraat op grafiekpapier en trek lyn $PQ$.
\begin{itemize}
 \item Vou die papier so dat punt $P$ presies op punt $Q$ val.
\item Merk die punt waar die voulyn die lyn $PQ$ kruis as $S$.
\item Tel die blokkies en vind presiese posisie van $S$.
\item Skryf die ko\"ordinate van $S$ neer.
\end{itemize}

\end{Investigation}
Die koördinate van die middelpunt $M(x;y)$ van ’n lyn tussen enige twee punte $A$ en $ B$ met koördinate $A(x_1;y_1)$ en $B(x_2;y_2)$ word as volg bereken:

\setcounter{subfigure}{0}
\begin{figure}[H] % horizontal\label{m39107*id63458}
\begin{center}
\scalebox{0.9}{
\begin{pspicture}(-5,-5)(5.5,5.5)
% \psgrid[gridcolor=lightgray,linecolor=lightgray,subgriddiv=1,gridlabels=0.0cm](0,0)(0,0)(5,5)
\psaxes[linewidth=1pt,labels=all,ticks=all]{<->}(0,0)(-2,-2)(5.5,5.5)
\psline[linewidth=1pt](-2,-1)(5,5)
\uput[l](-0.5,0.45){\Large{$A (x_{1};y_{1})$}}
\uput[r](-0.5,0.45){\qdisk(0,0){2pt}}
\uput[r](2.5,2.6){\Large{$M (x;y)$}}
\uput[u](2.5,2.7){\qdisk(0,0){2pt}}
\uput[ur](5,5){\Large{$B (x_{2};y_{2})$}}
\uput[u](5,4.8){\qdisk(0,0){2pt}}
\uput[l](6.1,0){\Large{$x$}}
\uput[d](0,6.2){\Large{$y$}}
\uput[d](-0.2,0.1){\Large{$0$}}
\end{pspicture}
}
\end{center}
\end{figure}      
\vspace*{-40pt}
\begin{eqnarray*}
x & = & \frac{{x}_{1} + {x}_{2}}{2} \\ 
y & = & \frac{{y}_{1} + {y}_{2}}{2} \\  
\end{eqnarray*}
Hieruit kry ons die middelpunt van 'n lynstuk:
% From this we obtain the mid-point of a line:
\Identity{
\vspace*{-3em}
\begin{center}{Middelpunt $M(x;y)=(\dfrac{{x}_{1} + {x}_{2}}{2};\dfrac{{y}_{1}+{y}_{2}}{2}) $}\end{center}}
\mindsetvid{the midpoint of a line segment}{VMbpr}
\vspace*{-30pt}
\begin{wex}{Berekening van middelpunt}
 {Bereken die ko\"ordinate van die middelpunt $F(x;y)$ van die lyn tussen punt $E(2;1)$ en punt $G(-2;-2)$.}
{
\westep{Teken 'n skets}
\setcounter{subfigure}{0}
\vspace*{-20pt}
\begin{figure}[H] % horizontal\label{m39107*id63458}
\begin{center}
\scalebox{1}{
\begin{pspicture}(-5,-5)(5.5,5.5)
% \psaxes{<->}(0,0)(5,5)
% \psgrid[gridcolor=lightgray,linecolor=lightgray,subgriddiv=1,gridlabels=0.0cm](0,0)(-3,-3)(3,3)
\psaxes[linewidth=1pt,labels=all,ticks=all]{<->}(0,0)(-3,-3)(3,3)
\psline[linewidth=1pt](-2,-2)(2,1)
% \psline[linewidth=.05cm](0,0)(3,3.5)
% \psline[linewidth=.05cm](0,0)(3,2)
% \psline[linewidth=.05cm](0,0)(3,0.5)
% \uput[ur](.9,3.5){\Large{$T$}}
\uput[dl](-1,-2){$G(-2,-2)$}
\uput[u](-2,-2.2){\qdisk(0,0){2pt}}
\uput[r](0,-.7){$F(x;y)$}
\uput[r](0,3.2){$y$}
\uput[r](3,0){$x$}
\uput[r](-0.4,-0.2){$0$}
% \uput[r](0,-1.2){\Large{mid-point}}
\uput[u](0,-.7){\qdisk(0,0){2pt}}
\uput[ur](2,1){$E (2;1)$}
\uput[u](2,.8){\qdisk(0,0){2pt}}
% \uput[l](4.8,0){\Large{$x$}}
% \uput[d](0,4.8){\Large{$y$}}
\end{pspicture}
}
\end{center}
\end{figure} 
\westep{Ken waardes toe aan $(x_1;y_1)$ en $(x_2;y_2)$}
\begin{equation*}
x_1 = -2 \hskip2em y_1 = -2 \hskip2em x_1 = 2 \hskip2em y_2 = 1
\end{equation*}
\westep{Skryf die formule vir middelpunt neer}
\begin{equation*}
F(x;y)=(\dfrac{{x}_{1} + {x}_{2}}{2};\dfrac{{y}_{1}+{y}_{2}}{2})
\end{equation*}
\westep{Vervang waardes in die middelpunt formule}
\begin{eqnarray*}
x & = & \frac{{x}_{1} + {x}_{2}}{2} \\ [5pt]
& = & \frac{-2 + 2}{2} \\ [5pt]
& = & 0 \\ 
y & = & \frac{{y}_{1} + {y}_{2}}{2} \\ [5pt]
& = & \frac{-2 + 1}{2} \\ [5pt]
& = & -\frac{1}{2} 
\end{eqnarray*}
\westep{Skryf die antwoord}
Die middelpunt is $F(0;-\frac{1}{2})$.
\westep{Bevestig die antwoord met die afstandformule}
Dit kan bewys word dat die afstande vanaf die eindpunte na die middelpunt gelyk is: 
\begin{eqnarray*}
PS & = & \sqrt{{({x}_{1} - {x}_{2})}^{2} + {({y}_{1} - {y}_{2})}^{2}} \\ 
& = & \sqrt{{(0 - 2)}^{2} + {(-0,5 - 1)}^{2}} \\ 
& = & \sqrt{{(-2)}^{2} + {(-1,5)}^{2}} \\ 
& = & \sqrt{4 + 2,25} \\ 
& = & \sqrt{6,25}
\end{eqnarray*}
en
\begin{eqnarray*}
QS & = & \sqrt{{({x}_{1} - {x}_{2})}^{2} + {({y}_{1} - {y}_{2})}^{2}} \\ 
& = & \sqrt{{(0 - (-2))}^{2} + {(-0,5 - (-2))}^{2}} \\ 
& = & \sqrt{{(0 + 2)}^{2}{+(-0,5 + 2)}^{2}} \\ 
& = & \sqrt{{(2)}^{2}{+(-1,5)}^{2}} \\ 
& = & \sqrt{4 + 2,25} \\ 
& = & \sqrt{6,25}
\end{eqnarray*}
Daar kan gesien word dat $PS=QS$ soos verwag is, dus is $F$ die middelpunt. 
}
\end{wex}

% 
% Die volgende video verskaf ’n opsomming oor die berekening van die middelpunt van ’n lyn.
% \setcounter{subfigure}{0}
% \begin{figure}[H] % horizontal\label{m39119*uid666}
% \textnormal{Khan academy video on mid-point of a line}\vspace{.1in} \nopagebreak
% \label{m39119*yt-media2}\label{m39119*yt-video2}
% \raisebox{-5 pt}{ \includegraphics[width=0.5cm]{col11306.imgs/summary_www.png}} { (Video:  MG10112 )}
% \vspace{2pt}
% \vspace{.1in}
% \end{figure}      
%          \section{ Summary \& Exercises}
%     \nopagebreak
%%%             \label{m39167} $ \hspace{-5pt}\begin{array}{cccccccccccc}   \end{array} $ \hspace{2 pt}\raisebox{-5 pt}{\includegraphics[width=0.5cm]{col11306.imgs/summary_www.png}} {(section shortcode: MG10113 )} \par 
%     
%     
% 
\begin{wex}{Berekening van die middelpunt}{Vind die middelpunt van lynstuk $AB$, as $A(6;2)$ en $B(-5;-1)$ is.}{
\westep{Maak 'n skets}
\begin{center}

\scalebox{1} % Change this value to rescale the drawing.
{
\begin{pspicture}(0,-3.0667188)(7.8490624,3.1067188)
\psset{xunit=1.2,yunit=1.2}
\rput(3.23,-0.06671875){\psaxes[linewidth=1pt,arrowsize=0.05291667cm 2.0,arrowlength=1.4,arrowinset=0.4,ticksize=0.10583333cm,dx=0.6cm,dy=0.6cm]{<->}(0,0)(-3,-3)(4,3)}
\psdots[dotsize=0.12](3.47,0.17328125)
\psdots[dotsize=0.12](6.23,0.91328126)
\psline[linewidth=1pt](0.79,-0.5467188)(6.25,0.91328126)
\usefont{T1}{ppl}{m}{n}
\rput(7.2,0.14328125){$x$}
\usefont{T1}{ppl}{m}{n}
\rput(3.5145311,2.9032812){$y$}
\usefont{T1}{ppl}{m}{n}
\rput(6.3745313,1.1232812){$A~(6;2)$}
\usefont{T1}{ppl}{m}{n}
\rput(3.1,-0.25){$0$}
\psdots[dotsize=0.12](0.75,-0.5467188)
\usefont{T1}{ppl}{m}{n}
\rput(4.0445313,0.65){$M(x;y)$}
\usefont{T1}{ppl}{m}{n}
\rput(0.84453124,-0.83671874){$B(-5;-1)$}
\end{pspicture} 
}


\end{center}

Vanaf die skets skat ons  $M$ sal in kwadrant I l\^e, met positiewe $x$- en $y$-ko\"ordinate.
\westep{Ken waardes toe aan $(x_1;y_1)$ en $(x_2;y_2)$}
Laat $M$ die punt $(x;y)$  wees
\begin{equation*}
x_1= 6 \hskip2em y_1=2 \hskip2em x_2=-5 \hskip2em y_2=-1
\end{equation*}
\westep{Skryf die middelpunt formule neer}
\begin{equation*}
M(x;y) = \Big(\frac{x_1+x_2}{2};\frac{y_1+y_2}{2}\Big)
\end{equation*}
\westep{Vervang waardes en vereenvoudig}
\begin{equation*}
\Big(\frac{6-5}{2};\frac{2-1}{2}\Big) = \Big(\frac{1}{2};\frac{1}{2}\Big)
\end{equation*}
\westep{Skryf die finale antwoord}
$M(\frac{1}{2};\frac{1}{2})$ is die middelpunt van lynstuk $AB$.
}
\end{wex}
\vspace*{-40pt}
\begin{wex}{Gebruik die middelpuntformule}{ Die lynstuk wat $C(-2;4)$ en $D(x;y)$ verbind, het middelpunt $M(1;-3)$. Vind punt $D$.\vspace*{-20pt}}{
\westep{Maak 'n skets}
\begin{center}\vspace*{-20pt}
\scalebox{1} % Change this value to rescale the drawing.
{
\begin{pspicture}(0,-4.224219)(7.5390625,4.224219)
\psset{xunit=1.2,yunit=1.2}
\rput(3.0,0.93078125){\psaxes[linewidth=1pt,arrowsize=0.05291667cm 2.0,arrowlength=1.4,arrowinset=0.4,ticksize=0.10583333cm,dx=0.6cm,dy=0.6cm]{<->}(0,0)(-3,-5.5)(4,3)}
\psdots[dotsize=0.12](3.5,-0.58921874)
\psdots[dotsize=0.12](5.0,-4.0492187)
\psline[linewidth=1pt](1.96,2.9107811)(5.0,-4.0292187)
\usefont{T1}{ppl}{m}{n}
\rput(7,1.2207812){$x$}
\usefont{T1}{ppl}{m}{n}
\rput(3.2,4.020781){$y$}
\usefont{T1}{ppl}{m}{n}
\rput(5.65,-3.9992187){$D(x;y)$}
\usefont{T1}{ppl}{m}{n}
\rput(3.1245313,0.72078127){$0$}
\psdots[dotsize=0.12](1.96,2.9107811)
\usefont{T1}{ppl}{m}{n}
\rput(4.4,-0.5992187){$M(1;-3)$}
\usefont{T1}{ppl}{m}{n}
\rput(1.7145313,3.1207812){$C(-2;4)$}
\psline[linewidth=1pt](2.638401,1.7359167)(2.3761587,1.5784051)
\psline[linewidth=1pt](2.687574,1.6269872)(2.4253316,1.4694756)
\psline[linewidth=1pt](4.478401,-2.4440832)(4.216159,-2.601595)
\psline[linewidth=1pt](4.547574,-2.5530128)(4.2853317,-2.7105243)
\end{pspicture} 
}
\end{center}
Vanaf die skets skat ons dat $D$ in kwadrant IV sal l\^e, met 'n positiewe $x$- en 'n negatiewe $y$-ko\"ordinaat.
\westep{Ken waardes toe aan $(x_1;y_1)$ en $(x_2;y_2)$}
Laat die ko\"ordinate van $C(x_1;y_1)$ en die ko\"ordinate van $D(x_2;y_2)$ wees.
\begin{equation*}
x_1=-2 \hskip2em y_1=4 \hskip2em x_2=x \hskip2em y_2=y
\end{equation*}
\westep{Skryf die middelpuntformule neer}
\begin{equation*}
M(x;y) = \Big(\frac{x_1+x_2}{2}; \frac{y_1+y_2}{2}\Big)
\end{equation*}
\westep{Vervang waardes en los op vir $x$ en $y$}
\begin{equation*}
\begin{array}{rllrl}
1&=\dfrac{-2+x_2}{2}&\hskip10em&-3&=\dfrac{4+y_2}{2}\\[5pt]
1\times 2&=-2+x_2&&-3\times 2&=4+y_2\\
2&=-2+x_2&&-6&=4+y_2\\
x_2&=2+2&&y_2&=-6-4\\
x_2&=4&&y_2&=-10\\
\end{array}
\end{equation*}
\westep{Skryf die finale antwoord neer}
Die ko\"ordinate van punt $D$ is $(4;-10)$.
}
\end{wex}
\vspace*{-40pt}
\begin{wex}{Gebruik middelpuntformule}{Punte $E(-1;0)$ , $F(0;3)$ , $G(8;11)$ en $H(x;y)$ is op die Cartesiese vlak. Vind $H(x;y)$ as $EFGH$ 'n parallelogram is.}
{
\westep{Maak 'n skets}
\begin{center}
\scalebox{1} % Change this value to rescale the drawing.
{
\begin{pspicture}(0,-3.6667187)(8.549063,3.7067187)
\psset{xunit=1.2,yunit=1.2}
\rput(1.87,-2.6667187){\psaxes[linewidth=0.04,arrowsize=0.05291667cm 2.0,arrowlength=1.4,arrowinset=0.4,ticksize=0.10583333cm,dx=0.6cm,dy=0.6cm]{<->}(0,0)(-1,-1)(5,6)}
\psdots[dotsize=0.12](5.87,2.8332813)
\psline[linewidth=0.04cm](1.89,-1.1667187)(5.85,2.8132813)
\psdots[dotsize=0.12](1.87,-1.1667187)
\usefont{T1}{ppl}{m}{n}
\rput(7,-2.4567187){$x$}
\usefont{T1}{ppl}{m}{n}
\rput(2.1545312,3.5032814){$y$}
\usefont{T1}{ppl}{m}{n}
\rput(6.534531,2.9832811){$G(8;11)$}
\usefont{T1}{ppl}{m}{n}
\rput(2,-2.8567188){$0$}
\usefont{T1}{ppl}{m}{n}
\rput(6.304531,1.4032812){$H(x;y)$}
\usefont{T1}{ppl}{m}{n}
\rput(0.76453125,-2.3967187){$E(-1;0)$}
\psline[linewidth=0.04cm](1.87,-1.1467187)(1.35,-2.7267187)
\psline[linewidth=0.04cm,linestyle=dashed,dash=0.16cm 0.16cm](1.43,-2.6267188)(5.55,1.3932812)
\psdots[dotsize=0.12](1.37,-2.6667187)
\usefont{T1}{ppl}{m}{n}
\rput(0.9,-1.1567187){$F(0;3)$}
\psline[linewidth=0.02cm](1.51,-2.4067187)(5.87,2.7932813)
\psline[linewidth=0.02cm](1.89,-1.1867187)(5.55,1.4532813)
\usefont{T1}{ppl}{m}{n}
\rput(3.5445313,-0.27671874){$M$}
\psline[linewidth=0.04cm,linestyle=dashed,dash=0.16cm 0.16cm](5.87,2.7732813)(5.57,1.4932812)
\end{pspicture} 
}
\end{center}

Metode: die diagonale van 'n parallelogram halveer mekaar, daarom sal die middelpunt van $EG$ ook die middelpunt van
$FH$ wees. Ons moet eerste die middelpunt van $EG$ kry en dit dan gebruik om die ko\"ordinate van $H$ te kry.
\westep{Ken waardes toe aan $(x_1;y_1)$ en $(x_2;y_2)$}
Laat $M(x;y)$ die middelpunt van $EG$ wees.
\begin{equation*}
x_1=-1 \hskip2em y_1=0 \hskip2em x_2=8 \hskip2em y_2=11
\end{equation*}
\westep{Skryf die middelpuntformule neer}
\begin{equation*}
 M(x;y) =\Big(\frac{x_1+x_2}{2}; \frac{y_1+y_2}{2}\Big)
\end{equation*}
\westep{Vervang waardes en bereken die ko\"ordinate van  $M$}
\begin{equation*}
M(x;y) =\Big(\frac{-1+8}{2}; \frac{0+11}{2}\Big) = \Big(\frac{7}{2};\frac{11}{2}\Big)
\end{equation*}

\westep{Gebruik die ko\"odinate van $M$ om $H$ te vind}
$M$ is ook 'n middelpunt van $FH$, so ons gebruik $M(\frac{7}{2};\frac{11}{2})$ en $F(0;3)$ om $H(x;y)$ te vind.
\westep{Vervang waardes en los op vir $x$ en $y$}
\begin{equation*}
\begin{array}{rllrl}
 M\left(\dfrac{7}{2};\dfrac{11}{2}\right) &=~\left(\dfrac{x_1+x_2}{2}; \dfrac{y_1+y_2}{2}\right)\\[8pt]
\dfrac{7}{2}&=~\dfrac{0+x}{2}&\hskip10em&\dfrac{11}{2}&=~\dfrac{3+y}{2}\\
7&=~x+0&&11&=~3+y\\
x&=~7&&y&=~8\\
\end{array}
\end{equation*}
\westep{Skryf die finale antwoord}

Die ko\"ordinate van $H$ is $(7;8)$.
}
\end{wex}
% \Tip{Remember to draw sketches!}
\begin{exercises}{}
\begin{enumerate}[itemsep=5pt, label=\textbf{\arabic*}. ]
\item Vind die middelpunt van die volgende lynstukke:
  \begin{enumerate}[noitemsep, label=\textbf{(\alph*)} ]
\item $A(2;5)$, $B(-4;7)$
\item $C(5;9)$, $D(23;55)$
\item $E(x+2;y-1)$, $F(x-5;y-4)$
\end{enumerate}

\item $M$, die middelpunt van $PQ$ is $(3;9)$. Vind $P$ as $Q$  $(-2;5)$ is.
\item $PQRS$ is 'n parallelogram met punte $P(5;3)$, $Q(2;1)$ en $R(7;-3)$. Vind punt $S$.

%\item $MPNO$ is 'n parallelogram met punte $M(5;3)$, $N(2;1)$ en $O(7;-3)$. Vind punt $P$.
\end{enumerate}
% Automatically inserted shortcodes - number to insert 3
\par \practiceinfo
\par \begin{tabular}[h]{cccccc}
% Question 1
(1.)	02ix	&
% Question 2
(2.)	02iy	&
% Question 3
(3.)	02iz	&
\end{tabular}
% Automatically inserted shortcodes - number inserted 3
\end{exercises}    
\clearpage
\summary{}
\begin{itemize}[noitemsep]
\item 'n Punt is 'n geordende getallepaar geskryf as $(x;y)$.
\item Afstand is 'n maatstaf van die lengte van die lyn tussen twee punte.
\item Die formule van die afstand tussen twee punte is: 
\begin{equation*}
d=\sqrt{{({x}_{1}-{x}_{2})}^{2}+{({y}_{1}-{y}_{2})}^{2}}
\end{equation*}
\item Die gradi\"ent tussen twee punte is die verhouding tussen die vertikale verandering en die horisontale verandering.

\item Die formule van die gradiënt van ’n lyn is: 
\begin{equation*}
m=\frac{{y}_{2}-{y}_{1}}{{x}_{2}-{x}_{1}}
\end{equation*}
\item 'n Reguitlyn is 'n versameling punte met 'n konstante gradi\"ent tussen enige twee van die punte.
\item Die standaardvorm van 'n reguitlynvergelyking is $y=mx+c$.
\item Die vergelyking van 'n reguitlyn kan ook geskryf word as:
\begin{equation*}
\dfrac{y-y_1}{x-x_1}=\dfrac{y_2-y_1}{x_2-x_1}\end{equation*}
\item As twee lyne parallel is, sal hulle dieselfde gradiënt hê.
\item As twee lyne loodreg op mekaar is,
dan het ons die produk van die gradi\"ente gelyk aan $-1$.
\item Die gradi\"ent van 'n horisontale lyn is $0$.
\item Die gradi\"ent van 'n vertikale lyn is ongedefinieerd.
\item Die formule om die middelpunt van die lyn tussen twee punte te vind: 
\begin{equation*}
M(x;y) = \Big(\frac{{x}_{1}+{x}_{2}}{2};\frac{{y}_{1}+{y}_{2}}{2}\Big)
\end{equation*}
\end{itemize}


\begin{eocexercises}{}
\begin{enumerate}[noitemsep, label=\textbf{\arabic*}. ] 
\item 
Stel die volgende vorms op die Cartesiese vlak voor: 
\begin{enumerate}[noitemsep, label=\textbf{(\alph*)} ]
\item Driehoek $DEF$ met $D(1;2)$, $E(3;2)$ en $F(2;4)$ 
\item Vierhoek $GHIJ$ met $G(2;-1)$, $H(0;2)$, $I(-2;-2)$ en $J(1;-3)$
\item Vierhoek $MNOP$ met $M(1;1)$, $N(-1;3)$, $O(-2;3)$ en $P(-4;1)$ 
\item Vierhoek $WXYZ$ met $W(1;-2)$, $X(-1;-3)$, $Y(2;-4)$ en $Z(3;-2)$
\end{enumerate}
\item 
In die gegewe diagram is die hoekpunte van ’n veelhoek $F(2;0)$, $G(1;5)$, $H(3;7)$ en $I(7;2)$.
    \setcounter{subfigure}{0}
    \begin{figure}[H] % horizontal\label{m39107*id63458}
      \begin{center}
        \scalebox{.8}{
          \begin{pspicture}(-5,-5)(5.5,5.5)
            % \psaxes{<->}(0,0)(5,5)
            % \psgrid[gridcolor=lightgray,linecolor=lightgray,subgriddiv=1](0,0)(-1,-1)(7,7)
            \psaxes[linewidth=1pt,labels=all,ticks=all]{<->}(0,0)(-1,-1)(7.5,7.5)
            \pspolygon[linewidth=1pt](2,0)(1,5)(3,7)(7,2)(2,0)
            \psdots(2,0)(1,5)(3,7)(7,2)
            % \psline[linewidth=.05cm](0,0)(3,3.5)
            % \psline[linewidth=.05cm](0,0)(3,2)
            % \psline[linewidth=.05cm](0,0)(3,0.5)
            % \uput[ur](.9,3.5){\Large{$T$}}
            \uput[d](1.3,.6){\normalsize{$F (2;0)$}}
            \uput[ul](1.4,5.1){\normalsize{$G (1;5)$}}
            \uput[u](3,7){\normalsize{$H (3;7)$}}
            \uput[r](7,2){\normalsize{$I (7;2)$}}
            \uput[l](8,0){\Large{$x$}}
            \uput[d](0,8){\Large{$y$}}
            \uput[d](-0.2,0){\Large{$0$}}
          \end{pspicture}
        }
      \end{center}
    \end{figure}  
\begin{enumerate}[noitemsep, label=\textbf{(\alph*)} ]
\item Vind die lengtes van die sye van $FGHI$.
\item Is die teenoorstaande sye van $FGHI$ parallel?
\item Toon deur berekening of die hoeklyne van $FGHI$ mekaar halveer.
\item Watter tipe veelhoek is  $FGHI$? Gee redes vir jou antwoord.
\end{enumerate}
\item  ’n Veelhoek $ABCD$ met hoekpunte $A(3;2)$, $B(4;5)$, $C(1;7)$ en $D(1;3)$ word gegee.
\begin{enumerate}[noitemsep, label=\textbf{(\alph*)} ]
\item  Teken die veelhoek.
\item  Bepaal die sylengtes van die veelhoek.
\end{enumerate}
\item $ABCD$ is ’n veelhoek met hoekpunte $A(0;3)$, $B(4;3)$, $C(5;-1)$ en $D(-1;-1)$.
\begin{enumerate}[noitemsep, label=\textbf{(\alph*)} ]
\item Wys dat:
\begin{enumerate}[noitemsep, label=\textbf{(\roman*)} ]
\item $AD = BC$
\item $AB \parallel DC$
\end{enumerate}
\item Benoem $ABCD$.
\item Wys dat die hoeklyne $AC$ en $BD$ mekaar nie halveer nie.
\end{enumerate}
\item $P$, $Q$, $R$ en $S$ is die punte $(-2;0)$, $(2;3)$, $(5;3)$ en $(-3;-3)$ onderskeidelik.
\begin{enumerate}[noitemsep, label=\textbf{(\alph*)} ]
\item Wys dat:
\begin{enumerate}[noitemsep, label=\textbf{(\roman*)} ]
\item $SR = 2PQ$
\item $SR \parallel PQ$
\end{enumerate}
\item Bereken die lengte van:
\begin{enumerate}[noitemsep, label=\textbf{(\roman*)} ]
\item $PS$
\item $QR$
\end{enumerate}
\item Watter tipe veelhoek is $PQRS$? Gee redes vir jou antwoord.
\end{enumerate}
\item $EFGH$ is ’n parallelogram met hoekpunte $E(-1;2)$, $F(-2;-1)$ en $G(2;0)$. Vind die koördinate van H deur gebruik te maak van die feit dat die hoeklyne van ’n parallelogram mekaar halveer.
\item $PQRS$ is ’n vierhoek met punte $P(0;-3)$ ; $Q(-2;5)$ ; $R(3;2)$ en $S(3;-2)$  in die Cartesiese vlak.
\begin{enumerate}[noitemsep, label=\textbf{(\alph*)} ]
\item Vind die lengte van $QR$.
\item Bepaal die gradiënt van $PS$.
\item Vind die middelpunt van $PR$.
\item Is $PQRS$  ’n parallelogram? Verskaf redes vir jou antwoord.
 \end{enumerate}
\item $A(-2;3)$ en $B(2;6)$ is punte op die Cartesiese vlak. $C(a;b)$ is die middelpunt van $AB$. Vind die waardes
van $a$ en $b$.
\item Beskou: $\triangle ABC$ met hoekpunte $A(1; 3)$ , $B(4;1)$ en $C (6; 4)$:
\begin{enumerate}[noitemsep, label=\textbf{(\alph*)} ]
\item Skets die driehoek $ABC$ op die Cartesiese vlak. 
\item Wys dat $ABC$ is ’n gelykbenige driehoek is.
\item Bepaal die koördinate van $M$, die middelpunt van $AC$.
\item Bepaal die gradiënt van $AB$.
\item Toon aan dat die volgende punte op dieselfde reguitlyn l\^e: $A$, $B$ en $D(7;-1)$
\end{enumerate}
\item In die diagram is $A$ die punt $(-6;1)$ en $B$ is die punt $(0;3)$:
    \setcounter{subfigure}{0}
    \begin{figure}[H] % horizontal\label{m39107*id63458}
      \begin{center}
        \scalebox{.8}{
          \begin{pspicture}(-5,-5)(5.5,5.5)
            % \psaxes{<->}(0,0)(5,5)
            % \psgrid[gridcolor=lightgray,linecolor=lightgray,subgriddiv=1](0,0)(-10,-1)(1,7)
            \psaxes[linewidth=1pt,labels=all,ticks=all]{<->}(0,0)(-10,-1)(1,7)
            \psline[linewidth=.05cm](-6,1)(0,3)
            % \psline[linewidth=.05cm](0,0)(3,3.5)
            % \psline[linewidth=.05cm](0,0)(3,2)
            % \psline[linewidth=.05cm](0,0)(3,0.5)
            % \uput[ur](.9,3.5){\Large{$T$}}
            \uput[d](-6,1){\Large{$A (-6;1)$}}
            \uput[d](-5.9,1.2){\qdisk(0,0){3pt}}
            \uput[r](.2,3){\Large{$B (0;3)$}}
            \uput[d](0,3.2){\qdisk(0,0){3pt}}
            \uput[l](1.5,0){\Large{$x$}}
            \uput[d](0,7.5){\Large{$y$}}
            \uput[d](-0.2,0.1){\Large{$0$}}
          \end{pspicture}
        }
      \end{center}
    \end{figure} 
\begin{enumerate}[noitemsep, label=\textbf{(\alph*)} ]
\item Vind die vergelyking van die lyn $AB$. 
\item Bereken die lengte van $AB$.
% \item  $A'$ is the image of $A$ and $B'$ is the image of $B$. Both these images are obtain by applying the transformation: $(x;y)\to(x - 4;y - 1)$. Give the coordinates of both $A'$ and $B'$
% \item Find the equation of $A'B'$
% \item Calculate the length of $A'B'$
% \item Can you state with certainty that $AA'B'B$ is a parallelogram? Justify your answer.
\end{enumerate}
\item $\triangle PQR$ het hoekpunte $P(1;8)$, $Q(8;7)$ en $R(7;0)$.  Wys deur berekening dat $\triangle PQR$ 'n reghoekige gelykbenige driehoek is.
\clearpage
\item $\triangle ABC$ het hoekpunte $A(-3;4)$, $B(3;-2)$ en $R(-5;-2)$. $M$ is die middelpunt van $AC$ en $N$ die middelpunt van $BC$. Gebruik $\triangle ABC$ om die middelpuntstelling deur analitiese meetkundige metodes te bewys. 
\end{enumerate}
% Automatically inserted shortcodes - number to insert 12
\par \practiceinfo
\par \begin{tabular}[h]{cccccc}
% Question 1
(1.)	02j0	&
% Question 2
(2.)	02j1	&
% Question 3
(3.)	02j2	&
% Question 4
(4.)	02j3	&
% Question 5
(5.)	02j4	&
% Question 6
(6.)	02j5	\\ % End row of shortcodes
% Question 7
(7.)	02j6	&
% Question 8
(8.)	02j7	&
% Question 9
(9.)	02j8	&
% Question 10
(10.)	02j9	&
% Question 11
(11.)	02ja	&
% Question 12
(12.)	02jb	\\ % End row of shortcodes
\end{tabular}
% Automatically inserted shortcodes - number inserted 12
% \
% \par \raisebox{-5 pt}{\includegraphics[width=0.5cm]{col11306.imgs/summary_www.png}} Kry die oplossing met die kortkodes:
% \par \begin{tabular}[h]{cccccc}
% (1.) lgv  &  (2.) liZ  &  (3.) liB  &  (4.) lac  &  (5.) lax  &  (6.) laa  &  (7.) laY  &  (8.) lag  &  (9.) la4  &  (10.) l4o  & \end{tabular}
\end{eocexercises}