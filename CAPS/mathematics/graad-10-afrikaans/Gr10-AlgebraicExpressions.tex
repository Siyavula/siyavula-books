\chapter{Algebra\"iese uitdrukkings}
\fancyfoot[LO,RE]{Fokus Area: Wiskunde}
\section{Die re\"ele getalsisteem}
\begin{figure}[H] % horizontal\label{m38348*id62548}
\begin{center}
% \label{m38348*id62548!!!underscore!!!media}\label{m38348*id62548!!!underscore!!!printimage}
%\includegraphics[width=9cm]{col11306.imgs/m38348_MG10C3_001.png} % m38348;MG10C3\_001.png;;;6.0;8.5;
\scalebox{0.6} % Change this value to rescale the drawing.
{
\begin{pspicture}(0,-4.764375)(14.481563,4.804375)
\psellipse[linewidth=0.04,dimen=outer](6.81,-0.484375)(6.81,4.28)
\psline[linewidth=0.04cm](8.18,3.695625)(8.26,-4.684375)
\psellipse[linewidth=0.04,dimen=outer](4.34,-1.364375)(3.8,2.5)
\psellipse[linewidth=0.04,dimen=outer](3.57,-1.854375)(2.57,1.61)
% \usefont{T1}{ppl}{b}{n}
\rput(6.735781,4.350625){\Huge Re\"eel $\mathbb{R}$}
% \usefont{T1}{ppl}{b}{n}
\rput(11.03875,-0.484375){\Large Irrasionaal $\mathbb{Q'}$}
% \usefont{T1}{ppl}{b}{n}
\rput(5.11875,1.975625){\Large Rasionaal $\mathbb{Q}$}
% \usefont{T1}{ppl}{b}{n}
\rput(4.06875,0.275625){\Large Heel $\mathbb{Z}$}
% \usefont{T1}{ppl}{b}{n}
\rput(3.21875,-2.324375){\Large Natuurlik $\mathbb{N}$}
\psellipse[linewidth=0.04,dimen=outer](3.14,-2.354375)(1.68,0.83)
% \usefont{T1}{ptm}{b}{n}
\rput(3.5735939,-1.024375){\Large Tel $\mathbb{N}_0$}
\end{pspicture} 
}
\vspace{2pt}
\vspace{.1in}
\end{center}
\end{figure}   

% \setcounter{subfigure}{1}

% ’n Getal is ’n manier om ’n hoeveelheid voor te stel. Die getalle wat op hoërskool gebruik sal word is almal reëel, maar daar is heelwat verskillende maniere om enige gegewe reële
% getal voor te stel.\par Hierdie hoofstuk beskryf rasionale getalle.\par 
 
% \par 
% \subsection*{Die groot prentjie van getalle}
% \addcontentsline{toc}{subsection}{Die oorhoofse beskouing van getalle}
% \setcounter{subfigure}{0}
    
\par 
Ons gebruik die volgende definisies:\par 
\begin{itemize}[itemsep=5pt]
\item $\mathbb{N}$: natuurlike getalle is $\{1; 2; 3; \ldots\}$
\item $\mathbb{N}_0$: telgetalle is $\{0; 1; 2; 3; \ldots\}$
\item $\mathbb{Z}$: heelgetalle is $\{\ldots -3; -2; -1; 0; 1; 2; 3; \ldots\}$
\end{itemize}

\par
\chapterstartvideo{VMabo}

\section{Rasionale en irrasionale getalle}
% \setcounter{figure}{1}
\nopagebreak

\Definition{Rasionale getal}{
’n Rasionale getal is enige getal wat geskryf kan word as: 
%       \label{m38348*uid6}\nopagebreak\noindent{}

\begin{equation*}
\frac{a}{b}
\end{equation*}
waar $a$ en $b$ heelgetale is en $b\ne 0$. \par 
} 


Die volgende getalle is almal rasionaal.\par 
\nopagebreak\noindent{}

\begin{equation*}
\frac{10}{1};\frac{21}{7};\frac{-1}{-3};\frac{10}{20};\frac{-3}{6}
\end{equation*}
Ons sien dat al die tellers en noemers heelgetalle is. Dit beteken dat alle heelgetalle rasionaal is, aangesien hulle geskryf kan word met ’n noemer van $1$.\par 
% Dus is
% 
% \begin{equation*}
% \frac{\sqrt{2}}{7} ; \frac{20}{\pi}
% \end{equation*}
% nie voorbeelde van rasionale getalle nie, want in elke geval is óf die teller óf die noemer nie ’n heelgetal nie.\par 
% ’n Getal wat nie geskryf word in die vorm van ’n heelgetal gedeel deur ’n heelgetal nie kan nogtans ’n rasionale
% getal wees. Dit is omdat die vereenvoudigde resultaat wel as ’n kwosiënt van heelgetalle geskryf kan word. Die
% reël is dat indien ’n getal geskryf kan word as ’n kwosiënt van heelgetalle, dit rasionaal is, selfs al kan dit op ’n
% manier geskryf word wat nie so ’n kwosiënt is nie. Hier is twee voorbeelde wat dalk nie na rasionale getalle lyk
% nie, maar nogtans is, omdat daar ekwivalente vorms gevind kan word wat bestaan uit ’n heelgetal gedeel deur ’n
% heelgetal:\par 
% \nopagebreak\noindent{}
% \begin{equation*}    
% \frac{-1,33}{-3}=\frac{133}{300}; ~~~~~~\frac{-3}{6,39}=\frac{-300}{639}=\frac{-100}{213}
% \end{equation*}
% 
% 
% \section{Irrasionale getalle}
% \setcounter{figure}{1}
% \setcounter{subfigure}{1}

% 
% Jy het reeds gesien dat baie ink en papier nodig sou wees om repeterende desimale getalle neer te skryf. Dis
% nie net onmoontlik om hierdie getalle neer te skryf nie, maar om énige getal tot baie desimale plekke of met hoë
% akkuraatheid neer te skryf, is gewoonlik onprakties. Daarom benader ons dikwels ’n getal tot ’n sekere aantal
% desimale plekke of, selfs beter, tot ’n sekere aantal beduidende syfers.\par 

\Definition{Irrasionale getalle}{Irrasionale getalle ($\mathbb{Q'}$) is getalle wat nie as ’n breuk met ’n heeltallige teller en noemer geskryf kan word nie. Dit
beteken dat enige getal wat nóg ’n eindige nóg ’n herhalende desimale getal is, irrasionaal is. 
}

Voorbeelde van irrasionale getalle is:\par 

\begin{equation*}
\sqrt{2};~\sqrt{3};~\sqrt[3]{4};~\pi ;
~\frac{1+\sqrt{5}}{2}
\end{equation*}

Hierdie is nie rasionale getalle nie, want óf die teller óf die noemer is nie heeltallig nie. 

% Wanneer irrasionale getalle in desimaalnotasie geskryf word, het hulle ’n oneindige aantal desimale syfers wat nooit herhaal nie. \\Die vierkantswortels van non-vierkante en die derde-magswortels van non-derdemagte is almal irrasionaal.

% \begin{activity}{Irrasionale getalle}
% \nopagebreak
% Watter van die volgende kan nie as 'n rasionale getalle geskryf word nie?\par \vspace{0.5cm}
% \textbf{Onthou}: ’n Rasionale getal is ’n breuk met ’n heeltallige teller en noemer. Eindige en herhalende desimale
% getalle is rasionaal.\par 
% \begin{enumerate}[itemsep=5pt, label=\textbf{\arabic*}. ] 
% \item $\pi =3,14159265358979323846264338327950288419716939937510\ldots$
% \item $1,4$
% \item $1,618\phantom{\rule{0.166667em}{0ex}}033\phantom{\rule{0.166667em}{0ex}}989\phantom{\rule{0.166667em}{0ex}}\ldots$
% \item $100$
% \item $1,7373737373\ldots$
% \item $0,\overline{02}$
% \end{enumerate}
% \end{activity}
% 


 \subsection*{Desimale getalle}
\addcontentsline{toc}{subsection}{Decimal numbers}
\nopagebreak

Alle heelgetalle en heeltallige kwosiënte is rasionaal. Daar is twee bykomende vorme van rasionale getalle.\par 
% 
% 
% \begin{activity}{Desimale getalle}
% \nopagebreak
% Jy kan die rasionale getal
% $\frac{1}{2}$ skryf as die desimale getal $0,5$. Skryf die volgende getalle as desimale getalle:\par 
% \begin{enumerate}[itemsep=5pt, label=\textbf{\arabic*}. ] 
% \item $\dfrac{1}{4}$
% \item $\dfrac{1}{10}$
% \item $\dfrac{2}{5}$
% \item $\dfrac{1}{100}$
% \item $\dfrac{2}{3}$
% \end{enumerate}
% Beskou die getalle na die desimale komma. Kom hulle tot ’n einde of gaan hulle voort? Indien hulle voortgaan, is
% daar ’n herhalende patroon in die getalle? \par 
% \end{activity}

Jy kan ’n rasionale getal as ’n desimale getal skryf. Twee tipes desimale getalle wat as rasionale getalle geskryf
kan word:\par 
\begin{itemize}
\item Desimale getalle waarvan die nie-nul getalle na die komma tot ’n einde kom of termineer, byvoorbeeld die breuk
 $\frac{4}{10}$ kan geskryf word as $0,4$.
\item Desimale getalle wat ’n nimmereindigende herhalende patroon van getalle na die komma het, byvoorbeeld die breuk $\frac{1}{3}$ kan geskryf word as
$0,\dot{3}$. 
Die kolletjie beteken dat die $3$’e repeteer, m.a.w. $0,333\ldots=0,\dot{3}$.
\item Desimale getalle wat 'n herhalende patroon van meer as een syfer het, byvoorbeeld die breuk $\frac{2}{11}$ kan geskryf word as 
$0,\overline{18}$. 
Die staaf stel 'n herhalende patroon van $1$ en $8$ voor, m.a.w.
$0,\overline{18} = 0,181818\ldots$.
\end{itemize}

\par
\textbf{Notasie:} Jy kan 'n kolletjie of 'n staaf bo die herhalende getalle gebruik om aan te dui dat die desimaal herhalend is. As die staaf bo meer as een getal is, beteken dit dat al die getalle onder die staaf herhalend is. 

\par
As jy gevra word om uit te werk of ’n getal rasionaal of irrasionaal is, skryf eers die getal in desimaalnotasie. As die desimaal eindig, is die getal rasionaal. As dit vir ewig aanhou, soek vir ’n herhalende syferpatroon. As daar geen patroon is nie, is die getal irrasionaal.
\par 
As jy ’n irrasionale getal in desimaalnotasie skryf, kan jy aanhou skryf vir baie, baie
syfers. Dit is egter ongerieflik en ’n mens rond gewoonlik af.\par
%english
\clearpage
\begin{wex}{Rasionale en irrasionale getalle}
{
\begin{minipage}{\textwidth}
Watter van die volgende getalle is nie rasionaal nie?\\

\begin{enumerate}[itemsep=5pt, label=\textbf{\arabic*}. ] 
\item $\pi =3,14159265358979323846264338327950288419716939937510\ldots$
\item $1,4$
\item $1,618033989\ldots$
\item $100$
\item $1,7373737373\ldots$
\item $0,\overline{02}$
\end{enumerate}
\end{minipage}
}
{
\begin{minipage}{\textwidth}
\begin{enumerate}[itemsep=5pt, label=\textbf{\arabic*}. ] 
\item Irrasionaal, desimaal termineer nie en het nie 'n herhalende patroon nie. 
\item Rasionaal, desimaal termineer.
\item Irrasionaal, desimaal termineer nie en het nie 'n herhalende patroon nie. 
\item Rasionaal, alle heelgetalle is rasionaal.
\item Rasionaal, desimaal het 'n herhalende patroon.
\item Rasionaal, desimaal het 'n herhalende patroon.
\end{enumerate}
\end{minipage}
}
\end{wex}

\subsection*{Omskakeling tussen eindigende desimale getalle en rasionale getalle}
\addcontentsline{toc}{subsection}{Omskakeling tussen terminerende desimale getalle en rasionale getalle}
’n Desimale getal het ’n heeltallige deel en ’n breukdeel. Byvoorbeeld, $10,589$ het ’n heeltallige deel van $10$ en ’n
breukdeel van $0,589$ omdat $10+0,589=10,589$. Die breukdeel kan geskryf word as ’n rasionale getal, m.a.w.
met ’n teller en ’n noemer wat heelgetalle is. \\
\\
Elke syfer na die desimale komma is ’n breuk met ’n noemer wat ’n vermeerderende mag van $10$ is. Byvoorbeeld: 
\begin{itemize}
 \item $0,1$ is $\dfrac{1}{10}$
\item $0,01$ is $\dfrac{1}{100}$
\item $0,001$ is $\dfrac{1}{1000}$
\end{itemize}

Dit beteken dat:
\begin{align*}
  10,589 &= 10 + \dfrac{5}{10} + \dfrac{8}{100} + \dfrac{9}{1~000} \\
  &= \dfrac{10~000}{1~000} + \dfrac{500}{1~000} + \dfrac{80}{1~000} + \dfrac{9}{1~000} \\
  &= \dfrac{10~589}{1~000} \\
\end{align*}

\subsection*{Omskakeling van repeterende desimale na \\rasionale getalle}


Wanneer die desimaal repeterend is, is daar ’n bietjie meer werk nodig om die breukdeel van die desimale getal
as ’n breuk te skryf.\par 
\
\begin{wex}
{%title
Omskakeling van  desimale getalle na breuke
}
{%question
\\
Skryf $0,\dot{3}$ in die vorm $\dfrac{a}{b}$ waar $a$ en $b$ heelgetalle is.
}
{%answer

\westep{Stel 'n vergelyking op}
Laat $x = 0,33333\ldots$


\westep{Vermenigvuldig met $10$ aan beide kante}

$$10x = 3,33333\ldots$$


\westep{Trek die tweede vergelyking van die eerste vergelyking af}

$$9x = 3 $$

\westep{Vereenvoudig}

$$ x = \dfrac{3}{9} = \dfrac{1}{3} $$
}
\end{wex}

\clearpage
\begin{wex}
{%title
Omskakeling van desimale getalle na breuke
}
{%question
\\Skryf $5,\dot{4}\dot{3}\dot{2}$ as 'n rasionale breuk.
}
{%answer

\westep{Stel 'n vergelyking op}

$$ x = 5,432432432\ldots $$

\westep{Vermenigvuldig met $1000$ aan beide kante}

$$ 1~000x = 5~432,432432432\ldots $$

\westep{Trek die tweede vergelyking van die eerste vergelyking af}

$$ 999x = 5~427 $$

\westep{Vereenvoudig}

$$ x = \dfrac{5~427}{999} = \dfrac{201}{37}=5\dfrac{16}{37} $$

}
\end{wex}

In die eerste voorbeeld is die desimaal met $10$  vermenigvuldig en in die tweede voorbeeld is dit met $1000$ vermenigvuldig. Dit is omdat daar in die eerste voorbeeld slegs een repeterende syfer (nl. $3$) was, terwyl die tweede voorbeeld drie repeterende syfers (nl. $432$) gehad het.\par 
In die algemeen, as jy een repeterende syfer het, vermenigvuldig jy met $10$.  As jy twee repeterende syfers het, vermenigvuldig jy met $100$.  Met drie syfers vermenigvuldig jy met $1000$ en so aan.\clearpage

Nie alle desimale getalle kan as rasionale getalle geskryf word nie. Hoekom nie? Irrasionale desimale getalle soos
$\sqrt{2}=1,4142135\ldots$
kan nie geskryf word met ’n heeltallige teller en noemer nie, omdat daar geen patroon van repeterende syfers is nie. Jy behoort egter, so ver moontlik, eerder rasionale getalle of breuke as desimale getalle te gebruik.






\begin{exercises}{}{
\begin{enumerate}[itemsep=5pt, label=\textbf{\arabic*}. ] 
\item Sê of die volgende getalle rasionaal of irrasionaal is. As die getal rasionaal is, sê of dit 'n natuurlike getal, telgetal of heelgetal is:
\begin{enumerate}[itemsep=5pt, label=\textbf{(\alph*)} ] 
    \item $-\dfrac{1}{3}$
    \item $0,651268962154862\ldots$
    \item $\dfrac{\sqrt{9}}{3}$
    \item $\pi^2$
\end{enumerate}
\item As $a$ 'n heelgetal is, $b$ 'n heelgetal is en $c$ irrasionaal is, watter van die volgende is rasionale getalle? 
  \begin{enumerate}[itemsep=5pt, label=\textbf{\alph*}. ] 
    \item $\dfrac{5}{6}$
    \item $\dfrac{a}{3}$
    \item $\dfrac{-2}{b}$
    \item $\dfrac{1}{c}$
    \end{enumerate}
\item Vir watter van die volgende waardes van $a$ is $\frac{a}{14}$ rasionaal of irrasionaal?
    \begin{enumerate}[itemsep=5pt, label=\textbf{\alph*}. ] 
    \item $1$
    \item $-10$
    \item $\sqrt{2}$
    \item $2,1$
    \end{enumerate}

\item Skryf die volgende as breuke:
    \begin{enumerate}[itemsep=5pt, label=\textbf{\alph*}. ] 
    \item $0,1$
    \item $0,12$
    \item $0,58$
    \item $0,2589$
    \end{enumerate}

\item Skryf die volgende in repeterende (herhalende) desimale notasie:
    \begin{enumerate}[itemsep=5pt, label=\textbf{\alph*}. ] 
    \item $0,11111111\ldots$
    \item $0,1212121212\ldots$
    \item $0,123123123123\ldots$
    \item $0,11414541454145\ldots$
    \end{enumerate}
\item Skryf die volgende in repeterende (herhalende) desimale notasie:
    \begin{enumerate}[itemsep=5pt, label=\textbf{\alph*}. ] 
    \item $\dfrac{2}{3}$
    \item $1\dfrac{3}{11}$
    \item $4\dfrac{5}{6}$
    \item $2\dfrac{1}{9}$
    \end{enumerate}
\item Skryf die volgende in breukvorm:
    \begin{enumerate}[itemsep=5pt, label=\textbf{\alph*}. ] 
    \item $0,\dot{5}$
    \item $0,6\dot{3}$
    \item $5,\overline{31}$
    \end{enumerate}
\end{enumerate}

% Automatically inserted shortcodes - number to insert 7
\par \practiceinfo
\par \begin{tabular}[h]{cccccc}
% Question 1
(1.)	02ee	&
% Question 2
(2.)	02ef	&
% Question 3
(3.)	02eg	&
% Question 4
(4.)	02eh	&
% Question 5
(5.)	02ei	&
% Question 6
(6.)	02ej	\\ % End row of shortcodes
% Question 7
(7.)	02ek	&
\end{tabular}
% Automatically inserted shortcodes - number inserted 7
}
\end{exercises}

\section{Afronding}

Afronding van ’n desimale getal tot ’n sekere aantal desimale plekke is ’n eenvoudige manier om die benaderde waarde van ’n desimale getal te vind. As jy byvoorbeeld $2,6525272$  tot drie desimale plekke wil afrond:

\begin{itemize}
\item tel drie plekke na die desimale komma en plaas 'n $|$ tussen die derde en vierde syfers
\item rond die derde syfer na bo af as die vierde syfer groter as of gelyk aan $5$ is
\item los die derde syfer onverandend as die derde syfer kleiner as $5$ is
\item as die derde syfer $9$ is en afgerond moet word, dan raak die $9$ 'n $0$ en die tweede syfer sal boontoe afgerond moet word 
%if the third digit is $9$ and needs to be round up, then the $9$ becomes a $0$ and the second digit rounded up
\end{itemize}
\par 
% 
% 
% \begin{equation*}
% 2,652|5272
% \end{equation*}
% Nadat jy vasgestel het of die syfer in die derde desimale plek na bo of na onder afgerond moet word, word al
% die syfers aan die regterkant van die $|$ geïgnoreer. Jy rond die finale syfer na bo af as die eerste syfer ná die $|$ groter of gelyk is aan $5$, andersins rond jy na onder af (los die syfer onveranderd). Wanneer die eerste syfer links
% van die $|$ 'n $9$ is en jy moet boontoe afrond, dan word die $9$ 'n $0$ en die tweede syfer links van die  $|$ word boontoe afgerond.
\par 
Dus, aangesien die eerste syfer na die $|$ 'n $5$ is, moet ons na bo afrond en die derde desimaal na die komma word $3$. Die antwoord vir $2,6525272$ afgerond na drie desimale plekke is dus $2,653$.
\par

\vspace*{-30pt}
\begin{wex}{Afronding}

{
\begin{minipage}{\textwidth}
Rond die volgende getalle af: 

\begin{enumerate}[itemsep=2pt, label=\textbf{\arabic*}. ] 

\item $\frac{120}{99}=1,212121212\dot{1}\dot{2}$ tot $3$ desimale plekke
\item $\pi =3,141592654\ldots$ tot $4$ desimale plekke
\item $\sqrt{3}=1,7320508\ldots$ tot $4$ desimale plekke
\item $2,78974526\ldots$ tot $3$ desimale plekke
\end{enumerate}
\end{minipage}\vspace*{-20pt}
}
{

\westep{Merk die gevraagde aantal desimale plekke af}

\begin{minipage}{\textwidth}
\vspace*{7pt}
\begin{enumerate}[itemsep=2pt, label=\textbf{\arabic*}. ] 
\item $\frac{120}{99}=1,212|121212\dot{1}\dot{2}$
\item $\pi =3,1415|92654\ldots$
\item $\sqrt{3}=1,7320|508\ldots$
\item $2,789|74526\ldots$
\end{enumerate}
\end{minipage}
\westep{Kyk na die volgende syfer om te sien of jy boontoe of ondertoe moet afrond}
\begin{minipage}{0.8\textwidth}
\vspace*{7pt}
\begin{enumerate}[itemsep=2pt, label=\textbf{\arabic*}. ]
\item Die laaste syfer van $\frac{120}{99}=1,212|121212\dot{1}\dot{2}$  moet na onder afgerond word.
\item Die laaste syfer van $\pi =3,1415|92654\ldots$ moet na bo afgerond word.
\item Die laaste syfer van $\sqrt{3}=1,7320|508\ldots$ moet na bo afgerond word.
\item Die laaste syfer van $2,789|74526\ldots$ moet na bo afgerond word.  

\end{enumerate}
\vspace*{7pt}
\end{minipage}
\newline Aangesien dit ’n $9$, is, vervang ons dit met $0$ en
rond die tweede laaste syfer boontoe af.
\westep{Skryf die finale antwoord}
\begin{minipage}{\textwidth}
\vspace*{7pt}
\begin{enumerate}[itemsep=2pt, label=\textbf{\arabic*}. ]
\item $\frac{120}{99}=1,212$ afgerond tot $3$ desimale plekke
\item $\pi =3,1416$  afgerond tot $4$ desimale plekke
\item $\sqrt{3}=1,7321$ afgerond tot $4$ desimale plekke
\item $2,790$ afgerond tot $3$ desimale plekke
\end{enumerate}
\end{minipage}
}  
\end{wex}


\begin{exercises}{}
{
Skryf die volgende getalle tot $3$ desimale plekke:
\begin{enumerate}[itemsep=5pt, label=\textbf{\arabic*}. ]
\item $12,56637061\ldots$ %$4\pi$
\item $3,31662479\ldots$ %$\sqrt{11}$
\item $0,26666666\ldots$ %$\dfrac{0,8}{3}$
\item $1,912931183\ldots$ %$\sqrt[3]{7}$
\item $6,32455532\ldots$ %$2\sqrt{10}$
\item $0,05555555\ldots$ %$\dfrac{1}{18}$
\end{enumerate}

% Automatically inserted shortcodes - number to insert 6
\par \practiceinfo
\par \begin{tabular}[h]{cccccc}
% Question 1
(1.-6.)	02em	&
% Question 2
	\\ % End row of shortcodes
\end{tabular}
% Automatically inserted shortcodes - number inserted 6
}
\end{exercises}


\section{Benadering van wortelgetalle}
% \setcounter{figure}{1}
% \setcounter{subfigure}{1}

Indien die $n^{de}$ magswortel van ’n getal nie as ’n rasionale getal geskryf kan word nie,  noem ons dit ’n wortelgetal . Byvoorbeeld, $\sqrt{2}$ en $\sqrt[3]{6}$ is wortelgetalle, maar $\sqrt{4}$ is nie ’n wortelgetal nie, aangesien ons dit kan vereenvoudig tot die rasionale getal $2$.\par 
In hierdie hoofstuk gaan ons slegs wortelgetalle van die vorm $\sqrt[n]{a}$, ondersoek, waar  $a$ ’n positiewe heelgetal is, byvoorbeeld $\sqrt{7}$ or $\sqrt[3]{5}$. Dit kom algemeen voor dat $n$ = $2$, daarom skryf ons $\sqrt{a}$ in plaas van $\sqrt[2]{a}$.\par 
Dit is soms nuttig om ’n wortelgetal te benader sonder om ’n sakrekenaar te gebruik. Ons wil byvoorbeeld kan benader waar $\sqrt{3}$ op die getallelyn lê. Met ’n sakrekenaar kan ons bereken $\sqrt{3}$ is gelyk aan $1,73205\ldots$. Dit is maklik on te sien dat $\sqrt{3}$ tussen $1$ en $2$ lê. Om sonder ’n sakrekenaar te bepaal waar ander wortelgetalle, soos $\sqrt{18}$, op die getallelyn lê, moet jy eers die volgende verstaan:\par 


\Identity{
\vspace*{-3em}
\begin{center}
As $a$ en $b$ positiewe getalle is met $a\lessthan{}b$, dan $\sqrt[n]{a}\lessthan{}\sqrt[n]{b}$. 
\end{center}
}
      
’n Volkome vierkant is die getal wat verkry word wanneer ’n heelgetal met homself vermenigvuldig word. Byvoorbeeld, $9$  is ’n volkome vierkant aangesien ${3}^{2}=9$. \\
Soortgelyk is ’n volkome derdemag die getal wat verkry word wanneer ’n heelgetal
tot die derde mag verhef word. Byvoorbeeld, $27$ is ’n volkome derdemag aangesien ${3}^{3}=27$.
\par 
Gegee die wortelgetal $\sqrt[3]{52}$, behoort jy te kan sien dat dit iewers tussen $3$ en $4$ op die getallelyn lê, aangesien $\sqrt[3]{27}=3$ en $\sqrt[3]{64}=4$ en $52$ tussen $27$ en $64$ lê.


\begin{wex}{Benadering van wortelgetalle}
{
Vind die opeenvolgende heelgetalle wat weerskante van $\sqrt{26}$ lê.
Onthou dat opeenvolgende heelgetalle met $1$ verskil, byvoorbeeld $5$ en $6$ of $8$ en $9$.
}
{
           
\westep{Gebruik die volkome vierkant om die laer heelgetal te skat}     
${5}^{2}=25$. Daarom $5\lessthan{}\sqrt{26}$.

\westep{Gebruik die volkome vierkant on die ho\"er heelgetal te skat}
${6}^{2}=36$. 
Daarom $\sqrt{26}\lessthan{}6$.

\westep{Die finale aantwoord is}
$5\lessthan{}\sqrt{26}\lessthan{}6$. 
}
\end{wex}


\begin{wex}{Benadering van wortelgetalle }{

Vind die opeenvolgende heelgetalle wat weerskante van $\sqrt[3]{49}$ lê.
}
{
\westep{Gebruik volkome derdemagte om die laer heelgetal te skat}   ${3}^{3}=27$, daarom $3\lessthan{}\sqrt[3]{49}$.

\westep{Gebruik volkome derdemagte om die ho\"er heelgetal te skat} ${4}^{3}=64$, daarom $\sqrt[3]{49}\lessthan{}4$. 

\westep{Die antwoord is}
$3\lessthan{}\sqrt[3]{49}\lessthan{}4$.

\westep{Kontroleer die antwoord deur al die getalle in die ongelykheid tot die derde mag te verhef en vereenvoudig}
$27<49<64$, wat waar is. Dus lê $\sqrt[3]{49}$ tussen $3$ en $4$.
%       \end{enumerate}
}
\end{wex}

\begin{exercises}{}
 {
Vind twee opeenvolgende heelgetalle wat weerskante van die getalle lê, sonder die gebruik van 'n sakrekenaar:
\begin{enumerate}[itemsep=5pt, label=\textbf{\arabic*}. ]
\item $\sqrt{18}$
\item $\sqrt{29}$
\item $\sqrt[3]{5}$
\item $\sqrt[3]{79}$

\end{enumerate}
% Automatically inserted shortcodes - number to insert 4
\par \practiceinfo
\par \begin{tabular}[h]{cccccc}
% Question 1
(1.-4.)	02et	&


\end{tabular}
% Automatically inserted shortcodes - number inserted 4
}
\end{exercises}



\section{Produkte}
\setcounter{figure}{1}
\setcounter{subfigure}{1}

%   
\nopagebreak
Wiskundige uitdrukkings is soos sinne en elke deel het ’n spesifieke naam. Jy behoort vertroud te wees met die volgende name wat die dele van wiskundige uitdrukkings beskryf.\par 

\begin{equation*}
3x^2 + 7xy -5^3 = 0
\end{equation*}


%english
\begin{table}[H]
\begin{center}
\begin{tabular}{|l|l|}
\hline
\textbf{Naam} & \textbf{Voorbeelde} \\
\hline
terme & $3x^2;~7xy;~-5^3$\\ \hline
uitdrukking & $3x^2 + 7xy -5^3$\\ \hline
koëffisiënt & $3;~7$\\ \hline
eksponent & $2;~1;~3$\\ \hline
basis & $x;~y;~5$\\ \hline
konstante & $3;~7;~5$\\ \hline
veranderlike & $x;~y$\\ \hline
vergelyking & $3x^2 + 7xy -5^3 = 0$\\ \hline

% \label{m39387*secfhsst!!!underscore!!!id1562}\vspace{.5cm}

\end{tabular}
\end{center}
\end{table} 

\par

\subsection*{Vermenigvuldig 'n eenterm en 'n tweeterm}
%english
’n Eenterm is ’n wiskundige uitdrukking met een term, byvoorbeeld $3x$ of $y^2$. ’n Binomiaal of tweeterm is ’n wiskundige uitdrukking met twee terme, soos $ax+b$ of $cx+d$.
\vspace*{-10pt}
\begin{wex}{Vereenvoudiging van hakies}
{Vereenvoudig: $2a(a-1) - 3(a^{2}-1)$. \vspace*{-10pt}}
{\vspace*{-10pt}
\begin{align*}
  2a(a-1) -3(a^{2}-1) &= 2a(a) + 2a(-1) + (-3)(a^{2})+(-3)(-1) \\
  &= 2a^{2} - 2a - 3a^{2} + 3 \\
  &= -a^{2} -2a + 3
\end{align*}
\vspace*{-15pt}
}
\end{wex}

\subsection*{Produk van twee binomiale}
\addcontentsline{toc}{subsection}{Vermenigvuldig twee binomiale (tweeterme)}

% ’n Binomiaal is ’n wiskundige uitdrukking met twee terme, soos $ax+b$ and $cx+d$. As hierdie twee binomiale
% vermenigvuldig word, is die volgende die resultaat:


Hieronder vermenigvuldig ons twee lineêre binomiale:


\begin{center}
\scalebox{1} % Change this value to rescale the drawing.
{
\begin{pspicture}(0,-1.0)(4.8421874,1.0)
\usefont{T1}{ptm}{m}{n}
\rput(2.3514063,-0.09){\LARGE $(ax+b)(cx+d)$}
\psbezier[linewidth=0.02,arrowsize=0.05291667cm 2.0,arrowlength=1.4,arrowinset=0.4]{<-}(4.0,0.15)(3.55625,0.99)(0.9782838,0.8882759)(0.8,0.19)
\psbezier[linewidth=0.02,arrowsize=0.05291667cm 2.0,arrowlength=1.4,arrowinset=0.4]{<-}(3.02,0.17142878)(2.6685817,0.65)(0.82,0.613871)(0.8,0.17)
\psbezier[linewidth=0.02,arrowsize=0.05291667cm 2.0,arrowlength=1.4,arrowinset=0.4]{->}(2.02,-0.4070588)(2.1441379,-0.84)(2.8579311,-0.7317647)(2.92,-0.38)
\psbezier[linewidth=0.02,arrowsize=0.05291667cm 2.0,arrowlength=1.4,arrowinset=0.4]{->}(2.0,-0.39)(2.1934042,-0.9844227)(3.6561701,-0.99)(4.0,-0.39557728)
\end{pspicture} 
}
\end{center}



\begin{align*}
  (ax+b)(cx+d) &= (ax)(cx)+(ax)d+b(cx)+bd \\
               &= ac{x}^{2}+adx +bcx+bd \\
               &= ac{x}^{2}+x(ad+bc)+bd
\end{align*}

% \begin{equation*}
% \begin{array}{ccc}\hfill (ax+b)(cx+d) & =& (ax)(cx)+(ax)d+b(cx)+bd\hfill  \\
% & =& ac{x}^{2}+adx +bcx+bd\hfill\\
% & =& ac{x}^{2}+x(ad+bc)+bd\hfill \end{array}
% \end{equation*}
% 
% \Note{Jy kan FOIL= F(firsts) O(outers) I(inners) L(lasts) gebruil om te help om hierdie proses te onthou}
\par

\begin{wex}{Produk van twee binomiale }
{Vind die produk: $(3x-2)(5x+8)$. }{
\begin{equation*}
\begin{array}{ccl}\hfill (3x-2)(5x+8)& =& (3x)(5x)+(3x)(8)+(-2)(5x)+(-2)(8)\hfill \\
 & =& 15{x}^{2}+24x-10x-16\hfill \\
 & =& 15{x}^{2}+14x-16\hfill 
\end{array}
\end{equation*}
} 
\end{wex}

Die produk van twee identiese binomiale is bekend as die kwadraat (of vierkant) van binomiale en word geskryf
as:

\begin{equation*}
{(ax+b)}^{2}={a}^{2}{x}^{2}+2abx+{b}^{2}
\end{equation*}
Gestel die twee terme is $ax+b$ en $ax-b$ dan is hulle produk:

\begin{align*}
(ax+b)(ax-b) &={a}^{2}{x}^{2}-{b}^{2}
\end{align*}

Dit staan bekend as die verskil van twee kwadrate (of vierkante).


\subsection*{Vermenigvuldig 'n tweeterm en 'n drieterm}
\addcontentsline{toc}{subsection}{Vermenigvuldig van 'n tweeterm en 'n drieterm}

\addtocounter{footnote}{-0}

’n Trinomiaal of drieterm is 'n uitdrukking met drie terme, byvoorbeeld $ax^{2} + bx + c$.
In hierdie gedeelte gaan ons leer hoe om ’n binomiaal, of tweeterm, met ’n trinomiaal, of drieterm, te vermenigvuldig.\par 
%english
Om die produk van 'n binomiaal en 'n trinomiaal te bereken, vermenigvuldig die terme in hakies:\\


\begin{equation*}
  (A+B)(C+D+E)= A(C+D+E)+B(C+D+E) 
\end{equation*}

% As jy dit onthou, sal jy nie ’n fout maak nie!

\mindsetvid{Products of polynomials}{MG10062}

\begin{wex}
{Vermenigvuldiging van binomiaal met trinomiaal 
}
{
Vind die produk: $(x-1)({x}^{2}-2x+1)$.
} 
{
\westep{Pas die distributiewe wet toe}
$(x-1)(x^2-2x+1) &= x(x^2-2x+1)-1(x^2-2x+1)\\
\phantom{(x-1)(x^2-2x+1)} &= x^3-2x^2+x-x^2+2x-1$

\westep{Vereenvoudig}
$\phantom{(x-1)(x^2-2x+1) } = x^3-3x^2 + 3x-1$
}       

\end{wex}



\begin{exercises}{}
{

Vind die produkte:

\begin{multicols}{2}
\begin{enumerate}[label=\textbf{\arabic*}., itemsep=5pt]
\item $2y(y+4)$ 
\item $(y+5)(y+2) $
\item $(2-t)(1-2t)$
\item $(x-4)(x+4)$
\item $ (2p+9)(3p+1)$
\item $(3k-2)(k+6)$
\item $(s+6)^2$
\item $-(7-x)(7+x)$
\item $(3x-1)(3x+1)$
\item $(7k+2)(3-2k)$
\item $(1-4x)^2$
\item $(-3-y)(5-y)$
\item $(8-x)(8+x)$
\item $(9+x)^2$
\item$(-2{y}^{2}-4y+11)(5y-12)$ 
\item$(7{y}^{2}-6y-8)(-2y+2)$% make-rowspan-placeholders
\item$(10y+3)(-2{y}^{2}-11y+2)$ 
\item$(-12y-3)(12{y}^{2}-11y+3)$% make-rowspan-placeholders
\item$(-10)(2{y}^{2}+8y+3)$ 
\item$(2{y}^{6}+3{y}^{5})(-5y-12)$% make-rowspan-placeholders
\item$(-7y+11)(-12y+3)$% make-rowspan-placeholders
\item$(7y+3)(7{y}^{2}+3y+10)$% make-rowspan-placeholders
\item$9(8{y}^{2}-2y+3)$ 
\item \small$(-6{y}^{4}+11{y}^{2}+3y)(10y+4)(4y-4)$ 
\end{enumerate}
\end{multicols}

% Automatically inserted shortcodes - number to insert 24
\par \practiceinfo
\par \begin{tabular}[h]{cccccc}
% Question 1
(1.-6.)	02ex	&
% Question 2
(7.-12.)	02ey	&
% Question 3
(13.-18.)	02ez	&
% Question 4
(19.-24.)	02f0	&

\end{tabular}
% Automatically inserted shortcodes - number inserted 24
}
\end{exercises}





\section{Faktorisering}

%            \subsubsubsection{ Factorisation}
%             \nopagebreak
Faktorisering is die omgekeerde proses van uitbreiding of die verwydering van hakies. Byvoorbeeld, as hakies uitgebrei word, word $2(x+1)$ geskryf as $2x+2$. Faktorisering sal dus begin met $2x+2$ en eindig met $2(x+1)$. 

\begin{center}
\scalebox{1} % Change this value to rescale the drawing.
{
\begin{pspicture}(0,-1.1042187)(3.6090624,1.1042187)
% \usefont{T1}{ppl}{m}{n}
\rput(0.78453124,-0.01921875){$2(x+1)$}
\psbezier[linewidth=0.02,arrowsize=0.093cm 2.4,arrowlength=1.4,arrowinset=0.4]{->}(2.85,0.23078126)(2.59,0.9507812)(1.0414754,0.73897433)(0.97,0.27078125)
\psbezier[linewidth=0.02,arrowsize=0.093cm 2.4,arrowlength=1.4,arrowinset=0.4]{->}(0.95,-0.20921876)(1.13,-0.8720759)(2.55,-0.8492187)(2.87,-0.2063616)
% \usefont{T1}{ppl}{m}{n}
\rput(2.8645313,-0.03921875){$2x+2$}
% \usefont{T1}{ptm}{m}{n}
\rput(1.9039062,0.9007813){faktorisering}
% \usefont{T1}{ptm}{m}{n}
\rput(1.9079688,-1){uitbreiding}
\end{pspicture} 
}
\end{center}

Die twee uitdrukkings $2(x+1)$ en $2x+2$ is ekwivalent; hulle het dieselfde waarde vir alle waardes van $x$.
\par
% 
% \Identity{}
% {
% \begin{center}
% \begin{small}\hspace{8pt}uitbreiding\end{small}\\
% \begin{Large}
% $2(x+1)$ \begin{Huge} $\leftrightharpoons$ \end{Huge} $2x+2$
% \end{Large}\\
% \begin{small}\hspace{8pt}faktorisering\end{small}
% \end{center}
% }
% 
% Die uitrukkings $2(x+1)$ en $2x+2$ is ekwivalente; hulle het die $x$

\par
In vorige grade het ons gefaktoriseer deur die uithaal van gemeenskaplike faktore en die verskil tussen twee vierkante.

\par 
\mindsetvid{Revision of factorisation}{VMabz}

\subsection*{Gemeenskaplike faktore}
\nopagebreak
Faktorisering deur die uithaal van gemeenskaplike faktore is gebaseer daarop dat daar faktore is wat in al die terme voorkom. \par

Byvoorbeeld, $2x-6{x}^{2}$ kan as volg gefaktoriseer word:\par 

\begin{equation*}
2x-6{x}^{2}=2x(1-3x)
\end{equation*}

% \Tip{Kontroleer jou antwoord deur die uitbreiding van die hakies en maak seker jy kry die oorspronklike uitdrukking. 
% \begin{equation*}
%  2x(1-3x) = 2x - 6x^{2}
% \end{equation*}
% is korrek.}

% 
% \begin{wex}{Faktorisering}
% {Faktoriseer volledig: ${b}^{2}{y}^{5}-3ab{y}^{3}$}
% {
% \westep{Haal die grootste gemeenskaplike faktoor, by $by^3$, uit}  
% \begin{equation*}
% {b}^{2}{y}^{5}-3ab{y}^{3}& =& b{y}^{3}(b{y}^{2}-3a)
% \end{equation*}
% }
% \end{wex}


\begin{wex}{ Faktorisering wat die omruiling van getalle in hakies benodig}{Faktoriseer: $5(a-2)-b(2-a)$. }{
Gebruik 'n omruilingstegniek om die gemene faktor te vind: \\let op dat $2-a = -(a-2)$.
\begin{equation*}
\begin{array}{ccl}
\hfill 5(a-2)-b(2-a)& =& \hfill 5(a-2)-[-b(a-2)]  \\
&= & 5(a-2)+b(a-2)\\ 
& =& (a-2)(5+b) \hfill
\end{array}
\end{equation*}
}
\end{wex}

% \begin{exercises}{}
% {
% Vind die grootste gemene faktore van die volgende pare getalle:\par
% \mindsetvid{Revision of factorisation}{VMabz}
% \begin{multicols}{2}
% \begin{enumerate}[label=\textbf{\arabic*}., itemsep=5pt]
% \item $6y;~18x$
% \item $12mn;~8n$
% \item $3st;~4su$ 
% \item $18kl;~9kp$
% \item $abc;~ac$% 
% \item $2xy;~4xyz$
% \item $3uv;~6u$ 
% \item $9xy;~15xz$
% \item $24xyz;~16yz$
% \item $3m;~45n$
% \end{enumerate}
% \end{multicols}
% 
% Automatically inserted shortcodes - number to insert 10
% \par \practiceinfo
% \par \begin{tabular}[h]{cccccc}
% % Question 1
% (1.)	02fm	&
% % Question 2
% (2.)	02fn	&
% % Question 3
% (3.)	02fp	&
% % Question 4
% (4.)	02fq	&
% % Question 5
% (5.)	02fr	&
% % Question 6
% (6.)	02fs	\\ % End row of shortcodes
% % Question 7
% (7.)	02ft	&
% % Question 8
% (8.)	02fu	&
% % Question 9
% (9.)	02fv	&
% % Question 10
% (10.)	02fw	&
% \end{tabular}
% Automatically inserted shortcodes - number inserted 10
% }
% \end{exercises}

\subsection* {Verskil van twee vierkante}
Ons het gesien dat: 
\begin{equation*}
(ax+b)(ax-b)~\mbox{ kan uit gebrei word as }~{a}^{2}{x}^{2}-{b}^{2}
\end{equation*}

Dus
\begin{equation*}
\mbox{ kan }{a}^{2}{x}^{2}-{b}^{2}~\mbox{ gefaktoriseer word as }~(ax+b)(ax-b)
\end{equation*}
Byvoorbeeld, ${x}^{2}-16$ kan geskryf word as $({x}^{2}-{4}^{2})$ wat die verskil is tussen twee kwadrate. Dus, die faktore van ${x}^{2}-16$ is $(x-4)$ en $(x+4)$.

Om te faktoriseer, kyk vir uitdrukkings:
\begin{itemize}[noitemsep]
\item wat uit twee terme bestaan 
\item met terme met verskillende tekens (een positief, een negatief)
\item met elke term 'n volkome vierkant
\end{itemize}
Byvoorbeeld: $a^{2}-1;~ 4x^{2}-y^{2};~ -49+p^{4}$
}
\vspace*{-20pt}
\begin{wex}{Faktorisering van die verskil tussen twee vierkante}
{Faktoriseer: $3a(a^2-4)-7(a^2-4)$.}
{
\westep{Haal uit die gemene faktor $(a^2-4)$}
\begin{equation*}
\begin{array}{ccc}\hfill 3a(a^2-4)-7(a^2-4)& =& (a^2-4)(3a-7)\hfill \end{array}
\end{equation*}
\westep{Faktoriseer die verskil van twee kwadrate $(a^2-4)$ }
$$(a^2-4)(3a-7) = (a-2)(a+2)(3a-7)$$\vspace*{-20pt}
}
\end{wex}

\begin{exercises}{}
{
Faktoriseer:
\begin{multicols}{2}
\begin{enumerate}[itemsep=5pt, label=\textbf{\arabic*}. ] 
\item $2l+2w$
\item $12x+32y$
\item $6{x}^{2}+2x+10{x}^{3}$
\item $2x{y}^{2}+x{y}^{2}z+3xy$
\item $-2a{b}^{2}-4{a}^{2}b$
\item $7a+4$ 
\item $20a-10$ 
\item $18ab-3bc$
\item $12kj+18kq$ 
\item $16{k}^{2}-4$ 
\item $3{a}^{2}+6a-18$
\item $-12a+24a^3$ 
\item $-2ab-8a$ 
\item $24kj-16{k}^{2}j$
\item $-{a}^{2}b-{b}^{2}a$ 
\item $12{k}^{2}j+24{k}^{2}{j}^{2}$ 
\item $72{b}^{2}q-18{b}^{3}{q}^{2}$
\item $4(y-3)+k(3-y)$ 
\item $a^2(a-1)-25(a-1)$ 
\item $bm(b+4)-6m(b+4)$
\item ${a}^{2}(a+7)+9(a+7)$ 
\item $3b(b-4)-7(4-b)$ 
\item ${a}^{2}{b}^{2}{c}^{2}-1$
\end{enumerate}
\end{multicols}

% Automatically inserted shortcodes - number to insert 23
\par \practiceinfo
\par \begin{tabular}[h]{cccccc}
% Question 1
(1.-7.)	02fx	&
% Question 2
(8.-12.)	02fy	&
% Question 3
(13.-18.)	02fz	&
% Question 4
(19.-23.)	02g0	&
% Question 5

\end{tabular}
% Automatically inserted shortcodes - number inserted 23
}
\end{exercises}

\subsection{Faktorisering deur groepering}
\nopagebreak

’n Verdere metode van faktorisering gebruik groepering van terme. Ons weet dat die faktore van $3x+3$ is $3$ en $(x+1)$ is. Soortelyk is die faktore van $2{x}^{2}+2x$ is $2x$ en $(x+1)$. Gevolglik as ons ’n uitdrukking het:

\begin{equation*}
2{x}^{2}+2x+3x+3
\end{equation*}
is daar geen gemene faktor van al vier terme nie, maar ons kan as volg faktoriseer:
\nopagebreak\noindent{}

\begin{equation*}
\begin{array}{ccl}
(2{x}^{2}+2x)+(3x+3) &=& 2x(x+1)+3(x+1) \hfill \hfill
\end{array}
\end{equation*}


Nou kan ons sien dat daar ’n ander gemene faktor $(x+1)$ is. Gevolglik kan ons nou skryf: 
\begin{equation*}
(x+1)(2x+3)
\end{equation*}
Ons kry dit deur die $(x+1)$ ’uit te haal’ (uit te deel) en te sien wat oorbly. Ons het $+2x$ uit die eerste term en $+3$ uit die tweede term. Dit word genoem faktorisering deur groepering.\par 

% \Tip{Kyk na die verhoudings van die ko\"effisi\"ente as 'n riglyn vir groepering.}


\begin{wex}{Faktorisering deur groepering }{Vind die faktore van $7x+14y+bx+2by$.}
{

\westep{Daar is geen algemene gemeenskaplike faktore nie}

\westep{Groepeer terme met gemene faktore saam}
 $7$ is ’n gemene faktor van die eerste twee terme en $b$ is ’n gemene faktor van die tweede twee terme. Ons sien die verhouding van die ko\"effisi\"ente $7:14$ is dieselfde as $b:2b$.
\begin{equation*}
 \begin{array}{ccl}

7x+14y+bx+2by&=& (7x+14y)+(bx+2by)  \hfill\\ 
&=&7(x+2y)+b(x+2y) \hfill 
\end{array}
\end{equation*}
\westep{Haal die gemeenskaplike faktor $(x+2y)$ uit}


\begin{equation*}
7(x+2y)+b(x+2y)=(x+2y)(7+b)
\end{equation*}
% 
% \westep{Skryf die finale antwoord}  
% Die faktore van $7x+14y+bx+2by$ is $(7+b)$ en $(x+2y)$.
\par 
OF
% 
% 
% \begin{wex}{Faktorisering deur groepering  }{Vind die faktore van $7x+14y+bx+2by$}
% {
% 
% \westep{Daar is geen algemene gemeenskaplike faktore nie.}

\setcounter{stepcounter}{1}
\westep{Groepeer terme met gemene faktore saam}$x$ is 'n gemene faktor van die eerste en derde terme en $2y$ is 'n gemene faktor van die tweede en vierde terme $(7:b~=~14:2b)$.\par 
\westep{Herrangskik die uitdrukking}

\begin{equation*}
 \begin{array}{ccl}

7x+14y+bx+2by&=& (7x+bx)+(14y+2by)  \hfill\\ 
&=&x(7+b)+2y(7+b) \hfill 
\end{array}
\end{equation*}

\westep{Haal die gemeenskaplike faktor $(7+b)$ uit}

\begin{equation*}
x(7+b)+2y(7+b) = (7+b)(x+2y)
\end{equation*}
\westep{Skryf die finale antwoord}  
Die faktore van $7x+14y+bx+2by$ is $(7+b)$ en $(x+2y)$.
}
\end{wex}


\begin{exercises}{}{
\nopagebreak
Faktoriseer die volgende:
\begin{multicols}{2}
\begin{enumerate}[itemsep=5pt, label=\textbf{\arabic*}. ] 
\item $6x+a+2ax+3$
\item ${x}^{2}-6x+5x-30$
\item $5x+10y-ax-2ay$
\item ${a}^{2}-2a-ax+2x$
\item $5xy-3y+10x-6$
\item $ab - a^{2} - a + b$
\end{enumerate}
\end{multicols}

% Automatically inserted shortcodes - number to insert 6
\par \practiceinfo
\par \begin{tabular}[h]{cccccc}
% Question 1
(1.)	02gk	&
% Question 2
(2.)	02gm	&
% Question 3
(3.)	02gn	&
% Question 4
(4.)	02gp	&
% Question 5
(5.)	02gq	&
% Question 6
(6.)	02gr	\\ % End row of shortcodes
\end{tabular}
% Automatically inserted shortcodes - number inserted 6
}
\end{exercises}

\subsection* {Faktorisering van 'n kwadratiese drieterm}

    
Faktorisering kan gesien word as die omgekeerde proses van die berekening van die produk van faktore. Om ’n kwadratiese uitdrukking te faktoriseer, is dit dus nodig om die faktore te vind wat, wanneer hulle met mekaar vermenigvuldig word, gelyk sal wees aan die oorspronklike kwadratiese uitdrukking.\par 

Beskou ’n kwadratiese uitdrukking van die vorm: $a{x}^{2}+bx$. Ons kan sien hier is  $x$ ’n gemeenskaplike faktor in beide terme. Dus, $a{x}^{2}+bx$  faktoriseer tot $x(ax+b)$. Byvoorbeeld, $8{y}^{2}+4y$ faktoriseer tot  $4y(2y+1)$.\par 
’n Ander tipe kwadratiese uitdrukking bestaan uit die verskil tussen kwadrate. Ons weet dat:
\begin{equation*}
(a+b)(a-b)={a}^{2}-{b}^{2}
\end{equation*}

So, $a^2-b^2$ kan in gefaktoriseerde vorm geskryf word as $(a+b)(a-b)$. \par

Dit beteken dat wanneer ons enige kwadratiese uitdrukking, wat bestaan uit die verskil tussen twee kwadrate, te\"ekom, ons onmiddellik die faktore kan neerskryf.


Hierdie soort kwadratiese uitdrukking is eenvoudig om te faktoriseer. Nie baie kwadratiese uitdrukkings val egter
in hierdie kategorie nie, en gevolglik het ons ’n meer algemene metode nodig vir kwadrate.
\par 
Ons kan leer hoe om kwadrate te faktoriseer deur twee tweeterme met mekaar te vermenigvuldig en so ’n kwadratiese uitdrukking te kry. Byvoorbeeld

\begin{equation*}
\begin{array}{ccl}\hfill (x+2)(x+3)& =& x^2+3x+2x+6 \\ & =& {x}^{2}+5x+6\hfill \end{array}
\end{equation*}
Ons kan sien dat die ${x}^{2}$ term in die kwadratiese uitdrukking die produk is van die $x$ -terme in elke hakie. Soortgelyk, die $6$ in die kwadratiese uitdrukking is die produk van $2$ en $3$ in die hakies. Gevolglik is die middelterm die
som van die twee terme.\par 
Hoe gebruik ons hierdie inligting om die kwadratiese uitdrukking te faktoriseer?\par 
Kom ons begin faktoriseer ${x}^{2}+5x+6$ en sien of ons kan besluit op sekere algemene reëls.\par
 Eerstens, skryf twee hakies neer met ’n $x$ in elke hakie en los spasie vir die oorblywende terme.\par 
\begin{equation*}
(x\phantom{\rule{2.em}{0ex}})(x\phantom{\rule{2.em}{0ex}})
\end{equation*}
Besluit nou op die faktore van $6$. Aangesien $6$ ’n positiewe getal is, sal dit wees:\par 
% \textbf{m39394*id275986}\par
\begin{table}[H]
% \begin{table}[H]
% \\ '' '0'
\begin{center}
% \label{m39394*id275986}
\noindent

\begin{tabular}[t]{|c|c|}\hline
% My position: 0
% my spanname: 
% my ct of spanspec: 0
% my column-count: 2
\multicolumn{2}{|c|}{\textbf{Faktore van $6$}}
\\ \hline
%--------------------------------------------------------------------
$1$ &
$6$% make-rowspan-placeholders
\\ \hline
%--------------------------------------------------------------------
$2$ &
$3$% make-rowspan-placeholders
\\ \hline
%--------------------------------------------------------------------
$-1$ &
$-6$% make-rowspan-placeholders
\\ \hline
%--------------------------------------------------------------------
$-2$ &
$-3$% make-rowspan-placeholders
\\ \hline
%--------------------------------------------------------------------
\end{tabular}
\end{center}
%     \begin{center}{\small\bfseries Table 8.6}\end{center}
%     \begin{caption}{Factors of 6}\end{caption}
\end{table}
\par
Dus het ons nou vier moontlikhede:\par 
% \textbf{m39394*id276099}\par
\begin{table}[H]
% \begin{table}[H]
% \\ '' '0'
\begin{center}
\label{m39394*id276099}
\noindent

\begin{tabular}{|l|l|l|l|}\hline
\textbf{Opsie 1} &
\textbf{Opsie 2} &
\textbf{Opsie 3} &
\textbf{Opsie 4}% make-rowspan-placeholders
\\ \hline
%--------------------------------------------------------------------
  $(x+1)(x+6)$
  &
  $(x-1)(x-6)$
  &
  $(x+2)(x+3)$
  &
  $(x-2)(x-3)$
% make-rowspan-placeholders
\\ \hline
%--------------------------------------------------------------------
\end{tabular}
\end{center}
%     \begin{center}{\small\bfseries Table 8.7}\end{center}
%     \begin{caption}{Factor Options}\end{caption}
\end{table}
\par
Vervolgens vermenigvuldig ons elke stel hakies uit om te sien watter stel gee die regte middelterm.\par 
% \textbf{m39394*id276265}\par
\begin{table}[H]
% \begin{table}[H]
% \\ '' '0'
\begin{center}
\label{m39394*id276265}
\noindent

\begin{tabular}[t]{|l|l|l|l|}\hline
\textbf{Opsie 1} &
\textbf{Opsie 2} &
\textbf{Opsie 3} &
\textbf{Opsie 4}% make-rowspan-placeholders
\\ \hline
%--------------------------------------------------------------------
  $(x+1)(x+6)$
  &
  $(x-1)(x-6)$
  &
  $(x+2)(x+3)$
  &
  $(x-2)(x-3)$
% make-rowspan-placeholders
\\ \hline
%--------------------------------------------------------------------
  ${x}^{2}+7x+6$
  &
  ${x}^{2}-7x+6$
  &
  \uline{
    ${x}^{2}+5x+6$
  }
  &
  ${x}^{2}-5x+6$
% make-rowspan-placeholders
\\ \hline
%--------------------------------------------------------------------
\end{tabular}
\end{center}
%     \begin{center}{\small\bfseries Table 8.8}\end{center}
%     \begin{caption}{Quadratic factors}\end{caption}
\end{table}
\par
Ons kan sien dat Opsie 3, $(x+2)(x+3)$, die korrekte oplossing is. \par
Die proses van faktorisering is hoofsaaklik ’n proses van opsies identifiseer en evalueer, maar daar is inligting wat die proses kan vergemaklik.\par 

\mindsetvid{Factorising a quadratic}{MG10064}

\subsection*{Algemene prosedure vir die faktorisering van 'n drieterm}
\addcontentsline{toc}{subsection}{General procedure for factorising a trinomial}

\begin{enumerate}[itemsep=5pt, label=\textbf{\arabic*}. ] 
\item Haal enige gemeenskaplike faktore van die koëffisïente uit om ’n uitdrukking te kry van die vorm $a{x}^{2}+bx+c$ waar $a$, $b$ en $c$ geen gemene faktore het nie en $a$ positief is.
\item Skryf twee hakies neer met ’n $x$ in elke hakie en plek vir die oorblywende terme:

\begin{equation*}
(x\phantom{\rule{2.em}{0ex}})(x\phantom{\rule{2.em}{0ex}})
\end{equation*}
\item Skryf ’n stel faktore neer vir $a$ en $c$.
\item Skryf ’n stel opsies neer vir die moontlike faktore van die kwadratiese drieterm deur die faktore van $a$ en $c$ te gebruik.
\item Brei al die opsies uit om te sien watter stel vir jou die korrekte antwoord gee.
\end{enumerate}
 \textbf{Let op:} as $c$ positief is, moet albei die faktore van $c$ positief of albei negatief wees. Die faktore is beide negatief indien $b$ negatief is, en beide positief indien $b$ positief is. As $c$ negatief is, beteken dit slegs een van die faktore van $c$ is negatief, en die ander een is positief. Wanneer jy ’n antwoord gekry het, brei weer jou hakies uit om te toets of dit reg uitwerk. 
\begin{wex}{Faktorisering van ’n kwadratiese drieterm}
{Faktoriseer: $3{x}^{2}+2x-1$.} 
{
\westep{Die kwadraat is in die regte vorm $ax^2+bx+c$}
\westep{Skryf die stel faktore neer van $a$ en $c$}  
\begin{equation*}
(x\phantom{\rule{2.em}{0ex}})(x\phantom{\rule{2.em}{0ex}})
\end{equation*}
\hspace*{-40pt}
\begin{minipage}{0.9\textwidth}

Die moontlike faktore van $a$ is: $(1;~3)$.\\
Die moontlike faktore van $c$ is: $(-1;~1)$ of $(1;~-1)$.\\
Skryf die groep opsies neer van die moontlike faktore van die kwadratiese uitdrukking deur die faktore van $a$ en $c$ te gebruik.
Daar is twee moontlike oplossings.
\end{minipage}
\par 

\begin{table}[H]

\begin{center}


\begin{tabular}{|l|l|}\hline
\textbf{Opsie 1} &
\textbf{Opsie 2}% make-rowspan-placeholders
\\ \hline
%--------------------------------------------------------------------
$(x-1)(3x+1)$
&
$(x+1)(3x-1)$
% make-rowspan-placeholders
\\ \hline
%--------------------------------------------------------------------
$3{x}^{2}-2x-1$
&
\uline{
$3{x}^{2}+2x-1$
}
% make-rowspan-placeholders
\\ \hline
%--------------------------------------------------------------------
\end{tabular}
\end{center}

\end{table}

\westep{Toets of jou oplossing korrek is deur uitbreiding van die faktore} 
\begin{equation*}
\begin{array}{ccl}  
(x+1)(3x-1)& =& 3{x}^{2}-x+3x-1\hfill \\ & =& 3{x}^{2}+2x-1\end{array}
\end{equation*}
\westep{Skryf die finale antwoord}
Die faktore van $3{x}^{2}+2x-1$ is $(x+1)$ en $(3x-1)$.
}
\end{wex}


\begin{exercises}{}
{
\begin{enumerate}[itemsep=5pt, label=\textbf{\arabic*}. ] 
\item Faktoriseer die volgende:
\begin{multicols}{2}
\begin{enumerate}[itemsep=5pt, label=\textbf{(\alph*)} ] 
\item ${x}^{2}+8x+15$
\item ${x}^{2}+10x+24$
\item ${x}^{2}+9x+8$
\item ${x}^{2}+9x+14$
\item ${x}^{2}+15x+36$
\item ${x}^{2}+12x+36$
\end{enumerate}
\end{multicols}


\item Ontbind die volgende in faktore:
\begin{multicols}{2}
\begin{enumerate}[itemsep=5pt, label=\textbf{(\alph*)} ] 
% \setcounter{enumi}{6}
\item ${x}^{2}-2x-15$
\item ${x}^{2}+2x-3$
\item ${x}^{2}+2x-8$
\item ${x}^{2}+x-20$
\item ${x}^{2}-x-20$
\item $2{x}^{2}+22x+20$
\end{enumerate}
\end{multicols}


\item Vind die faktore van die volgende drieterme:
\begin{multicols}{2}
\begin{enumerate}[itemsep=5pt, label=\textbf{(\alph*)} ] 
% \setcounter{enumi}{11}
\item $3{x}^{2}+19x+6$
\item $6{x}^{2}+7x+2$
\item $12{x}^{2}+8x+1$
\item $8{x}^{2}+6x+1$
\end{enumerate}
\end{multicols}

\item Faktoriseer:
\begin{multicols}{2}
\begin{enumerate}[itemsep=5pt, label=\textbf{(\alph*)} ] 
% \setcounter{enumi}{16}
\item $3{x}^{2}+17x-6$
\item $7{x}^{2}-6x-1$
\item $8{x}^{2}-6x+1$
\item $6{x}^{2}-15x-9$
\end{enumerate}
\end{multicols}

% Automatically inserted shortcodes - number to insert 4
\par \practiceinfo
\par \begin{tabular}[h]{cccccc}
% Question 1
(1.)	02gs	&
% Question 2
(2.)	02gt	&
% Question 3
(3.)	02gu	&
% Question 4
(4.)	02gv	&
\end{tabular}
% Automatically inserted shortcodes - number inserted 4
}
\end{exercises}



\subsection*{Som en verskil van twee derdemagte}      
Ons kyk vervolgens na twee spesiale resultate wat verkry word as 'n tweeterm en 'n drieterm met mekaar vermenigvuldig word.\par

Som van twee derdemagte:
\begin{align*}
  (x+y)({x}^{2}-xy+{y}^{2}) &= x({x}^{2}-xy+{y}^{2})+y({x}^{2}- xy+{y}^{2})\\
  &= \left[x({x}^{2})+x(-xy)+x({y}^{2})\right]+\left[y({x}^{2})+y(-xy)+y({y}^{2})\right]\\
  &= {x}^{3}-{x}^{2}y+x{y}^{2}+{x}^{2}y-x{y}^{2}+{y}^{3} \\
  %&= {x}^{3}+(-{x}^{2}y+{x}^{2}y)+(x{y}^{2}-x{y}^{2})+{y}^{3} \\
  &= {x}^{3}+{y}^{3}\\
\end{align*}

\par
Verskil van twee derdemagte:
\begin{align*}
(x-y)({x}^{2}+xy+{y}^{2}) &= x({x}^{2}+xy+{y}^{2})-y({x}^{2}+ xy+{y}^{2})\\
  &= \left[x({x}^{2})+x(xy)+x({y}^{2})\right]-\left[y({x}^{2})+y(xy)+y({y}^{2})\right]\\
  &= {x}^{3}+{x}^{2}y+x{y}^{2}-{x}^{2}y-x{y}^{2}-{y}^{3} \\
  %&= {x}^{3}+({x}^{2}y-{x}^{2}y)+(x{y}^{2}-x{y}^{2})-{y}^{3} \\
  &= {x}^{3}-{y}^{3}\\
\end{align*}

Ons het gesien dat:
\begin{center}
${x}^{3}+{y}^{3}=(x+y)({x}^{2}-xy+{y}^{2})$\\
${x}^{3}-{y}^{3}=(x-y)({x}^{2}+xy+{y}^{2})$\\
\end{center}
%English
\par
Ons gebruik hierdie twee basiese uitdrukkings om meer komplekse voorbeelde te faktoriseer.
\clearpage

\begin{wex}{Faktoriseer 'n verskil van twee derdemagte}{Faktoriseer: $x^{3} - 1$. \vspace*{-10pt}}
{
\westep{Neem die derdemagswortel van die terme wat volkome derdemagte is}
Let op $\sqrt[3]{x^{3}} = x$ en $\sqrt[3]{1} = 1$. Dit gee die terme in die eerste hakie.

\westep{Gebruik inspeksie om die terme in die tweede hakie te vind}
\begin{equation*}
  (x^{3} - 1) = (x-1)(x^{2}+x+1)
\end{equation*}

\westep{Brei die hakies uit om te kontroleer dat die uitdrukking reg gefaktoriseer is}
\begin{align*}
  (x-1)(x^{2}+x+1) &= x(x^{2}+x+1)-1(x^{2}+x+1)\\
		   &=x^{3}+x^{2}+x-x^{2}-x-1\\
		   &=x^{3}-1\\
\end{align*}
\vspace*{-50pt}
}
\end{wex}

% \begin{wex}{Faktoriseer 'n verskil van derdemagte}{Faktoriseer volledig: $16y^{2} - 432$}
% {
% \westep{Haal die ’n gemeenskaplike faktor $16$ uit}
% \begin{equation*}
% \begin{array}{ccc}16y^{3} - 432\hfill & =& 16(y^{3} - 27)\hfill & \\
% \end{array}
% \end{equation*}
% 
% \westep{Neem die derdemagswortel van die terme wat volkome derdemagte is}
% Let op $\sqrt[3]{y^{3}} = y$ en $\sqrt[3]{27} = 3$. Dit gee die terme in die eerste hakie.
% \newline
% \westep{Gebruik inspeksie om die terme in die tweede hakie te vind}
% \begin{equation*}
% \begin{array}{cll} 16(y^{3} - 27) & =& 16(y-3)(y^{2}+3y+9)\hfill & \\
% \end{array}
% \end{equation*}
% 
% \westep{Brei die hakies uit om te kontroleer dat die uitdrukking reg gefaktoriseer is}
% }
% \end{wex}

\begin{wex}{Faktoriseer 'n som van twee derdemagte}{Faktoriseer: $x^{3}+8$.}
{
\westep{Neem die derdemagswortel van die terme wat volkome derdemagte is}
Let op $\sqrt[3]{x^{3}} = x$ en $\sqrt[3]{8} = 2$. Dit gee die terme in die eerste hakie.

\westep{Gebruik inspeksie om die terme in die tweede hakie te vind}
\begin{equation*}
  (x^{3} +8) = (x+2)(x^{2}-2x+4)
\end{equation*}

\westep{Brei die hakies uit om te kontroleer dat die uitdrukking reg gefaktoriseer is}
\begin{align*}
  (x+2)(x^{2}-2x+4) &= x(x^{2}-2x+4)+2(x^{2}-2x+4)\\
		   &=x^{3}-2x^{2}+4x+2x^{2}-4x+8\\
		   &=x^{3}+8\\
\end{align*}
\vspace*{-50pt}}
\end{wex}

% \begin{wex}{Som van derdemagte}{Vind die produk van $x+y$ en ${x}^{2}-xy+{y}^{2}$, en let noukeurig op die antwoord.}
% {
% \westep{Pas die distributiewe wet toe}{ 
% \begin{equation*}
% \begin{array}{ccc}(x+y)({x}^{2}-xy+{y}^{2})\hfill & =& x({x}^{2}-xy+{y}^{2})+y({x}^{2}- xy+{y}^{2})\hfill\end{array}
% \end{equation*}}
% \westep{Brei die hakies uit} { 
% \begin{equation*}
% \begin{array}{ccc}& =& [x({x}^{2})+x(-xy)+x({y}^{2})]+[y({x}^{2})+y(-xy)+y({y}^{2})]\hfill & \\
% \end{array}
% \end{equation*}}
% 
% \westep{vereenvoudig die terme}{
% \begin{equation*}
% \begin{array}{cll} & =& {x}^{3}-{x}^{2}y+x{y}^{2}+{x}^{2}y-x{y}^{2}+{y}^{3}\hfill & \hfill & \\ & =& {x}^{3}+(-{x}^{2}y+{x}^{2}y)+(x{y}^{2}-x{y}^{2})+{y}^{3}\hfill & \hfill & \\ & =& {x}^{3}+{y}^{3}\hfill & \hfill & \end{array}
% \end{equation*}}
% \westep{Sktryf die finale antwoord}{
% \label{m39387*id273290}Die produk van $x+y$ en ${x}^{2}-xy+{y}^{2}$ is ${x}^{3}+{y}^{3}$. \\
% Hierdie tipe tweeterm word die som van derdemagte genoem. }
% }
% \end{wex}
% 
% 
% \Tip{ons het gesien dat:
% $(x+y)({x}^{2}-xy+{y}^{2})= \\
% {x}^{3}+{y}^{3}$
% Dit staan bekend as die som van derdemagte. 
% So $x^{3}+y^{3}$ het faktore $(x+y)$ en $(x^{2} -xy+y^{2})$.\\
% \\
% Ons het ook gesien dat:
% $(x-y)({x}^{2}+xy+{y}^{2})= \\
% {x}^{3}-{y}^{3}$
% 
% Dit staan bekend as die verskil van derdemagte. So $x^{3}-y^{3}$ word deur die produk van $(x-y)$ en $(x^{2} +xy+y^{2})$ gegee.
% } 
% 
% 
% \begin{wex}{Verskil van derdemagte }{Vind die produk van $x-y$ en ${x}^{2}+xy+{y}^{2}$, en let op die antwoord.}{
% \westep{Pas die distributiewe wet toe}  {
% \begin{equation*}
% \begin{array}{ccc}(x-y)({x}^{2}+xy+{y}^{2})\hfill & =& x({x}^{2}+xy+{y}^{2})-y({x}^{2}+ xy+{y}^{2})\hfill\end{array}
% \end{equation*}}
% \westep{Brei die hakies uit}  {
% \begin{equation*}
% \begin{array}{ccc}& =& [x({x}^{2})+x(xy)+x({y}^{2})]-[y({x}^{2})+y(xy)+y({y}^{2})]\hfill & \\
% \end{array}
% \end{equation*}}
% \westep{Vereenvoudig die terme}{
% 
% \begin{equation*}
% \begin{array}{cll} & =& {x}^{3}+{x}^{2}y+x{y}^{2}-{x}^{2}y-x{y}^{2}-{y}^{3}\hfill & \hfill & \\ & =& {x}^{3}+({x}^{2}y-{x}^{2}y)+(x{y}^{2}-x{y}^{2})-{y}^{3}\hfill & \hfill & \\ & =& {x}^{3}-{y}^{3}\hfill & \hfill & \end{array}
% \end{equation*}}
% \westep{Skryf die finale produk}{
% \ Die produk van $x-y$ en ${x}^{2}+xy+{y}^{2}$ is ${x}^{3}-{y}^{3}$. \\
% Hierdie tipe tweeterm is die verskil van derdemagte. }
% }
% \end{wex}
% 
% 
% \begin{activity}{Ondersoek die produkte van 'n tweeterme en drieterme}
%  Ons het gesien dat in spesifieke gevalle gee die produk van 'n tweeterm en 'n drieterm die som of die verskil tussen derdemagte. Bepaal die produk van $x+y$ en ${x}^{2}+xy+{y}^{2}$. Maak 'n gevolgtrekking.
% \end{activity}
% 


\begin{wex}{Faktoriseer 'n verskil van twee derdemagte}{Faktoriseer: $16y^{3} - 432$.}
{
\westep{Haal die ’n gemeenskaplike faktor $16$ uit}
\begin{equation*}
\begin{array}{ccc}16y^{3} - 432\hfill & =& 16(y^{3} - 27)\hfill & \\
\end{array}
\end{equation*}

\westep{Neem die derdemagswortel van die terme wat volkome derdemagte is}
Let op $\sqrt[3]{y^{3}} = y$ en $\sqrt[3]{27} = 3$. Dit gee die terme in die eerste hakie.
\newline
\westep{Gebruik inspeksie om die terme in die tweede hakie te vind}
\begin{equation*}
\begin{array}{cll} 16(y^{3} - 27) & =& 16(y-3)(y^{2}+3y+9)\hfill & \\
\end{array}
\end{equation*}
\westep{Brei die hakies uit om te kontroleer dat die uitdrukking reg gefaktoriseer is}
\begin{align*}
  16(y-3)(y^{2}+3y+9) &= 16[(y(y^{2}+3y+9)-3(y^{2}+3y+9)]\\
		   &= 16[y^{3}+3y^{2}+9y-3y^{2}-9y-27]\\
		   &= 16y^{3}-432
\end{align*}
\vspace*{-50pt}}
\end{wex}


\begin{wex}{Faktoriseer 'n som van twee derdemagte}{Faktoriseer: $8t^{3} +125p^{3}$.}
{
\westep{Daar is geen gemeenskaplike faktor nie}

\westep{Neem die derdemagswortels}
Let op dat $\sqrt[3]{8t^{3}} = 2t$ en $\sqrt[3]{125p^{3}} = 5p$. Dit gee die terme in die eerste hakie.
\newline

\westep{Gebruik inspeksie om die drie terme in die tweede hakie te vind}
\begin{align*}
  (8t^{3} +125p^{3}) &= (2t + 5p)\left[(2t)^{2} - (2t)(5p)+(5p)^{2}\right] \\
                     &= (2t+5p)(4t^{2} - 10tp + 25p^{2})
\end{align*}

\westep{Brei die hakies uit om te kontroleer dat die uitdrukking korrek gefaktoriseer is}
\begin{align*}
  (2t+5p)(4t^{2} - 10tp + 25p^{2}) &= 2t(4t^{2} - 10tp + 25p^{2})+5p(4t^{2} - 10tp + 25p^{2})\\
		   &= 8t^{3} - 20pt^{2} + 50p^{2}t+ 20pt^{2} - 50p^{2}t + 125p^{3}\\
		   &= 8t^{3} +125p^{3}\\
\end{align*}
\vspace*{-50pt}}
\end{wex}

% \Tip{Let op die patroon:
% $
% a^{3} + b^{3} \rightleftharpoons \\ (a+b)(a^{2}-ab+b^{2})\\
% \\
% a^{3} - b^{3} \rightleftharpoons \\ (a-b)(a^{2}+ab+b^{2})\\
% $
% }

\begin{exercises}{}
{

Faktoriseer:
\begin{multicols}{2}
\begin{enumerate}[itemsep=5pt, label=\textbf{\arabic*}. ] 
\item ${x}^{3}+8$
\item $27-m^{3}$
\item $2x^{3}-2y^{3}$
\item $3k^{3} + 27q^{3}$
\item $64t^{3}-1$
\item $64x^{2} -1$
\item $125x^{3} +1$
\item $25x^{2} +1$
\item $z-125z^4{}$
\item $8m^{6} + n^{9}$
\item $p^{15} - \frac{1}{8}y^{12}$
\item $1- (x-y)^3$
\end{enumerate}
\end{multicols}

% Automatically inserted shortcodes - number to insert 12
\par \practiceinfo
\par \begin{tabular}[h]{cccccc}
% Question 1
(1.-5.)	02gw	&
% Question 2
(6.-10.)	02gx	&
% Question 3
(11.-12.)	02gy	&
\\ % End row of shortcodes
\end{tabular}
% Automatically inserted shortcodes - number inserted 12
}
\end{exercises}

\section{Vereenvoudiging van breuke}
\nopagebreak

In vorige grade is die prosedures vir bewerkings met breuke bestudeer.

\Identity{
\vspace*{-3em}
\begin{flushleft}
\begin{enumerate}[itemsep=5pt, label=\textbf{\arabic*}. ] 
\item $\dfrac{a}{b} \times \dfrac{c}{d} = \dfrac{ac}{bd}  ~~~~~~~~~~ (b \neq 0\mbox{; } d \neq 0)$ 
\item $\dfrac{a}{b} + \dfrac{c}{b}   = \dfrac{a+c}{b} ~~~~~~~~~~~(b \neq 0) $
\item $\dfrac{a}{b} \div \dfrac{c}{d}   = \dfrac{a}{b} \times \dfrac{d}{c}  = \dfrac{ad}{bc} ~~~~~~~~~~ (b \neq 0)$; $(c \neq 0)$; $(d \neq 0)$  
\end{enumerate}
\end{flushleft}
}
\textbf{Let op:} om te deel deur 'n breuk is dieselfde as om te vermenigvuldig met die resiprook.\par

In sommige gevalle van die vereenvoudiging van ’n algebraïese uitdrukking, sal die uitdrukking ’n breuk wees. Byvoorbeeld,

\begin{equation*}
\dfrac{{x}^{2}+3x}{x+3}
\end{equation*}
het ’n tweeterm in die noemer en ’n kwadratiese tweeterm in die teller. Jy kan die verskillende metodes van faktorisering gebruik om die uitdrukking te vereenvoudig.\par 

\begin{equation*}
\begin{array}{llll} \dfrac{{x}^{2}+3x}{x+3}\hfill & =& \dfrac{x(x+3)}{x+3}\hfill & \\ & =& x\hfill & (x\neq -3)\hfill \end{array}
\end{equation*}
As $x = -3$, sal die noemer, $x+3 = 0$, wees en die breuk ongedefinieer.\par

% \Note{In breuke kan slegs faktore uitdeel, nie terme nie.}
% \label{m39392*secfhsst!!!underscore!!!id3026}\vspace{.5cm} 
%       \noindent
%       \hspace*{-30pt}\includegraphics[width=0.5in]{col11306.imgs/pspencil2.png}   \raisebox{25mm}{   

\mindsetvid{Simplification of fractions}{VMacc}

\begin{wex}{Vereenvoudiging van breukuitdrukkings }{Vereenvoudig: $\dfrac{ax-b+x-ab}{a{x}^{2}-abx}, ~~(x \neq 0;x \neq b)$.}
{
\westep{Gebruik groepering vir die teller en haal ’n gemene faktor van $ax$ uit in die noemer.}
\begin{equation*}
\begin{array}{lll}\dfrac{(ax-ab)+(x-b)}{a{x}^{2}-abx}\hfill & =& \dfrac{a(x-b)+(x-b)}{ax(x-b)}\hfill \\ \end{array}
\end{equation*}
\westep{Haal gemene faktor $(x -b)$ uit in die teller }  
\begin{equation*}
\begin{array}{lll}
\\& =& \dfrac{(x-b)(a+1)}{ax(x-b)}\hfill \end{array}
\end{equation*}

\westep{Deel die gemeenskaplike faktor uit in die noemer en die teller om die finale antwoord te gee} 
\begin{equation*}
\begin{array}{ccc}& =& \dfrac{a+1}{ax}\hfill \end{array}
\end{equation*}
}
\end{wex}


\begin{wex}{Vereenvoudig breukuitdrukkings }
{Vereenvoudig: $\dfrac{{x}^{2}-x-2}{{x}^{2}-4}÷\dfrac{{x}^{2}+x}{{x}^{2}+2x}, ~~(x \neq 0;x \neq \pm2)$.} 
{
\westep{Faktoriseer die noemer en die teller}
\begin{equation*}
\begin{array}{ccc}& =& \dfrac{(x+1)(x-2)}{(x+2)(x-2)}÷\dfrac{x(x+1)}{x(x+2)}\hfill \end{array}
\end{equation*}
\westep{Verander deling na vermenigvuldig met die resipook}


\begin{equation*}
\begin{array}{ccc}& =& \dfrac{(x+1)(x-2)}{(x+2)(x-2)}\ensuremath{\times}\dfrac{x(x+2)}{x(x+1)}\hfill \end{array}
\end{equation*}
\westep{Die vereenvoudigde antwoord is}
\begin{equation*}
\begin{array}{ccc}& =& 1\hfill \end{array}
\end{equation*}
}
\end{wex}


%     \noindent
%  \noindent
%       \hspace*{-30pt}\includegraphics[width=0.5in]{col11306.imgs/pspencil2.png}   \raisebox{25mm}{   
%      
\begin{wex}{Vereenvoudiging van breuke}{Vereenvoudig: $\dfrac{x-2}{{x}^{2}-4}+\dfrac{{x}^{2}}{x-2}-\dfrac{{x}^{3}+x-4}{{x}^{2}-4}, ~~(x \neq \pm2)$}
{
\westep{Faktoriseer die noemers}
\begin{equation*}
\frac{x-2}{(x+2)(x-2)}+\frac{{x}^{2}}{x-2}-\frac{{x}^{3}+x-4}{(x+2)(x-2)}
\end{equation*}
\westep{Maak al die noemers dieselfde sodat die breuke kan opgestel en afgetrek word} Die kleinste gemene noemer is $(x-2)(x+2)$\par 

\begin{equation*}
\frac{x-2}{(x+2)(x-2)}+\frac{({x}^{2})
(x+2)}{(x+2)(x-2)}-\frac{{x}^{3}+x-4}{(x+2)(x-2)}
\end{equation*}
\westep {Skryf as een breuk}
\begin{equation*}
\frac{x-2+({x}^{2})(x+2)-(x^{3}+x-4)}{(x+2)(x-2)}
\end{equation*}
\westep{Vereenvoudig}

\begin{equation*}
 \begin{array}{llll}
\dfrac{x-2+{x}^{3}+ 2x^{2}-x^{3} - x+4}{(x+2)(x-2)} & = & \dfrac{2x^{2} + 2}{(x+2)(x-2)}\\
\end{array}
\end{equation*}
\westep{Haal die gemene faktoor uit en skryf die finale antwoord}

\begin{equation*}
\dfrac{2({x}^{2}
+1)}{(x+2)(x-2)}
\end{equation*}
}
\end{wex}

\begin{wex}{Vereenvoudig breukuitdrukkings}{Vereenvoudig: $\dfrac{2}{{x}^{2}-x}+\dfrac{x^{2}+x+1}{x^{3}-1}-\dfrac{x}{{x}^{2}-1}, ~~(x \neq 0;x \neq \pm1)$.}
{
\westep{Fakotoriseer die noemers en die tellers}
\begin{equation*}
\dfrac{2}{x(x-1)}+ \dfrac{({x}^{2} + x + 1)}{(x-1)(x^{2}+x+1)}-\dfrac{x}{(x-1)(x+1)}
\end{equation*}
\westep{Skryf as een breuk} 
\begin{equation*}
\dfrac{2(x+1)+x(x+1)-x^{2}}{x(x-1)(x+1)}
\end{equation*}
\westep {Skryf die finale antwoord}
\begin{equation*}
\dfrac{2x+2 + x^{2} + x - x^{2}}{x(x-1)(x+1)} = \dfrac{3x+2}{x(x-1)(x+1)}
\end{equation*}



}
\end{wex}


\begin{exercises}{}
{

Vereenvoudig (aanvaar alle noemers is ongelyk aan nul):
\begin{multicols}{2}
\begin{enumerate}[itemsep=5pt, label=\textbf{\arabic*}. ] 
\item$\dfrac{3a}{15}$
\item $\dfrac{2a+10}{4}$
\item $\dfrac{5a+20}{a+4}$
\item $\dfrac{{a}^{2}-4a}{a-4}$
\item $\dfrac{3{a}^{2}-9a}{2a-6}$
\item $\dfrac{9a+27}{9a+18}$
\item $\dfrac{6ab+2a}{2b}$
\item $\dfrac{16{x}^{2}y-8xy}{12x-6}$
\item $\dfrac{4xyp-8xp}{12xy}$
\item $\dfrac{3a+9}{14}÷\dfrac{7a+21}{a+3}$
\item $\dfrac{{a}^{2}-5a}{2a+10} \times \dfrac{4a}{3a+15}$
\item $\dfrac{3xp+4p}{8p}÷\dfrac{12{p}^{2}}{3x+4}$
\item $\dfrac{24a-8}{12}÷\dfrac{9a-3}{6}$
\item $\dfrac{{a}^{2}+2a}{5}÷\dfrac{2a+4}{20}$
\item $\dfrac{{p}^{2}+pq}{7p} \times \dfrac{21q}{8p+8q}$
\item $\dfrac{5ab-15b}{4a-12}÷\dfrac{6{b}^{2}}{a+b}$
\item $\dfrac{{f}^{2}a-f{a}^{2}}{f-a}$
\item $\dfrac{2}{xy} + \dfrac{4}{xz}+\dfrac{3}{yz}$
\item $\dfrac{5}{t-2} - \dfrac{1}{t-3}$
\item $\dfrac{k+2}{k^{2} +2} - \dfrac{1}{k+2}$
\item $\dfrac{t+2}{3q} + \dfrac{t+1}{2q}$
\item $\dfrac{3}{p^{2}-4}+\dfrac{2}{(p-2)^{2}}$
\item $\dfrac{x}{x+y}+\dfrac{x^{2}}{y^{2} - x^{2}}$
\item $\dfrac{1}{m+n} + \dfrac{3mn}{m^{3} + n^{3}}$
\item $\dfrac{h}{h^{3}-f^{3}} - \dfrac{1}{h^{2} + hf + f^{2}}$
\item $\dfrac{{x}^{2}-1}{3}\times\dfrac{1}{x-1}-\dfrac{1}{2}$
\item $\dfrac{x^2-2x+1}{(x-1)^3} - \dfrac{x^2+x+1}{x^3-1}$
\item $\dfrac{1}{(x-1)^2} - \dfrac{2x}{x^3-1}$
\item $\dfrac{p^3 + q^3}{p^2} \times \dfrac{3p-3q}{p^2-q^2}$
\item $\frac{1}{a^2-4ab+4b^2} + \frac{a^2+2ab+b^2}{a^3-8b^3} - \frac{1}{a^2-4b^2}$
\end{enumerate}
\end{multicols}

% Automatically inserted shortcodes - number to insert 30
\par \practiceinfo
\par \begin{tabular}[h]{cccccc}
% Question 1
(1.-7.)	02h8	&
% Question 2
(8.-13.)	02h9	&
% Question 3
(14.-20.)	02ha	&
% Question 4
(21.-26.)	02hb	&
% Question 5
(27.-30.)	02hc	&
% Question 6
\\ % End row of shortcodes
\end{tabular}
% Automatically inserted shortcodes - number inserted 30
}
\end{exercises}

\summary{VMcyl}

% \label{m38348*cid8} $ \hspace{-5pt}\begin{array}{cccccccccccc}   \end{array} $ \hspace{2 pt}\raisebox{-5 pt}{\includegraphics[width=0.5cm]{col11306.imgs/summary_www.png}} {(subsection shortcode: MG10040 )} \par \label{m38348*eip-280}

\begin{itemize}[itemsep=5pt, label=\textbullet{}]

\item n Rasionale getal is enige getal wat geskryf kan word as $\dfrac{a}{b}$
waar $a$ en $b$ heelgetalle is en $b\ne 0$.
\item Die volgende is rasionale getalle:
    \begin{itemize}[noitemsep]
	\item Breuke waarvan beide die teller en die noemer heeltallig is
	\item Heelgetalle
	\item Desimale getalle wat eindig
	\item Desimale getalle wat repeteer
    \end{itemize}
% \label{m38349*eip-361} $ \hspace{-5pt}\begin{array}{cccccccccccc}   \end{array} $ \hspace{2 pt}\raisebox{-5 pt}{\includegraphics[width=0.5cm]{col11306.imgs/summary_www.png}} {(subsection shortcode: MG10058 )} \par \label{m38349*uid0821}
\item Irrasionale getalle is getalle wat nie as ’n breuk met ’n heeltallige teller en noemer geskryf kan word nie.
% \label{m38347*eip-194} $ \hspace{-5pt}\begin{array}{cccccccccccc}   \end{array} $ \hspace{2 pt}\raisebox{-5 pt}{\includegraphics[width=0.5cm]{col11306.imgs/summary_www.png}} {(subsection shortcode: MG10053 )} \par \label{m38347*eip-50}\begin{itemize}[itemsep=5pt]
\item Indien die ${n}^{\mathrm{de}}$ magswortel van ’n getal nie as ’n rasionale getal geskryf kan word nie, noem ons dit ’n wortelgetal.
\item As $a$ en $b$ positiewe getalle is met $a\lessthan{}b$, dan $\sqrt[n]{a}\lessthan{}\sqrt[n]{b}$.
\item ’n Binomiaal is ’n wiskundige uitdrukking met twee terme.
\item Die produk van twee identiese binomiale staan bekend as die vierkant of kwadraat van die binomiaal. 
\item Die verskil tussen twee kwadrate kry ons wanneer ons $(ax+b)$ met $(ax-b)$ vermenigvuldig.
\item Faktorisering is die omgekeerde proses van die uitbreiding van hakies.
\item $(A+B)(C+D+E)=A(C+D+E)+B(C+D+E)$ help ons om ’n binomiaal en ’n trinomiaal te vermenigvuldig.
\item Uithaal van 'n gemeenskaplike faktor is die basiese metode van faktorisering.
\item Ons gebruik dikwels groepering om polinome te faktoriseer.
\item Om ’n kwadratiese drieterm te faktoriseer, moet ons die twee tweeterme vind wat met mekaar vermenigvuldig is om die kwadratiese drieterm te gee.
\item Die som van derdemagte is: ${x}^{3}+{y}^{3}=(x+y)({x}^{2}-xy+{y}^{2})$. 
\item Die verskil van derdemagte is: ${x}^{3}-{y}^{3}=(x-y)({x}^{2}+xy+{y}^{2})$.
\item Ons kan breuke vereenvoudig deur metodes van faktorisering toe te pas.
\item Slegs faktore kan uitgekanselleer word in breuke, nooit terme nie.
\item Breuke kan bymekaargetel of van mekaar afgetrek word. Om dit te kan doen, moet al die breuke dieselfde
noemers hê.

\end{itemize}


\begin{eocexercises}{}
% ----------------------------------------------------------------------------------------------
% RATIONAL NUMBERS

% \label{m38348*cid9} $ \hspace{-5pt}\begin{array}{cccccccccccc}   \end{array} $ \hspace{2 pt}\raisebox{-5 pt}{\includegraphics[width=0.5cm]{col11306.imgs/summary_www.png}} {(subsection shortcode: MG10041 )} \par \label{m38348*id64954}

\begin{enumerate}[itemsep=5pt, label=\textbf{\arabic*}. ] 
\item Indien $a$ ’n heelgetal is, $b$ ’n heelgetal is en $c$ irrasionaal is, watter van die volgende is rasionaal?
    \begin{enumerate}[itemsep=5pt, label=\textbf{\alph*}. ] 
    \item $\dfrac{-b}{a}$
    \item $c \div c$
    \item $\dfrac{a}{c}$
    \item $\dfrac{1}{c}$
    \end{enumerate}
\item Skryf elkeen van die volgende as ’n onegte breuk:
    \begin{enumerate}[itemsep=5pt, label=\textbf{\alph*}. ] 
    \item $0,12$
    \item $0,006$
    \item $1,59$
    \item $12,27\dot{7}$
    \end{enumerate}
\item Wys dat die desimaal $3,21\dot{1}\dot{8}$ ’n rasionale getal is.
\item Druk  $0,7\dot{8}$ uit as ’n breuk $\dfrac{a}{b}$ waar $a,b\in \mathbb{Z}$ (wys alle stappe).


% \par \raisebox{-5 pt}{\includegraphics[width=0.5cm]{col11306.imgs/summary_www.png}} Find the answers with the shortcodes:
% \par \begin{tabular}[h]{cccccc}
% (1.) l3v  &  (2.) l3f  &  (3.) l3G  &  (4.) lOf  & \end{tabular}
% -----------------------------------------------------------------------------------

% IRRATIONAL NUMBERS & ROUNDING OFF
%        \subsection{End of Section Exercises}

% \label{m38349*cid5} $ \hspace{-5pt}\begin{array}{cccccccccccc}   \end{array} $ \hspace{2 pt}\raisebox{-5 pt}{\includegraphics[width=0.5cm]{col11306.imgs/summary_www.png}} {(subsection shortcode: MG10059 )} \par \label{m38349*id325742}\begin{enumerate}[itemsep=5pt, label=\textbf{\arabic*}. ] 

\item Skryf die volgende rasionalle getalle tot $2$ desimale plekke:
    \begin{enumerate}[itemsep=5pt, label=\textbf{\alph*}. ] 
    \item $\dfrac{1}{2}$
    \item $1$
    \item $0,11111\overline{1}$
    \item $0,99999\overline{1}$
    \end{enumerate}

\item Rond die volgende getalle af tot $3$ desimale plekke:
\begin{multicols}{2}
    \begin{enumerate}[itemsep=5pt, label=\textbf{\alph*}. ] 
    \item $3,141592654\ldots$
    \item $1,618\phantom{\rule{0.166667em}{0ex}}033\phantom{\rule{0.166667em}{0ex}}989\phantom{\rule{0.166667em}{0ex}}\ldots$
    \item $1,41421356\ldots$
    \item $2,71828182845904523536\ldots$
    \end{enumerate}
\end{multicols}
\item Gebruik jou sakrekenaar om die volgende irrasionale getalle tot $4$ desimale plekke te skryf:
\begin{multicols}{2}
    \begin{enumerate}[itemsep=5pt, label=\textbf{\alph*}. ] 
    \item $\sqrt{2}$
    \item $\sqrt{3}$
    \item $\sqrt{5}$
    \item $\sqrt{6}$
    \end{enumerate}
\end{multicols}

\item Gebruik jou sakrekenaar (waar nodig) om die volgende getalle tot  $5$ desimale plekke te skryf en dui aan of die getal rasionaal of irrasionaal is:
\begin{multicols}{2}
    \begin{enumerate}[itemsep=5pt, label=\textbf{\alph*}. ] 
    \item $\sqrt{8}$
    \item $\sqrt{768}$
    \item $\sqrt{0,49}$
    \item $\sqrt{0,0016}$
    \item $\sqrt{0,25}$
    \item $\sqrt{36}$
    \item $\sqrt{1960}$
    \item $\sqrt{0,0036}$
    \item $-8\sqrt{0,04}$
    \item $5\sqrt{80}$
    \end{enumerate}
\end{multicols}

\item Skryf die volgende irrasionale getalle tot $3$ desimale plekke, en skryf die afgeronde getal dan as ’n rasionale getal om ’n benadering van die irrasionale getal te verkry.
\begin{multicols}{2}
\begin{enumerate}[itemsep=5pt, label=\textbf{\alph*}. ] 
    \item $3,141592654\ldots$
    \item $1,618\phantom{\rule{0.166667em}{0ex}}033\phantom{\rule{0.166667em}{0ex}}989\phantom{\rule{0.166667em}{0ex}}\ldots$
    \item $1,41421356\ldots$
    \item $2,71828182845904523536\ldots$
    \end{enumerate}
\end{multicols}

% \par \raisebox{-5 pt}{\includegraphics[width=0.5cm]{col11306.imgs/summary_www.png}} Find the answers with the shortcodes:
% \par \begin{tabular}[h]{cccccc}
% (1.) llN  &  (2.) llR  &  (3.) lln  &  (4.) llQ  &  (5.) llU  & \end{tabular}

% ESTIMATING SURDS
%           \subsubsubsection \subsection{ End of Chapter Exercises}

% \label{m38347*cid4} $ \hspace{-5pt}\begin{array}{cccccccccccc}   \end{array} $ \hspace{2 pt}\raisebox{-5 pt}{\includegraphics[width=0.5cm]{col11306.imgs/summary_www.png}} {(subsection shortcode: MG10054 )} \par \label{m38347*id260269}\begin{enumerate}[itemsep=5pt, label=\textbf{\arabic*}. ] 
\item Bepaal sonder 'n sakrekenaar tussen watter twee opeenvolgende heelgetalle die volgende irasionale getalle l\^{e}:
\begin{multicols}{2}
    \begin{enumerate}[itemsep=5pt, label=\textbf{\alph*}. ] 
    \item $\sqrt{5}$ 
    \item $\sqrt{10}$ 
    \item $\sqrt{20}$ 
    \item $\sqrt{30}$ 
    \item $\sqrt[3]{5}$ 
    \item $\sqrt[3]{10}$ 
    \item $\sqrt[3]{20}$ 
    \item $\sqrt[3]{30}$ 
    \end{enumerate}
\end{multicols}

\item  Vind twee opeenvolgende heelgetalle wat weerskante van $\sqrt{7}$ lê op die getallelyn.          
\item Vind twee opeenvolgende heelgetalle wat weerskante van $\sqrt{15}$ lê op die getallelyn.          

% \par \raisebox{-5 pt}{\includegraphics[width=0.5cm]{col11306.imgs/summary_www.png}} Find the answers with the shortcodes:
% \par \begin{tabular}[h]{cccccc}
% (1.a) lqr  &  (1.b) lqY  &  (1.c) lqg  &  (1.d) lq4  &  (1.e) lq2  &  (1.f) lqT  &  (1.g) lqb  &  (1.h) ll5  &  (2.) lqW  &  (3.) lq1  & \end{tabular}
% ---------------------------------------------------------------------------------------

% PRODUCTS AND FACTORS
%             \subsubsection{ End of Chapter Exercises}


\item Faktoriseer:
\begin{multicols}{2}
\begin{enumerate}[itemsep=5pt, label=\textbf{\alph*}. ] 
\item ${a}^{2}-9$
\item ${m}^{2}-36$
\item $9{b}^{2}-81$
\item $16{b}^{6}-25{a}^{2}$
\item ${m}^{2}-\frac{1}{9}$
\item $5-5{a}^{2}{b}^{6}$
\item $16b{a}^{4}-81b$
\item ${a}^{2}-10a+25$
\item $16{b}^{2}+56b+49$
\item $2{a}^{2}-12ab+18{b}^{2}$
\item $-4{b}^{2}-144{b}^{8}+48{b}^{5}$
\item $(16-{x}^{4})$
\item ${7x}^{2}-14x+7xy-14y$
\item ${y}^{2}-7y-30$
\item $1-x-{x}^{2}+{x}^{3}$
\item $-3(1-{p}^{2})+p+1$
\item $x-x^{3} + y - y^{3}$
\item $x^{2} - 2x + 1 - y^{4}$
\item $4b(x^{3} - 1) + x(1-x^{3})$
\item $3p^{3} - \frac{1}{9}$
\item $8x^6-125y^9$
\item $(2+p)^3- 8(p+1)^3$
\end{enumerate}
\end{multicols}


\item Vereenvoudig die volgende:
\begin{multicols}{2}
\begin{enumerate}[itemsep=5pt, label=\textbf{\alph*}. ] 

\item ${(a-2)}^{2}-a(a+4)$
\item $(5a-4b)(25{a}^{2}+20ab+16{b}^{2})$
\item $(2m-3)(4{m}^{2}+9)(2m+3)$
\item $(a+2b-c)(a+2b+c)$
\item $\dfrac{{p}^{2}-{q}^{2}}{p}÷\dfrac{p+q}{{p}^{2}-pq}$
\item $\dfrac{2}{x}+\dfrac{x}{2}-\dfrac{2x}{3}$
\item $\dfrac{1}{a+7}-\dfrac{a+7}{a^{2}-49}$
\item $\dfrac{x+2}{2x^{3}} + 16$
\item $\dfrac{1-2a}{4a^{2} -1} - \dfrac{a-1}{2a^{2}-3a+1} - \dfrac{1}{1-a}$
\item $\dfrac{x^{2} + 2x}{x^{2}+ x + 6} \times \dfrac{x^{2} + 2x + 1}{x^{2} + 3x +2}$
\end{enumerate}
\end{multicols}


\item Wys dat ${(2x-1)}^{2}-{(x-3)}^{2}$ vereenvoudig kan word tot  $(x+2)(3x-4)$.

\item Bepaal wat moet by ${x}^{2}-x+4$ getel word sodat dit gelyk is aan ${(x+2)}^{2}$.
\item Evalueer $\dfrac{x^{3}+1}{x^{2}-x+1}$ as $x=7,85$ sonder om 'n sakrekenaar te gebruik. Toon jou bewerkings.
%English
\item Met watter uitdrukking moet $(a-2b)$ vermenigvuldig word om 'n produk van $a^3-8b^3$ te kry?
\item Met watter uitdrukking moet $27x^3+1$ gedeel word om 'n kwosi\"ent van $3x+1$ te kry?
\end{enumerate}

% \par \raisebox{-5 pt}{\includegraphics[width=0.5cm]{col11306.imgs/summary_www.png}} Find the answers with the shortcodes:
% \par \begin{tabular}[h]{cccccc}
% (1.) liM  &  (2.) lTY  &  (3.) lTg  &  (4.) lT4  &  (5.) lib  &  (6.) liT  & \end{tabular}

% Automatically inserted shortcodes - number to insert 19
\par \practiceinfo
\par \begin{tabular}[h]{cccccc}
% Question 1
(1.)	02i4	&
% Question 2
(2.)	02i5	&
% Question 3
(3.)	02i6	&
% Question 4
(4.)	02i7	&
% Question 5
(5.)	02i8	&
% Question 6
(6.)	02i9	\\ % End row of shortcodes
% Question 7
(7.)	02ia	&
% Question 8
(8.)	02ib	&
% Question 9
(9.)	02ic	&
% Question 10
(10.)	02id	&
% Question 11
(11.)	02ie	&
% Question 12
(12.)	02if	\\ % End row of shortcodes
% Question 13
(13a-k.)	02ig	&
 (13l-p.) 02sp &
(13q-t.) 02sq &
(13u-v.) 02sr &
% Question 14
(14a-d.)	02ih	&
(14e-f.) 02ss	\\ % End row of shortcodes
(14g-j.) 02te &
% Question 15
(15.)	02ii	&
% Question 16
(16.)	02ij	&
% Question 17
(17.)	02ik	&
% Question 18
(18.)	02im	&
% Question 19
(19.)	02in	\\ % End row of shortcodes
\end{tabular}
% Automatically inserted shortcodes - number inserted 19

\end{eocexercises}
