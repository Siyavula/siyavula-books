\chapter{Eksponensiale}\fancyfoot[LO,RE]{Fokus Area: Wiskunde}
\setcounter{figure}{1}
\setcounter{subfigure}{1}

Eksponensiaalnotasie is ’n kort manier om te skryf dat ’n getal meermale met homself vermenigvuldig word. Laat ons beter definieer hoe om eksponensiaalnotasie te gebruik.

%  \begin{Large}
% \begin{center}
% $ _{\mbox{grondtal}~\leftarrow} $\begin{Large} $ ~a^{n~\rightarrow~$ \end{Large}$\mbox{eksponent / indeks}} $
% \end{center}
%  \end{Large}

\begin{center}
\scalebox{1} % Change this value to rescale the drawing.
{
\begin{pspicture}(0,-0.27015626)(5.1884375,0.27015626)
% \usefont{T1}{ptm}{m}{n}
\rput(0,-0.11328125){grondtal}
% \usefont{T1}{ptm}{m}{n}
\rput(1.5,-0.07828125){\Large $a^n$}
% \usefont{T1}{ptm}{m}{n}
\rput(3.85,0.06671875){eksponent/indeks}
\psline[linewidth=0.01cm,arrowsize=0.05291667cm 2.0,arrowlength=1.4,arrowinset=0.4]{->}(1.2871875,-0.15)(0.7871875,-0.15)
\psline[linewidth=0.01cm,arrowsize=0.05291667cm 2.0,arrowlength=1.4,arrowinset=0.4]{->}(1.8071876,0.03671875)(2.3471875,0.03671875)
\end{pspicture} 
}
\end{center}
Vir enige re\"ele getal $a$ en natuurlike getal $n$, kan ons $a$ wat $n$ keer vermenigvuldig word met homself skryf as $a^n$.

 
\Identity{
\vspace*{-3em}
\begin{flushleft}
\begin{enumerate}[itemsep=5pt, label=\textbf{\arabic*}.]
 \item $a^n = a \times a \times a \times \ldots \times a ~~ (n ~ \mbox{keer}) ~~~~ (a \in \mathbb{R}, n \in \mathbb{N})$
 \item $ a^0 = 1 \hspace{1cm}$  ($a \ne 0 $ want $0^0$ is ongedefinieer)
 \item $a^{-n} &=& \frac{1}{a^n}\hspace{0.5cm}$ ($a \ne 0 $ want $\frac{1}{0}$ is ongedefinieer)
\end{enumerate}
\end{flushleft}
}


Voorbeelde:
\begin{enumerate}[noitemsep, label=\textbf{\arabic*.}]
\item $3 \times 3 = 3^2$
\item $5 \times 5 \times 5 \times 5 = 5^4 $
\item $p \times p \times p = p^3$
\item $(3^x)^0 = 1$
\item $ 2^{-4} = \dfrac{1}{2^4} = \frac{1}{16}$
\item $ \dfrac{1}{5^{-x}} = 5^x$
\end{enumerate}

% We now expand the definition to include 0 and the negative integers.
% 
% If exponent $n$ is 0, then we have 
% 
% 
% 
% \Tip{The restriction applies because $0^0$ is undefined }
% 
% We also define what it means if we have a negative exponent $a^{-n}$.
% \begin{eqnarray*}
%     a^{-n} &=& \frac{1}{a^n} \\
%            &=& \frac{1}{a \times a \times a \times \ldots \times a ~~ (n ~ \textrm{times})} 
% \end{eqnarray*}
% 
% 
% \Tip{Here $a$ cannot equal 0 because $\frac{1}{0}$ is undefined}      
      
\textbf{Nota:} Ons skryf altyd die finale antwoord met positiewe eksponente.


% If exponent $n$ is an even integer, then ${a}^{n}$ will always be positive for any non-zero real number $a$. 
% 
% For example, although $-3$ is negative, but $(-3)^2=-3 \times -3 = 9$ which is positive and $(-3)^{-2} = \frac{1}{-3 \times -3} = \frac{1}{9} $ is also positive.
% 
% If the exponent $n$ is an odd integer, then for any non-zero real number $a$ 
% 
% \begin{eqnarray*}
% a^n ~~ \mbox{is positive if} ~~ a > 0 \\
% a^n ~~ \mbox{is negative if} ~~ a < 0
% \end{eqnarray*}
% 
% For example, $(-2)^3 = -2 \times -2 \times -2 = -8$ and $(2)^{-5} = \dfrac{1}{2 \times 2 \times 2 \times 2 \times 2} = \dfrac{1}{32}$.

\par
\chapterstartvideo{VMald}  
    


\section {Eksponentwette}
\nopagebreak

Daar is 'n aantal eksponentwette wat ons kan gebruik om getalle met eksponente te vereenvoudig. Sommige
van hierdie wette het ons reeds in vorige grade teëgekom, maar hier is die volledige lys:

\Identity{
\vspace*{-3em}
\begin{center}
\begin{itemize}

% \begin{align*}
  \item $a^{m} \times a^{n} = a^{m+n}$ 
  \item $\frac{a^{m}}{a^{n}} = a^{m-n}$ 
  \item ${(ab)}^{n} = a^{n}b^{n}$ 
  \item $\left(\frac{a}{b}\right)^n = \frac{a^n}{b^n}$ 
  \item ${({a}^{m})}^{n} = a^{mn}$
% \end{align*}

\end{itemize}
\end{center}
waar $a > 0, b > 0$ en $m,n \in \mathbb{Z}$
}



\begin{wex}
{ %title
Toepassing van eksponentwette
}
{%question
Vereenvoudig:\\
\\
\begin{minipage}{\textwidth}
\begin{enumerate}[itemsep=6pt, label=\textbf{\arabic*}.]
\item  $2^{3x} \times 2^{4x}$
\item $\frac{12p^2t^5}{3pt^3}$
\item $ (3x)^2 $
\item $(3^4 5^2)^3$
\end{enumerate}
\end{minipage}
}
{%answer
\begin{minipage}{\textwidth}
\begin{enumerate}[itemsep=6pt, label=\textbf{\arabic*}.]
\item  $2^{3x} \times 2^{4x} = 2^{3x+4x} = 2^{7x}$
 \item $\dfrac{12p^2t^5}{3pt^3} = 4p^{(2-1)}t^{(5-3)} = 4pt^2$
 \item $ (3x)^2 = 3^2x^2 = 9x^2$
 \item $(3^4\times5^2)^3 = 3^{(4\times3)}\times5^{(2\times3)} = 3^{12}\times5^6 $
\end{enumerate}
\end{minipage}
}
\end{wex}


\begin{wex}
{
Eksponentuitdrukkings
}
{
Vereenvoudig: $\dfrac{2^{2n} \times 4^n \times 2}{16^n}$
}
{

\westep{Verander grondtalle na priemgetalle}
\begin{equation*}
  \dfrac{2^{2n} \times 4^n \times 2}{16^n} = \dfrac{2^{2n} \times (2^2)^n \times 2^1}{(2^4)^n} 
\end{equation*}
\westep{Vereenvoudig die eksponente}
\begin{align*}
  &= \dfrac{2^{2n} \times 2^{2n} \times 2^1}{2^{4n}} \\
  &= \dfrac{2^{2n + 2n +1}}{2^{4n}} \\
  &= \dfrac{2^{4n+1}}{2^{4n}} \\
  &= 2^{4n+1-(4n)} \\
  &= 2
\end{align*}
}
\end{wex}


     \clearpage
\begin{wex}{Eksponentuitdrukkings}
{Vereenvoudig: $\dfrac{{5}^{2x-1}{9}^{x-2}}{{15}^{2x-3}}$}
{

\westep{Verander grondgetalle na priemgetalle}
\begin{equation*}
\begin{array}{lcl} \dfrac{{5}^{2x-1}  {9}^{x-2}}{{15}^{2x-3}}& =& \dfrac{{5}^{2x-1}  {({3}^{2})}^{x-2}}{{(5\times3)}^{2x-3}}\hfill \vspace{5pt}\\
		  & =& \dfrac{{5}^{2x-1}  {3}^{2x-4}}{{5}^{2x-3}  {3}^{2x-3}}\hfill 
\end{array}
\end{equation*}
\westep{Trek eksponente af (dieselfde grondtal)}
\begin{equation*}
\begin{array}{lcl}
& =& {5}^{(2x-1)-(2x-3)} \times {3}^{(2x-4)-(2x-3)}\hfill \\ 
& =& 5^{2x-1-2x+3} \times 3^{2x-4 - 2x+3} \\
& =& {5}^{2} \times {3}^{-1}\hfill \end{array}
\end{equation*}
\westep{Skryf die antwoord as 'n breuk}  
\begin{align*}
  &= \frac{25}{3} \\
  &= 8\frac{1}{3}
\end{align*}
}
\end{wex}

\textbf{Nota:} Wanneer ons met eksponentuitdrukkings werk, geld al die re\"els vir algebrai\"ese bewerkings nog steeds.

\begin{wex}{Vereenvoudig deur die uithaal van 'n gemene faktor}
{Vereenvoudig: $\dfrac{2^t-2^{t-2}}{3 \times 2^t - 2^t}$}
{
\westep{Vereenvoudig tot 'n faktoriseerbare vorm}
\begin{equation*}
  \dfrac{2^t-2^{t-2}}{3 \times 2^t-2^t} =
  \dfrac{2^t-(2^t \times 2^{-2})}{3 \times 2^t - 2^t}
\end{equation*}
\westep{Haal die gemeenskaplike faktor uit}
\begin{equation*}
  \phantom{\frac{2^t-2^{t-2}}{3 \times 2^t-2^t}} = \frac{2^t(1-2^{-2})}{2^t(3-1)}
\end{equation*}
\westep{Kanselleer die gemeenskaplike faktor en vereenvoudig}
\begin{align*}
  \phantom{\frac{2^t-2^{t-2}}{3.2^t-2^t}}
  &= \dfrac{1- \frac{1}{4}}{2} \\
  &= \dfrac{\frac{3}{4}}{2} \\
  &= \dfrac{3}{8} 
\end{align*}
} 
\end{wex}

\clearpage
\begin{wex}
{Vereenvoudig met die gebruik van die verskil van twee kwadrate}
{Vereenvoudig: $\dfrac{9^x-1}{3^x+1}$}
{
\westep{Verander grondtalle na priemgetalle}
\begin{eqnarray*}
 \frac{9^x-1}{3^x+1} & = & \frac{(3^2)^x -1}{3^x+1} \\
		     & = & \frac{(3^x)^2-1}{3^x+1} 
\end{eqnarray*}
\westep{Faktoriseer deur die verskil van twee vierkante te gebruik}
\begin{eqnarray*}
 \phantom{\frac{9^x-1}{3^x+1}} & = & \frac{(3^x-1)(3^x+1)}{3^x+1}\\
\end{eqnarray*}
\westep{Vereenvoudig}
\begin{eqnarray*}
 \phantom{\frac{9^x-1}{3^x+1}} & = & 3^x-1\\
\end{eqnarray*}
}
\end{wex}


\begin{exercises}{Vereenvoudig eksponentuitdrukkings}
{
Vereenvoudig sonder om ’n sakrekenaar te gebruik:
\begin{multicols}{2}
\begin{enumerate}[noitemsep, label=\textbf{\arabic*}., itemsep=5pt]
 \item $16^0$
 \item $16a^0$
 \item $\dfrac{2^{-2}}{3^2}$
 \item $ \dfrac{5}{2^{-3}}$
 \item $ \left(\dfrac{2}{3}\right)^{-3} $
 \item $ x^2 x^{3t+1} $
 \item $ 3 \times 3^{2a} \times 3^2$
 \item $ \dfrac{a^{3x}}{a^x} $
 \item $ \dfrac{32p^2}{4p^8}$
 \item $ (2t^4)^3$
 \item $ (3^{n+3})^2$
 \item $ \dfrac{3^n 9^{n-3}}{27^{n-1}}$
\end{enumerate}
\end{multicols}

 
% Automatically inserted shortcodes - number to insert 1
\par \practiceinfo
\par \begin{tabular}[h]{cccccc}
% Question 1
(1.-12.)	02kc	&
\end{tabular}
% Automatically inserted shortcodes - number inserted 1

}
\end{exercises}




\section{Rasionale eksponente}

Ons kan ook die eksponentwette toepas op uitdrukkings met rasionale eksponente.
\par
\mindsetvid{Fractions with exponents}{VMaln}
\clearpage
\begin{wex}{Vereenvoudig rasionale eksponente }
{Vereenvoudig: $2x^{\frac{1}{2}}\times 4x^{-\frac{1}{2}}$}
{
\begin{eqnarray*}
 2x^{\frac{1}{2}} \times 4x^{-\frac{1}{2}} & = & 8x^{\frac{1}{2}-\frac{1}{2}} \\
					  & = & 8x^0 \\
					  & = & 8(1) \\
					  & = & 8 
\end{eqnarray*}
}
\end{wex}


\begin{wex}{Vereenvoudig rasionale eksponente} 
{Vereenvoudig: $(0,008)^{\frac{1}{3}}$}
{% answer
\westep{Skryf as 'n breuk en verander grondtalle na priemgetalle}

\begin{eqnarray*}
 0,008 & = & \frac{8}{1~000} \\
       & = & \frac{2^3}{10^3} \\
       & = & \left(\frac{2}{10}\right)^3\\
\end{eqnarray*}

\westep{Dus}
\begin{eqnarray*}
 (0,008)^{\frac{1}{3}} & = & \left[\left(\frac{2}{10}\right)^3\right]^{\frac{1}{3}} \\
		 & = & \frac{2}{10} \\
		 & = & \frac{1}{5}
\end{eqnarray*}
}
\end{wex}

\begin{exercises}{Vereenvoudig rasionale eksponenete}
{
Vereenvoudig sonder om ’n sakrekenaar te gebruik:
\begin{multicols}{2}
\begin{enumerate}[noitemsep, label=\textbf{\arabic*}., itemsep=5pt]
 \item $ t^{\frac{1}{4}} \times 3t^{\frac{7}{4}} $
 \item $ \dfrac{16x^2}{(4x^2)^{\frac{1}{2}}} $
 \item $ (0,25)^{\frac{1}{2}} $
 \item $ (27)^{-\frac{1}{3}} $
 \item $ (3p^2)^{\frac{1}{2}} \times (3p^4)^{\frac{1}{2}} $
\end{enumerate}
\end{multicols}

% Automatically inserted shortcodes - number to insert 1
\par \practiceinfo
\par \begin{tabular}[h]{cccccc}
% Question 1
(1.-5.)	02kd	&
\end{tabular}
% Automatically inserted shortcodes - number inserted 1
}
\end{exercises}







\section{Eksponentvergelykings}

Eksponentvergelykings het die onbekende veranderlike in die eksponent. Hier is voorbeelde:
\begin{eqnarray*}
 3^{x+1} & = & 9 \\
5^t + 3 \,.\, 5^{t-1} & = & 400
\end{eqnarray*}

Om eksponentvergelykings op te los, moet ons die eksponentwette toepas. Dit beteken dat as ons weerskante van die gelykaanteken 'n enkele term het met dieselfde grondtal, kan ons die eksponente gelykstel.
\par
\textbf{Nota:} Indien $a>0$ en $a \ne 0$ en
\begin{center}
 $ a^x &= a^y $ \\
dan sal $ x &= y ~~\mbox{(dieselfde grondtal)}$\\
\end{center}
\par
Indien $a=1$, dan kan $x$ en $y$ verskil.

\begin{wex}{Stel eksponente gelyk}
{Los op vir $x$: $3^{x+1} = 9$.}
{%answer
\westep{Verander grondtalle na priemgetalle}
\begin{eqnarray*}
 3^{x+1} & = & 3^2 \\
\end{eqnarray*}
\westep{Grondtalle is dieselfde, dus kan ons eksponente gelykstel}
\begin{eqnarray*}
 {x+1} & = & 2 \\
\therefore x & = & 1
\end{eqnarray*}
}
\end{wex}

\textbf{Nota:} Om eksponentvergelykings op te los, gebruik ons al die prosedures vir oplos van line\^ere en kwadratiese vergelykings.



\begin{wex}
{%title
Oplos van vergelykings deur die uithaal van 'n gemene faktor
}
{%question
Los op vir $t$: $5^t + 3 \times 5^{t+1} = 400$.}
{%answer

\westep{Herskryf die vergelyking}
\begin{equation*}
  5^t + 3 ( 5^t \times 5) = 400 
\end{equation*}

\westep{Haal 'n gemeenskaplike faktor uit}
\begin{equation*}
 5^t(1 + 15) = 400 
\end{equation*}


\westep{Vereenvoudig}
\begin{align*}
 5^t(16) &= 400 \\
  5^t &= 25 
\end{align*}



\westep{Verander grondtalle na priemgetalle}
\begin{equation*}
  5^t = 5^2 
\end{equation*}


\westep{Grondtalle is dieselfde, dus kan ons eksponente gelykstel}
\begin{equation*}
\therefore t = 2
\end{equation*}

}
\end{wex}

\begin{wex}
{Oplos van vergelykings deur die faktorisering van \\'n kwadratiese drieterm}
{Los op: $ p-13 p^{\frac{1}{2}} + 36 =  0$.}
{ % Answer

\westep{Let op dat $(p^{\frac{1}{2}})^2=p$ so ons kan die vergelyking herskryf as}

$$ (p^{\frac{1}{2}})^2 -13p^{\frac{1}{2}} + 36 = 0 $$

\westep{Faktoriseer as 'n drieterm}

$$ (p^{\frac{1}{2}} -9)(p^{\frac{1}{2}}-4) = 0 $$

\westep{Los op om beide wortels te vind}

%Left Column				%right column
\begin{align*}
p^{\frac{1}{2}} - 9 &= 0			&   p^{\frac{1}{2}} - 4 &= 0		\\
p^{\frac{1}{2}} &= 9				&   p^{\frac{1}{2}} &= 4		\\		
(p^{\frac{1}{2}})^2 &= (9)^2			&   (p^{\frac{1}{2}})^2 &= (4)^2\\
p &= 81				&   p &= 16\\
\end{align*} 
$\therefore p=81$ of $p=16$.
}
\end{wex}
\clearpage
\begin{exercises}{}
{
\begin{enumerate}[noitemsep, label=\textbf{\arabic*}., itemsep=5pt]
\item Los op:
\begin{enumerate}[label=\textbf{(\alph*)}, itemsep=5pt]
\begin{multicols}{2}
\item $ 2^{x+5} = 32 $
\item $ 5^{2x+2} = \dfrac{1}{125} $
\item $ 64^{y+1} = 16^{2y+5} $
\item $ 3^{9x-2} = 27 $
\item $ 81^{k+2} = 27^{k+4} $

\item $ 25^{(1-2x)}-5^4 = 0 $
\item $ 27^x \times 9^{x-2} = 1 $
\item $ 2^t + 2^{t+2} = 40 $
\item $ 2 \times 5^{2-x} = 5+ 5^x $
\item $ 9^m + 3^{3-2m} = 28 $
\item $ y - 2y^{\frac{1}{2}} + 1 = 0 $
\item $4^{x+3} = 0,5$
\item $2^a = 0,125$
\item $10^x = 0,001$
\item $2^{x^2-2x-3} = 1$
\end{multicols}
\end{enumerate}

\item Die groei van alge kan gemodelleer word met die funksie $f(t) = 2^t$. Vind die waarde van $t$ as $f(t)=128$.   
\end{enumerate}

% Automatically inserted shortcodes - number to insert 2
\par \practiceinfo
\par \begin{tabular}[h]{cccccc}
% Question 1
(1a-m.)	02ke	&
% Question 2
(2.)	02kf	&
\end{tabular}
% Automatically inserted shortcodes - number inserted 2
}
\end{exercises}




\summary{VMdgh}



\begin{itemize}[noitemsep, label=\textbullet{}]
    \item Eksponensiaalnotasie verwys na ’n getal wat geskryf word as ${a}^{n}$ waar $n$ ’n heelgetal is en $a$ enige reële getal is.
    \item $a$ is die \textit{grondtal} en $n$ is die \textit{eksponent} of \textit{indeks}.
    \item Definisie: 
	  \begin{itemize}[noitemsep]
	   \item ${a}^{n}=a\ensuremath{\times}a\ensuremath{\times}\cdots \ensuremath{\times}a\phantom{\rule{2.em}{0ex}}\left(\mbox{$n$ kere}\right)$
	   \item ${a}^{0}=1,~a\ne 0$
	   \item ${a}^{-n}=\frac{1}{{a}^{n}},~a\ne 0$
	  \end{itemize}

    
    \item  Die eksponentwette: 
	\begin{itemize}[itemsep=4pt]
	    \item  ${a}^{m}\ensuremath{\times}{a}^{n}={a}^{m+n}$
	    \item  ${\dfrac{{a}^{m}}{{a}^{n}}={a}^{m-n}$
	    \item  ${\left(ab\right)}^{n}={a}^{n}{b}^{n}$
            \item  $\left(\dfrac{a}{b}\right)^n = \dfrac{a^n}{b^n}$
	    \item  ${\left({a}^{m}\right)}^{n}={a}^{mn}$
	\end{itemize}
    \end{itemize}


\begin{eocexercises}{}

  \begin{enumerate}[label=\textbf{\arabic*}., itemsep=5pt]
  \item Vereenvoudig:
    \begin{multicols}{2}
      \begin{enumerate}[label=\textbf{(\alph*)}, itemsep=7pt]
      \item $ t^3 \times 2t^0 $
      \item $ 5^{2x+y} 5^{3(x+z)} $
      \item $ (b^{k+1})^k $
      \item $ \dfrac{6^{5p}}{9^p} $
      \item $ m^{-2t} \times (3m^t)^3 $
      \item $\dfrac{3{x}^{-3}}{{(3x)}^{2}}$
      \item $\dfrac{{5}^{b-3}}{{5}^{b+1}}$
      \item $\dfrac{{2}^{a-2} {3}^{a+3}}{{6}^{a}}$
      \item $\dfrac{{3}^{n} {9}^{n-3}}{{27}^{n-1}}$
      \item ${\left(\dfrac{2{x}^{2a}}{{y}^{-b}}\right)}^{3}$
      \item $\dfrac{{2}^{3x-1} {8}^{x+1}}{{4}^{2x-2}}$
      \item $\dfrac{{6}^{2x} {11}^{2x}}{{22}^{2x-1} {3}^{2x}}$
      \item $\dfrac{{(-3)}^{-3} {(-3)}^{2}}{{(-3)}^{-4}}$
      \item ${({3}^{-1}+{2}^{-1})}^{-1}$
      \item $\dfrac{{9}^{n-1} {27}^{3-2n}}{{81}^{2-n}}$
      \item $\dfrac{{2}^{3n+2} {8}^{n-3}}{{4}^{3n-2}}$
      \item $\dfrac{3^{t+3} + 3^t}{2 \times 3^t} $
      \item $\dfrac{2^{3p} +1}{2^p + 1} $
      \end{enumerate}
    \end{multicols}

\item Los op:
    \begin{multicols}{2}
      \begin{enumerate}[label=\textbf{(\alph*)}, itemsep=7pt]
        \setcounter{enumi}{18}
      \item $ 3^x = \dfrac{1}{27} $
      \item $ 5^{t-1} = 1 $
      \item $ 2 \times 7^{3x} = 98 $
      \item $ 2^{m+1} = (0,5)^{m-2}$
      \item $ 3^{y+1} = 5^{y+1} $
      \item $ z^{\frac{3}{2}} = 64 $
      \item $ 16x^{\frac{1}{2}} - 4 = 0 $
      \item $ m^0 + m^{-1} = 0 $
      \item $ t^{\frac{1}{2}} - 3t^{\frac{1}{4}} + 2 = 0 $
      \item $ 3^p + 3^p + 3^p = 27 $
      \item $ k^{-1} - 7k^{-\frac{1}{2}} -18 = 0 $
      \item $ x^{\frac{1}{2}}+3x^{\frac{1}{4}}-18 = 0 $
      \end{enumerate}
    \end{multicols}
  \end{enumerate}
\end{eocexercises}

% Automatically inserted shortcodes - number to insert 2
\par \practiceinfo
\par \begin{tabular}[h]{cccccc}
% Question 1
(1.)	02kg	&
% Question 2
(2.)	02kh	&
\end{tabular}
% Automatically inserted shortcodes - number inserted 2

% \begin{enumerate}[noitemsep, label=\textbf{\arabic*}. ] 
% %     \item Simplify as far as possible:
% % 	\begin{enumerate}[noitemsep, label=\textbf{\alph*}. ] 
% % 	    \item ${302}^{0}$
% % 	    \item ${1}^{0}$
% % 	    \item ${\left(xyz\right)}^{0}$
% % 	    \item ${\left[{\left(3{x}^{4}{y}^{7}{z}^{12}\right)}^{5}{\left(-5{x}^{9}{y}^{3}{z}^{4}\right)}^{2}\right]}^{0}$
% % 	    \item ${\left(2x\right)}^{3}$
% % 	    \item ${\left(-2x\right)}^{3}$
% % 	    \item ${\left(2x\right)}^{4}$
% % 	    \item ${\left(-2x\right)}^{4}$
% % 	\end{enumerate}
%     \item Simplify
%     \item Simplify without using a calculator. Leave your answers with positive exponents.
% 	\begin{enumerate}[noitemsep, label=\textbf{\alph*}. ] 
% 	    \item $\dfrac{3{x}^{-3}}{{\left(3x\right)}^{2}}$
% 	    \item $5{x}^{0}+{8}^{-2}-{\left(\frac{1}{2}\right)}^{-2}\ensuremath{\,.\,}{1}^{x}$
% 	    \item $\dfrac{{5}^{b-3}}{{5}^{b+1}}$
% 	\end{enumerate}
%     \item Simplify, showing all steps:
% 	\begin{enumerate}[noitemsep, label=\textbf{\alph*}. ] 
% 	    \item $\dfrac{{2}^{a-2}.{3}^{a+3}}{{6}^{a}}$
% 	    \item $\dfrac{{a}^{2m+n+p}}{{a}^{m+n+p}\ensuremath{\,.\,}{a}^{m}}$
% 	    \item $\dfrac{{3}^{n}\ensuremath{\,.\,}{9}^{n-3}}{{27}^{n-1}}$
% 	    \item ${\left(\dfrac{2{x}^{2a}}{{y}^{-b}}\right)}^{3}$
% 	    \item $\dfrac{{2}^{3x-1}\ensuremath{\,.\,}{8}^{x+1}}{{4}^{2x-2}}$
% 	    \item $\dfrac{{6}^{2x}\ensuremath{\,.\,}{11}^{2x}}{{22}^{2x-1}\ensuremath{\,.\,}{3}^{2x}}$
% 	\end{enumerate}
%     \item Simplify, without using a calculator:
% 	\begin{enumerate}[noitemsep, label=\textbf{\alph*}. ] 
% 	    \item $\dfrac{{\left(-3\right)}^{-3}\ensuremath{\,.\,}{\left(-3\right)}^{2}}{{\left(-3\right)}^{-4}}$
% 	    \item ${\left({3}^{-1}+{2}^{-1}\right)}^{-1}$
% 	    \item $\dfrac{{9}^{n-1}\ensuremath{\,.\,}{27}^{3-2n}}{{81}^{2-n}}$
% 	    \item $\dfrac{{2}^{3n+2}\ensuremath{\,.\,}{8}^{n-3}}{{4}^{3n-2}}$
% 	\end{enumerate}
% \end{enumerate}
% 
% \raisebox{-5 pt}{\includegraphics[width=0.5cm]{col11306.imgs/summary_www.png}} Find the answers with the shortcodes:
% \begin{tabular}[h]{cccccc}
% (1.) lOJ  &  (2.) lOu  &  (3.) lOS  &  (4.) lOh  & 
% \end{tabular}
