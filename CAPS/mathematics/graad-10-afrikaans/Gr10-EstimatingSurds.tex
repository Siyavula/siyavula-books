         \section{Estimating surds}
    \setcounter{figure}{1}
    \setcounter{subfigure}{1}
    \label{m38347}
    \subsection{ Introduction}
            \nopagebreak
            \label{m38347*cid1} $ \hspace{-5pt}\begin{array}{cccccccccccc}   \includegraphics[width=0.75cm]{col11306.imgs/summary_fullmarks.png} &   \end{array} $ \hspace{2 pt}\raisebox{-5 pt}{} {(subsection shortcode: MG10052 )} \par 
      \label{m38347*id258007}You should know by now what the ${n}^{\mathrm{th}}$ root of a number means. If the ${n}^{\mathrm{th}}$ root of a number cannot be simplified to a rational number, we call it a $\mathit{surd}$. For example, $\sqrt{2}$ and $\sqrt[3]{6}$ are surds, but $\sqrt{4}$ is not a surd because it can be simplified to the rational number 2.\par 
      \label{m38347*id258405}In this chapter we will only look at surds that look like $\sqrt[n]{a}$, where $a$ is any positive number, for example $\sqrt{7}$ or $\sqrt[3]{5}$. It is very common for $n$ to be 2, so we usually do not write $\sqrt[2]{a}$. Instead we write the surd as just $\sqrt{a}$, which is much easier to read.\par 
      \label{m38347*id258479}It is sometimes useful to know the approximate value of a surd without having to use a calculator. For example, we want to be able to estimate where a surd like $\sqrt{3}$ is on the number line. So how do we know where surds lie on the number line? From a calculator we know that $\sqrt{3}$ is equal to $1,73205...$. It is easy to see that $\sqrt{3}$ is above 1 and below 2. But to see this for other surds like $\sqrt{18}$ without using a calculator, you must first understand the following fact:\par 
\label{m38347*notfhsst!!!underscore!!!id71}
\begin{tabular}{cc}
	\hspace*{-50pt}\raisebox{-8 mm}{\hspace{-0.2in}\includegraphics[width=0.75in]{col11306.imgs/psfact2.png} } & 
	\begin{minipage}{0.85\textwidth}
	\begin{note}
      {note: } 
If $a$ and $b$ are positive whole numbers, and $a\lessthan{}b$, then $\sqrt[n]{a}\lessthan{}\sqrt[n]{b}$. (Challenge: Can you explain why?)
	\end{note}
	\end{minipage}
	\end{tabular}
	\par
      \label{m38347*id258599}If you don't believe this fact, check it for a few numbers to convince yourself it is true.\par 
      \label{m38347*id258603}How do we use this fact to help us guess what $\sqrt{18}$ is? Well, you can easily see that $18\lessthan{}25$. Using our rule, we also know that $\sqrt{18}\lessthan{}\sqrt{25}$. But we know that ${5}^{2}=25$ so that $\sqrt{25}=5$. Now it is easy to simplify to get $\sqrt{18}\lessthan{}5$. Now we have a better idea of what $\sqrt{18}$ is.\par 
      \label{m38347*id258700}Now we know that $\sqrt{18}$ is less than 5, but this is only half the story. We can use the same trick again, but this time with 18 on the right-hand side. You will agree that $16\lessthan{}18$. Using our rule again, we also know that $\sqrt{16}\lessthan{}\sqrt{18}$. But we know that 16 is a perfect square, so we can simplify $\sqrt{16}$ to 4, and so we get $4\lessthan{}\sqrt{18}$!\par 
      \label{m38347*id258766}As you can see, we have shown that $\sqrt{18}$ is between 4 and 5. If we check on our calculator, we can see that $\sqrt{18}=4,1231...$, and the idea was right! You will notice that our idea used perfect squares that were close to the number 18. We found the largest perfect square smaller than 18 was ${4}^{2}=16$, and the smallest perfect square greater than 18 was ${5}^{2}=25$. Here is a quick summary of what a perfect square or cube is:\par 
\label{m38347*notfhsst!!!underscore!!!id78}
\begin{tabular}{cc}
	\hspace*{-50pt}\raisebox{-8 mm}{\hspace{-0.2in}\includegraphics[width=0.75in]{col11306.imgs/psfact2.png} } & 
	\begin{minipage}{0.85\textwidth}
	\begin{note}
      {note: } A perfect square is the number obtained when an integer is squared. For example, 9 is a perfect square since ${3}^{2}=9$. Similarly, a perfect cube is a number which is the cube of an integer. For example, 27 is a perfect cube, because ${3}^{3}=27$.
	\end{note}
	\end{minipage}
	\end{tabular}
	\par
      \label{m38347*id258890}To make it easier to use our idea, we will create a list of some of the perfect squares and perfect cubes. The list is shown in Table 6.1.\par 
    % \textbf{m38347*uid1}\par
          \begin{table}[H]
    % \begin{table}[H]
    % \\ '' '0'
        \begin{center}
      \label{m38347*uid1}
    \noindent
    \tabletail{%
        \hline
        \multicolumn{3}{|p{\mytableboxwidth}|}{\raggedleft \small \sl continued on next page}\\
        \hline
      }
      \tablelasttail{}
      \begin{xtabular}[t]{|l|l|l|}\hline
        Integer &
        Perfect Square &
        Perfect Cube% make-rowspan-placeholders
     \tabularnewline\cline{1-1}\cline{2-2}\cline{3-3}
      %--------------------------------------------------------------------
        0 &
        0 &
        0% make-rowspan-placeholders
     \tabularnewline\cline{1-1}\cline{2-2}\cline{3-3}
      %--------------------------------------------------------------------
        1 &
        1 &
        1% make-rowspan-placeholders
     \tabularnewline\cline{1-1}\cline{2-2}\cline{3-3}
      %--------------------------------------------------------------------
        2 &
        4 &
        8% make-rowspan-placeholders
     \tabularnewline\cline{1-1}\cline{2-2}\cline{3-3}
      %--------------------------------------------------------------------
        3 &
        9 &
        27% make-rowspan-placeholders
     \tabularnewline\cline{1-1}\cline{2-2}\cline{3-3}
      %--------------------------------------------------------------------
        4 &
        16 &
        64% make-rowspan-placeholders
     \tabularnewline\cline{1-1}\cline{2-2}\cline{3-3}
      %--------------------------------------------------------------------
        5 &
        25 &
        125% make-rowspan-placeholders
     \tabularnewline\cline{1-1}\cline{2-2}\cline{3-3}
      %--------------------------------------------------------------------
        6 &
        36 &
        216% make-rowspan-placeholders
     \tabularnewline\cline{1-1}\cline{2-2}\cline{3-3}
      %--------------------------------------------------------------------
        7 &
        49 &
        343% make-rowspan-placeholders
     \tabularnewline\cline{1-1}\cline{2-2}\cline{3-3}
      %--------------------------------------------------------------------
        8 &
        64 &
        512% make-rowspan-placeholders
     \tabularnewline\cline{1-1}\cline{2-2}\cline{3-3}
      %--------------------------------------------------------------------
        9 &
        81 &
        729% make-rowspan-placeholders
     \tabularnewline\cline{1-1}\cline{2-2}\cline{3-3}
      %--------------------------------------------------------------------
        10 &
        100 &
        1000% make-rowspan-placeholders
     \tabularnewline\cline{1-1}\cline{2-2}\cline{3-3}
      %--------------------------------------------------------------------
    \end{xtabular}
      \end{center}
    \begin{center}{\small\bfseries Table 6.1}: Some perfect squares and perfect cubes\end{center}
    \begin{caption}{\small\bfseries Table 6.1}: Some perfect squares and perfect cubes\end{caption}
\end{table}
    \par
      \label{m38347*id259412}When given the surd $\sqrt[3]{52}$ you should be able to tell that it lies somewhere between 3 and 4, because $\sqrt[3]{27}=3$ and $\sqrt[3]{64}=4$ and 52 is between 27 and 64. In fact $\sqrt[3]{52}=3,73...$ which is indeed between 3 and 4.\par 
\label{m38347*secfhsst!!!underscore!!!id162}\vspace{.5cm} 
      \noindent
      \hspace*{-30pt}\includegraphics[width=0.5in]{col11306.imgs/pspencil2.png}   \raisebox{25mm}{   
      \begin{mdframed}[linewidth=4, leftmargin=40, rightmargin=40]  
      \begin{exercise}
    \noindent\textbf{Exercise 6.1:  Estimating Surds }
      \label{m38347*probfhsst!!!underscore!!!id163}
      \label{m38347*id259741}Find the two consecutive integers such that $\sqrt{26}$ lies between them.\par 
      \label{m38347*id259757}(Remember that consecutive numbers are two numbers one after the other, like 5 and 6 or 8 and 9.) \par 
      \vspace{5pt}
      \label{m38347*solfhsst!!!underscore!!!id167}\noindent\textbf{Solution to Exercise } \label{m38347*listfhsst!!!underscore!!!id167}\begin{enumerate}[noitemsep, label=\textbf{Step} \textbf{\arabic*}. ] 
            \leftskip=20pt\rightskip=\leftskip\item  
      \label{m38347*id259781}This is ${5}^{2}=25$. Therefore $5\lessthan{}\sqrt{26}$.\par 
      \item  
      \label{m38347*id259824}This is ${6}^{2}=36$. Therefore $\sqrt{26}\lessthan{}6$.\par 
      \item  
      \label{m38347*id259866}Our answer is $5\lessthan{}\sqrt{26}\lessthan{}6$. \par 
      \end{enumerate}
    \end{exercise}
    \end{mdframed}
    }
    \noindent
\label{m38347*secfhsst!!!underscore!!!id176}\vspace{.5cm} 
      \noindent
      \hspace*{-30pt}\includegraphics[width=0.5in]{col11306.imgs/pspencil2.png}   \raisebox{25mm}{   
      \begin{mdframed}[linewidth=4, leftmargin=40, rightmargin=40]  
      \begin{exercise}
    \noindent\textbf{Exercise 6.2: Estimating Surds }\label{m38347*probfhsst!!!underscore!!!id177}
      \label{m38347*id259913}$\sqrt[3]{49}$ lies between: \label{m38347*id7432}\begin{enumerate}[noitemsep, label=\textbf{\alph*}. ] 
            \leftskip=20pt\rightskip=\leftskip\item 1 and 2\item 2 and 3\item 3 and 4\item 4 and 5\end{enumerate}
        \par 
      \vspace{5pt}
      \label{m38347*solfhsst!!!underscore!!!id193}\noindent\textbf{Solution to Exercise } \label{m38347*listfhsst!!!underscore!!!id193}\begin{enumerate}[noitemsep, label=\textbf{Step} \textbf{\arabic*}. ] 
            \leftskip=20pt\rightskip=\leftskip\item  
      \label{m38347*id259980}If $1\lessthan{}\sqrt[3]{49}\lessthan{}2$ then cubing all terms gives $1\lessthan{}49\lessthan{}{2}^{3}$. Simplifying gives $1\lessthan{}49\lessthan{}8$ which is false. So $\sqrt[3]{49}$ does not lie between 1 and 2.\par 
      \item  
      \label{m38347*id260068}If $2\lessthan{}\sqrt[3]{49}\lessthan{}3$ then cubing all terms gives ${2}^{3}\lessthan{}49\lessthan{}{3}^{3}$. Simplifying gives $8\lessthan{}49\lessthan{}27$ which is false. So $\sqrt[3]{49}$ does not lie between 2 and 3.\par 
      \item  
      \label{m38347*id260160}If $3\lessthan{}\sqrt[3]{49}\lessthan{}4$ then cubing all terms gives ${3}^{3}\lessthan{}49\lessthan{}{4}^{3}$. Simplifying gives $27\lessthan{}49\lessthan{}64$ which is true. So $\sqrt[3]{49}$ lies between 3 and 4.
 \par 
      \end{enumerate}
    \end{exercise}
    \end{mdframed}
    }
    \noindent
    \subsection{ Summary}
            \nopagebreak
            \label{m38347*eip-194} $ \hspace{-5pt}\begin{array}{cccccccccccc}   \end{array} $ \hspace{2 pt}\raisebox{-0.2em}{\includegraphics[height=1em]{../icons/www.pdf}} {(subsection shortcode: MG10053 )} \par \label{m38347*eip-50}\begin{itemize}[noitemsep]
            \item If the ${n}^{\mathrm{th}}$ root of a number cannot be simplified to a rational number, we call it a $\mathit{surd}$\item If $a$ and $b$ are positive whole numbers, and $a\lessthan{}b$, then $\sqrt[n]{a}\lessthan{}\sqrt[n]{b}$\item Surds can be estimated by finding the largest perfect square (or perfect cube) that is less than the surd and the smallest perfect square (or perfect cube) that is greater than the surd. The surd lies between these two numbers.\end{itemize}
        \subsection{ End of Chapter Exercises}
            \nopagebreak
            \label{m38347*cid4} $ \hspace{-5pt}\begin{array}{cccccccccccc}   \end{array} $ \hspace{2 pt}\raisebox{-0.2em}{\includegraphics[height=1em]{../icons/www.pdf}} {(subsection shortcode: MG10054 )} \par \label{m38347*id260269}\begin{enumerate}[noitemsep, label=\textbf{\arabic*}. ] 
            \item Answer the following multiple choice questions:
            \label{m38347*id7221}\begin{enumerate}[noitemsep, label=\textbf{\alph*}. ] 
            \item $\sqrt{5}$ lies between:
\label{m38347*id7241}\begin{enumerate}[noitemsep, label=\textbf{\roman*}. ] 
            \item 1 and 2\item 2 and 3\item 3 and 4\item 4 and 5\end{enumerate}
        \item 
              $\sqrt{10}$ lies between:
\label{m38347*id72245}\begin{enumerate}[noitemsep, label=\textbf{\roman*}. ] 
            \item  1 and 2\item  2 and 3\item  3 and 4\item  4 and 5\end{enumerate}
        \item 
              $\sqrt{20}$ lies between:
\label{m38347*id72345}\begin{enumerate}[noitemsep, label=\textbf{\roman*}. ] 
            \item 2 and 3\item 3 and 4\item 4 and 5\item 5 and 6\end{enumerate}
                      \item $\sqrt{30}$ lies between:
\label{m38347*id722643}\begin{enumerate}[noitemsep, label=\textbf{\roman*}. ] 
            \item 3 and 4\item 4 and 5\item 5 and 6\item 6 and 7\end{enumerate}
                      \item $\sqrt[3]{5}$ lies between:
\label{m38347*id7351}\begin{enumerate}[noitemsep, label=\textbf{\roman*}. ] 
            \item  1 and 2\item  2 and 3\item  3 and 4\item  4 and 5\end{enumerate}
                     \item $\sqrt[3]{10}$ lies between:
\label{m38347*id76451}\begin{enumerate}[noitemsep, label=\textbf{\roman*}. ] 
            \item  1 and 2\item  2 and 3\item  3 and 4\item  4 and 5\end{enumerate}
                      \item $\sqrt[3]{20}$ lies between:
\label{m38347*id7334}\begin{enumerate}[noitemsep, label=\textbf{\roman*}. ] 
            \item 2 and 3\item 3 and 4\item 4 and 5\item 5 and 6\end{enumerate}
                      \item $\sqrt[3]{30}$ lies between:
\label{m38347*id73224}\begin{enumerate}[noitemsep, label=\textbf{\roman*}. ] 
            \item 3 and 4\item 4 and 5\item 5 and 6\item 6 and 7\end{enumerate}
                      \end{enumerate}
        \item  Find two consecutive integers such that $\sqrt{7}$ lies between them.          \item  Find two consecutive integers such that $\sqrt{15}$ lies between them.          \end{enumerate}
  \label{m38347**end}
\par \raisebox{-0.2em}{\includegraphics[height=1em]{../icons/www.pdf}} Find the answers with the shortcodes:
 \par \begin{tabular}[h]{cccccc}
 (1.) lqr  &  (2.) lqY  &  (3.) lqg  &  (4.) lq4  &  (5.) lq2  &  (6.) lqT  &  (7.) lqb  &  (8.) ll5  &  (9.) lqW  &  (10.) lq1  & \end{tabular}
