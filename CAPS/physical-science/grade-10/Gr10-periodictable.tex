         \chapter{The periodic table}\fancyfoot[LO,RE]{Chemistry: Matter and materials}
%     \setcounter{figure}{1}
%     \setcounter{subfigure}{1}
 \label{m38760*cid9}
            \section{The arrangement of the elements}
            \nopagebreak
%            \label{m38760} $ \hspace{-5pt}\begin{array}{cccccccccccc}   \includegraphics[width=0.75cm]{col11305.imgs/summary_video.png} &   \end{array} $ \hspace{2 pt}\raisebox{-5 pt}{} {(section shortcode: P10025 )} \par 
      \label{m38760*id261491}The \textbf{periodic table of the elements} is a method of showing the chemical elements in a table with the elements arranged in order of increasing atomic number. Most of the work that was done to arrive at the periodic table that we know can be attributed to a Russian chemist named \textbf{Dmitri Mendeleev}. Mendeleev designed the table in 1869 in such a way that recurring ("periodic") trends or patterns in the properties of the elements could be shown. Using the trends he observed, he left gaps for those elements that he thought were ``missing''. He also predicted the properties that he thought the missing elements would have when they were discovered. Many of these elements were indeed discovered and Mendeleev's predictions were proved to be correct.\\
\chapterstartvideo{VParg}
      \label{m38760*id261511}To show the recurring properties that he had observed, Mendeleev began new rows in his table so that elements with similar properties were in the same vertical columns, called \textbf{groups}. Each row was referred to as a \textbf{period}. Figure~\ref{fig:atom:periodic} shows a simplified version of the periodic table. The full periodic table is reproduced at the front of this book. You can view an online periodic table at \textsl{http://periodictable.com/}. 
%     \setcounter{subfigure}{0}
	\begin{figure}[H] % horizontal\label{m38760*uid133}
 \begin{center}
\begin{pspicture}(-7.8,-1)(6.8,4)
%\psgrid[gridcolor=gray]
\psset{unit=0.75}
\pspolygon[fillstyle=solid,fillcolor=green!50!blue](-9,3)(-9,4)(-8,4)(-8,3)(-9,3)
\pspolygon[fillstyle=solid,fillcolor=lightgray](-9,0)(-9,3)(-8,3)(-7,3)(-7,1)(3,1)(3,2)(4,2)(4,0)(-9,0)
\pspolygon[fillstyle=solid,fillcolor=cyan](3,2)(3,3)(4,3)(4,2)(3,2)
\pspolygon[fillstyle=solid,fillcolor=cyan](4,0)(4,2)(5,2)(5,1)(6,1)(6,0)(4,0)
\pspolygon[fillstyle=solid,fillcolor=green!50!blue](5,2)(4,2)(4,3)(8,3)(8,4)(9,4)(9,0)(6,0)(6,1)(5,1)(5,2)
\psline(-8,3)(-8,0)
\psline(-7,1)(-7,0)
\psline(-6,1)(-6,0)
\psline(-5,1)(-5,0)
\psline(-4,1)(-4,0)
\psline(-3,1)(-3,0)
\psline(-2,1)(-2,0)
\psline(-1,1)(-1,0)
\psline(0,1)(0,0)
\psline(1,1)(1,0)
\psline(2,1)(2,0)
\psline(3,1)(3,0)
\psline(5,3)(5,2)
\psline(5,1)(5,0)
\psline(6,3)(6,1)
\psline(7,3)(7,0)
\psline(8,3)(8,0)
\psline(-9,3)(-8,3)
\psline(-9,2)(-7,2)
\psline(-9,1)(9,1)
\psline(5,2)(9,2)
\psline(8,3)(9,3)
\rput(-8.5,3.5){\textbf{H}}
\rput(-8.5,2.5){\textbf{Li}}
\rput(-8.5,1.5){\textbf{Na}}
\rput(-8.5,0.5){\textbf{K}}
\rput(-7.5,2.5){\textbf{Be}}
\rput(-7.5,1.5){\textbf{Mg}}
\rput(-7.5,0.5){\textbf{Ca}}
\rput(-6.5,0.5){\textbf{Sc}}
\rput(-5.5,0.5){\textbf{Ti}}
\rput(-4.5,0.5){\textbf{V}}
\rput(-3.5,0.5){\textbf{Cr}}
\rput(-2.5,0.5){\textbf{Mn}}
\rput(-1.5,0.5){\textbf{Fe}}
\rput(-0.5,0.5){\textbf{Co}}
\rput(0.5,0.5){\textbf{Ni}}
\rput(1.5,0.5){\textbf{Cu}}
\rput(2.5,0.5){\textbf{Zn}}
\rput(3.5,2.5){\textbf{B}}
\rput(3.5,1.5){\textbf{Al}}
\rput(3.5,0.5){\textbf{Ga}}
\rput(4.5,2.5){\textbf{C}}
\rput(4.5,1.5){\textbf{Si}}
\rput(4.5,0.5){\textbf{Ge}}
\rput(5.5,2.5){\textbf{N}}
\rput(5.5,1.5){\textbf{P}}
\rput(5.5,0.5){\textbf{As}}
\rput(6.5,2.5){\textbf{O}}
\rput(6.5,1.5){\textbf{S}}
\rput(6.5,0.5){\textbf{Se}}
\rput(7.5,2.5){\textbf{F}}
\rput(7.5,1.5){\textbf{Cl}}
\rput(7.5,0.5){\textbf{Br}}
\rput(8.5,3.5){\textbf{He}}
\rput(8.5,2.5){\textbf{Ne}}
\rput(8.5,1.5){\textbf{Ar}}
\rput(8.5,0.5){\textbf{Kr}}
\psline[linewidth=0.1,arrows=<->](-9.5,4)(-9.5,0)
\pcline[linestyle=none](-9.5,0)(-9.5,4)
\aput{:U}{Period}
\rput(-8.5,5){group number}
\rput(-8.5,4.6){1}
\rput(-7.5,3.5){2}
\rput(3.5,3.5){13}
\rput(4.5,3.5){14}
\rput(5.5,3.5){15}
\rput(6.5,3.5){16}
\rput(7.5,3.5){17}
\rput(8.5,4.5){18}
\psline[linewidth=0.1,arrows=->](-7,-0.5)(5,-0.5)
\pcline[linestyle=none](-7,-0.5)(5,-0.5)
\bput{:U}{Group}
\end{pspicture}
\end{center}
\caption{A simplified diagram showing part of the periodic table. Metals are given in gray, metalloids in light blue and non-metals in turquoise.}
\label{fig:atom:periodic}
 \end{figure}       
            
\subsection*{Definitions and important concepts}
Before we can talk about the trends in the periodic table, we first need to define some terms that are used:
\begin{itemize}[noitemsep]
\item \textbf{Atomic radius} \\
The atomic radius is a measure of the size of an atom. 
\item \textbf{Ionisation energy}\\
The first ionisation energy is the energy needed to remove one electron from an atom in the gas phase. The ionisation energy is different for each element. We can also define second, third, fourth, etc.\@{} ionisation energies. These are the energies needed to remove the second, third, or fourth electron respectively. 
\item \textbf{Electron affinity}\\
Electron affinity can be thought of as how much an element wants electrons.
\item \textbf{Electronegativity}\\
Electronegativity is the tendency of atoms to attract electrons. The electronegativity of the elements starts from about 0.7 (Francium ($\text{Fr}$)) and goes up to 4 (Fluorine ($\text{F}$))
\item A \textbf{group} is a vertical column in the periodic table and is considered to be the most important way of classifying the elements. If you look at a periodic table, you will see the groups numbered at the top of each column. The groups are numbered from left to right starting with 1 and ending with 18. This is the convention that we will use in this book. On some periodic tables you may see that the groups are numbered from left to right  as follows: 1, 2, then an open space which contains the \textbf{transition elements}, followed by groups 3 to 8. Another way to label the groups is using Roman numerals.
\item A \textbf{period} is a horizontal row in the periodic table of the elements. The periods are labelled from top to bottom, starting with 1 and ending with 7.
\end{itemize}
For each element on the periodic table we can give its period number and its group number. For example, $\text{B}$ is in period 2 and group 13. We can also determine the electronic structure of an element from its position on the periodic table. In chapter~\ref{chap:atom} you worked out the electronic configuration of various elements. Using the periodic table we can easily give the electronic configurations of any element. To see how this works look at the following: \\
\begin{figure}[H]
 \begin{center}
\scalebox{0.6}{
  \begin{pspicture}(0,0)(10,10)
{
   \pspolygon[fillstyle=solid,fillcolor=cyan](0.0)(0,7)(1,7)(1,6)(2,6)(2,0)(0,0)
   \pspolygon[fillstyle=solid,fillcolor=white](2,0)(2,4)(12,4)(12,0)(2,0)
   \pspolygon[fillstyle=solid,fillcolor=teal](12,0)(12,6)(17,6)(17,7)(18,7)(18,0)(12,0)
   \rput(1,3.5){\Large{\textbf{s-block}}}
   \rput(5.5,2){\Large{\textbf{d-block}}}
   \rput(13.5,3.5){\Large{\textbf{p-block}}}
\rput(0,8){\Large{group number}}
\rput(0.5,7.5){\Large{1}}
\rput(1.5,6.5){\Large{2}}
\rput(12.5,6.5){\Large{13}}
\rput(13.5,6.5){\Large{14}}
\rput(14.5,6.5){\Large{15}}
\rput(15.5,6.5){\Large{16}}
\rput(16.5,6.5){\Large{17}}
\rput(17.5,7.5){\Large{18}}
}
  \end{pspicture}
}
 \end{center}
\end{figure}
We also note that the period number gives the energy level that is being filled. For example, phosphorus ($\text{P}$) is in the third period and group 15. Looking at the figure above, we see that the p-orbital is being filled. Also the third energy level is being filled. So its electron configuration is: $\text{[Ne]}3s^{2}3p^{3}$. (Phosphorus is in the third group in the p-block, so it must have 3 electrons in the p shell.) 
      \label{m38760*uid146}
            \subsection*{Periods in the periodic table}
            \nopagebreak

            \label{m38760*id261855} The following diagram illustrates some of the key trends in the periods: \\
\begin{figure}[H]

\begin{center}
\scalebox{0.7}{
\begin{pspicture}(0,0)(10,10)
%\psgrid[gridcolor=gray]
\rput(5,5){
\psset{unit=0.75}
\pspolygon(0,0)(0,7)(1,7)(1,6)(2,6)(2,4)(12,4)(12,6)(17,6)(17,7)(18,7)(18,0)(0,0)
\rput(1,8){\Large{period number}}
\rput(-1,6.5){\Large{1}}
\rput(-1,5.5){\Large{2}}
\rput(-1,4.5){\Large{3}}
\rput(-1,3.5){\Large{4}}
\rput(-1,2.5){\Large{5}}
\rput(-1,1.5){\Large{6}}
\rput(-1,0.5){\Large{7}}
\psline[linewidth=0.08,linecolor=red,arrowinset=0]{<-}(0.2,0.5)(17.8,0.5)
\rput(8.8,1.1){\Large{Atomic radius}}
\psline[linewidth=0.08,linecolor=blue,arrowinset=0]{->}(0.2,1.8)(17.8,1.8)
\rput(8.8,2.3){\Large{Ionization energy}}
\psline[linewidth=0.08,linecolor=blue,arrowinset=0]{->}(0.2,3.1)(17.8,3.1)
\rput(8.8,3.5){\Large{Electronegativity}}
}
\end{pspicture}
}
\end{center}

\caption{Trends on the periodic table.}
\label{fig:atom:periodic2}
 \end{figure} 
\mindsetvid{Ionization energy}{VParr}
 Table~\ref{tab:period3trends} summarises the patterns or trends in the properties of the elements in period 3. Similar trends are observed in the other periods of the periodic table. The chlorides are compounds with chlorine and the oxides are compounds with oxygen. \\
\begin{table}[H]
 \begin{center}
  \begin{tabular}{|p{2cm}|p{1cm}|p{1cm}|p{1.4cm}|p{1.2cm}|p{1.4cm}|p{1.4cm}|p{1.4cm}|} \hline
\textbf{Element} & $^{23}_{11}\text{Na}$ & $^{24}_{12}\text{Mg}$ &  $^{27}_{13}\text{Al}$ & $^{28}_{14}\text{Si}$ &  $^{31}_{15}\text{P}$ & $^{32}_{16}\text{S}$ & $^{35}_{17}\text{Cl}$ \\ \hline
   \textbf{Chlorides} & $\text{NaCl}$ & $\text{MgCl}_2$ & $\text{AlCl}_{3}$ & $\text{SiCl}_{4}$ & $\text{PCl}_{5}$ or $\text{PCl}_{3}$ & $\text{S}_{2}\text{Cl}_{2}$ & no chlorides  \\ \hline
\textbf{Oxides} & $\text{Na}_{2}\text{O}$ & $\text{MgO}$ & $\text{Al}_{2}\text{O}_{3}$ & $\text{SiO}_{2}$ & $\text{P}_{4}\text{O}_{6}$ or $\text{P}_{4}\text{O}_{10}$ & $\text{SO}_{3}$ or $\text{SO}_{4}$ & $\text{Cl}_{2}\text{O}_{7}$ or $\text{Cl}_{2}\text{O}$ \\ \hline
\textbf{Valence electrons} & $3s^{1}$ & $3s^{2}$ & $3s^{2}3p^{1}$ & $3s^{2}3p^{2}$ & $3s^{2}3p^{3}$ & $3s^{2}3p^{4}$ & $3s^{2}3p^{5}$  \\ \hline
\textbf{Atomic radius} & \multicolumn{7}{p{8cm}|}{Decreases across a period.} \\ \hline
\textbf{First Ionization energy} & \multicolumn{7}{p{8cm}|}{The general trend is an increase across the period.} \\ \hline
\textbf{Electro-negativity} & \multicolumn{7}{p{8cm}|}{Increases across the period.} \\ \hline
\textbf{Melting and boiling point} & \multicolumn{7}{p{8cm}|}{Increases to silicon and then decreases to argon.} \\ \hline
\textbf{Electrical conductivity} & \multicolumn{7}{p{10cm}|}{Increases from sodium to aluminium. Silicon is a semi-conductor. The rest are insulators.} \\ \hline
  \end{tabular}
\caption{Summary of the trends in period 3}
\label{tab:period3trends}
 \end{center}

\end{table}
Note that we have left argon ($^{40}_{18}\text{Ar}$) out. Argon is a noble gas with electron configuration: $\text{[Ne]}3s^{2}3p^{6}$. Argon does not form any compounds with oxygen or chlorine. 
\begin{exercises}{Periods on the periodic table} \noindent
\begin{enumerate}[noitemsep, label=\textbf{\arabic*}. ]
%Question 1
\item Use Table~\ref{tab:period3trends} and Figure~\ref{fig:atom:periodic2} to help you produce a similar table for the elements in period 2.
%Question 2
\item Refer to the data table below which gives the ionisation energy (in $\text{kJ} \cdot \text{mol}^{-1}$) and atomic number (Z) for a number of elements in the periodic table:\\
\begin{center}
\begin{tabular}{|l|l|c|l|l|c|}\hline
\textbf{Z} & \textbf{Name of element} & \textbf{Ionization energy} & \textbf{Z} & \textbf{Name of element} & \textbf{Ionization energy} \\\hline
$1$        &                 & $1310$              & $10$         &                 & $2072$              \\\hline
$2$        &                 & $2360$              & $11$         &                 & $494$               \\\hline
$3$        &                 & $517$               & $12$         &                 & $734$               \\\hline
$4$        &                 & $895$               & $13$         &                 & $575$               \\\hline
$5$        &                 & $797$               & $14$         &                 & $783$               \\\hline
$6$        &                 & $1087$              & $15$         &                 & $1051$              \\\hline
$7$        &                 & $1397$              & $16$         &                 & $994$               \\\hline
$8$        &                 & $1307$              & $17$         &                 & $1250$              \\\hline
$9$        &                 & $1673$              & $18$         &                 & $1540$              \\\hline
\end{tabular}
\end{center}

\begin{enumerate}[noitemsep, label=\textbf{\alph*}. ]
 \item Fill in the names of the elements.
\item Draw a line graph to show the relationship between atomic number (on the x-axis) and ionisation energy (y-axis).
\item Describe any trends that you observe.
\item Explain why:
	\begin{enumerate}[noitemsep, label=\textbf{\roman*}. ]
	\item the ionisation energy for $Z=2$ is higher than for $Z=1$
	\item the ionisation energy for $Z=3$ is lower than for $Z=2$
	\item the ionisation energy increases between $Z=5$ and $Z=7$
	\end{enumerate}

\end{enumerate}
\end{enumerate}
      \label{m38757*secfhsst!!!underscore!!!id936}
\practiceinfo
\par 
 \par \begin{tabular}[h]{cccccc}
 (1.) 001p  & (2.) 001q \end{tabular}
\end{exercises}
\section{Chemical properties of the groups}
 \label{m38760*secfhsst!!!underscore!!!id1062}
            \nopagebreak
            \label{m38760*id261554} In some groups, the elements display very similar chemical properties and some of the groups are even given special names to identify them. The characteristics of each group are mostly determined by the electron configuration of the atoms of the elements in the group. The names of the groups are summarised in Figure~\ref{fig:atom:periodic}\par
\begin{figure}[H]

\begin{center}
\scalebox{0.7}{
\begin{pspicture}(-7.8,-1)(6.8,4)
%\psgrid[gridcolor=gray]
\psset{unit=0.75}
%alkali metals
\psline(-9,0)(-9,7)(-8,7)(-8,0)(-9,0)
\uput[90]{90}(-8.5,1){\Large{Alkali metals}}
%alkali earth metals
\psline(-8,0)(-8,6)(-7,6)(-7,0)(-8,0)
\uput[90]{90}(-7.5,0.1){\Large{Alkali earth metals}}
%transition metals
\psline(-7,0)(-7,4)(3,4)(3,0)(-7,0)
\uput[0](-5,2){\Large{Transition metals}}
%group 13
\psline(3,0)(3,6)(4,6)(4,0)(3,0)
\uput[90]{90}(3.5,1){\Large{Group 13}}
%group 14
\psline(4,0)(4,6)(5,6)(5,0)(4,0)
\uput[90]{90}(4.5,1){\Large{Group 14}}
%pnictogens (group 15)
\psline(5,0)(5,6)(6,6)(6,0)(5,0)
\uput[90]{90}(5.5,1){\Large{Group 15}}
%chalcogens (group 16)
\psline(6,0)(6,6)(7,6)(7,0)(6,0)
\uput[90]{90}(6.5,1){\Large{Group 16}}
%halogens
\psline(7,0)(7,6)(8,6)(8,0)(7,0)
\uput[90]{90}(7.5,1){\Large{Halogens}}
%noble gases
\psline(8,0)(8,7)(9,7)(9,0)(8,0)
\uput[90]{90}(8.5,1){\Large{Noble gases}}
\end{pspicture}
}
\end{center}

\caption{Groups on the periodic table}
\label{fig:atom:periodic}
\end{figure}  
A few points to note about the groups are:
        \label{m38757*id261581}\begin{itemize}[noitemsep]
            \label{m38757*uid135}\item Although hydrogen appears in group 1, it is not an alkali metal.
\item Group 15 elements are sometimes called the pnictogens.
\label{m38757*id6232}\item Group 16 elements are sometimes known as the chalcogens.
\label{m38757*uid142}\item The \textbf{halogens} and the \textbf{alkali earth metals} are very reactive groups.
\label{m38757*uid143}\item The \textbf{noble gases} are \textsl{inert} (unreactive).   
\end{itemize}            
\mindsetvid{Groups in the periodic table}{VPasj}
        \label{m38760*id261833}The following diagram illustrates some of the key trends in the groups of the periodic table: \\
\begin{figure}[H]

\begin{center}
\scalebox{0.7}{
\begin{pspicture}(0,0)(20,20)
%\psgrid[gridcolor=gray]
\rput(5,5){
\psset{unit=0.75}
\pspolygon(0,0)(0,7)(1,7)(1,6)(2,6)(2,4)(12,4)(12,6)(17,6)(17,7)(18,7)(18,0)(0,0)
\rput(0,8){\Large{group number}}
\rput(0.5,7.5){\Large{1}}
\rput(1.5,6.5){\Large{2}}
\rput(12.5,6.5){\Large{13}}
\rput(13.5,6.5){\Large{14}}
\rput(14.5,6.5){\Large{15}}
\rput(15.5,6.5){\Large{16}}
\rput(16.5,6.5){\Large{17}}
\rput(17.5,7.5){\Large{18}}
\psline[linewidth=0.1,arrowinset=0,linecolor=red]{->}(0.8,6.8)(0.8,0.2)
\uput[0](0.8,3){\Large{Atomic}}
\uput[0](0.8,2.3){\Large{radius}}
\psline[linewidth=0.1,arrowinset=0,linecolor=blue]{<-}(4.2,6.8)(4.2,0.2)
\uput[0](4.2,3){\Large{Ionization}}
\uput[0](4.2,2.3){\Large{energy}}
\psline[linewidth=0.1,arrowinset=0,linecolor=blue]{<-}(7.5,6.8)(7.5,0.2)
\uput[0](7.5,3){\Large{Electro-}}
\uput[0](7.5,2.3){\Large{negativity}}
\psline[linewidth=0.1,arrowinset=0,linecolor=blue]{<-}(10.9,6.8)(10.9,0.2)
\uput[0](10.9,3){\Large{Melting and}}
\uput[0](10.9,2.3){\Large{boiling point}}
\psline[linewidth=0.1,arrowinset=0,linecolor=red]{->}(15,6.8)(15,0.2)
\uput[0](15,3){\Large{Density}}
}
\end{pspicture}
}
\end{center}
\caption{Trends in the groups on the periodic table.}
\label{fig:atom:periodic1}
 \end{figure} 
Table~\ref{tab:group1trends} summarises the patterns or trends in the properties of the elements in group 1. Similar trends are observed for the elements in the other groups of the periodic table. 
We can use the information in \ref{tab:group1trends} to predict the chemical properties of unfamiliar elements. For example, given the element Francium ($\text{Fr}$) we can say that its electronic structure will be $\text{[Rn]}7s^{1}$, it will have a lower first ionisation energy than caesium ($\text{Cs}$) and its melting and boiling point will also be lower than caesium. \\
You should also recall from chapter~\ref{chap:classification} that the metals are found on the left of the periodic table, non-metals are on the right and metalloids are found on the zig-zag line that starts at boron.
\begin{table}[H]
 \begin{center}
  \begin{tabular}{|l|p{1cm}|p{1cm}|p{1cm}|p{1cm}|p{1cm}|} \hline
\textbf{Element} & $^{7}_{3}\text{Li}$ & $^{7}_{3}\text{Na}$ &  $^{7}_{3}\text{K}$ & $^{7}_{3}\text{Rb}$ &  $^{7}_{3}\text{Cs}$ \\ \hline
\textbf{Electron structure} & $\text{[He]}2s^{1}$ & $\text{[Ne]}3s^{1}$ & $\text{[Ar]}4s^{1}$ & $\text{[Kr]}4s^{1}$ & $\text{[Xe]}5s^{1}$ \\ \hline
   \multirow{2}{*}{\textbf{Group 1 chlorides}} & $\text{LiCl}$ & $\text{NaCl}$ & $\text{KCl}$ & $\text{RbCl}$ & $\text{CsCl}$ \\ \cline{2-6}
   & \multicolumn{5}{p{6cm}|}{Group 1 elements all form halogen compounds in a 1:1 ratio} \\ \hline
\multirow{2}{*}{\textbf{Group 1 oxides}} & $\text{Li}_{2}\text{O}$ & $\text{Na}_{2}\text{O}$ & $\text{K}_{2}\text{O}$ & $\text{Rb}_{2}\text{O}$ & $\text{Cs}_{2}\text{O}$\\ \cline{2-6}
   & \multicolumn{5}{l|}{Group 1 elements all form oxides in a 2:1 ratio} \\ \hline
\textbf{Atomic radius} & \multicolumn{5}{p{6cm}|}{Increases as you move down the group.} \\ \hline
\textbf{First ionisation energy} & \multicolumn{5}{p{6cm}|}{Decreases as you move down the group.} \\ \hline
\textbf{Electronegativity} & \multicolumn{5}{p{6cm}|}{Decreases as you move down the group.} \\ \hline
\textbf{Melting and boiling point} & \multicolumn{5}{p{6cm}|}{Decreases as you move down the group.} \\ \hline
\textbf{Density} & \multicolumn{5}{p{6cm}|}{Increases as you move down the group.} \\ \hline
  \end{tabular}
\caption{Summary of the trends in group 1}
\label{tab:group1trends}
 \end{center}

\end{table}


\begin{exercises}{Groups in the periodic table}
            \nopagebreak \noindent
\begin{enumerate}[noitemsep, label=\textbf{\arabic*}. ]
%Question 1
 \item Use Table~\ref{tab:group1trends} and Figure~\ref{fig:atom:periodic1} to help you produce similar tables for group 2 and group 17.
%Question 2 
\item The following two elements are given. Compare these elements in terms of the following properties. Explain the differences in each case.
$^{24}_{12}\text{Mg}$ and $^{40}_{20}\text{Ca}$. 
\begin{enumerate}[noitemsep, label=\textbf{\alph*}. ]
 \item Size of the atom (atomic radius)
\item Electronegativity
\item First ionisation energy
\item Boiling point
\end{enumerate}
%Question 3
 \item Study the following graph and explain the trend in electronegativity of the group 2 elements.\\
\begin{pspicture}(-2.25,-2)(29,12)
  \psaxes[axesstyle=axes,Dx=1,Dy=0.5,labels=y,ticks=y]{-}(5,1.5)
  \listplot[plotstyle=bar,barwidth=0.5cm]{0.5 1.5
1.5 1.3
2.5 1
3.5 1
4.5 0.9}
\rput{0}(0.5,-0.2){$\text{Be}$}
\rput{0}(1.5,-0.2){$\text{Mg}$}
\rput{0}(2.5,-0.2){$\text{Ca}$}
\rput{0}(3.5,-0.2){$\text{Sr}$}
\rput{0}(4.5,-0.2){$\text{Ba}$}
\end{pspicture}
 
%Question 4
\item            \label{m38760*id262476}Refer to the elements listed below: \label{m38760*id7632}
\begin{multicols}{2}
\begin{itemize}[noitemsep]
            \item Lithium ($\mathrm{Li}$)
\item Chlorine ($\mathrm{Cl}$)
\item Magnesium ($\mathrm{Mg}$)
\item Neon ($\mathrm{Ne}$)
\item Oxygen ($\mathrm{O}$)
\item Calcium ($\mathrm{Ca}$)
\item Carbon ($\mathrm{C}$)
\end{itemize}
\end{multicols}
         Which of the elements listed above:
% \begin{multicols}{2}
        \label{m38760*id262499}\begin{enumerate}[noitemsep, label=\textbf{\alph*}. ] 
            \label{m38760*uid158}\item belongs to Group $1$
\label{m38760*uid159}\item is a halogen
\label{m38760*uid160}\item is a noble gas
\label{m38760*uid161}\item is an alkali metal
\label{m38760*uid162}\item has an atomic number of $12$
\label{m38760*uid163}\item has four neutrons in the nucleus of its atoms
\label{m38760*uid164}\item contains electrons in the 4th energy level
\label{m38760*uid166}\item has all its energy orbitals full
\label{m38760*uid167}\item will have chemical properties that are most similar
\end{enumerate}
% \end{multicols}
\end{enumerate}
         
\label{m38760**end}
\practiceinfo
\par 
 \par \begin{tabular}[h]{cccccc}
 (1.) 001r  & (2.) 001s & (3.) 001t & (4.) 001u \end{tabular}

\end{exercises}
%         \par 
%         \label{m38757*eip-6}
%     \setcounter{subfigure}{0}
% 	\begin{figure}[H] % horizontal\label{m38757*periodictable-3}
%     \textnormal{Khan academy video on periodic table - 2} \nopagebreak
%   \label{m38757*yt-media3}\label{m38757*yt-video3}
%             \raisebox{-5 pt}{ \includegraphics[width=0.5cm]{col11305.imgs/summary_www.png}} { (Video:  P10028 )}
%  \end{figure}       \par 
% \label{m38760*eip-148}
%     \setcounter{subfigure}{0}
% 	\begin{figure}[H] % horizontal\label{m38760*periodictable-1}
%     \textnormal{Khan academy video on the periodic table - 1} \nopagebreak
%   \label{m38760*yt-media1}\label{m38760*yt-video1}
%             \raisebox{-5 pt}{ \includegraphics[width=0.5cm]{col11305.imgs/summary_www.png}} { (Video:  P10026 )}
%  \end{figure}       \par 
% The following presentation provides a summary of the periodic table
%     \setcounter{subfigure}{0}
% 	\begin{figure}[H] % horizontal\label{m38757*slidesharefigure}
%     \label{m38757*slidesharemedia}\label{m38757*slideshareflash}\raisebox{-5 pt}{ \includegraphics[width=0.5cm]{col11305.imgs/summary_www.png}} { (Presentation:  P10029 )}
%  \end{figure}       \par 
%     \label{m38757*eip-572}

\begin{activity}{Inventing your own periodic table}
            \nopagebreak
            \label{m38760*eip-603}
\begin{minipage}{.5\textwidth}
You are the official chemist for the planet Zog. You have discovered all the same elements that we have here on Earth, but you don't have a periodic table. The citizens of Zog want to know how all these elements relate to each other. How would you invent the periodic table? Think about how you would organise the data that you have and what properties you would include. Do not simply copy Mendeleev's ideas, be creative and come up with some of your own. Research other forms of the periodic table and make one that makes sense to you. Present your ideas to your class. 
\end{minipage}
\begin{minipage}{.5\textwidth}
\begin{center}
\textbf{Circular periodic table}\\
\includegraphics[width=.9\textwidth]{photos/Circular_periodic_table.png}\\
\textsl{Image from Wikimedia commons}
\end{center}
\end{minipage}

\end{activity}
\summary{VPddg}
            \nopagebreak
            \label{m38757*uid0123}\begin{itemize}[noitemsep]
\item Elements are arranged in periods and groups on the periodic table. The elements are arranged according to increasing atomic number. 
\item A \textbf{group} is a column on the periodic table containing elements with similar properties. A \textbf{period} is a row on the periodic table.
\item The atomic radius is a measure of the size of the atom.
\item The first ionisation energy is the energy needed to remove one electron from an atom in the gas phase.
\item Electronegativity is the tendency of atoms to attract electrons.
\item Across a period the ionisation energy and electronegativity increase. The atomic radius decreases across a period.
\item The groups on the periodic table are labelled from 1 to 18. Group 1 is known as the alkali metals, group 2 is known as the alkali earth metals, group 17 is known as the halogens and the group 18 is known as the noble gases. The elements in a group have similar properties.
\item The atomic radius and the density both increase down a group. The ionisation energy, electronegativity, and melting and boiling points all decrease down a group.
\end{itemize}
        \label{m38757*eip-219}

\begin{eocexercises}{ Periodic table}
            \nopagebreak \noindent
\label{m38757*uid091221}\begin{enumerate}[noitemsep, label=\textbf{\arabic*}. ] 
%Question 1
\item For the following questions state whether they are true or false. If they are false, correct the statement
  \label{m38757*id073324}\begin{enumerate}[noitemsep, label=\textbf{\alph*}. ] 
  \item The group $1$ elements are sometimes known as the alkali earth metals.
  \item The group $8$ elements are known as the noble gases.
  \item Group $7$ elements are very unreactive.
  \item The transition elements are found between groups $3$ and $4$.\end{enumerate}
%Question 2
\item Give one word or term for each of the following:
   \label{m38757*id0734}\begin{enumerate}[noitemsep, label=\textbf{\alph*}. ] 
   \item The energy that is needed to remove one electron from an atom
   \item A horizontal row on the periodic table
   \item A very reactive group of elements that is missing just one electron from their outer shells.
   \end{enumerate}
%Question 3
\item Given $^{80}_{35}\text{Br}$ and $^{35}_{17}\text{Cl}$. Compare these elements in terms of the following properties. Explain the differences in each case.
  \begin{enumerate}[noitemsep, label=\textbf{\alph*}. ]
   \item Atomic radius
   \item Electronegativity
   \item First ionisation energy
   \item Boiling point
  \end{enumerate}
%Question 4
\item Given the following table:
  \begin{table}[H]
   \begin{center}
    \begin{tabular}{|l|l|l|l|l|l|l|l|l|} \hline
     \textbf{Element} & $\text{Na}$ & $\text{Mg}$ & $\text{Al}$ & $\text{Si}$ & $\text{P}$ & $\text{S}$ & $\text{Cl}$ & $\text{Ar}$ \\ \hline
     \textbf{Atomic number} & 11 & 12 & 13 & 14 & 15 & 16 & 17 & 18 \\ \hline
     \textbf{Density ($g \cdot cm^{-3}$)} & 0,97 & 1,74 & 2,70 & 2,33 & 1,82 & 2,08 & 3,17 & 1,78 \\ \hline
     \textbf{Melting point ($^{\circ} C$)} & 370,9 & 923,0 & 933,5 & 1687 & 317,3 & 388,4 & 171,6 & 83,8 \\ \hline
     \textbf{Boiling point ($^{\circ} C$)} & 1156 & 1363 & 2792 & 3538 & 550 & 717,8 & 239,1 & 87,3 \\ \hline
     \textbf{Electronegativity} & 0.93 & 1.31 & 1.61 & 1.90 & 2.19 & 2.58 & 3.16 & - \\ \hline
    \end{tabular}
   \end{center}
  \end{table}
Draw graphs to show the patterns in the following physical properties:
  \begin{enumerate}[noitemsep, label=\textbf{\alph*}. ]
  \item Density
  \item Boiling point
  \item Melting point
  \item Electronegativity
  \end{enumerate}
%Question 5
\item A graph showing the pattern in first ionisation energy for the elements in period 3 is shown below:\\
\begin{pspicture}(-2.25,-2)(29,12)
  \psaxes[axesstyle=axes,Dx=1,Dy=.5,ticks=none,labels=none]{-}(8,2)
  \listplot[plotstyle=bar,barwidth=0.5cm]{0.5 .514
1.5 .765
2.5 .599
3.5 .816
4.5 1.049
5.5 1.036
6.5 1.297
7.5 1.576}
\rput{0}(0.5,-0.2){$\text{Na}$}
\rput{0}(1.5,-0.2){$\text{Mg}$}
\rput{0}(2.5,-0.2){$\text{Al}$}
\rput{0}(3.5,-0.2){$\text{Si}$}
\rput{0}(4.5,-0.2){$\text{P}$}
\rput{0}(5.5,-0.2){$\text{S}$}
\rput{0}(6.5,-0.2){$\text{Cl}$}
\rput{0}(7.5,-0.2){$\text{Ar}$}
\end{pspicture}
  \begin{enumerate}[noitemsep, label=\textbf{\alph*}. ]
  \item Explain the pattern by referring to the electron configuration
  \item Predict the pattern in the first ionisation energy for the elements in period 2.
  \item Draw a rough graph to show the pattern predicted in the previous question.
  \end{enumerate}
\end{enumerate}
  \label{4e3d8e3d8992782b4e5d6fd958df32f9**end}
\practiceinfo
\par 
 \par \begin{tabular}[h]{cccccc}
 (1.) 001v  &  (2.) 001w  &  (3.) 001x  &  (4.) 001y  & (5.) 001z \end{tabular}
\end{eocexercises}
