         \chapter{Reactions in aqueous solution}\fancyfoot[LO,RE]{Chemistry: Chemical change} \label{chap:rxnsaq}
    \setcounter{figure}{1}
    \setcounter{subfigure}{1}
    \label{4c7ba3bfe702f176850b0d58ba743465}
         \section{Introduction}
    \nopagebreak
%            \label{m38720} $ \hspace{-5pt}\begin{array}{cccccccccccc}   \includegraphics[width=0.75cm]{col11305.imgs/summary_fullmarks.png} &   \end{array} $ \hspace{2 pt}\raisebox{-5 pt}{} {(section shortcode: P10082 )} \par 
\label{m38720*eip-56}
            \label{m38720*eip-869}Many reactions in chemistry and all biological reactions (reactions in living systems) take place in water. We say that these reactions take place in aqueous solution. Water has many unique properties and is plentiful on Earth. For these reasons reactions in aqueous solutions occur frequently. In this chapter we will look at some of these reactions in detail. Almost all the reactions that occur in aqueous solutions involve ions. We will look at three main types of reactions that occur in aqueous solutions, namely precipitation reactions, acid-base reactions and redox reactions. Before we can learn about the types of reactions, we need to first look at ions in aqueous solutions and electrical conductivity.
\chapterstartvideo{VPbls}
\par \label{m38720*cid6}
            \section{Ions in aqueous solution}
            \nopagebreak
      \label{m38720*id335294}Water is seldom pure. Because of the structure of the water molecule, substances can dissolve easily in it. This is very important because if water wasn't able to do this, life would not be possible on Earth. In rivers and the oceans for example, dissolved oxygen means that organisms (such as fish) are able to respire (breathe). For plants, dissolved nutrients are available in a form which they can absorb. In the human body, water is able to carry dissolved substances from one part of the body to another.\par 
      \label{m38720*uid19}
            \subsection*{Dissociation in water}
            \nopagebreak
        \label{m38720*id335324}Water is a \textbf{polar molecule}. If we represent water using Lewis structures we will get the following:
\begin{figure}[H]
\begin{center}
\scalebox{.8}{
\begin{pspicture}(-0.2,-0.4)(2,0.4)
%\psgrid[gridcolor=gray]
\rput(0.1,0){\Large \textbf{$\text{H}$}}
\rput(1,0){\Large \textbf{$\text{O}$}}
\uput{9pt}[d](1,0){$\times$ $\bullet$}
\rput{90}(1,0){\uput{9pt}[d](0,0){$\times$ $\times$}}
\rput{180}(1,0){\uput{9pt}[d](0,0){$\times$ $\times$}}
\rput{270}(1,0){\uput{9pt}[d](0,0){$\times$ $\bullet$}}
\rput(1,-0.8){\Large \textbf{$\text{H}$}}
\end{pspicture}
}
\end{center}
\end{figure}

You will notice that there are two electron pairs that do not take part in bonding. This side of the water molecule has a higher \textsl{electron density} than the other side where the hydrogen atoms are bonded. This side of the water molecule is \textsl{more negative} than the side where the hydrogen atoms are bonded. We say this side is the delta negative ($\delta -$) side and the hydrogen side is the delta positive ($\delta +$) side. This means that one part of the molecule has a slightly \textsl{positive} charge (positive pole) and the other part has a slightly \textsl{negative} charge (negative pole). We say such a molecule is a \textbf{dipole}. It has two poles. Figure~\ref{fig:hydrosphere:water} shows this.\par 
    \setcounter{subfigure}{0}
\begin{figure}[H]
\begin{center}
\scalebox{1}{
\begin{pspicture}(0,0)(10,2.6)
\rput(1,1){{\psset{unit=0.25}\rput{150}{\pscircle[fillcolor=red,fillstyle=solid](0,0){2}
\psarc[fillcolor=white,fillstyle=solid](-1.5,1){1.5}{30}{260}
\psarc[fillcolor=white,fillstyle=solid](1.5,1){1.5}{280}{150}
\rput(-1.5,1){\pscurve(1.5;30)(-1;142.5)(1.5;260)}
\rput(1.5,1){\pscurve(1.5;150)(-1;37.5)(1.5;280)}}}}
\rput(0.5,0.2){$\delta +$}
\rput(1.6,1.5){$\delta -$}
\end{pspicture}
}
\end{center}
\caption{Water is a polar molecule}
\label{fig:hydrosphere:water}
\end{figure}
\mindsetvid{special properties of water}{VPbmr}
\subsection*{Dissociation of sodium chloride in water}       
        \label{m38720*id335349}It is the polar nature of water that allows ionic compounds to dissolve in it. In the case of sodium chloride ($\text{NaCl}$) for example, the positive sodium ions (${\text{Na}}^{+}$) are attracted to the negative pole of the water molecule, while the negative chloride ions (${\text{Cl}}^{-}$) are attracted to the positive pole of the water molecule. When sodium chloride is dissolved in water, the polar water molecules are able to work their way in between the individual ions in the lattice. The water molecules surround the negative chloride ions and positive sodium ions and pull them away into the solution. This process is called \textbf{dissociation}. Note that the positive side of the water molecule will be attracted to the negative chlorine ion and the negative side of the water molecule to the positive sodium ions. A simplified representation of this is shown in Figure~\ref{fig:hydrosphere:ions dissolving}. We say that dissolution of a substance has occurred when a substance dissociates or dissolves. Dissolving is a physical change that takes place. It can be reversed by removing (evaporating) the water.\par 
\label{m38720*fhsst!!!underscore!!!id155}
\Definition{ Dissociation}{Dissociation is a general process in which ionic compounds separate into smaller ions, usually in a reversible manner.} 
\Definition{Dissolution}{Dissolution or dissolving is the process where ionic crystals break up into ions in water.}
\Definition{Hydration}{Hydration is the process where ions become surrounded with water molecules.}
    \setcounter{subfigure}{0}
	\begin{figure}[H] % horizontal\label{m38720*uid21}
    \begin{center}
\scalebox{.7} % Change this value to rescale the drawing.
{
\begin{pspicture}(0,-1.737183)(8.252103,1.737183)
\definecolor{color57b}{rgb}{0.6666666666666666,0.0,1.0}
\definecolor{color54b}{rgb}{1.0,0.6666666666666666,0.0}
\pscircle[linewidth=0.04,dimen=outer](2.461634,1.0271829){0.21}
\pscircle[linewidth=0.04,linecolor=red,dimen=outer,fillstyle=solid,fillcolor=red](2.0316339,1.377183){0.36}
\pscircle[linewidth=0.04,dimen=outer](1.6016339,1.0271829){0.21}
\pscircle[linewidth=0.04,linecolor=color54b,dimen=outer,fillstyle=solid,fillcolor=color54b](2.0316339,0.09718295){0.36}
\pscircle[linewidth=0.04,linecolor=color57b,dimen=outer,fillstyle=solid,fillcolor=color57b](6.511634,-0.06281705){0.36}
\rput{18.878832}(0.08622949,-0.33395928){\pscircle[linewidth=0.04,dimen=outer](1.0474685,0.0923481){0.21}}
\rput{18.878832}(0.12038696,-0.1553355){\pscircle[linewidth=0.04,linecolor=red,dimen=outer,fillstyle=solid,fillcolor=red](0.5273516,0.28438565){0.36}}
\rput{18.878832}(-0.04758419,-0.085629836){\pscircle[linewidth=0.04,dimen=outer](0.23373225,-0.18592027){0.21}}
\rput{203.82613}(5.847363,1.364404){\pscircle[linewidth=0.04,dimen=outer](3.0676064,0.065389454){0.21}}
\rput{203.82613}(6.930438,1.2999665){\pscircle[linewidth=0.04,linecolor=red,dimen=outer,fillstyle=solid,fillcolor=red](3.6023467,-0.08107813){0.36}}
\rput{203.82613}(7.2133904,2.3474119){\pscircle[linewidth=0.04,dimen=outer](3.8543134,0.41279718){0.21}}
\rput{180.27708}(3.245362,-1.8619516){\pscircle[linewidth=0.04,dimen=outer](1.6204299,-0.93489945){0.21}}
\rput{180.27708}(4.1104145,-2.5556927){\pscircle[linewidth=0.04,linecolor=red,dimen=outer,fillstyle=solid,fillcolor=red](2.0521173,-1.2828159){0.36}}
\rput{180.27708}(4.9653115,-1.8494748){\pscircle[linewidth=0.04,dimen=outer](2.4804199,-0.9307405){0.21}}
\rput{94.233444}(8.956811,-7.522198){\pscircle[linewidth=0.04,dimen=outer](7.9713893,0.3980573){0.21}}
\rput{94.233444}(8.162662,-7.693989){\pscircle[linewidth=0.04,linecolor=red,dimen=outer,fillstyle=solid,fillcolor=red](7.654087,-0.05660657){0.36}}
\rput{94.233444}(8.169669,-8.506476){\pscircle[linewidth=0.04,dimen=outer](8.034875,-0.45959625){0.21}}
\rput{3.5407977}(-0.07731536,-0.43452772){\pscircle[linewidth=0.04,dimen=outer](6.990459,-1.467952){0.21}}
\rput{3.5407977}(-0.058241528,-0.40607083){\pscircle[linewidth=0.04,linecolor=red,dimen=outer,fillstyle=solid,fillcolor=red](6.5396643,-1.1451766){0.36}}
\rput{3.5407977}(-0.082234114,-0.38161755){\pscircle[linewidth=0.04,dimen=outer](6.132101,-1.521065){0.21}}
\rput{264.35886}(6.212653,4.596659){\pscircle[linewidth=0.04,dimen=outer](5.1888213,-0.5162836){0.21}}
\rput{264.35886}(6.250011,5.4175353){\pscircle[linewidth=0.04,linecolor=red,dimen=outer,fillstyle=solid,fillcolor=red](5.5793943,-0.122770205){0.36}}
\rput{264.35886}(5.4538083,5.620747){\pscircle[linewidth=0.04,dimen=outer](5.273357,0.33955148){0.21}}
\rput{179.04214}(12.34774,2.5576375){\pscircle[linewidth=0.04,dimen=outer](6.1631804,1.3304267){0.21}}
\rput{179.04214}(13.189889,1.8363191){\pscircle[linewidth=0.04,linecolor=red,dimen=outer,fillstyle=solid,fillcolor=red](6.5872693,0.9732873){0.36}}
\rput{179.04214}(14.06714,2.5145116){\pscircle[linewidth=0.04,dimen=outer](7.0230603,1.31605){0.21}}
\usefont{T1}{ptm}{m}{n}
\rput(6.503509,-0.05281705){Na}
\usefont{T1}{ptm}{m}{n}
\rput(2.0113213,0.08718295){Cl}
\end{pspicture} 
}
\caption{Sodium chloride dissolves in water}
\label{fig:hydrosphere:ions dissolving}
\end{center}
\end{figure}     
        \label{m38720*id335421}The dissolution of sodium chloride can be represented by the following equation:\par 
        \label{m38720*uid3241}$\text{NaCl (s)} \to {\text{Na}}^{+}\text{(aq)} + {\text{Cl}}^{-}\text{(aq)}$
        \par 
        \label{m38720*id333999}The dissolution of potassium sulphate into potassium and sulphate ions is shown below as another example:\par 
        \label{m38720*uid971321}${\text{K}}_{2}{\text{SO}}_{4}\text{(s)}\to 2{\text{K}}^{+}\text{(aq)}+\text{SO}_{4}^{2-}\text{(aq)}$
        \par 
        \label{m38720*id335781}Remember that \textbf{molecular} substances (e.g.\@{} covalent compounds) may also dissolve, but most will not form ions. One example is glucose.\par 
        \label{m38720*uid922381}${\text{C}}_{6}{\text{H}}_{12}{\text{O}}_{6}\text{(s)}\rightarrow{\text{C}}_{6}{\text{H}}_{12}{\text{O}}_{6}\text{(aq)}$
        \par 
        \label{m38720*id335863}There are exceptions to this and some molecular substances \textsl{will} form ions when they dissolve. Hydrogen chloride for example can ionise to form hydrogen and chloride ions.\par 
        \label{m38720*uid98732}$\text{HCl (g)} + \text{H}_{2}\text{O} (\ell) \to \text{H}_{3}\text{O}^{+} \text{(aq)} + {\text{Cl}}^{-}\text{(aq)}$
        \par 
    \noindent
You can try dissolving ionic compounds such as potassium permanganate, sodium hydroxide and potassium nitrate in water and observing what happens. 
  \label{m38720*secfhsst!!!underscore!!!id338}
\begin{exercises}{Ions in solution}
\label{m38720*id336094}\begin{enumerate}[noitemsep, label=\textbf{\arabic*}. ]
\vspace{-1cm}
%Q1
\label{m38720*uid22}\item For each of the following, say whether the substance is ionic or molecular.
\label{m38720*id336110}\begin{enumerate}[noitemsep, label=\textbf{\alph*}. ] 
\label{m38720*uid23}\item potassium nitrate ($\text{KNO}_{3}$)
\label{m38720*uid24}\item ethanol ($\text{C}_{2}\text{H}_{5}\text{OH}$)
\label{m38720*uid25}\item sucrose (a type of sugar) ($\text{C}_{12}\text{H}_{22}\text{O}_{11}$)
\label{m38720*uid26}\item sodium bromide ($\text{NaBr}$)
\end{enumerate}
%Q2
\label{m38720*uid27}\item Write a balanced equation to show how each of the following ionic compounds dissociate in water.
\label{m38720*id336252}\begin{enumerate}[noitemsep, label=\textbf{\alph*}. ] 
            \label{m38720*uid28}\item sodium sulphate ($\text{Na}_{2}\text{SO}_{4}$)
\label{m38720*uid29}\item potassium bromide ($\text{KBr}$)
\label{m38720*uid30}\item potassium permanganate ($\text{KMnO}_{4}$)
\label{m38720*uid31}\item sodium phosphate ($\text{Na}_{3}\text{PO}_{4}$)
\end{enumerate}
%Q3
\item Draw a diagram to show how $\text{KCl}$ dissolves in water.
\end{enumerate}
\practiceinfo
\par 
 \par \begin{tabular}[h]{cccccc}
 (1.) 0075  &  (2.) 0076  & (3.) 0077 \end{tabular}
\end{exercises}
%             \subsection*{Applications}
%             \nopagebreak
% \begin{activity}{Acid rain and water hardness}
% The following two topics are given: acid rain or water hardness. Choose one of the two topics and prepare a poster or class presentation on it. Some information is given below about the topics. You will have to read further on the topic as well. Some guideline questions to answer are: 
% \begin{enumerate}[noitemsep,label=\textbf{\arabic*}.]
%  \item What is hard water/acid rain?
% \item Why is it a problem?
% \item Where in South Africa is this a problem?
% \item What is the chemistry involved?
% \item What is the impact on our lives and the environment?
% \item What can be done to to improve the situation?
% \end{enumerate}
% \begin{minipage}{.5\textwidth}
% \textbf{Hard water} is water that has a high mineral content. Water that has a low mineral content is known as \textbf{soft water}. If water has a high mineral content, it usually contains high levels of metal ions, mainly calcium ($\text{Ca}^{2+}$) and magnesium ($\text{Mg}^{2+}$). The calcium ions enter the water from either ${\text{CaCO}}_{3}$ (limestone or chalk) or from mineral deposits of ${\text{CaSO}}_{4}$. The main source of magnesium is a sedimentary rock called dolomite, ${\text{CaMg(CO}}_{3}\text{)}_{2}$. Hard water may also contain other metals as well as bicarbonates and sulphates.\par 
% \label{m38720*notfhsst!!!underscore!!!id362}
% \Tip{The simplest way to check whether water is hard or soft is to use the lather/froth test. If the water is very soft, soap will make bubbles more easily when it is rubbed against the skin. With hard water this won't happen. Toothpaste will also not froth well in hard water.}
% Hard water causes scaling on kettles (scaling is a build up of mainly calcium ions). Hard water also leads to blocked water pipes. 
% \end{minipage}
% \begin{minipage}{.5\textwidth}
% The acidity of rainwater comes from the natural presence of three substances ($\text{CO}_{2}$, $\text{NO}$, and $\text{SO}_{2}$) in the lowest layer of the atmosphere. These gases are able to dissolve in water and therefore make rain more acidic than it would otherwise be. Of these gases, carbon dioxide ($\text{CO}_{2}$) has the highest concentration and therefore contributes the most to the natural acidity of rainwater. \par 
% Acid rain refers to the deposition of acidic components in rain, snow and dew. Acid rain occurs when sulphur dioxide and nitrogen oxides are emitted into the atmosphere, undergo chemical transformations and are absorbed by water droplets in clouds. The droplets then fall to earth as rain, snow, mist, dry dust, hail, or sleet. This increases the acidity of the soil and affects the chemical balance of lakes and streams. 
% \label{m38720*id338300}Although these reactions do take place naturally, human activities can greatly increase the concentration of these gases in the atmosphere, so that rain becomes far more acidic than it would otherwise be. The burning of fossil fuels in industries, vehicles etc is one of the biggest culprits. If the acidity of the rain drops to below 5, it is referred to as \textbf{acid rain}.\par 
% \label{m38720*id338311}If the water in rivers, dams and lakes becomes too acidic some plants and animals may not be able to survive. Acid rain can also dissolve some minerals from the soil and these ions can get washed into the rivers and lakes. The soil can also become to acidic which influences the ability of the soil to produce crops.\par 
% \label{m38720*id338337}Acid rain can also affect buildings and monuments, many of which are made from marble and limestone. A chemical reaction takes place between ${\text{CaCO}}_{3}$ (limestone) and sulphuric acid to produce aqueous ions which can be easily washed away. The same reaction can occur in the lithosphere where limestone rocks are present e.g.\@{} limestone caves can be eroded by acidic rainwater.
%         \label{m38720*id7435}\nopagebreak\noindent{}       
%     \begin{equation}
%     {\text{H}}_{2}{\text{SO}}_{4}+{\text{CaCO}}_{3}\to {\text{CaSO}}_{4}\ensuremath{\cdot}\text{H}{}_{2}\text{O}+{\text{CO}}_{2}\tag{17.2}
%       \end{equation}
% \end{minipage}
% \end{activity}
    \label{m38720*cid7}
            \section{Electrolytes, ionisation and conductivity}
            \nopagebreak
      \label{m38720*id338608}You have learnt that water is a polar molecule and that it can dissolve ionic substances in water. When ions are present in water, the water is able to conduct electricity. The solution is known as an electrolyte.\par 
\label{m38720*fhsst!!!underscore!!!id635}
\Definition{Electrolyte} {  An electrolyte is a substance that contains free ions and behaves as an electrically conductive medium.   } 
Because electrolytes generally consist of ions in solution, they are also known as ionic solutions. A strong electrolyte is one where many ions are present in the solution and a weak electrolyte is one where few ions are present. Strong electrolytes are good conductors of electricity and weak electrolytes are weak conductors of electricity. Non-electrolytes do not conduct electricity at all. 
\textbf{Conductivity} in aqueous solutions, is a measure of the ability of water to conduct an electric current. The more \textbf{ions} there are in the solution, the higher its conductivity. Also the more ions there are in solution, the stronger the electrolyte. 
      \label{m38720*uid56}
            \subsection*{Factors that affect the conductivity of electrolytes}
            \nopagebreak
            \label{m38720*id339287}The conductivity of an electrolyte is therefore affected by the following factors:\par 
        \label{m38720*id339291}\begin{itemize}[noitemsep]
\label{m38720*uid58}\item The \textbf{concentration of ions} in solution. The higher the concentration of ions in solution, the higher its conductivity will be.
\label{m38720*uid57}\item The \textbf{type of substance} that dissolves in water. Whether a material is a strong electrolyte (e.g.\@{} potassium nitrate, ${\text{KNO}}_{3}$), a weak electrolyte (e.g.\@{} acetic acid, ${\text{CH}}_{3}\text{COOH}$) or a non-electrolyte (e.g.\@{} sugar, alcohol, oil) will affect the conductivity of water because the concentration of ions in solution will be different in each case. Strong electrolytes form ions easily, weak electrolytes do not form ions easily and non-electrolytes do not form ions in solution.
\mindsetvid{investigating solutions}{VPbon}
\label{m38720*uid59}\item \textbf{Temperature.}
The warmer the solution, the higher the solubility of the material being dissolved and therefore the higher the conductivity as well.
\end{itemize} \nopagebreak
\label{m38720*secfhsst!!!underscore!!!id739}
            \begin{g_experiment}{Electrical conductivity }
            \nopagebreak
            \label{m38720*id339425}\noindent{}\textbf{Aim: } To investigate the electrical conductivities of different substances and solutions.\\ 
        \label{m38720*id339438}\noindent{}\textbf{Apparatus:}
\begin{itemize}[noitemsep]
\item Solid salt ($\text{NaCl}$) crystals
\item different liquids such as distilled water, tap water, seawater, sugar, oil and alcohol
\item solutions of salts e.g.\@{} $\text{NaCl}$, $\text{KBr}$, $\text{CaCl}_{2}$, $\text{NH}_{4}\text{Cl}$
\item a solution of an acid (e.g.\@{} $\text{HCl}$) and a solution of a base (e.g.\@{} $\text{NaOH}$)
\item torch cells
\item ammeter
\item conducting wire, crocodile clips and 2 carbon rods.
\end{itemize}
        \label{m38720*eip-456}
      \label{m38720*id334346}\noindent{}\textbf{Method:}
\begin{enumerate}[noitemsep, label=\textbf{\arabic*}.]
\item Set up the experiment by connecting the circuit as shown in the diagram below. In the diagram, X represents the substance or solution that you will be testing.
\item When you are using the solid crystals, the crocodile clips can be attached directly to each end of the crystal. When you are using solutions, two carbon rods are placed into the liquid and the clips are attached to each of the rods.
\item In each case, complete the circuit and allow the current to flow for about 30 seconds. 
\item Observe whether the ammeter shows a reading.
\end{enumerate}
        \label{m38720*id334362}
\begin{minipage}{.5\textwidth}
    \setcounter{subfigure}{0}
	\begin{figure}[H] % horizontal\label{m38720*id334366}
\begin{center}
\scalebox{0.7}{
\begin{pspicture}(0,-0.6)(5,6.2)
\SpecialCoor
%\psgrid[gridcolor=lightgray]
\pnode(0,0){A}
\pnode(0,5){B}
\pnode(5,5){C}
\pnode(5,0){D}
\pnode(3.5,0){E}
\pnode(1.5,0){F}
\battery(B)(C){battery}
\psline(C)(D)
\psline[arrowsize=10pt,arrowinset=0,arrowlength=2.5]{->}(D)(E)
\psframe(1.5,-0.5)(3.5,0.5)
\uput[u](2.5,0.5){test substance}
\rput(2.5,0){X}
\psline(4,0)(4,-0.4)(4.6,-0.4)
\uput[r](4.6,-0.4){crocodile clip}
\psline[arrowsize=10pt,arrowinset=0,arrowlength=2.5]{<-}(F)(A)
\psellipse(0,2.5)(0.5,0.5)
\rput(0,2.5){\textbf{A}}
\psline(0,5)(0,3)
\psline(0,2)(0,0)
\rput(-1.5,2.5){Ammeter}
\end{pspicture}
}
\end{center}
 \end{figure}
\end{minipage}
\begin{minipage}{.5\textwidth}   
	\begin{figure}[H] % horizontal\label{m38720*id334366}
\begin{center}
\scalebox{0.7}{
\begin{pspicture}(0,-0.6)(5,6.2)
\SpecialCoor
%\psgrid[gridcolor=lightgray]
\pnode(0,0){A}
\pnode(0,5){B}
\pnode(5,5){C}
\pnode(5,0){D}
\pnode(3.5,0){E}
\pnode(1.5,0){F}
\battery(B)(C){battery}
\psline(C)(D)
\psline[arrowsize=10pt,arrowinset=0,arrowlength=2.5]{->}(D)(E)
\psline[linewidth=.1](1.5,0)(1.5,-1.2)
\psline[linewidth=.1](3.5,0)(3.5,-1.2)
\psline(1.2,-0.5)(1.2,-1.5)(3.7,-1.5)(3.7,-0.5)
\psline(1.2,-0.7)(3.7,-0.7)
\uput[u](2.5,-2){test substance}
\rput(2.5,-1.2){X}
\psline(4,0)(4,-0.4)(4.6,-0.4)
\uput[r](4.6,-0.4){crocodile clip}
\psline[arrowsize=10pt,arrowinset=0,arrowlength=2.5]{<-}(F)(A)
\psellipse(0,2.5)(0.5,0.5)
\rput(0,2.5){\textbf{A}}
\psline(0,5)(0,3)
\psline(0,2)(0,0)
\rput(-1.5,2.5){Ammeter}
\end{pspicture}
}
\end{center}
 \end{figure}
\end{minipage}    
\\
        \label{m38720*id334372}\noindent{}\textbf{Results: } Record your observations in a table similar to the one below:
    % \textbf{m38720*id334385}\par
          \begin{table}[H]
        \begin{center}
    \noindent
      \begin{tabular}{|l|l|}\hline
        Test substance &
        Ammeter reading \\ \hline
         &
       \\ \hline
         &
       \\ \hline
         &
       \\ \hline
         &
    \\ \hline
    \end{tabular}
      \end{center}
\end{table}
        \label{m38720*id339669}What do you notice? Can you explain these observations?\\ 
        \label{m38720*id339864}\noindent{}\textbf{Conclusions: } Solutions that contain free-moving ions are able to conduct electricity because of the movement of charged particles. Solutions that do not contain free-moving ions do not conduct electricity.
\end{g_experiment}
        \label{m38720*id339672}Remember that for electricity to flow, there needs to be a movement of charged particles e.g.\@{} ions. With the solid $\text{NaCl}$ crystals, there was no flow of electricity recorded on the ammeter. Although the solid is made up of ions, they are held together very tightly within the crystal lattice and therefore no current will flow. Distilled water, oil and alcohol also don't conduct a current because they are \textsl{covalent compounds} and therefore do not contain ions.\par 
        \label{m38720*id339687}The ammeter should have recorded a current when the salt solutions and the acid and base solutions were connected in the circuit. In solution, salts \textsl{dissociate} into their ions, so that these are free to move in the solution. Look at the following examples: \\
Dissociation of potassium bromide:
        \label{m38720*id339701}\nopagebreak\noindent        
    \begin{equation*}
    \text{KBr (s)} \to {\text{K}}^{+} \text{(aq)} + {\text{Br}}^{-} \text{(aq)}
      \end{equation*} \\
Dissociation of table salt:\\
        \label{m38720*id339737}\nopagebreak\noindent          
    \begin{equation*}
    \text{NaCl (s)}\to {\text{Na}}^{+} \text{(aq)} + {\text{Cl}}^{-} \text{(aq)}
      \end{equation*}\\
Ionisation of hydrochloric acid:\\
        \label{m38720*id339770}\nopagebreak\noindent          
    \begin{equation*}
    \text{HCl} (\ell)  +{\text{H}}_{2}\text{O} (\ell) \to {\text{H}}_{3}{\text{O}}^{+} \text{(aq)} +{\text{Cl}}^{-} \text{(aq)}
      \end{equation*}\\
        \label{m38720*id339831}\nopagebreak\noindent
Dissociation of sodium hydroxide:\\          
    \begin{equation*}
    \text{NaOH (s)} \to {\text{Na}}^{+} \text{(aq)} + {\text{OH}}^{-} \text{(aq)}
      \end{equation*}
 \par 
\label{m38720**end}
            \section{Precipitation reactions}
            \nopagebreak
      \label{m38719*id339907}Sometimes, ions in solution may react with each other to form a new substance that is \textsl{insoluble}. This is called a \textbf{precipitate}. The reaction is called a precipitation reaction. \par 
\mindsetvid{preparing precipitates}{VPbpr}
\label{m38719*fhsst!!!underscore!!!id887}
\Definition{ Precipitate } {   A precipitate is the solid that forms in a solution during a chemical reaction.} 
\label{m38719*secfhsst!!!underscore!!!id890}
            \begin{g_experiment}{The reaction of ions in solution }
            \nopagebreak
\textbf{Aim: } To investigate the reactions of ions in solutions.\\
            \label{m38719*id339954}\noindent{}\textbf{Apparatus:} 4 test tubes; copper(II) chloride solution; sodium carbonate solution; sodium sulphate solution
      \label{m38719*id339975}
%     \setcounter{subfigure}{0}
\begin{figure}[H]

\begin{center}
\scalebox{0.8} % Change this value to rescale the drawing.
{
\begin{pspicture}(-5,-5)(5,5)
\psset{unit=1cm}
\newpsstyle{white} {linestyle=solid,linewidth=.1,fillstyle=solid,fillcolor=white}
\rput(-4,0){\pstTubeEssais[niveauLiquide1=40]}
\psline[linewidth=0.04]{->}(-3.8,-1)(-3,-1)
\uput[r](-3,-1){\large{$\text{CuCl}_2$}}
\rput(0,0){\pstTubeEssais[niveauLiquide1=40,aspectLiquide1=white]}
\psline[linewidth=0.04]{->}(0.2,-1)(1,-1)
\uput[r](1,-1){\large{$\text{Na}_{2}\text{CO}_3$}}
\rput(4,0){\pstTubeEssais[niveauLiquide1=40]}
\psline[linewidth=0.04]{->}(4.2,-1)(5,-1)
\uput[r](5,-1){\large{$\text{CuCl}_2$}}
\rput(8,0){\pstTubeEssais[niveauLiquide1=40,aspectLiquide1=white]}
\psline[linewidth=0.04]{->}(8.2,-1)(9,-1)
\uput[r](9,-1){\large{$\text{Na}_{2}\text{SO}_4$}}
\end{pspicture}
}
\end{center}
\end{figure}       

      \label{m38719*id339985}\noindent{}\textbf{Method:}
      \label{m38719*id339992}\begin{enumerate}[noitemsep, label=\textbf{\arabic*}. ] 
            \label{m38719*uid60}\item Prepare 2 test tubes with approximately $5~\text{ml}$ of dilute copper(II) chloride solution in each
\label{m38719*uid61}\item Prepare 1 test tube with $5~\text{ml}$ sodium carbonate solution
\label{m38719*uid62}\item Prepare 1 test tube with $5~\text{ml}$ sodium sulphate solution
\label{m38719*uid63}\item Carefully pour the sodium carbonate solution into one of the test tubes containing copper(II) chloride and observe what happens
\label{m38719*uid64}\item Carefully pour the sodium sulphate solution into the second test tube containing copper(II) chloride and observe what happens
\end{enumerate}
      \label{m38719*id340060}\noindent{}\textbf{Results:}
      \label{m38719*id340067}\begin{enumerate}[noitemsep, label=\textbf{\arabic*}. ] 
            \label{m38719*uid65}\item A light blue precipitate forms when sodium carbonate reacts with copper(II) chloride.
\label{m38719*uid66}\item No precipitate forms when sodium sulphate reacts with copper(II) chloride. The solution is light blue.
\end{enumerate}
\end{g_experiment}
      \label{m38719*id340106}It is important to understand what happened in the previous demonstration. We will look at what happens in each reaction, step by step.\\
For \textbf{reaction 1} you have the following ions in your solution: ${\text{Cu}}^{2+}$, ${\text{Cl}}^{-}$, ${\text{Na}}^{+}$ and $\text{CO}_{3}^{2-}$. A precipitate will form if any combination of cations and anions can become a solid. The following table summarises which combination will form solids (precipitates) in solution. 
          \begin{table}[H]
    % \begin{table}[H]
    % \\ '' '0'
        \begin{center}
      \label{m38719*uid69}
    \noindent
      \begin{tabular}{|l|p{8cm}|}\hline
                \textbf{Salt}
               &
                \textbf{Solubility} \\ \hline
        Nitrates &
        All are \textbf{soluble} \\ \hline
        Potassium, sodium and ammonium salts &
        All are \textbf{soluble} \\ \hline
        Chlorides, bromides and iodides &
        All are \textbf{soluble} except silver, lead(II) and mercury(II) salts (e.g.\@{} silver chloride) \\ \hline
        Sulphates &
        All are \textbf{soluble} except lead(II) sulphate, barium sulphate and calcium sulphate \\ \hline
        Carbonates &
        All are \textbf{insoluble} except those of potassium, sodium and ammonium \\ \hline
        Compounds with fluorine &
        Almost all are \textbf{soluble} except those of magnesium, calcium, strontium (II), barium (II) and lead (II) \\ \hline
        Perchlorates and acetates &
        All are \textbf{soluble} \\ \hline
        Chlorates &
        All are \textbf{soluble} except potassium chlorate \\ \hline
        Metal hydroxides and oxides &
        Most are \textbf{insoluble} \\ \hline
    \end{tabular}
      \end{center}
    \caption{General rules for the solubility of salts}
\label{tab:solubility}
\end{table}
\Tip{Salts of carbonates, phosphates, oxalates, chromates and sulphides are generally insoluble.} 
If you look under carbonates in the table it states that all carbonates are \textbf{insoluble} except potassium sodium and ammonium. This means that $\text{Na}_{2}\text{CO}_3$ will dissolve in water or remain in solution, but $\text{CuCO}_3$ will form a precipitate. The precipitate that was observed in the reaction must therefore be $\text{CuCO}_3$. The balanced chemical equation is:
\begin{equation*}
2\text{Na}^{+} \text{(aq)} + \text{CO}_{3}^{2-} \text{(aq)} + \text{Cu}^{2+} \text{(aq)} + 2\text{Cl}^{-} \text{(aq)} \to \text{CuCO}_{3} \text{(s)} +  2\text{Na}^{+} \text{(aq)} + 2\text{Cl}^{-} \text{(aq)}
\end{equation*}
Note that sodium chloride does not precipitate and we write it as ions in the equation.
For \textbf{reaction 2} we have ${\text{Cu}}^{2+}$, ${\text{Cl}}^{-}$, ${\text{Na}}^{+}$ and $\text{SO}_{4}^{2-}$ in solution. Most chlorides and sulphates are soluble according to the table. The balanced chemical equation is: 
\begin{equation*}
2{\text{Na}}^{+} \text{(aq)} + \text{SO}_{4}^{2-} \text{(aq)} + {\text{Cu}}^{2+} \text{(aq)} + 2{\text{Cl}}^{-} \text{(aq)} \to 2{\text{Na}}^{+} \text{(aq)} + \text{SO}_{4}^{2-} \text{(aq)} + {\text{Cu}}^{2+} \text{(aq)} + 2{\text{Cl}}^{-} \text{(aq)} 
\end{equation*}
Both of these reactions are ion exchange reactions.
\subsection*{Tests for anions}
We often want to know which ions are present in solution. If we know which salts precipitate, we can derive tests to identify ions in solution. Given below are a few such tests.\\
\mindsetvid{test for halides}{VPbqd}
% \Warning{As always when working with chemicals, you must work carefully as you can easily get bad chemical burns if you spill the chemicals on yourself.}
      \label{m38719*uid70}
            \subsubsection*{Test for a chloride}
            \nopagebreak
        \label{m38719*id341138}Prepare a solution of the unknown salt using distilled water and add a small amount of \textbf{silver nitrate} solution. If a white precipitate forms, the salt is either a chloride or a carbonate.
        \label{m38719*id341148}\nopagebreak\noindent{}
    \begin{equation*}
    {\text{Cl}}^{-} \text{(aq)} +{\text{Ag}}^{+} \text{(aq)} + \text{NO}_{3}^{-} \text{(aq)} \to \text{AgCl} \text{(s)} +\text{NO}_{3}^{-} \text{(aq)}
      \end{equation*}
     ($\text{AgCl}$ is white precipitate)
        \label{m38719*id341211}\nopagebreak\noindent{}
    \begin{equation*}
    \text{CO}_{3}^{2-} \text{(aq)} + 2{\text{Ag}}^{+} \text{(aq)} + 2\text{NO}_{3}^{-} \text{(aq)} \to {\text{Ag}}_{2}{\text{CO}}_{3} \text{(s)} + 2\text{NO}_{3}^{-} \text{(aq)}
      \end{equation*}
    (${\text{Ag}}_{2}{\text{CO}}_{3}$ is white precipitate)\par 
        \label{m38719*id341323}The next step is to treat the precipitate with a small amount of \textbf{concentrated nitric acid}. If the precipitate remains unchanged, then the salt is a chloride. If carbon dioxide is formed and the precipitate disappears, the salt is a carbonate.\par 
        \label{m38719*id341334}$\text{AgCl} \text{(s)} + {\text{HNO}}_{3} (\ell) \to $ (no reaction; precipitate is unchanged)\par 
        \label{m38719*id341361}${\text{Ag}}_{2}{\text{CO}}_{3} \text{(s)} + 2{\text{HNO}}_{3} (\ell) \to 2{\text{Ag}^{+}} \text{(aq)} + 2\text{NO}_{3}^{-} \text{(aq)} + {\text{H}}_{2}\text{O} (\ell) + {\text{CO}}_{2} \text{(g)} $ (precipitate disappears)\par 
            \subsubsection*{Test for bromides and iodides}
            \nopagebreak
        \label{m38719*id341894}As was the case with the chlorides, the bromides and iodides also form precipitates when they are reacted with silver nitrate. Silver chloride is a white precipitate, but the silver bromide and silver iodide precipitates are both pale yellow. To determine whether the precipitate is a bromide or an iodide, we use chlorine water and carbon tetrachloride (${\text{CCl}}_{4}$).\par 
        \label{m38719*id341914}Chlorine water frees bromine gas from the bromide and colours the carbon tetrachloride a reddish brown.\\
$2\text{Br}^{-} \text{(aq)} + \text{Cl}_{2} \text{(aq)} \to 2 \text{Cl}^{-} \text{(aq)} + \text{Br}_{2} \text{(g)}$
\par 
        \label{m38719*id341920}Chlorine water frees iodine gas from an iodide and colours the carbon tetrachloride purple.\\
$2\text{I}^{-} \text{(aq)} + \text{Cl}_{2} \text{(aq)} \to 2 \text{Cl}^{-} \text{(aq)} + \text{I}_{2} \text{(g)}$
\par 
      \label{m38719*uid71}
            \subsubsection*{Test for a sulphate}
            \nopagebreak
        \label{m38719*id341449}Add a small amount of barium chloride solution to a solution of the test salt. If a white precipitate forms, the salt is either a sulphate or a carbonate.\par 
        \label{m38719*id341454}$\text{SO}_{4}^{2-} \text{(aq)} + {\text{Ba}}^{2+} \text{(aq)} + {\text{Cl}}^{-} \text{(aq)} \to {\text{BaSO}}_{4} \text{(s)} + {\text{Cl}}^{-} \text{(aq)} $ (${\text{BaSO}}_{4}$ is a white precipitate)\par 
        \label{m38719*id341538}$\text{CO}_{3}^{2-} \text{(aq)} + {\text{Ba}}^{2+} \text{(aq)} + {\text{Cl}}^{-} \text{(aq)} \to {\text{BaCO}}_{3} \text{(s)} + {\text{Cl}}^{-} \text{(aq)} $ (${\text{BaCO}}_{3}$ is a white precipitate)\par 
        \label{m38719*id341622}If the precipitate is treated with nitric acid, it is possible to distinguish whether the salt is a sulphate or a carbonate (as in the test for a chloride).\par 
        \label{m38719*id341627}${\text{BaSO}}_{4} \text{(s)} + {\text{HNO}}_{3} (\ell) \to $ (no reaction; precipitate is unchanged)\par 
        \label{m38719*id341658}${\text{BaCO}}_{3} \text{(s)} + 2{\text{HNO}}_{3} (\ell) \to \text{Ba}^{2+} \text{(aq)} + 2\text{NO}_{3}^{-} \text{(aq)} + {\text{H}}_{2}\text{O} (\ell) + {\text{CO}}_{2} \text{(g)} $ (precipitate disappears)\par 
      \label{m38719*uid72}
            \subsubsection*{Test for a carbonate}
            \nopagebreak
        \label{m38719*id341749}If a sample of the dry salt is treated with a small amount of acid, the production of carbon dioxide is a positive test for a carbonate.
\begin{equation*}
2\text{HCl} + \text{K}_{2}\text{CO}_{3} \text{(aq)} \to {\text{CO}}_{2} \text{(g)} + 2\text{KCl} \text{ (aq)} + \text{H}_{2}\text{O } \ell
\end{equation*}
If the gas is passed through limewater (an aqueous solution of calcium hydroxide) and the solution becomes milky, the gas is carbon dioxide.\\
$\text{Ca}^{2+} \text{(aq)} + 2\text{OH}^{-} \text{(aq)} + {\text{CO}}_{2} \text{(g)} \to {\text{CaCO}}_{3} \text{(s)} + \text{H}_{2}\text{O} (\ell)$ (It is the insoluble ${\text{CaCO}}_{3}$ precipitate that makes the limewater go milky)

            \begin{exercises}{Precipitation reactions and ions in solution }
            \nopagebreak \vspace{-1.5cm}
            \label{m38719*id341939}\begin{enumerate}[noitemsep, label=\textbf{\arabic*}. ] 
%Q1
            \label{m38719*uid74}\item Silver nitrate (${\text{AgNO}}_{3}$) reacts with potassium chloride ($\text{KCl}$) and a white precipitate is formed.
\label{m38719*id341969}\begin{enumerate}[noitemsep, label=\textbf{\alph*}. ] 
            \label{m38719*uid75}\item Write a balanced equation for the reaction that takes place. Include the state symbols. 
\label{m38719*uid76}\item What is the name of the insoluble salt that forms?
\label{m38719*uid77}\item Which of the salts in this reaction are soluble?
\end{enumerate}
%Q2
\label{m38719*uid78}\item Barium chloride reacts with sulphuric acid to produce barium sulphate and hydrochloric acid.
\label{m38719*id342022}\begin{enumerate}[noitemsep, label=\textbf{\alph*}. ] 
            \label{m38719*uid79}\item Write a balanced equation for the reaction that takes place. Include the state symbols.
\label{m38719*uid80}\item Does a precipitate form during the reaction? 
\label{m38719*uid81}\item Describe a test that could be used to test for the presence of barium sulphate in the products.
\end{enumerate}
%Q3
\label{m38719*uid82}\item A test tube contains a clear, colourless salt solution. A few drops of silver nitrate solution are added to the solution and a pale yellow precipitate forms. Chlorine water and carbon tetrachloride were added, which resulted in a purple solution. Which one of the following salts was dissolved in the original solution? Write the balanced equation for the reaction that took place between the salt and silver nitrate.
\label{m38719*id342078}\begin{enumerate}[noitemsep, label=\textbf{\alph*}. ] 
            \label{m38719*uid83}\item $\text{NaI}$
\label{m38719*uid84}\item $\text{KCl}$
\label{m38719*uid85}\item ${\text{K}}_{2}{\text{CO}}_{3}$\label{m38719*uid86}\item ${\text{Na}}_{2}{\text{SO}}_{4}$\end{enumerate}

\end{enumerate}
\vspace{-.5cm}
\practiceinfo
\begin{tabular}[h]{cccccc}
 (1.) 0078  &  (2.) 0079  &  (3.) 007a  & \end{tabular}
\end{exercises}
         \section{Other types of reactions}
    \nopagebreak
%            \label{m38719} $ \hspace{-5pt}\begin{array}{cccccccccccc}   \end{array} $ \hspace{2 pt}\raisebox{-0.2em}{\includegraphics[height=1em]{../icons/www.pdf}} {(section shortcode: P10083 )} \par 
            \label{m38719*uid2131}
	We will look at two types of reactions that occur in aqueous solutions. These are ion-exchange reactions and redox reactions. Ion exchange reactions include precipitation reactions, gas forming reactions and acid-base reactions. Redox reactions are electron transfer reactions. It is important to remember the difference between these two types of reactions. In ion exchange reactions ions are exchanged, in electron transfer reactions electrons are transferred. These terms will be explained further in the following sections. 
      \par 
\subsection*{Ion exchange reactions}
      \label{m38719*uid78332}
	Ion exchange reactions can be represented by:
	  \label{m38719*eid071534}\nopagebreak\noindent{}
	    
    \begin{equation*}
    \text{AB(aq)}+\text{CD (aq)}\to \text{AD}+\text{CB}
      \end{equation*}
	  Either $\text{AD}$ or $\text{CB}$ may be a solid or a gas. When a solid forms this is known as a precipitation reaction. If a gas is formed then this may be called a gas forming reaction. Acid-base reactions are a special class of ion exchange reactions and we will look at them separately. 
      \par 
      \label{m38719*eip-179}The formation of a precipitate or a gas helps to make the reaction happen. We say that the reaction is driven by the formation of a precipitate or a gas. All chemical reactions will only take place if there is something to make them happen. For some reactions this happens easily and for others it is harder to make the reaction occur.  \par 
\label{m38719*id7583}
 \Definition{   \label{id2492977}Ion exchange reaction} { \label{m38719*eip-id1168354893169}A type of reaction where the positive ions exchange their respective negative ions due to a driving force. } 
\label{m38719*uid10825}
\IFact{Ion exchange reactions are used in ion exchange chromatography. Ion exchange chromatography is used to purify water and as a means of softening water. Often when chemists talk about ion exchange, they mean ion exchange chromatography. }
	\par
We have already looked at precipitation reactions.
\subsubsection*{Gas forming reactions}
These reactions are similar to precipitation reactions with the exception that instead of a precipitate forming, a gas is formed instead. An example of a gas forming reaction is sodium carbonate in hydrochloric acid. The balanced equation for this reaction is: \\
$\text{Na}_{2}\text{CO}_{3} \text{(s)} + 2\text{HCl} \text{(aq)} \to \text{CO}_{2} \text{(g)} + 2\text{NaCl} \text{(aq)} + \text{H}_{2}\text{O} (\ell)$ 
            \subsubsection*{Acid-base reactions}
            \nopagebreak
\label{m38719*id0821354}Acid-base reactions take place between acids and bases. In general, the products will be water and a salt (i.e.\@{} an ionic compound). An example of this type of reaction is: \label{m38719*eid1534}\nopagebreak\noindent{}
	    
    \begin{equation*}
    \text{NaOH (aq)}+\text{HCl (aq)}\to \text{NaCl (aq)}+{\text{H}}_{2}\text{O} (\ell)
      \end{equation*}
\par \label{m38719*eip-588}This is an special case of an ion exchange reaction since the sodium in the sodium hydroxide swaps places with the hydrogen in the hydrogen chloride forming sodium chloride. At the same time the hydroxide and the hydrogen combine to form water. \par 
\label{m38719*eip-454}
            \subsection*{Redox reactions}
            \nopagebreak
            \label{m38719*eip-585}Redox reactions involve the exchange of electrons. One ion loses electrons and becomes more positive, while the other ion gains electrons and becomes more negative. To decide if a redox reaction has occurred we look at the charge of the atoms, ions or molecules involved. If one of them has become more positive and the other one has become more negative then a redox reaction has occurred. For example, sodium metal is oxidised to form sodium oxide (and sometimes sodium peroxide as well). The balanced equation for this is:
\label{m38719*id624}\nopagebreak\noindent{}
    \begin{equation*}
    4\text{Na}+{\text{O}}_{2}\to 2{\text{Na}}_{2}{\text{O}}
      \end{equation*}
\par \label{m38719*eip-815}In the above reaction sodium and oxygen are both neutral and so have no charge. In the products however, the sodium atom has a charge of $+1$ and the oxygen atom has a charge of $-2$. This tells us that the sodium has lost electrons and the oxygen has gained electrons. Since one species has become more positive and one more negative we can conclude that a redox reaction has occurred. We could also say that electrons have been transferred from one species to the other. (In this case the electrons were transferred from the sodium to the oxygen).\par \label{m38719*eip-878}
\vspace{-.5cm}
            \begin{g_experiment}{Demonstration: Oxidation of sodium metal}
            \nopagebreak
            \label{m38719*eip-355}
\begin{minipage}{.6\textwidth}
You will need a Bunsen burner, a small piece of sodium metal and a metal spatula. Light the Bunsen burner. Place the sodium metal on the spatula. Place the sodium in the flame. When the reaction finishes, you should observe a white powder on the spatula. This is a mixture of sodium oxide (${\text{Na}}_{2}\text{O}$) and sodium peroxide (${\text{Na}}_{2}{\text{O}}_{2}$). 
\par \label{m38719*eip-980}
      \Warning{Sodium metal is very reactive. Sodium metal reacts vigorously with water and should never be placed in water. Be very careful when handling sodium metal.}
\end{minipage}
\begin{minipage}{.4\textwidth}
 \begin{center}
  \includegraphics[width=.5\textwidth]{photos/sodium_flame_soren_wedel_nielsen_wikipedia.jpg}
 \end{center}

\end{minipage}

\end{g_experiment}
\nopagebreak 
            \begin{g_experiment}{Reaction types}
            \nopagebreak
            \label{m38719*eip-190}\noindent{}\textbf{Aim: } To use experiments to determine what type of reaction occurs.\\
\label{m38719*eip-1901}\noindent{}\textbf{Apparatus: } Soluble salts (e.g.\@{} potassium nitrate, ammonium chloride, sodium carbonate, silver nitrate, sodium bromide); hydrochloric acid ($\text{HCl}$); sodium hydroxide ($\text{NaOH}$); bromothymol blue; zinc metal; copper (II) sulphate; beakers; test-tubes \\
\label{m38719*eip-1902}\noindent{}\textbf{Method: }
\begin{enumerate}[noitemsep, label=\textbf{\arabic*}.]
            \item For each of the salts, dissolve a small amount in water and observe what happens.
\item Try dissolving pairs of salts (e.g.\@{} potassium nitrate and sodium carbonate) in water and observe what happens.
\item Dissolve some sodium carbonate in hydrochloric acid and observe what happens.
\item Carefully measure out $20{\text{cm}}^{3}$ of sodium hydroxide into a beaker. 
\item Add some bromothymol blue to the sodium hydroxide\item Carefully add a few drops of hydrochloric acid to the sodium hydroxide and swirl. Repeat until you notice the colour change.\item Place the zinc metal into the copper sulphate solution and observe what happens.
\end{enumerate}

\noindent{}\textbf{Results: }  Answer the following questions:
\begin{enumerate}
[noitemsep, label=\textbf{\arabic*}.]
\item What did you observe when you dissolved each of the salts in water?
\item What did you observe when you dissolved pairs of salts in the water?
\item What did you observe when you dissolved sodium carbonate in hydrochloric acid?
\item Why do you think we used bromothymol blue when mixing the hydrochloric acid and the sodium hydroxide? Think about the kind of reaction that occurred.
\item What did you observe when you placed the zinc metal into the copper sulphate?
\item Classify each reaction as either precipitation, gas forming, acid-base or redox.
\item What makes each reaction happen (i.e.\@{} what is the driving force)? Is it the formation of a precipitate or something else?
\item What criteria would you use to determine what kind of reaction occurs?\item Try to write balanced chemical equations for each reaction
\end{enumerate}

\label{m38719*eip-1904}\noindent{}\textbf{Conclusion: }  We can see how we can classify reactions by performing experiments.  
\end{g_experiment}
\label{m38719*eip-761}In the experiment above, you should have seen how each reaction type differs from the others. For example, a gas forming reaction leads to bubbles in the solution, a precipitation reaction leads to a precipitate forming, an acid-base reaction can be seen by adding a suitable indicator and a redox reaction can be seen by one metal disappearing and a deposit forming in the solution.\par  
\label{m38719*eip-796}
\summary{VPenn}
            \nopagebreak
            \label{m38719*eip-903}\begin{itemize}[noitemsep]
\item The \textbf{polar} nature of water means that \textbf{ionic compounds} dissociate easily in aqueous solution into their component ions.
% \label{m38719*uid96}\item \textbf{Ions} in solution play a number of roles. In the human body for example, ions help to regulate the internal environment (e.g.\@{} controlling muscle function, regulating blood pH). Ions in solution also determine water hardness and pH.
\item Dissociation is a general process in which ionic compounds separate into smaller ions, usually in a reversible manner.
\item Dissolution or dissolving is the process where ionic crystals break up into ions in water.
\item Hydration is the process where ions become surrounded with water molecules.
\item \textbf{Conductivity} is a measure of a solution's ability to conduct an electric current.
\item An \textbf{electrolyte} is a substance that contains free ions and is therefore able to conduct an electric current. Electrolytes can be divided into \textbf{strong} and \textbf{weak} electrolytes, based on the extent to which the substance ionises in solution.
\item A \textbf{non-electrolyte} cannot conduct an electric current because it does not contain free ions.
\item The \textbf{type of substance}, the \textbf{concentration of ions} and the \textbf{temperature} of the solution affect its conductivity.
\item There are three main types of reactions that occur in aqueous solutions. These are precipitation reactions, acid-base reactions and redox reactions.
\label{m38719*uid8923}\item 
Precipitation and acid-base reactions are sometimes known as ion exchange reactions. Ion exchange reactions also include gas forming reactions. Ion exchange reactions are a type of reaction where the positive ions exchange their respective negative ions due to a driving force.
\label{m38719*uid104}\item A \textbf{precipitate} is formed when ions in solution react with each other to form an insoluble product. Solubility rules help to identify the precipitate that has been formed.
\label{m38719*uid105}\item A number of tests can be used to identify whether certain \textbf{anions} (chlorides, bromides, iodides, carbonates, sulphates) are present in a solution.
\label{m38719*id813}\item An acid-base reaction is one in which an acid reacts with a base to form a salt and water.
\label{m38719*uid823}\item A redox reaction is one in which electrons are transferred from one substance to another. 
\end{itemize}
\label{m38719*eip-896}
            \begin{eocexercises}{Reactions in aqueous solutions}
            \nopagebreak
            \label{m38719*id342869}\begin{enumerate}[noitemsep, label=\textbf{\arabic*}. ] 
%Q1
            \label{m38719*uid107}\item Give one word for each of the following descriptions:
\label{m38719*id342885}\begin{enumerate}[noitemsep, label=\textbf{\alph*}. ] 
            \label{m38719*uid108}\item the change in phase of water from a gas to a liquid
\label{m38719*uid109}\item a charged atom
\label{m38719*uid110}\item a term used to describe the mineral content of water
\label{m38719*uid111}\item a gas that forms sulphuric acid when it reacts with water
\end{enumerate}
%Q2
\label{m38719*uid112}\item Match the information in column A with the information in column B by writing only the letter (A to I) next to the question number (1 to 7)
    % \textbf{m38719*id342952}\par
          \begin{table}[H]
    % \begin{table}[H]
    % \\ 'id2965514' '1'
        \begin{center}
      \label{m38719*id342952}
    \noindent
      \begin{tabular}{|l|l|}\hline
        \textbf{Column A} &
        \textbf{Column B} \\ \hline
        1. A polar molecule &
        A. ${\text{H}}_{2}{\text{SO}}_{4}$ \\ \hline
        2. Molecular solution &
        B. ${\text{CaCO}}_{3}$ \\ \hline
        3. Mineral that increases water hardness &
        C. $\text{NaOH}$ \\ \hline
        4. Substance that increases the hydrogen ion concentration &
        D. salt water \\ \hline
        5. A strong electrolyte &
        E. calcium \\ \hline
        6. A white precipitate &
        F. carbon dioxide \\ \hline
        7. A non-conductor of electricity &
        G. potassium nitrate \\ \hline
         &
        H. sugar water \\ \hline
         &
        I. ${\text{O}}_{2}$ \\ \hline
    \end{tabular}
      \end{center}
\end{table}
    \par
%Q3
        \item Explain the difference between a weak electrolyte and a strong electrolyte. Give a generalised equation for each.\newline
%Q4
            \item What factors affect the conductivity of water? How do each of these affect the conductivity?\newline
%Q5
            \item For each of the following substances state whether they are molecular or ionic. If they are ionic, give a balanced reaction for the dissociation in water.\label{m38719*id7342}\begin{enumerate}[noitemsep, label=\textbf{\alph*}. ] 
            \item methane (${\text{CH}}_{4}$)\item potassium bromide\item carbon dioxide\item hexane (${\text{C}}_{6}{\text{H}}_{14}$)\item lithium fluoride ($\text{LiF}$)\item magnesium chloride\end{enumerate}
%Q6
\label{m38719*uid127}\item Three test tubes (X, Y and Z) each contain a solution of an unknown potassium salt. The following observations were made during a practical investigation to identify the solutions in the test tubes:\\
A: A white precipitate formed when silver nitrate (${\text{AgNO}}_{3}$) was added to test tube Z.\\
B: A white precipitate formed in test tubes X and Y when barium chloride (${\text{BaCl}}_{2}$) was added.\\
C: The precipitate in test tube X dissolved in hydrochloric acid ($\text{HCl}$) and a gas was released.\\
D: The precipitate in test tube Y was insoluble in hydrochloric acid.
\label{m38719*id343466}\begin{enumerate}[noitemsep, label=\textbf{\alph*}. ] 
            \label{m38719*uid128}\item Use the above information to identify the solutions in each of the test tubes X, Y and Z.
\label{m38719*uid129}\item Write a chemical equation for the reaction that took place in test tube X before hydrochloric acid was added.
\end{enumerate}
(DoE Exemplar Paper 2 2007)
\end{enumerate}
\practiceinfo
\par 
 \par \begin{tabular}[h]{cccccc}
 (1.) 007b  &  (2.) 007c  &  (3.) 007d  &  (4.) 007e  &  (5.) 007f  &  (6.) 007g  &    & \end{tabular}
\end{eocexercises}
