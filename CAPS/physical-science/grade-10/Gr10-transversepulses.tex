         \chapter{Transverse pulses}\fancyfoot[LO,RE]{Physics: Waves, Sound and Light}
    \setcounter{figure}{1}
    \setcounter{subfigure}{1}
    \label{21d48a6f8839b4b265192acd9ea3d978}
         \section{Introduction and key concepts}
    \nopagebreak
%            \label{m38801} $ \hspace{-5pt}\begin{array}{cccccccccccc}   \includegraphics[width=0.75cm]{col11305.imgs/summary_fullmarks.png} &   \end{array} $ \hspace{2 pt}\raisebox{-5 pt}{} {(section shortcode: P10037 )} \par 
    \label{m38801*cid2}
%             \subsection*{Introduction}
%             \nopagebreak
      \label{m38801*id312450}This chapter forms the basis of the discussion into mechanical waves in the following chapters. We begin by discussing pulses. Pulses are disturbances in a medium. If you tap water in a bucket with your finger, notice that a ripple moves away from the point where you touched the water. The ripple is a pulse moving away from where you touched the water. 
\chapterstartvideo{VPchw}
%Waves do not only occur in water, they occur in any kind of medium. Earthquakes release enough energy to create waves that are powerful enough to travel through the rock of the Earth. When your friend speaks to you \textsl{sound waves} are produced that travel through the air to your ears. Light is made up of electromagnetic waves. A wave is simply the disturbance of a medium by moving energy.\par 

    \label{m38801*cid3}
            \subsection*{What is a \textsl{medium}?}
            \nopagebreak
\begin{minipage}{.5\textwidth}
      \label{m38801*id312816}A medium is the substance or material through which a pulse moves. The medium carries the pulse from one place to another. The medium does not create the pulse and the medium is not the pulse. Therefore the medium does not travel with the pulse as the pulse moves through it. \par %Air is a medium for sound waves, water is a medium for water waves and rock is a medium for earthquakes. Air, water and rock are therefore examples of media (media is the plural of medium).\par 
\label{m38801*id312841}In each medium, the particles that make up the medium are moved \textsl{temporarily} from their rest position. In order for a pulse to travel, the different parts of the medium must be able to interact with each other.\par 

\end{minipage}
\begin{minipage}{.5\textwidth}
\begin{center}
\textbf{A disturbance in water}\par
 \includegraphics[width=.8\textwidth]{photos/waveby-mikeyskatie-flickr.jpg}\par
\textit{\small Picture by mikeyskatie on Flickr.com}
\end{center}
\end{minipage}

\Definition{Medium } { A medium is the substance or material in which a pulse will move. } 
           
\section{Pulses: amplitude and length}

\subsection*{What is a \textsl{pulse}?}
    \nopagebreak
    \label{m38801*secfhsst!!!underscore!!!id58}
    %\begin{investigation}{Investigation: Observation of Pulses }
\begin{Investigation}{Observation of pulses }

    \nopagebreak
    \label{m38801*id312873}Take a heavy rope. Have two people hold the rope stretched out horizontally. Flick the rope at one end only once.\par 
    \label{m38801*id312879}
    \begin{figure}[H]
	\nonumber
        \begin{center}
            \begin{pspicture}(-0.2,0)(5,1.4)
                \rput(0,0.8){\rope}
                \uput[d](2.5,0.8){flick rope upwards at one end, once only}
                \rput(-0.2,0){\psline{->}(0,0.6)(0,1.2)}
            \end{pspicture}
        \end{center}
    \end{figure}
    
    \par 
    \label{m38801*id312888}What happens to the disturbance that you created in the rope? Does it stay at the place where it was created or does it move down the length of the rope? \par 

    %}% End of Investigation
    \end{Investigation}

    \label{m38801*id312898}In the activity, we created a \textsl{pulse}. A pulse is a \textsl{single} disturbance that moves through a medium. In a transverse pulse the displacement of the medium is perpendicular to the direction of motion of the pulse. Figure~\ref{m38801*uid2!!!underscore!!!media}
 shows an example of a transverse pulse. In the activity, the rope or spring was held horizontally and the pulse moved the rope up and down. This was an example of a transverse pulse.\par

    \Definition{Pulse} {A pulse is a single disturbance that moves through a medium.   } 

    \Definition{Transverse Pulse} {A pulse where all of the particles disturbed by the pulse move perpendicular (at a right angle) to the direction in which the pulse is moving.    } 
     
    \label{m38801*uid1}

\subsubsection*{Pulse length and amplitude}
\nopagebreak
\label{m38801*id312946}The amplitude of a pulse is a measurement of how far the medium is displaced momentarily from a position of rest. The pulse length is a measurement of how long the pulse is. Both these quantities are shown in Figure~\ref{m38801*uid2!!!underscore!!!media}.
\Definition{Amplitude} {The amplitude of a pulse is the maximum disturbance or distance the medium is displaced from its rest (equilibrium) position. \\
Quantity: Amplitude (A) \hspace{1cm} Unit name: metre \hspace{1cm} Unit symbol: $m$} 
\begin{figure}[h]
    \begin{center}
        \begin{pspicture}(0,-0.6)(5,2.2)
            \pnode(1,0){b}
            \pnode(2.5,0){c}
            \pnode(3.5,0){d}
            \pnode(5,0){e}
            \psline(b)(c)
            \rput(c){\psplot[xunit=0.034]{0}{30}{6 x mul sin 2 mul}}
            \psline(d)(e)
            \psline[linestyle=dotted]{<->}(2.4,0)(2.4,2)
            \uput[l](2.3,1){amplitude (A)}
            \psline[linestyle=dotted]{<->}(2.5,-0.1)(3.5,-0.1)
            \uput[d](3,0){pulse length}
            \psline{->}(-0.3,0)(0.7,0)
            \uput[l](-0.3,0){position of rest}
        \end{pspicture}
    \end{center}
\caption{Example of a transverse pulse}
\label{m38801*uid2!!!underscore!!!media}
\end{figure}


\label{m38801*eip-400}The position of rest is the position the medium would be in if it were undisturbed. This is also called the equilibrium position. People will often use rest and equilibrium interchangeably.\par \label{m38801*secfhsst!!!underscore!!!id87}
            \begin{Investigation}{Pulse length and amplitude }
            \nopagebreak
        \label{m38801*id312993}The graphs below show the positions of a pulse at different times.\par 
        \label{m38801*id312998}
    \setcounter{subfigure}{0}
\begin{figure}[H] % horizontal\label{m38801*id313002}
%    \begin{center}
%    \label{m38801*id313002!!!underscore!!!media}\label{m38801*id313002!!!underscore!!!printimage}\includegraphics[width=300px]{col11305.imgs/m38801_PG10C4_003.png} % m38801;PG10C4\_003.png;;;6.0;8.5;
%      \vspace{2pt}
%    \vspace{.1in}
%    \end{center}
\begin{center}
\begin{pspicture}(0,-4.2)(5,1.2)
%\psgrid[gridcolor=lightgray]
\def\pulse{\psplot[xunit=0.017]{0}{30}{6 x mul sin}
\pcline{|-|}(0,0)(0,1)\uput[l](0,0.5){$A$}
\pcline[offset=-5pt]{|-|}(0,0)(0.5,0)\uput[d](0.25,-0.1){$p$}}
\rput(0,0){\pulse}\psline(0.5,0)(5,0)
\psline(0,-1.2)(1,-1.2)\rput(1,-1.2){\pulse}\psline(1.5,-1.2)(5,-1.2)
\psline(0,-2.4)(2,-2.4)\rput(2,-2.4){\pulse}\psline(2.5,-2.4)(5,-2.4)
\psline(0,-3.6)(3,-3.6)\rput(3,-3.6){\pulse}\psline(3.5,-3.6)(5,-3.6)
\uput[ur](5,0){$t$=0~s}
\uput[ur](5,-1.2){$t$=1~s}
\uput[ur](5,-2.4){$t$=2~s}
\uput[ur](5,-3.6){$t$=3~s}
\end{pspicture}
\end{center}
\end{figure}       
        \par 
        \label{m38801*id313008}Use your ruler to measure the lengths of $a$ and $p$. Fill your answers in the table.\par 
    % \textbf{m38801*id313027}\par
          \begin{table}[H]
    % \begin{table}[H]
    % \\ '' '0'
        \begin{center}
      \label{m38801*id313027}
      \tablelasttail{}
      \begin{xtabular}[t]{|l|l|l|}\hline
        Time &
                  $A$
                 &
                  $p$
                % make-rowspan-placeholders
     \tabularnewline\cline{1-1}\cline{2-2}\cline{3-3}
      %--------------------------------------------------------------------
        $t=0$~s &
         &
        % make-rowspan-placeholders
     \tabularnewline\cline{1-1}\cline{2-2}\cline{3-3}
      %--------------------------------------------------------------------
        $t=1$~s &
         &
        % make-rowspan-placeholders
     \tabularnewline\cline{1-1}\cline{2-2}\cline{3-3}
      %--------------------------------------------------------------------
        $t=2$~s &
         &
        % make-rowspan-placeholders
     \tabularnewline\cline{1-1}\cline{2-2}\cline{3-3}
      %--------------------------------------------------------------------
        $t=3$~s &
         &
        % make-rowspan-placeholders
     \tabularnewline\cline{1-1}\cline{2-2}\cline{3-3}
      %--------------------------------------------------------------------
    \end{xtabular}
      \end{center}
\end{table}
    \par
        \label{m38801*id313222}What do you notice about the values of $A$ and $p$?
 \par 
        \label{m38801*id313246}In the activity, we found that the values for how high the pulse ($A$) is and how wide the pulse ($p$) is the same at different times. \textsl{Pulse length} and \textsl{amplitude} are two important quantities of a pulse.\par 
      \label{m38801*uid3}

\end{Investigation}

            \subsubsection*{Pulse speed}
            \nopagebreak

           \Definition{Pulse speed}{Pulse speed is the distance a pulse travels per unit time. \\\noindent
	    Quantity: Pulse speed ($v$) \hspace{1cm} Unit name: metre per second \hspace{1cm} Unit symbol: $\text{m}\cdot \text{s}^{-1}$  } 
        
\label{m38801*id313303}Speed is defined as the distance travelled per unit time (this will be covered in more detail in Motion in One Dimension). If the pulse travels a distance $D$ in a time $t$, then the pulse speed $v$ is:\par 
        \label{m38801*uid4}\nopagebreak\noindent{}
    \begin{equation}
    \boxed{v=\frac{D}{t}}\nonumber
      \end{equation}

    \begin{wex}{Pulse speed}{A pulse covers a distance of $2\phantom{\rule{3pt}{0ex}}\text{m}$ in $4\phantom{\rule{3pt}{0ex}}\text{s}$ on a heavy rope. Calculate the pulse speed.}{% Answer
        \westep{Analyse the question}{ 
        \label{m38801*id313393}We are given:\par 
        
\begin{minipage}{\textwidth}
        \label{m38801*id313396}
        \begin{itemize}[noitemsep]
            \item the distance travelled by the pulse: $D=2\phantom{\rule{2pt}{0ex}}\text{m}$
            \item the time taken to travel $2\phantom{\rule{2pt}{0ex}}\text{m}$: $t=4\phantom{\rule{2pt}{0ex}}\text{s}$
        \end{itemize}
        \label{m38801*id313439}We are required to calculate the speed of the pulse.
\end{minipage}
        }
        
        \westep{Apply the relevant principles}{
        \label{m38801*id313447}We can use:\par 
        \label{m38801*id313450}\nopagebreak\noindent{}
    \begin{equation}
    v=\frac{D}{t}\nonumber
    \end{equation}
        \label{m38801*id313470}to calculate the speed of the pulse.\par }

    
        \westep{Do the calculation}{  
        \label{m38801*id313479}\nopagebreak\noindent{}
    \begin{align*}
    v&= \frac{D}{t}\hfill \\ 
    &= \frac{2\phantom{\rule{0.166667em}{0ex}}\text{m}}{4\phantom{\rule{0.166667em}{0ex}}\text{s}}\hfill \\ 
    &= 0,5\phantom{\rule{0.166667em}{0ex}}\text{m}\ensuremath{\cdot}{\text{s}}^{-1}
      \end{align*}}

        \westep{Quote the final result}{  
        \label{m38801*id313763}The pulse speed is $0,5\phantom{\rule{2pt}{0ex}}\text{m}\ensuremath{\cdot}\text{s}{}^{-1}$.
    } 
    }   
    \end{wex}


    \noindent
\label{m38801*notfhsst!!!underscore!!!id259}
      \Tip{The pulse speed depends on the properties of the medium and not on the amplitude or pulse length of the pulse.}
\label{m38801*secfhsst!!!underscore!!!id260}
            \begin{exercises}{  Pulse Speed }\noindent\vspace{-1cm}
        \label{m38801*id313813}\begin{enumerate}[noitemsep, label=\textbf{\arabic*}. ] 
            \label{m38801*uid7}\item A pulse covers a distance of $5\phantom{\rule{2pt}{0ex}}\text{m}$ in $15\phantom{\rule{2pt}{0ex}}\text{s}$. Calculate the speed of the pulse.\newline
\label{m38801*uid8}\item A pulse has a speed of $5\phantom{\rule{2pt}{0ex}}\text{cm}\ensuremath{\cdot}\text{s}{}^{-1}$. How far does it travel in $2,5\phantom{\rule{2pt}{0ex}}\text{s}$?\newline
\label{m38801*uid9}\item A pulse has a speed of $0,5\phantom{\rule{2pt}{0ex}}\text{m}\ensuremath{\cdot}\text{s}{}^{-1}$. How long does it take to cover a distance of $25\phantom{\rule{2pt}{0ex}}\text{cm}$?\newline
\label{m38801*uid10}\item How long will it take a pulse moving at $0,25\phantom{\rule{2pt}{0ex}}\text{m}\ensuremath{\cdot}\text{s}{}^{-1}$ to travel a distance of $20\phantom{\rule{2pt}{0ex}}\text{m}$?\newline
\label{m38801*uid11}\item The diagram shows two pulses in the same medium. Which has the higher speed? Explain your answer.
	\begin{figure}[H] % horizontal\label{m38801*id313945}
   \begin{center}
\begin{pspicture*}(-0.6,-0.6)(4,1.8)
\psgrid[gridcolor=lightgray]
\psset{xunit=0.0055}
\rput(0,-0.4){\psplot{0}{180}{x sin 0.5 mul}\uput[l](0,0){A}\psline(180,0)(720,0)}
\rput(0,0.6){\psplot{0}{180}{x sin 1.1 mul}\uput[l](0,0){B}\psline(180,0)(720,0)}
\end{pspicture*}
\end{center}
 \end{figure}               

\end{enumerate}
  \label{m38801**end}
\par \practiceinfo
 \par \begin{tabular}[h]{cccccc}
 (1.) 002q  &  (2.) 002r  &  (3.) 002s  &  (4.) 002t  &  (5.) 002u  & \end{tabular}

\end{exercises}



         \section{Superposition of pulses}
    \nopagebreak
%            \label{m38802} $ \hspace{-5pt}\begin{array}{cccccccccccc}   \includegraphics[width=0.75cm]{col11305.imgs/summary_fullmarks.png} &   \includegraphics[width=0.75cm]{col11305.imgs/summary_video.png} &   \end{array} $ \hspace{2 pt}\raisebox{-5 pt}{} {(section shortcode: P10038 )} \par 
\label{m38802*fs-id1166232432154}
%             \subsection*{Superposition of pulses}
%             \nopagebreak
\mindsetvid{Superposition}{VPciu}
      \label{m38802*id316136}Two or more pulses can pass through the same medium at that same time in the same place. When they do they interact with each other to form a different disturbance at that point. The resulting pulse is obtained by using the \textsl{principle of superposition}. 
\Definition{Principle of superposition}{
The principle of superposition states that when two disturbance occupy the same space at the same time the resulting disturbance is the sum of two disturbances.} 

After pulses pass through each other, each pulse continues along its original direction of travel, and their original amplitudes remain unchanged.\par 
      \label{m38802*id316148}Constructive interference takes place when two pulses meet each other to create a larger pulse. The amplitude of the resulting pulse is the sum of the amplitudes of the two initial pulses. This could be two crests meeting or two troughs meeting. This is shown in Figure~\ref{m38802*uid53!!!underscore!!!media}.\par 

\Definition{Constructive interference} { \label{m38802*meaningfhsst!!!underscore!!!id567}
      Constructive interference is when two pulses meet, resulting in a bigger pulse. 
       } 
     \setcounter{subfigure}{0}
	\begin{figure}[H] % horizontal\label{m38802*uid53}
\begin{center}
    \scalebox{0.7}
    {
    \begin{pspicture}(0,-5)(4,1.8)
    %\psgrid[gridcolor=lightgray]
    \def\pulse{\psline[linecolor=white,linestyle=solid,linewidth=2pt](0,0)(1.02,0)\psplot[xunit=0.034]{0}{30}{6 x mul sin}}
\psline(2.,1.2,)(2.,-5.)
    \rput(-3,0){\uput[u](2,1.2){pulses move towards each other}
    \psline(0,0)(4,0)
    \rput(0,0){\psline{->}(0,1.2)(0.5,1.2)\pulse}
    \psline{<-}(3.5,1.2)(4,1.2)\rput(3,0){\pulse}}

    \rput(-3,-3){\uput[u](2,2.2){pulses constructively interfere}
    \psline(0,0)(4,0)
    \rput(1.5,0){\psset{yunit=2}\pulse}}

    \rput(-3,-5){\uput[u](2,1.2){pulses move away from other}
    \psline(0,0)(4,0)
    \rput(0,0){\psline{<-}(0,1.2)(0.5,1.2)\pulse}
    \psline{->}(3.5,1.2)(4,1.2)\rput(3,0){\pulse}}
%     \end{pspicture}
%     }
% \end{minipage}
% \begin{minipage}{.5\textwidth}
%     \scalebox{0.7}
%     {
%     \begin{pspicture}(0,-5)(4,1.8)
%     %\psgrid[gridcolor=lightgray]
    \def\pulse{\psline[linecolor=white,linestyle=solid,linewidth=2pt](0,0)(1.02,0)\psplot[xunit=0.034]{0}{30}{-6 x mul sin}}

    \rput(3,0){\uput[u](2,1.2){pulses move towards each other}
    \psline(0,1)(4,1)
    \rput(0,1){\psline{->}(0,.2)(0.5,0.2)\pulse}
    \psline{<-}(3.5,1.2)(4,1.2)\rput(3,1){\pulse}}

    \rput(3,-3){\uput[u](2,2.2){pulses constructively interfere}
    \psline(0,2)(4,2)
    \rput(1.5,2){\psset{yunit=2}\pulse}}

    \rput(3,-5){\uput[u](2,1.2){pulses move away from other}
    \psline(0,1)(4,1)
    \rput(0,1){\psline{<-}(0,.2)(0.5,0.2)\pulse}
    \psline{->}(3.5,1.2)(4,1.2)\rput(3,1){\pulse}}
    \end{pspicture}
    }
\end{center}    
\caption{Superposition of two pulses: constructive interference.}
\label{m38802*uid53!!!underscore!!!media}
 \end{figure}       


      \label{m38802*id316190}Destructive interference takes place when two pulses meet and result in a smaller amplitude disturbance. The amplitude of the resulting pulse is the sum of the amplitudes of the two initial pulses, but the one amplitude will be a negative number. This is shown in Figure~\ref{m38802*uid54!!!underscore!!!media}. In general, amplitudes of individual pulses are summed together to give the amplitude of the resultant pulse.\par 

\Definition{Destructive interference} { \label{m38802*meaningfhsst!!!underscore!!!id578}
      Destructive interference is when two pulses meet, resulting in a smaller pulse. 
       } 
    

\begin{figure}[H] % horizontal\label{m38802*uid54}
    \begin{center}
    \scalebox{0.7}
    {
        \begin{pspicture}(0,-7)(10,1.8)
            %\psgrid[gridcolor=lightgray]
            \def\pulse{\psline[linecolor=white,linestyle=solid,linewidth=2pt](0,0)(1.02,0)\psplot[xunit=0.034]{0}{30}{6 x mul sin}}
            \def\dotpulse{\psplot[linestyle=dashed,xunit=0.034]{0}{30}{6 x mul sin}}
	    \psline(5,1.2,)(5,-6)
            \uput[u](2,1.2){pulses move towards each other}
            \psline(0,0)(4,0)
            \psline{->}(0,1.2)(0.5,1.2)
            \rput(0,0){\pulse}
            \psline{<-}(3.5,1.2)(4,1.2)
            \rput{180}(4,0){\pulse}

            \rput(0,-3){
            \uput[u](2,1.2){pulses destructively interfere}
            \psline(0,0)(4,0)
            \rput{180}(2.5,0){\dotpulse}
            \rput(1.5,0){\dotpulse}
            }

            \rput(0,-6){
            \uput[u](2,1.2){pulses move away from other}
            \psline(0,0)(4,0)
            \psline{<-}(0,1.2)(0.5,1.2)
            \rput{180}(1,0){\pulse}
            \psline{->}(3.5,1.2)(4,1.2)
            \rput(3,0){\pulse}
            }
            %% second pic

            \uput[u](8,1.2){pulses move towards each other}
            \psline(6,-0.5)(10,-0.5)
            \psline{->}(6,1.1)(6.5,1.1)
            \rput(6,-0.5){\psset{yunit=1.5}\pulse}
            \psline{<-}(9.5,1.1)(10,1.1)
            \rput{180}(10,-0.5){\pulse}

            \rput(6,-3){\uput[u](2,0.6){pulses interfere}
            \psline(0,0)(4,0)
            \rput(1.5,0){\psset{yunit=0.5}\pulse}}

            \rput(6,-6){
            \uput[u](2,1.6){pulses move away from other}
            \psline(0,0)(4,0)
            \psline{<-}(0,1.6)(0.5,1.6)
            \rput{180}(1,0){\pulse}
            \psline{->}(3.5,1.6)(4,1.6)
            \rput(3,0){\psset{yunit=1.5}\pulse}}

        \end{pspicture}
    }
    \end{center}
\caption{Superposition of two pulses. The left-hand series of images demonstrates destructive interference, since the pulses cancel each other. The right-hand series of images demonstrate a partial cancellation of two pulses, as their amplitudes are not the same in magnitude.}
\label{m38802*uid54!!!underscore!!!media}
 \end{figure}       


\begin{wex}{Superposition of pulses}{The two pulses shown below approach each other at 1~\ms. Draw what the waveform would look like after 1~s, 2~s and 5~s.
\begin{center}
\begin{pspicture}(-1,-1)(8.6,2.6)
%\psgrid[gridcolor=lightgray]
\psaxes{<->}(0,0)(8.5,2.6)
\pcline[offset=0.4cm,linestyle=none](0,0)(0,2.6)
\aput{:U}{amplitude (m)}
\pcline[offset=-0.4cm,linestyle=none](0,0)(8.5,0)
\bput{:U}{distance (m)}
\rput(2,0){\psline(0,0)(0,1)(1,1)(1,0)\pcline{->}(0.2,1.2)(0.8,1.2)\aput{:U}{A}}
\rput(6,0){\psline(0,0)(0,1)(1,1)(1,0)\pcline{<-}(0.2,1.2)(0.8,1.2)\aput{:U}{B}}
\end{pspicture}
\end{center}}{\westep{After 1~s}
After 1~s, pulse A has moved 1~m to the right and pulse B has moved 1~m to the left.
\begin{center}
\begin{pspicture}(-1,-1)(8.6,2.6)
%\psgrid[gridcolor=lightgray]
\psaxes{<->}(0,0)(8.5,2.6)
\pcline[offset=0.4cm,linestyle=none](0,0)(0,2.6)
\aput{:U}{amplitude (m)}
\pcline[offset=-0.4cm,linestyle=none](0,0)(8.5,0)
\bput{:U}{distance (m)}
\rput(3,0){\psline(0,0)(0,1)(1,1)(1,0)\pcline{->}(0.2,1.2)(0.8,1.2)\aput{:U}{A}}
\rput(5,0){\psline(0,0)(0,1)(1,1)(1,0)\pcline{<-}(0.2,1.2)(0.8,1.2)\aput{:U}{B}}
\end{pspicture}
\end{center}

\westep{After 2~s}
After 1~s more, pulse A has moved 1~m to the right and pulse B has moved 1~m to the left.
\begin{center}
\begin{pspicture}(-1,-1)(8.6,2.6)
%\psgrid[gridcolor=lightgray]
\psaxes{<->}(0,0)(8.5,2.6)
\pcline[offset=0.4cm,linestyle=none](0,0)(0,2.6)
\aput{:U}{amplitude (m)}
\pcline[offset=-0.4cm,linestyle=none](0,0)(8.5,0)
\bput{:U}{distance (m)}
\rput(4,0){\psline(0,0)(0,2)(1,2)(1,0)\pcline[linestyle=none](0.2,2.2)(0.8,2.2)\aput{:U}{A+B}}
\end{pspicture}
\end{center}

\westep{After 5~s}
After 5~s more, pulse A has moved 5~m to the right and pulse B has moved 5~m to the left.
\begin{center}
\begin{pspicture}(-1,-1)(8.6,2.6)
%\psgrid[gridcolor=lightgray]
\psaxes{<->}(0,0)(8.5,2.6)
\pcline[offset=0.4cm,linestyle=none](0,0)(0,2.6)
\aput{:U}{amplitude (m)}
\pcline[offset=-0.4cm,linestyle=none](0,0)(8.5,0)
\bput{:U}{distance (m)}
\rput(7,0){\psline(0,0)(0,1)(1,1)(1,0)\pcline{->}(0.2,1.2)(0.8,1.2)\aput{:U}{A}}
\rput(1,0){\psline(0,0)(0,1)(1,1)(1,0)\pcline{<-}(0.2,1.2)(0.8,1.2)\aput{:U}{B}}
\end{pspicture}
\end{center}}\end{wex}
    
%    \noindent
%\label{m38802*notfhsst!!!underscore!!!id635}
%\Tip{The idea of superposition is one that occurs often in physics. You will see \textsl{much, much more} of superposition!}
	

\begin{g_experiment}{Constructive and destructive interference}

\textbf{Aim:} To demonstrate constructive and destructive interference
\par 
\label{m38802*eip7241}\noindent{}\textbf{Apparatus: } 
Ripple tank apparatus
    \setcounter{subfigure}{0}
	\begin{figure}[H] % horizontal\label{m38802*id63458}
    \begin{center}
    \label{m38802*id63458!!!underscore!!!media}\label{m38802*id63458!!!underscore!!!printimage}\includegraphics[width=0.8\columnwidth]{col11305.imgs/m38802_rippletray.png} % m38802;rippletray.png;;;6.0;8.5;
    \end{center}
 \end{figure}       \par 
\label{m38802*eip7474}\noindent{}\textbf{Method:}
\label{m38802*id6242}\begin{enumerate}[noitemsep, label=\textbf{\arabic*}. ] 
            \item Set up the ripple tank	
	    \item Produce a single pulse and observe what happens (you can do this any means, tapping the water with a finger, dropping a small object into the water, tapping a ruler or even using a electronic vibrator)
	  \item Produce two pulses simultaneously and observe what happens
	  \item Produce two pulses at slightly different times and observe what happens\end{enumerate}
\par 
\label{m38802*id614134}\noindent{}\textbf{Results and conclusion:} You should observe that when you produce two pulses simultaneously you see them interfere constructively and when you produce two pulses at slightly different times you see them interfere destructively.
\par \label{m38802*secfhsst!!!underscore!!!id636}
\end{g_experiment}


            \begin{exercises}{ Problems involving superposition of pulses }\noindent\vspace{-1cm}
            \label{m38802*id316401}\begin{enumerate}[noitemsep, label=\textbf{\arabic*}. ] 
            \label{m38802*uid55}\item For the following pulse, draw the resulting wave forms after $1\phantom{\rule{2pt}{0ex}}\text{s}$, $2\phantom{\rule{2pt}{0ex}}\text{s}$, $3\phantom{\rule{2pt}{0ex}}\text{s}$, $4\phantom{\rule{2pt}{0ex}}\text{s}$ and $5\phantom{\rule{2pt}{0ex}}\text{s}$. Each pulse is travelling at $1\phantom{\rule{2pt}{0ex}}\text{m}\ensuremath{\cdot}\text{s}{}^{-1}$. Each block represents $1\phantom{\rule{2pt}{0ex}}\text{m}$. The pulses are shown as thick black lines and the undisplaced medium as dashed lines.
    \setcounter{subfigure}{0}
	\begin{figure}[H] % horizontal\label{m38802*id316460}
    \begin{center}
\begin{pspicture}(-0.5,-1.4)(10,1.4)
\rput{90}(-0.5){Amplitude (m)}
\rput[r](-0.05,0){0}
\rput[r](-0.05,1){1}
\rput[l](1.0,-.15){1}
\rput[l](2.0,-.15){2}
\rput[l](3.0,-.15){3}
\rput[l](4.0,-.15){4}
\rput[l](5.0,-.15){5}
\rput[l](6.0,-.15){6}
\rput[l](7.0,-.15){7}
\rput[l](8.0,-.15){8}
\rput[l](9.0,-.15){9}
\rput(8.5,-1.3){Position (m)}
\psgrid[gridcolor=lightgray,gridlabels=0,subgriddiv=1](0,-1)(10,1)
\psline[linestyle=dashed](0,0)(2,0)(2,1)(4,1)(4,0)(6,0)(6,1)(8,1)(8,0)(10,0)
\psline[linewidth=0.08cm](2,0)(2,1)(4,1)(4,0)
\psline[linewidth=0.08cm](6,0)(6,1)(8,1)(8,0)
\psline{->}(2,1.2)(2.5,1.2)
\psline{<-}(7.5,1.2)(8,1.2)
\uput[ur](0,0.4){$t$=0~s}
\end{pspicture}
\end{center}
 \end{figure}               \label{m38802*uid57}\item For the following pulse, draw the resulting wave forms after $1\phantom{\rule{2pt}{0ex}}\text{s}$, $2\phantom{\rule{2pt}{0ex}}\text{s}$, $3\phantom{\rule{2pt}{0ex}}\text{s}$, $4\phantom{\rule{2pt}{0ex}}\text{s}$ and $5\phantom{\rule{2pt}{0ex}}\text{s}$. Each pulse is travelling at $1\phantom{\rule{2pt}{0ex}}\text{m}\ensuremath{\cdot}\text{s}{}^{-1}$. Each block represents $1\phantom{\rule{2pt}{0ex}}\text{m}$. The pulses are shown as thick black lines and the undisplaced medium as dashed lines.
    \setcounter{subfigure}{0}
	\begin{figure}[H] % horizontal\label{m38802*id316477}
    \begin{center}
\begin{pspicture}(-0.5,-1.4)(10,1.4)
\rput{90}(-0.5){Amplitude (m)}
\rput[r](-0.05,0){0}
\rput[r](-0.05,1){1}
\rput[l](1.0,-.15){1}
\rput[l](2.0,-.15){2}
\rput[l](3.0,-.15){3}
\rput[l](4.0,-.15){4}
\rput[l](5.0,-.15){5}
\rput[l](6.0,-.15){6}
\rput[l](7.0,-.15){7}
\rput[l](8.0,-.15){8}
\rput[l](9.0,-.15){9}
\rput(8.5,-1.3){Position (m)}
\psgrid[gridcolor=lightgray,gridlabels=0,subgriddiv=1](0,-1)(10,1)
\psline[linestyle=dashed](0,0)(2,0)(2,1)(4,1)(4,0)(6,0)(6,-1)(8,-1)(8,0)(10,0)
\psline[linewidth=0.08cm](2,0)(2,1)(4,1)(4,0)
\psline[linewidth=0.08cm](6,0)(6,-1)(8,-1)(8,0)
\psline{->}(2,1.2)(2.5,1.2)
\psline{<-}(7.5,1.2)(8,1.2)
\uput[ur](0,0.4){$t$=0~s}
\end{pspicture}
\end{center}
 \end{figure}               \label{m38802*uid58}\item For the following pulse, draw the resulting wave forms after $1\phantom{\rule{2pt}{0ex}}\text{s}$, $2\phantom{\rule{2pt}{0ex}}\text{s}$, $3\phantom{\rule{2pt}{0ex}}\text{s}$, $4\phantom{\rule{2pt}{0ex}}\text{s}$ and $5\phantom{\rule{2pt}{0ex}}\text{s}$. Each pulse is travelling at $1\phantom{\rule{2pt}{0ex}}\text{m}\ensuremath{\cdot}\text{s}{}^{-1}$. Each block represents $1\phantom{\rule{2pt}{0ex}}\text{m}$. The pulses are shown as thick black lines and the undisplaced medium as dashed lines.
    \setcounter{subfigure}{0}
	\begin{figure}[H] % horizontal\label{m38802*id316495}
    \begin{center}
\begin{pspicture}(-0.5,-1.4)(10,1.4)
\rput{90}(-0.5){Amplitude (m)}
\rput[r](-0.05,0){0}
\rput[r](-0.05,1){1}
\rput[l](1.0,-.15){1}
\rput[l](2.0,-.15){2}
\rput[l](3.0,-.15){3}
\rput[l](4.0,-.15){4}
\rput[l](5.0,-.15){5}
\rput[l](6.0,-.15){6}
\rput[l](7.0,-.15){7}
\rput[l](8.0,-.15){8}
\rput[l](9.0,-.15){9}
\rput(8.5,-1.3){Position (m)}
\psgrid[gridcolor=lightgray,gridlabels=0,subgriddiv=1](0,-1)(10,1)
\psline[linestyle=dashed](0,0)(2,0)(2,1)(3,1)(3,-1)(4,-1)(4,0)(6,0)(6,-1)(7,-1)(7,1)(8,1)(8,0)(10,0)
\psline[linewidth=0.08cm](2,0)(2,1)(3,1)(3,-1)(4,-1)(4,0)
\psline[linewidth=0.08cm](6,0)(6,-1)(7,-1)(7,1)(8,1)(8,0)
\psline{->}(2,1.2)(2.5,1.2)
\psline{<-}(7.5,1.2)(8,1.2)
\uput[ur](0,0.4){$t$=0~s}
\end{pspicture}
\end{center} \end{figure}               \label{m38802*uid59}\item For the following pulse, draw the resulting wave forms after $1\phantom{\rule{2pt}{0ex}}\text{s}$, $2\phantom{\rule{2pt}{0ex}}\text{s}$, $3\phantom{\rule{2pt}{0ex}}\text{s}$, $4\phantom{\rule{2pt}{0ex}}\text{s}$ and $5\phantom{\rule{2pt}{0ex}}\text{s}$. Each pulse is travelling at $1\phantom{\rule{2pt}{0ex}}\text{m}\ensuremath{\cdot}\text{s}{}^{-1}$. Each block represents $1\phantom{\rule{2pt}{0ex}}\text{m}$. The pulses are shown as thick black lines and the undisplaced medium as dashed lines.
    \setcounter{subfigure}{0}
	\begin{figure}[H] % horizontal\label{m38802*id316512}
    \begin{center}
\begin{pspicture}(-0.5,-1.4)(10,1.4)
\rput{90}(-0.5){Amplitude (m)}
\rput[r](-0.05,0){0}
\rput[r](-0.05,1){1}
\rput[l](1.0,-.15){1}
\rput[l](2.0,-.15){2}
\rput[l](3.0,-.15){3}
\rput[l](4.0,-.15){4}
\rput[l](5.0,-.15){5}
\rput[l](6.0,-.15){6}
\rput[l](7.0,-.15){7}
\rput[l](8.0,-.15){8}
\rput[l](9.0,-.15){9}
\rput(8.5,-1.3){Position (m)}
\psgrid[gridcolor=lightgray,gridlabels=0,subgriddiv=1](0,-1)(10,1)
\psline[linestyle=dashed](0,0)(2,0)(2,1)(3,1)(3,-1)(4,-1)(4,0)(6,0)(6,1)(7,1)(7,-1)(8,-1)(8,0)(10,0)
\psline[linewidth=0.08cm](2,0)(2,1)(3,1)(3,-1)(4,-1)(4,0)
\psline[linewidth=0.08cm](6,0)(6,1)(7,1)(7,-1)(8,-1)(8,0)
\psline{->}(2,1.2)(2.5,1.2)
\psline{<-}(7.5,1.2)(8,1.2)
\uput[ur](0,0.4){$t$=0~s}
\end{pspicture}
\end{center}

 \end{figure}               \label{m38802*uid60}\item For the following pulse, draw the resulting wave forms after $1\phantom{\rule{2pt}{0ex}}\text{s}$, $2\phantom{\rule{2pt}{0ex}}\text{s}$, $3\phantom{\rule{2pt}{0ex}}\text{s}$, $4\phantom{\rule{2pt}{0ex}}\text{s}$ and $5\phantom{\rule{2pt}{0ex}}\text{s}$. Each pulse is travelling at $1\phantom{\rule{2pt}{0ex}}\text{m}\ensuremath{\cdot}\text{s}{}^{-1}$. Each block represents $1\phantom{\rule{2pt}{0ex}}\text{m}$. The pulses are shown as thick black lines and the undisplaced medium as dashed lines.
    \setcounter{subfigure}{0}
	\begin{figure}[H] % horizontal\label{m38802*id316530}
    \begin{center}
\begin{pspicture}(-0.5,-1.4)(10,1.4)
\rput{90}(-0.5){Amplitude (m)}
\rput[r](-0.05,0){0}
\rput[r](-0.05,1){1}
\rput[l](1.0,-.15){1}
\rput[l](2.0,-.15){2}
\rput[l](3.0,-.15){3}
\rput[l](4.0,-.15){4}
\rput[l](5.0,-.15){5}
\rput[l](6.0,-.15){6}
\rput[l](7.0,-.15){7}
\rput[l](8.0,-.15){8}
\rput[l](9.0,-.15){9}
\rput(8.5,-1.3){Position (m)}
\psgrid[gridcolor=lightgray,gridlabels=0,subgriddiv=1](0,-1)(10,1)
\psline[linestyle=dashed](0,0)(2,0)(2,1)(5,1)(5,0)(7,0)(7,1)(8,1)(8,0)(10,0)
\psline[linewidth=0.08cm](2,0)(2,1)(5,1)(5,0)
\psline[linewidth=0.08cm](7,0)(7,1)(8,1)(8,0)
\psline{->}(2,1.2)(2.5,1.2)
\psline{<-}(7.5,1.2)(8,1.2)
\uput[ur](0,0.4){$t$=0~s}
\end{pspicture}
\end{center}

 \end{figure}               \label{m38802*uid61}\item For the following pulse, draw the resulting wave forms after $1\phantom{\rule{2pt}{0ex}}\text{s}$, $2\phantom{\rule{2pt}{0ex}}\text{s}$, $3\phantom{\rule{2pt}{0ex}}\text{s}$, $4\phantom{\rule{2pt}{0ex}}\text{s}$ and $5\phantom{\rule{2pt}{0ex}}\text{s}$. Each pulse is travelling at $1\phantom{\rule{2pt}{0ex}}\text{m}\ensuremath{\cdot}\text{s}{}^{-1}$. Each block represents $1\phantom{\rule{2pt}{0ex}}\text{m}$. The pulses are shown as thick black lines and the undisplaced medium as dashed lines.
    \setcounter{subfigure}{0}
	\begin{figure}[H] % horizontal\label{m38802*id316547}
   \begin{center}
\begin{pspicture}(-0.5,-1.4)(10,1.4)
\rput{90}(-0.5){Amplitude (m)}
\rput[r](-0.05,0){0}
\rput[r](-0.05,1){1}
\rput[l](1.0,-.15){1}
\rput[l](2.0,-.15){2}
\rput[l](3.0,-.15){3}
\rput[l](4.0,-.15){4}
\rput[l](5.0,-.15){5}
\rput[l](6.0,-.15){6}
\rput[l](7.0,-.15){7}
\rput[l](8.0,-.15){8}
\rput[l](9.0,-.15){9}
\rput(8.5,-1.3){Position (m)}
\psgrid[gridcolor=lightgray,gridlabels=0,subgriddiv=1](0,-1)(10,1)
\psline[linestyle=dashed](0,0)(2,0)(2,1)(5,1)(5,0)(7,0)(7,-1)(8,-1)(8,0)(10,0)
\psline[linewidth=0.08cm](2,0)(2,1)(5,1)(5,0)
\psline[linewidth=0.08cm](7,0)(7,-1)(8,-1)(8,0)
\psline{->}(2,1.2)(2.5,1.2)
\psline{<-}(7.5,1.2)(8,1.2)
\uput[ur](0,0.4){$t$=0~s}
\end{pspicture}
\end{center}

 \end{figure}               \label{m38802*uid62}\item 
          What is superposition of waves?\newline
\label{m38802*uid64}\item What is constructive interference?\newline
\label{m38802*uid65}\item What is destructive interference?\newline
        \end{enumerate}
% \label{m38802*fs-id1165499443114} The following presentation provides a summary of the work covered in this chapter. Although the presentation is titled waves, the presentation covers pulses only.
%     \setcounter{subfigure}{0}
% 	\begin{figure}[H] % horizontal\label{m38802*slidesharefigure}
%     \label{m38802*slidesharemedia}\label{m38802*slideshareflash}\raisebox{-5 pt}{ \includegraphics[width=0.5cm]{col11305.imgs/summary_www.png}} { (Presentation:  P10039 )}
%  \end{figure}       
% \par 
  \label{m38802*eip-812}
\par \practiceinfo
 \par \begin{tabular}[h]{cccccc}
 (1.) 002v  &  (2.) 002w  &  (3.) 002x  &  (4.) 002y  &  (5.) 002z  &  (6.) 0030  &  (7.) 0031  &  (8.) 0032  &  (9.) 0033  & \end{tabular}

\end{exercises}
%\pagebreak
            \summary{VPcjo}
            \nopagebreak
            \label{m38802*eip-404}\begin{itemize}[noitemsep]
            \item A medium is the substance or material in which a pulse will move.
	    \item A pulse is a single disturbance that moves through a medium.
	    \item The amplitude of a pulse is the maximum disturbance or distance the medium is displaced from its equilibrium position (rest).
	    \item Pulse speed is the distance a pulse travels per unit time.
	    \item Constructive interference is when two pulses meet and result in a bigger pulse.
	    \item Destructive interference is when two pulses meet and and result in a smaller pulse.
	    \end{itemize}
        \label{m38802*cid9}

%\begin{table}[H]
%\begin{center}
%\begin{tabular}{|l|c|c|c|}\hline \hline 
%\multicolumn{4}{|c|}{\textbf{Units}}\\ \hline \hline
%\textbf{Quantity} & \textbf{Symbol} & \textbf{Unit} & \textbf{S.I. Units}  \\ \hline
%Amplitude & $A$ & \multicolumn{2}{c|}{m} \\ \hline
%Pulse speed & $v$ & \multicolumn{2}{c|}{$\text{m} \cdot \text{s}^{-1}$} \\ \hline
%\end{tabular}
%\end{center}
%\caption{Units used in \textbf{transverse pulses} }
%\label{table:electricity::units}
%\end{table}

\begin{table}[H]
\begin{center}
\begin{tabular}{|l|c|c|}\hline \hline 
\multicolumn{3}{|c|}{\textbf{Physical Quantities}}\\ \hline \hline
\multicolumn{1}{|c|}{\textbf{Quantity}} & \textbf{Unit name} & \textbf{Unit symbol}\\ \hline
Amplitude ($A$) & metre & m \\ \hline
Pulse speed ($v$) & metre per second & $\text{m} \cdot \text{s}^{-1}$ \\ \hline
\end{tabular}
\end{center}
\caption{Units used in \textbf{transverse pulses} }
\label{table:electrostatics::units}
\end{table}

\begin{eocexercises}{Transverse pulses}
            \nopagebreak
      \label{m38802*id316647}\begin{enumerate}[noitemsep, label=\textbf{\arabic*}. ] 
            \label{m38802*uid66}\item A heavy rope is flicked upwards, creating a single pulse in the rope. Make a drawing of the rope and indicate the following in your drawing:
\label{m38802*id316663}\begin{enumerate}[noitemsep, label=\textbf{\alph*}. ] 
            \label{m38802*uid67}\item The direction of motion of the pulse
\label{m38802*uid68}\item Amplitude
\label{m38802*uid69}\item Pulse length
\label{m38802*uid70}\item Position of rest
\end{enumerate}
                \label{m38802*uid71}\item A pulse has a speed of $2,5\phantom{\rule{2pt}{0ex}}\text{m}\ensuremath{\cdot}\text{s}{}^{-1}$. How far will it have travelled in $6\phantom{\rule{2pt}{0ex}}\text{s}$?\newline
\label{m38802*uid72}\item A pulse covers a distance of $75\phantom{\rule{2pt}{0ex}}\text{cm}$ in $2,5\phantom{\rule{2pt}{0ex}}\text{s}$. What is the speed of the pulse?\newline
\label{m38802*uid73}\item How long does it take a pulse to cover a distance of $200\phantom{\rule{2pt}{0ex}}\text{mm}$ if its speed is $4\phantom{\rule{2pt}{0ex}}\text{m}\ensuremath{\cdot}\text{s}{}^{-1}$?\newline
% \label{m38802*uid74}\item The following position-time graph for a pulse in a slinky spring is given. Draw an accurate sketch graph of the velocity of the pulse against time.
%     \setcounter{subfigure}{0}
% 	\begin{figure}[H] % horizontal\label{m38802*id316803}
%     \begin{center} 
% \scalebox{1} % Change this value to rescale the drawing. 
% {
% \begin{pspicture}(0,-1.72)(6.626875,1.7) \psline[linewidth=0.04cm,arrowsize=0.0829cm 2.04,arrowlength=1.46,arrowinset=0.0]{->}(2.2671876,-1.32)(2.2671876,1.68) \psline[linewidth=0.04cm,arrowsize=0.0829cm 2.04,arrowlength=1.46,arrowinset=0.0]{->}(2.2471876,-1.32)(5.2471876,-1.32) \psline[linewidth=0.04cm](2.2671876,-1.32)(4.2871876,1.16) \psline[linewidth=0.04cm,linestyle=dotted,dotsep=0.16cm](4.2871876,1.12)(4.2671876,-1.3) \psline[linewidth=0.04cm,linestyle=dotted,dotsep=0.16cm](4.2871876,1.14)(2.2471876,1.14) %\usefont{T1}{ptm}{m}{n} 
% \rput(4.259375,-1.57){4} 
% %\usefont{T1}{ptm}{m}{n} 
% \rput(2.0140624,1.15){8} 
% %\usefont{T1}{ptm}{m}{n} 
% \rput(5.97125,-1.335){\small time (s)} 
% %\usefont{T1}{ptm}{m}{n} 
% \rput(1.160625,0.645){\small position $\Delta$x} 
% %\usefont{T1}{ptm}{m}{n} 
% \rput(1.106875,0.285){\small (m)} 
% \end{pspicture} 
% } 
% \end{center} 
% 
%  \end{figure}               \label{m38802*uid75}\item The following velocity-time graph for a particle in a medium is given. Draw an accurate sketch graph of the position of the particle vs. time.
%     \setcounter{subfigure}{0}
% 	\begin{figure}[H] % horizontal\label{m38802*id316826}
%     \begin{center}
% \scalebox{1} % Change this value to rescale the drawing. 
% { 
% \begin{pspicture}(0,-1.7232813)(6.779375,1.7032813) \psline[linewidth=0.04cm,arrowsize=0.0829cm 2.04,arrowlength=1.46,arrowinset=0.0]{->}(2.4196875,-1.3167187)(2.4196875,1.6832813) \psline[linewidth=0.04cm,arrowsize=0.0829cm 2.04,arrowlength=1.46,arrowinset=0.0]{->}(2.3996875,-1.3167187)(5.3996873,-1.3167187) \psline[linewidth=0.04cm,linestyle=dotted,dotsep=0.16cm](4.3996873,-0.37671876)(4.4196873,-1.3367188) %\usefont{T1}{ptm}{m}{n} 
% \rput(4.3903127,-1.5667187){5} 
% %\usefont{T1}{ptm}{m}{n} 
% \rput(2.171875,0.6532813){4} 
% %\usefont{T1}{ptm}{m}{n}
%  \rput(6.12375,-1.3317188){\small time (s)} 
% %\usefont{T1}{ptm}{m}{n} 
% \rput(1.1551563,0.44828126){\small velocity v}
% %\usefont{T1}{ptm}{m}{n} 
% \rput(1.039375,0.06828125){\small (m.s$^{-1}$)} \psline[linewidth=0.04cm](2.3996875,0.66328126)(3.5996876,0.66328126) \psline[linewidth=0.04cm](3.5996876,-0.31671876)(4.3996873,-0.31671876) \psline[linewidth=0.04cm,linestyle=dotted,dotsep=0.16cm](3.5796876,0.66328126)(3.5796876,-1.2967187) \psline[linewidth=0.04cm,linestyle=dotted,dotsep=0.16cm](3.6396875,-0.31671876)(2.3996875,-0.31671876) \psline[linewidth=0.04cm,linestyle=dotted,dotsep=0.16cm](6.7396874,0.34328124)(6.7396874,0.34328124) %\usefont{T1}{ptm}{m}{n} 
% \rput(3.5354688,-1.5667187){3} 
% %\usefont{T1}{ptm}{m}{n} 
% \rput(2.1176562,-0.32671875){2} 
% \end{pspicture} 
% } 
% \end{center} 
%  \end{figure}               \label{m38802*uid76}\item Describe what happens to a pulse in a slinky spring when:
% \label{m38802*id316845}\begin{enumerate}[noitemsep, label=\textbf{\alph*}. ] 
%             \label{m38802*uid77}\item the slinky spring is tied to a wall.
% \label{m38802*uid78}\item the slinky spring is loose, i.e. not tied to a wall.
% \end{enumerate}
% (Draw diagrams to explain your answers.)\newline
% \item The following diagrams each show two approaching pulses. Redraw the diagrams to show what type of interference takes place, and label the type of interference. \begin{enumerate} 
% \item 
% \begin{center} 
% \scalebox{1} % Change this value to rescale the drawing. 
% { \begin{pspicture}(0,-1.18)(6.4385505,1.18) \psbezier[linewidth=0.04](0.02,-1.14)(0.26574183,-1.1328717)(0.5114836,-1.14)(0.77211887,-1.132665)(1.0327542,-1.1253302)(1.1642967,-0.73940814)(1.2263689,-0.6164384)(1.288441,-0.4934686)(1.38275,-0.12080005)(1.6135985,-0.12778087) \psbezier[linewidth=0.04](3.1997502,-1.14)(2.9540083,-1.1328717)(2.7082665,-1.14)(2.4476314,-1.132665)(2.186996,-1.1253302)(2.0554533,-0.73940814)(1.9933813,-0.6164384)(1.9313091,-0.4934686)(1.837,-0.12080005)(1.6061517,-0.12778087) \psbezier[linewidth=0.04](3.0,-1.14)(3.264197,-1.1267147)(3.5283942,-1.14)(3.8086033,-1.1263297)(4.0888124,-1.1126592)(4.230234,-0.39340037)(4.2969675,-0.16421652)(4.3637013,0.064967334)(4.465093,0.7595252)(4.7132783,0.74651474) \psbezier[linewidth=0.04](6.4185505,-1.14)(6.1543536,-1.1267147)(5.8901563,-1.14)(5.609947,-1.1263297)(5.329738,-1.1126592)(5.188317,-0.39340037)(5.121583,-0.16421652)(5.054849,0.064967334)(4.9534574,0.7595252)(4.705272,0.74651474) \psline[linewidth=0.04cm,arrowsize=0.0929cm 2.05,arrowlength=1.42,arrowinset=0.0]{->}(5.06,1.16)(4.26,1.16) \psline[linewidth=0.04cm,arrowsize=0.0929cm 2.05,arrowlength=1.42,arrowinset=0.0]{->}(1.24,0.24)(2.04,0.24) \psline[linewidth=0.04cm,linestyle=dashed,dash=0.16cm 0.16cm,arrowsize=0.05291667cm 2.0,arrowlength=1.4,arrowinset=0.4]{<->}(4.7,0.76)(4.7,-1.14) \psline[linewidth=0.04cm,linestyle=dashed,dash=0.16cm 0.16cm,arrowsize=0.05291667cm 2.0,arrowlength=1.4,arrowinset=0.4]{<->}(1.58,-0.14)(1.58,-1.16) %\usefont{T1}{ptm}{m}{n} 
% \rput(1.7054688,-0.695){\small 1} 
% %\usefont{T1}{ptm}{m}{n} 
% \rput(4.8715625,-0.255){\small 3}
% \end{pspicture} 
% } 
% \end{center} 
% \item 
% \begin{center} 
% \scalebox{1} % Change this value to rescale the drawing. 
% { 
% \begin{pspicture}(0,-1.5298992)(6.5997505,1.5298992) \psbezier[linewidth=0.04](3.4,0.23055542)(3.6457417,0.23950318)(3.8914835,0.23055542)(4.1521187,0.23976253)(4.412754,0.24896964)(4.5442967,0.73339576)(4.606369,0.8877528)(4.668441,1.0421097)(4.76275,1.5098996)(4.9935985,1.501137) \psbezier[linewidth=0.04](6.57975,0.23055542)(6.334008,0.23950318)(6.0882664,0.23055542)(5.8276315,0.23976253)(5.566996,0.24896964)(5.4354534,0.73339576)(5.373381,0.8877528)(5.3113093,1.0421097)(5.217,1.5098996)(4.9861517,1.501137) \psbezier[linewidth=0.04](0.02,0.23008056)(0.28419712,0.21791111)(0.5483942,0.23008056)(0.82860327,0.21755837)(1.1088123,0.20503618)(1.2502339,-0.45381054)(1.3169676,-0.66374475)(1.3837013,-0.8736789)(1.4850931,-1.5098994)(1.7332783,-1.4979817) \psbezier[linewidth=0.04](3.4385505,0.23008056)(3.1743534,0.21791111)(2.9101562,0.23008056)(2.6299472,0.21755837)(2.3497381,0.20503618)(2.2083166,-0.45381054)(2.141583,-0.66374475)(2.0748491,-0.8736789)(1.9734575,-1.5098994)(1.7252723,-1.4979817) \psline[linewidth=0.04cm,linestyle=dashed,dash=0.16cm 0.16cm,arrowsize=0.05291667cm 2.0,arrowlength=1.4,arrowinset=0.4]{<->}(4.96,1.4905554)(4.94,0.21055542) \psline[linewidth=0.04cm,linestyle=dashed,dash=0.16cm 0.16cm,arrowsize=0.05291667cm 2.0,arrowlength=1.4,arrowinset=0.4]{<->}(1.78,0.23055542)(1.74,-1.5094446) \psline[linewidth=0.04cm,arrowsize=0.0929cm 2.05,arrowlength=1.42,arrowinset=0.0]{->}(4.52,1.2305554)(3.72,1.2305554) \psline[linewidth=0.04cm,arrowsize=0.0929cm 2.05,arrowlength=1.42,arrowinset=0.0]{->}(2.26,-1.1494446)(3.06,-1.1494446) %\usefont{T1}{ptm}{m}{n} 
% \rput(1.9115624,-0.4844446){\small 3} 
% %\usefont{T1}{ptm}{m}{n} 
% \rput(5.11375,0.77555543){\small 2} \end{pspicture} 
% } 
% \end{center} 
\end{enumerate}
%                 \label{m38802*uid82}\item Two pulses, A and B, of identical shape and amplitude are simultaneously generated in two identical wires of equal mass and length. Wire A is, however, pulled tighter than wire B. Which pulse will arrive at the other end first, or will they both arrive at the same time?\newline
  \label{m38802**end}
  \label{21d48a6f8839b4b265192acd9ea3d978**end}
\par \practiceinfo
 \par \begin{tabular}[h]{ccccc}
 (1.) 0034  &  (2.) 0035  &  (3.) 0036  &  (4.) 0037  \end{tabular}
\end{eocexercises}
