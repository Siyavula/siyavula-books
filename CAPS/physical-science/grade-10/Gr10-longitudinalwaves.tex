\chapter{Longitudinal waves}\fancyfoot[LO,RE]{Physics: Waves, Sound and Light}

    \setcounter{figure}{1}
    \setcounter{subfigure}{1}
    \label{e91550bed2a1600e0ddb2572d580bf8e}
         \section{Introduction and key concepts}
    \nopagebreak
%            \label{m38782} $ \hspace{-5pt}\begin{array}{cccccccccccc}   \includegraphics[width=0.75cm]{col11305.imgs/summary_fullmarks.png} &   \end{array} $ \hspace{2 pt}\raisebox{-5 pt}{} {(section shortcode: P10046 )} \par 
    \label{m38782*id291765}We have already studied transverse pulses and waves. In this chapter we look at another type of wave called a \textsl{longitudinal} wave. In transverse waves, the motion of the particles in the medium was perpendicular to the direction of the wave. In longitudinal waves, the particles in the medium move \textsl{parallel} (in the \textsl{same} direction as) to the motion of the wave. Examples of transverse waves (discussed in the previous chapter) are water waves. An example of a longitudinal wave is a sound wave.
\chapterstartvideo{VPdim}
    \label{m38782*cid3}
            \subsection*{What is a \textsl{longitudinal wave}?}
            \nopagebreak
\Definition{Longitudinal waves} {A longitudinal wave is a wave where the particles in the medium move parallel to the direction of propagation of the wave.} 
      
\label{m38782*id292159}When we studied transverse waves we looked at two different motions: the motion of the particles of the medium and the motion of the wave itself. We will do the same for longitudinal waves.\par 
      \label{m38782*id292164}The question is how do we construct such a wave?\par 
%       \label{m38782*id292167}To create a transverse wave, we flick the end of for example a rope up and down. The particles move up and down and return to their equilibrium position. The wave moves from horizontally and will be displaced.\par 
%       \label{m38782*id292172}
%     \setcounter{subfigure}{0}
% 	\begin{figure}[H] % horizontal\label{m38782*id292175}
%     \begin{center}
% \begin{pspicture}(-0.2,-2)(5,1.4)
% %\psgrid[gridcolor=lightgray]
% \rput(0,0.8){\psline[linewidth=2pt](0,0)(5,0)}
% \uput[d](2.5,0.8){flick rope up and down at one end}
% \rput(-0.2,0){\psline{<->}(0,0.6)(0,1.2)}
% \rput(0,-1){\psplot[xunit=0.0055,linewidth=2pt]{0}{900}{x sin}}
% \end{pspicture}
% \end{center}
%  \end{figure}       
%       \par 

      \label{m38782*id292181}A longitudinal wave is seen best in a slinky spring. Do the following investigation to find out more about longitudinal waves.\par 
\label{m38782*secfhsst!!!underscore!!!id79}
\clearpage
\begin{activity}{Investigating longitudinal waves}
\begin{minipage}{.5\textwidth}
%\begin{enumerate}[noitemsep,  label=\textbf{\arabic*}. ]
%\item 
Take a slinky spring and lay it on a table. Hold one end and pull the free end of the spring and flick it back and forth once in the direction of the spring. Observe what happens.

In which direction does the disturbance move?
\end{minipage}
\begin{minipage}{.5\textwidth}
\begin{center}
\textbf{A slinky spring}\\
 \includegraphics[width=.8\textwidth]{photos/Slinky_Flickr_Tim_Ebbs.jpg}\\
\textsl{Photograph by Tim Ebbs on Flickr.com}
\end{center}
\end{minipage}

\begin{center}
\begin{pspicture}(-1,-3)(1,3)
%\psgrid[gridcolor=gray,subgriddiv=10]
\psset{unit=0.6}
%\rput{90}(0,0){\pccoil[coilarm=0,coilwidth=0.5,coilheight=0.6](0,0)(4.9,0)}
\rput(-2.5,0){\pccoil[coilarm=0,coilwidth=0.5,coilheight=0.6](0,0)(4.9,0)}
\rput(-2.5,.7){\psline{<->}(0,0)(1,0)}
\rput(-0.0,0.2){
\psbezier[linecolor=blue,linewidth=0.075](0.26595977,0.036545016)(0.2546078,-0.08423818)(0.02,0.013059396)(0.10703192,0.10029171)(0.19406384,0.187524)(0.26595977,0.07680608)(0.26217577,0.043255195)(0.25839177,0.009704308)(0.48164758,-0.060752556)(0.4475916,0.070095904)(0.41353562,0.20094436)(0.2508238,0.13719767)(0.26595977,0.036545016)(0.28109577,-0.06410764)(0.41353562,-0.013781314)(0.38704768,-0.18489084)
\psbezier[linewidth=0.04](0.26595977,0.02647975)(0.23515671,-0.03849855)(0.10703192,-0.09094836)(0.21205442,-0.2009443)
}
\rput(.3,1){ribbon}
\psline{-}(0.2,.4)(0.2,0.8)
\uput[r](-10,0.7){flick spring back and forth}
\end{pspicture}
\end{center}

%\item 

%\item 
Tie a ribbon to the middle of the spring. Watch carefully what happens to the ribbon when the end of the spring is flicked. Describe the motion of the ribbon.\\%\item 
Flick the spring back and forth continuously to set up a train of pulses, a longitudinal wave.
%\end{enumerate}
\end{activity}



\label{m38782*id292264}From the investigation you will have noticed that the disturbance moves parallel to the direction in which the spring was pulled. The ribbon in the investigation represents one particle in the medium. The particles in the medium move in the same direction as the wave.
\mindsetvid{Veritasium video on slinky springs}{VPdkf}
	\begin{figure}[H] % horizontal\label{m38782*uid5}
    \begin{center}
\begin{pspicture}(0,-1)(10,1)
%\psgrid[gridcolor=gray,subgriddiv=10]
\psline{->}(0,0.75)(1,0.75)\uput[r](1,0.75){direction of motion of wave}
\pccoil[coilarm=0,coilwidth=0.5,coilheight=0.4](0,0)(1,0)
\pccoil[coilarm=0,coilwidth=0.5,coilheight=0.8](1,0)(3,0)
\pccoil[coilarm=0,coilwidth=0.5,coilheight=0.4](3,0)(4,0)
\pccoil[coilarm=0,coilwidth=0.5,coilheight=0.8](4,0)(6,0)
\pccoil[coilarm=0,coilwidth=0.5,coilheight=0.4](6,0)(7,0)
\pccoil[coilarm=0,coilwidth=0.5,coilheight=0.8](7,0)(9,0)
\pccoil[coilarm=0,coilwidth=0.5,coilheight=0.4](9,0)(10,0)
\psline{<->}(0,-0.75)(1,-0.75)\uput[r](1,-0.75){motion of particles in spring is back and forth}
\end{pspicture}
\caption{Longitudinal wave through a spring}
\label{fig:p:wsl:lw11:lw}
\end{center}

 \end{figure}       
    \label{m38782*cid4}
          
      \label{m38782*id292291}As in the case of transverse waves the following properties can be defined for longitudinal waves:
wavelength, amplitude, period, frequency and wave speed. 
      \label{m38782*uid6}
            \section{Compression and rarefaction}
            \nopagebreak
However instead of crests and troughs, longitudinal waves have \textsl{compressions} and \textsl{rarefactions}.\par 
\Definition{Compression} {A \textbf{compression} is a region in a longitudinal wave where the particles are closest together. } 
\par
\Definition{ Rarefaction } {A \textbf{rarefaction} is a region in a longitudinal wave where the particles are furthest apart.} 
\mindsetvid{Animation of longitudinal waves}{VPdml}        
\label{m38782*id292360}As seen in Figure~\ref{fig:p:wsl:lw11:cr}, there are regions where the medium is compressed and other regions where the medium is spread out in a longitudinal wave.\par 
        \label{m38782*id292369}The region where the medium is compressed is known as a \textbf{compression} and the region where the medium is spread out is known as a \textbf{rarefaction}.\par 
    \setcounter{subfigure}{0}
	\begin{figure}[H] % horizontal\label{m38782*uid7}
    \begin{center}
\begin{pspicture}(0,-1.4)(10,1.4)
%\psgrid[gridcolor=gray,subgriddiv=10]
\psline(0.5,0.75)(9.5,0.75)
\psline{->}(0.5,0.75)(0.5,0.3)
\rput(3,0){\psline{->}(0.5,0.75)(0.5,0.3)}
\rput(6,0){\psline{->}(0.5,0.75)(0.5,0.3)}
\rput(9,0){\psline{->}(0.5,0.75)(0.5,0.3)}
\uput[u](5,0.75){compressions}

\psline(2,-0.75)(8,-0.75)
\rput(2,0){\psline{->}(0,-0.75)(0,-0.3)}
\rput(5,0){\psline{->}(0,-0.75)(0,-0.3)}
\rput(8,0){\psline{->}(0,-0.75)(0,-0.3)}
\uput[d](5,-0.75){rarefactions}

\pccoil[coilarm=0,coilwidth=0.5,coilheight=0.4](0,0)(1,0)
\pccoil[coilarm=0,coilwidth=0.5,coilheight=0.8](1,0)(3,0)
\pccoil[coilarm=0,coilwidth=0.5,coilheight=0.4](3,0)(4,0)
\pccoil[coilarm=0,coilwidth=0.5,coilheight=0.8](4,0)(6,0)
\pccoil[coilarm=0,coilwidth=0.5,coilheight=0.4](6,0)(7,0)
\pccoil[coilarm=0,coilwidth=0.5,coilheight=0.8](7,0)(9,0)
\pccoil[coilarm=0,coilwidth=0.5,coilheight=0.4](9,0)(10,0)
\end{pspicture}
\caption{Compressions and rarefactions on a longitudinal wave}
\label{fig:p:wsl:lw11:cr}
\end{center}
 \end{figure}       
      \label{m38782*uid8}
            \section{Wavelength and amplitude}
            \nopagebreak
\par
 \Definition{ Wavelength } {The \textbf{wavelength} in a longitudinal wave is the distance between two consecutive points that are in phase.} 
        
\label{m38782*id292427}The wavelength in a longitudinal wave refers to the distance between two consecutive compressions or between two consecutive rarefactions.\par 
\Definition{Amplitude }{The \textbf{amplitude} is the maximum displacement from equilibrium. For a longitudinal wave which is a pressure wave this would be the maximum increase (or decrease) in pressure from the equilibrium pressure that is cause when a compression (or rarefaction) passes a point.} 
    \setcounter{subfigure}{0}
	\begin{figure}[H] % horizontal\label{m38782*uid9}
    \begin{center}
\begin{pspicture}(0,-1.4)(10,1.4)
%\psgrid[gridcolor=gray,subgriddiv=10]
\multirput(0,0)(3,0){3}{\psline{<->}(0,0.75)(3,0.75)}
\multirput(0,0)(3,0){4}{\psline{->}(0,0.75)(0,0.3)}
\multirput(1.5,0)(3,0){3}{\uput[u](0,0.75){$\lambda$}}
\multirput(1,0)(3,0){3}{\psline{<->}(0,-0.75)(3,-0.75)}
\multirput(1,0)(3,0){4}{\psline{->}(0,-0.75)(0,-0.3)}
\multirput(2.5,0)(3,0){3}{\uput[d](0,-0.75){$\lambda$}}
\multirput(0,0)(3,0){3}{
\pccoil[coilarm=0,coilwidth=0.5,coilheight=0.4](0,0)(1,0)
\pccoil[coilarm=0,coilwidth=0.5,coilheight=0.8](1,0)(3,0)}
\pccoil[coilarm=0,coilwidth=0.5,coilheight=0.4](9,0)(10,0)
\end{pspicture}
\caption{Wavelength of a longitudinal wave}
\label{fig:p:wsl:lw11:w}
\end{center}
 \end{figure}       
        \label{m38782*id292465}The amplitude is the distance from the equilibrium position of the medium to a compression or a rarefaction.\par 
      \label{m38782*uid10}
            \section{Period and frequency}
            \nopagebreak
            \par
\Definition{Period } {The \textbf{period} of a wave is the time taken by the wave to move one wavelength.} 
\par
 \Definition{Frequency} {The \textbf{frequency} of a wave is the number of wavelengths per second.} 
        \label{m38782*id292523}The \textsl{period} of a longitudinal wave is the time taken by the wave to move one wavelength. As for transverse waves, the symbol $T$ is used to represent period and period is measured in seconds (s).\par 
        \label{m38782*id292542}The \textsl{frequency}$f$ of a wave is the number of wavelengths per second. Using this definition and the fact that the period is the time taken for 1 wavelength, we can define:\par 
        \label{m38782*id291687}\nopagebreak\noindent{}
          
    \begin{equation*}
    f=\frac{1}{T}
      \end{equation*}
        \label{m38782*id291706}or alternately,\par 
        \label{m38782*id292764}\nopagebreak\noindent{}
    \begin{equation*}
    T=\frac{1}{f}
      \end{equation*}
      \label{m38782*uid11}
            \section{Speed of a longitudinal wave}
            \nopagebreak
            \label{m38782*id292794}The speed of a longitudinal wave is defined in the same was as the speed of transverse waves:\par 
%         \label{m38782*id292798}\nopagebreak\noindent{}
%           
%     \begin{equation*}
%     v=f\ensuremath{\cdot}\lambda 
%       \end{equation*}
%         \label{m38782*id292818}where
% \label{m38782*eip-id1170811315120}\begin{itemize}[noitemsep]
%             \item $v=\text{speed\; in\; m}\ensuremath{\cdot}\text{s}{}^{-1}$\item $f=\text{frequency\; in\; Hz}$\item $\lambda =\text{wavelength\; in\; m}$\end{itemize}
%         \par 
% \par

\Definition{Wave speed}{Wave speed is the distance a wave travels per unit time.\\
Quantity: Wave speed ($v$) \hspace{1cm} Unit name: speed \hspace{1cm} Unit: $\text{m}\cdot \text{s}^{-1}$} 
   
        \label{m38806*id319706}The distance between two successive compressions is 1 wavelength, $\lambda$. Thus in a time of 1 period, the wave will travel 1 wavelength in distance. Thus the speed of the wave, $v$, is:\par 
        \label{m38806*id319732}\nopagebreak\noindent{}
    \begin{equation*}
    v=\frac{\text{distance}\phantom{\rule{4.pt}{0ex}}\text{travelled}}{\text{time}\phantom{\rule{4.pt}{0ex}}\text{taken}}=\frac{\lambda }{T}
      \end{equation*}
        \label{m38806*id319776}However, $f=\frac{1}{T}$. Therefore, we can also write:\par 
        \label{m38806*id319802}\nopagebreak\noindent{}
          
    \begin{equation*}
    \begin{array}{ccc}\hfill v& =& \frac{\lambda }{T}\hfill \\ & =& \lambda \ensuremath{\cdot}\frac{1}{T}\hfill \\ & =& \lambda \ensuremath{\cdot}f\hfill \end{array}
      \end{equation*}
        \label{m38806*id319870}We call this equation the \textsl{wave equation}. To summarise, we have that $v=\lambda \ensuremath{\cdot}f$ where\par 
        \label{m38806*id319901}\begin{itemize}[noitemsep]
            \label{m38806*uid22}\item $v=$ speed in $\text{m}\ensuremath{\cdot}\text{s}{}^{-1}$\label{m38806*uid23}\item $\lambda =$ wavelength in $\text{m}$
\item $f=$ frequency in $\text{Hz}$
\end{itemize}
\par

\begin{wex}
{Speed of longitudinal waves}{The musical note ``A'' is a sound wave. The note has a frequency of 440 Hz and a wavelength of 0,784~m. Calculate the speed of the musical note.}{
\westep{Determine what is given and what is required}
Using:
\begin{eqnarray*}
f &=& 440 \ \text{Hz} \\
\lambda &=& 0,784\ \text{m}
\end{eqnarray*}
We need to calculate the speed of the musical note ``A''.

\westep{Determine how to approach based on what is given}
We are given the frequency and wavelength of the note. We can therefore use:
\begin{equation*}
v=f\cdot \lambda 
\end{equation*}

\westep{Calculate the wave speed}
\begin{eqnarray*}
v&=&f\cdot \lambda\\
&=&(440\;\text{Hz})(0,784~\text{m})\\
&=&345~\text{m}\cdot\text{s}^{-1}
\end{eqnarray*}

\westep{Write the final answer}
The musical note ``A'' travels at 345~\ms.
}
\end{wex}

\begin{wex}
{Speed of longitudinal waves}{A longitudinal wave travels into a medium in which its speed increases.
How does this affect its... (write only \emph{increases, decreases, stays the same}).\\
\begin{minipage}{\textwidth}
\begin{enumerate}[noitemsep, label=\textbf{\arabic*}. ]\item period?
\item wavelength?
\end{enumerate}
\end{minipage}
}{
\westep{Determine what is required}
We need to determine how the period and wavelength of a longitudinal wave change when its speed increases.

\westep{Determine how to approach based on what is given}
We need to find the link between period, wavelength and wave speed.

\westep{Discuss how the period changes}
We know that the frequency of a longitudinal wave is dependent on the frequency of the vibrations that lead to the creation of the longitudinal wave. Therefore, the frequency is always unchanged, irrespective of any changes in speed. Since the period is the inverse of the frequency, the period remains the same.

\westep{Discuss how the wavelength changes}
The frequency remains unchanged. According to the wave equation
\begin{equation*}
v = f\cdot\lambda
\end{equation*}
if $f$ remains the same and $v$ increases, then $\lambda$, the wavelength, must also increase.
}
\end{wex}

    \noindent
\vspace{-1cm}
\summary{VPduk}
            \nopagebreak
      \label{m38783*id293550}\begin{itemize}[noitemsep] 
            \label{m38783*uid20}\item A longitudinal wave is a wave where the particles in the medium move parallel to the direction in which the wave is travelling.
\label{m38783*uid21}\item Most longitudinal waves consist of areas of higher pressure, where the particles in the medium are closest together (compressions) and areas of lower pressure, where the particles in the medium are furthest apart (rarefactions).
\label{m38783*uid22}\item The wavelength of a longitudinal wave is the distance between two consecutive compressions, or two consecutive rarefactions.
\label{m38783*uid23}\item The relationship between the period ($T$) and frequency ($f$) is given by
\label{m38783*id293619}\nopagebreak\noindent{}
    \begin{equation*}\nonumber
    T=\frac{1}{f}\phantom{\rule{3pt}{0ex}}\text{or}\phantom{\rule{3pt}{0ex}}f=\frac{1}{T}
      \end{equation*}
    \label{m38783*uid24}\item The relationship between wave speed ($v$), frequency ($f$) and wavelength ($\lambda $) is given by
\label{m38783*id293694}\nopagebreak\noindent{}
    \begin{equation*}\nonumber
    v=f\lambda
      \end{equation*}
%     \label{m38783*uid25}\item Graphs of position vs time, velocity vs time and acceleration vs time can be drawn and are summarised in figures
% \label{m38783*uid26}\item Sound waves are examples of longitudinal waves. The speed of sound depends on the medium, temperature and pressure. Sound waves travel faster in solids than in liquids, and faster in liquids than in gases. Sound waves also travel faster at higher temperatures and higher pressures.
\end{itemize}

%\begin{table}[H]
%\begin{center}
%\begin{tabular}{|l|c|c|c|}\hline \hline 
%\multicolumn{4}{|c|}{\textbf{Units}}\\ \hline \hline
%\textbf{Quantity} & \textbf{Symbol} & \textbf{Unit} & \textbf{S.I. Units}\\ \hline
%Amplitude & $A$ & \multicolumn{2}{c|}{m} \\ \hline
%Wavelength & $\lambda$ & \multicolumn{2}{c|}{m}  \\ \hline
%Period & $T$ & \multicolumn{2}{c|}{s}  \\ \hline
%Frequency & $f$ & \multicolumn{2}{c|}{$\text{s}^{-1}$}  \\ \hline
%Wave speed & $v$ & \multicolumn{2}{c|}{$\text{m} \cdot \text{s}^{-1}$} \\ \hline
%\end{tabular}
%\end{center}
%\caption{Units used in \textbf{longitudinal waves} }
%\label{table:electricity::units}
%\end{table}
%   \label{m38783*cid9}

\begin{table}[H]
\begin{center}
\begin{tabular}{|l|c|c|}\hline \hline 
\multicolumn{3}{|c|}{\textbf{Physical Quantities}}\\ \hline \hline
\multicolumn{1}{|c|}{\textbf{Quantity}} & \textbf{Unit name} & \textbf{Unit symbol}\\ \hline
Amplitude ($A$) & metre & m \\ \hline
Wavelength ($\lambda$) & metre & m \\ \hline
Period ($T$) & second & s \\ \hline
Frequency ($f$) & hertz & Hz \ \ ($s^{-1}$) \\ \hline
Wave speed ($v$)& metre per second & $\text{m} \cdot \text{s}^{-1}$ \\ \hline
\end{tabular}
\end{center}
\caption{Units used in \textbf{longitudinal waves} }
\label{table:electrostatics::units}
\end{table}

\begin{eocexercises}{Longitudinal waves}
            \noindent\vspace{-1cm}
      \label{m38783*id293753}\begin{enumerate}[noitemsep, label=\textbf{\arabic*}. ] 
            \label{m38783*uid27}\item Which of the following is not a longitudinal wave?
\label{m38783*id293768}\begin{enumerate}[noitemsep, label=\textbf{\alph*}. ] 
\label{m38783*uid29}\item light
\label{m38783*uid30}\item sound
\label{m38783*uid31}\item ultrasound
\end{enumerate}
                \label{m38783*uid32}\item Which of the following media can a longitudinal wave like sound not travel through?
\label{m38783*id293834}\begin{enumerate}[noitemsep,label=\textbf{\alph*}. ] \vspace{-.5cm}
            \label{m38783*uid33}\item solid
\label{m38783*uid34}\item liquid
\label{m38783*uid35}\item gas
\label{m38783*uid36}\item vacuum
\end{enumerate}
          \label{m38783*uid38}\item A longitudinal wave has a compression to compression distance of 10~m. It takes the wave 5 s to pass a point.
\label{m38783*id294078}\begin{enumerate}[noitemsep, label=\textbf{\alph*}. ] 
            \label{m38783*uid39}\item What is the wavelength of the longitudinal wave?
\label{m38783*uid40}\item What is the speed of the wave?
\end{enumerate}
                \label{m38783*uid41}\item A flute produces a musical sound travelling at a speed of $320\phantom{\rule{2pt}{0ex}}\text{m}\ensuremath{\cdot}\text{s}{}^{-1}$. The frequency of the note is 256 Hz. Calculate:
\label{m38783*id294137}\begin{enumerate}[noitemsep, label=\textbf{\alph*}. ] 
            \label{m38783*uid42}\item the period of the note
\label{m38783*uid43}\item the wavelength of the note
\end{enumerate}
                \end{enumerate}
  \label{m38783**end}
  \label{e91550bed2a1600e0ddb2572d580bf8e**end}
\par \practiceinfo
 \par \begin{tabular}[h]{cccccc}
 (1.) 003p  &  (2.) 003q  &  (3.) 003r  &  (4.) 003s  &    &    &    & \end{tabular}
\end{eocexercises}
