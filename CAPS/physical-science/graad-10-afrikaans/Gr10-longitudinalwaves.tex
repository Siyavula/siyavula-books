\chapter{Longitudinale golwe}\fancyfoot[LO,RE]{Fisika: Golwe, klank en lig}

    \setcounter{figure}{1}
    \setcounter{subfigure}{1}
    \label{e91550bed2a1600e0ddb2572d580bf8e}
         \section{Inleiding en sleutelkonsepte}
    \nopagebreak
    \label{m38782*id291765}Ons het alreeds transversale pulse en golwe bestudeer.
    In hierdie hoofstuk kyk ons na 'n ander soort golf, naamlik 'n \textsl{longitudinale} golf.
    Met transversale golwe beweeg die deeltjies van die medium loodreg teenoor die rigting van beweging van die golf.
    Met longitudinale golwe beweeg die deeltjies van die medium in 'n rigting \textsl{parallel} aan die rigting van beweging van die golf. 
    Golwe in water (soos in die vorige hoofstuk bespreek) is voorbeelde van transversale golwe.
    Klankgolwe is 'n voorbeeld van longitudinale golwe.\par 
    \label{m38782*cid3}


\chapterstartvideo{VPdim}


            \subsection*{Wat is 'n \textsl{longitudinale golf}?}
            \nopagebreak
\par
\Definition{ Longitudinale golwe } {  'n Longitudinale golf is 'n golf waar die deeltjies van die medium in 'n rigting parallel aan die voortplantingsrigting van die golf beweeg.} 
      
\label{m38782*id292159}
  Toe ons transversale golwe bestudeer het, het ons gekyk na twee verskillende bewegings: Die beweging van die deeltjies van die medium en die beweging van die golf self. Ons sal longitudinale golwe op dieselfde manier bestudeer.\par 
      \label{m38782*id292164}Die vraag is hoe bou (konstrukteer) ons so 'n golf?\par 
%       \label{m38782*id292167}To create a transverse wave, we flick the end of for example a rope up and down. The particles move up and down and return to their equilibrium position. The wave moves from horizontally and will be displaced.\par 
%       \label{m38782*id292172}
%     \setcounter{subfigure}{0}
% 	\begin{figure}[H] % horizontal\label{m38782*id292175}
%     \begin{center}
% \begin{pspicture}(-0.2,-2)(5,1.4)
% %\psgrid[gridcolor=lightgray]
% \rput(0,0.8){\psline[linewidth=2pt](0,0)(5,0)}
% \uput[d](2.5,0.8){flick rope up and down at one end}
% \rput(-0.2,0){\psline{<->}(0,0.6)(0,1.2)}
% \rput(0,-1){\psplot[xunit=0.0055,linewidth=2pt]{0}{900}{x sin}}
% \end{pspicture}
% \end{center}
%  \end{figure}       
%       \par 

      \label{m38782*id292181}
      'n Longitudinale golf word beste waargeneem in 'n veer (``slinky''). Voer die volgende ondersoek uit om meer oor longitudinale golwe te leer.\pagebreak 
\label{m38782*secfhsst!!!underscore!!!id79}
\begin{activity}{Ondersoek longitudinale golwe}
\begin{minipage}{.5\textwidth}
\begin{enumerate}[noitemsep,  label=\textbf{\arabic*}. ]
\item Neem 'n veer en plaas dit op 'n tafel. Hou die een punt van die veer vas en beweeg die ander kant een keer vinnig in en weer uit. Let op wat gebeur.
\begin{center}
\begin{pspicture}(-1,-3)(1,3)
%\psgrid[gridcolor=gray,subgriddiv=10]
\psset{unit=0.6}
%\rput{90}(0,0){\pccoil[coilarm=0,coilwidth=0.5,coilheight=0.6](0,0)(4.9,0)}
\rput(-2.5,0){\pccoil[coilarm=0,coilwidth=0.5,coilheight=0.6](0,0)(4.9,0)}
\rput(-2.5,.7){\psline{<->}(0,0)(1,0)}
\rput(-0.0,0.2){
\psbezier[linecolor=blue,linewidth=0.075](0.26595977,0.036545016)(0.2546078,-0.08423818)(0.02,0.013059396)(0.10703192,0.10029171)(0.19406384,0.187524)(0.26595977,0.07680608)(0.26217577,0.043255195)(0.25839177,0.009704308)(0.48164758,-0.060752556)(0.4475916,0.070095904)(0.41353562,0.20094436)(0.2508238,0.13719767)(0.26595977,0.036545016)(0.28109577,-0.06410764)(0.41353562,-0.013781314)(0.38704768,-0.18489084)
\psbezier[linewidth=0.04](0.26595977,0.02647975)(0.23515671,-0.03849855)(0.10703192,-0.09094836)(0.21205442,-0.2009443)
}
\rput(.3,1){lint}
\psline{-}(0.2,.4)(0.2,0.8)
\uput[r](-10,0.7){beweeg veer vinnig in en uit}
\end{pspicture}
\end{center}

\item In watter rigting beweeg die versteuring?

\item Bind 'n lint in die middel van die veer vas. Kyk noukeurig wat gebeur as die punt van die veer beweeg word. Beskryf die beweging van die lint.
 
\item Beweeg die punt van die veer aanhoudend in en uit om 'n reeks pulse - 'n longitudinale golf - in die veer te skep.
\end{enumerate}
\end{minipage}
\begin{minipage}{.5\textwidth}
\vspace*{2cm}
\begin{center}
 \includegraphics[width=.8\textwidth]{photos/Slinky_Flickr_Tim_Ebbs.jpg}
\end{center}
\end{minipage}
\end{activity}



\label{m38782*id292264} Tydens die ondersoek sou jy opgemerk het dat die versteuring parallel aan die rigting waarin die veer beweeg word, beweeg. Die lint in die ondersoek verteenwoordig 'n deeltjie van die medium. Die deeltjies van die medium beweeg in dieselfde rigting as die golf.

\mindsetvid{Veritasium video oor slinky vere}{VPdkf}
\setcounter{subfigure}{0}
	\begin{figure}[H] % horizontal\label{m38782*uid5}
    \begin{center}
\begin{pspicture}(0,-1)(10,1)
%\psgrid[gridcolor=gray,subgriddiv=10]
\psline{->}(0,0.75)(1,0.75)\uput[r](1,0.75){bewegingsrigting van die golf}
\pccoil[coilarm=0,coilwidth=0.5,coilheight=0.4](0,0)(1,0)
\pccoil[coilarm=0,coilwidth=0.5,coilheight=0.8](1,0)(3,0)
\pccoil[coilarm=0,coilwidth=0.5,coilheight=0.4](3,0)(4,0)
\pccoil[coilarm=0,coilwidth=0.5,coilheight=0.8](4,0)(6,0)
\pccoil[coilarm=0,coilwidth=0.5,coilheight=0.4](6,0)(7,0)
\pccoil[coilarm=0,coilwidth=0.5,coilheight=0.8](7,0)(9,0)
\pccoil[coilarm=0,coilwidth=0.5,coilheight=0.4](9,0)(10,0)
\psline{<->}(0,-0.75)(1,-0.75)\uput[r](1,-0.75){bewegingsrigting van die deeltjies van die veer}
\end{pspicture}
\caption{Longitudinale golf wat deur 'n veer beweeg.}
\label{fig:p:wsl:lw11:lw}
\end{center}

 \end{figure}       
    \label{m38782*cid4}
            \subsection*{Eienskappe van Longitudinale Golwe}
            \nopagebreak
      \label{m38782*id292291}Soos met transversale golwe kan die volgende eienskappe vir longitudinale golwe gedefinieer word:
golflengte, amplitude, periode, frekwensie en golfspoed. 
      \label{m38782*uid6}
            \section{Verdigtings en Verdunnings}
            \nopagebreak
In plaas van kruine en buike (soos met transversale golwe), het longitudinale golwe \textsl{ver\-dig\-tings} en \textsl{verdunnings}.\par

\Definition{Verdigtings} {  'n \textbf{Verdigting} is 'n deel van 'n longitudinale golf waar die deeltjies van die medium naaste aan mekaar is.} 
\par
\Definition{ Verdunnings} {  'n \textbf{Verdunning} is 'n deel van 'n longitudinale golf waar die deeltjies van die medium verste van mekaar is.} 

\mindsetvid{Animasie van longitudinale golwe}{VPdml}        

        \label{m38782*id292360}Figuur~\ref{fig:p:wsl:lw11:cr} wys hoe sommige dele van die medium saamgetrek en ander dele uitgesprei is as 'n longitudinale golf deur die medium beweeg.\par 
        \label{m38782*id292369}Die deel waar die medium saamgetrek is staan beken as 'n \textbf{verdigting} en die deel waar die medium uitgesprei is staan bekend as 'n \textbf{verdunning}.\par 
    \setcounter{subfigure}{0}
	\begin{figure}[H] % horizontal\label{m38782*uid7}
    \begin{center}
\begin{pspicture}(0,-1.4)(10,1.4)
%\psgrid[gridcolor=gray,subgriddiv=10]
\psline(0.5,0.75)(9.5,0.75)
\psline{->}(0.5,0.75)(0.5,0.3)
\rput(3,0){\psline{->}(0.5,0.75)(0.5,0.3)}
\rput(6,0){\psline{->}(0.5,0.75)(0.5,0.3)}
\rput(9,0){\psline{->}(0.5,0.75)(0.5,0.3)}
\uput[u](5,0.75){Verdigtings}

\psline(2,-0.75)(8,-0.75)
\rput(2,0){\psline{->}(0,-0.75)(0,-0.3)}
\rput(5,0){\psline{->}(0,-0.75)(0,-0.3)}
\rput(8,0){\psline{->}(0,-0.75)(0,-0.3)}
\uput[d](5,-0.75){Verdunnings}

\pccoil[coilarm=0,coilwidth=0.5,coilheight=0.4](0,0)(1,0)
\pccoil[coilarm=0,coilwidth=0.5,coilheight=0.8](1,0)(3,0)
\pccoil[coilarm=0,coilwidth=0.5,coilheight=0.4](3,0)(4,0)
\pccoil[coilarm=0,coilwidth=0.5,coilheight=0.8](4,0)(6,0)
\pccoil[coilarm=0,coilwidth=0.5,coilheight=0.4](6,0)(7,0)
\pccoil[coilarm=0,coilwidth=0.5,coilheight=0.8](7,0)(9,0)
\pccoil[coilarm=0,coilwidth=0.5,coilheight=0.4](9,0)(10,0)
\end{pspicture}
\caption{Verdigtings en verdunnings in 'n longitudinale golf.}
\label{fig:p:wsl:lw11:cr}
\end{center}
 \end{figure}       
      \label{m38782*uid8}
            \section{Golflengte en Amplitude}
            \nopagebreak
\par
 \Definition{ Golflengte} {  Die \textbf{golflengte} van 'n longitudinale golf is die afstand tussen twee opeenvolgende punte in fase.} 
        
\label{m38782*id292427}Die golflengte van 'n longitudinale golf verwys na die afstand tussen twee opeenvolgende verdigtings of tussen twee opeenvolgende verdunnings.\par 
\Definition{  Amplitude } { Die \textbf{amplitude} is die maksimum verplasing van die deeltjies van die medium vanaf die ewewigsposisie. Vir 'n longitudinale golf (wat 'n drukgolf is) sal die amplitude die maksimum drukverhoging (of -verlaging) in die medium vanaf ewewigsdruk wees, wat deur verdigtings (of verdunnings) wat verby 'n punt beweeg, veroorsaak word.  } 
    \setcounter{subfigure}{0}
	\begin{figure}[H] % horizontal\label{m38782*uid9}
    \begin{center}
\begin{pspicture}(0,-1.4)(10,1.4)
%\psgrid[gridcolor=gray,subgriddiv=10]
\multirput(0,0)(3,0){3}{\psline{<->}(0,0.75)(3,0.75)}
\multirput(0,0)(3,0){4}{\psline{->}(0,0.75)(0,0.3)}
\multirput(1.5,0)(3,0){3}{\uput[u](0,0.75){$\lambda$}}
\multirput(1,0)(3,0){3}{\psline{<->}(0,-0.75)(3,-0.75)}
\multirput(1,0)(3,0){4}{\psline{->}(0,-0.75)(0,-0.3)}
\multirput(2.5,0)(3,0){3}{\uput[d](0,-0.75){$\lambda$}}
\multirput(0,0)(3,0){3}{
\pccoil[coilarm=0,coilwidth=0.5,coilheight=0.4](0,0)(1,0)
\pccoil[coilarm=0,coilwidth=0.5,coilheight=0.8](1,0)(3,0)}
\pccoil[coilarm=0,coilwidth=0.5,coilheight=0.4](9,0)(10,0)
\end{pspicture}
\caption{Die golflengte van 'n longitudinale golf}
\label{fig:p:wsl:lw11:w}
\end{center}
 \end{figure}       

Die amplitude is die afstand vanaf die ewewigsposisie van die medium na 'n verdigting of 'n verdunning.\par 
      \label{m38782*uid10}
            \section{Periode en Frekwensie}
            \nopagebreak
            \par
\Definition{ Periode } {  Die \textbf{periode} van 'n golf is die tyd wat die golf neem om 'n afstand gelykstaande aan een golflengte deur die medium te beweeg.     } 
\par
 \Definition{  Frekwensie } {   Die \textbf{frekwensie} van 'n golf is die aantal golflengtes wat verby 'n vasgestelde punt in die medium per tydseenheid beweeg.  } 
        \label{m38782*id292523} Die \textsl{periode} van 'n golf is die tyd wat die golf neem om een golflengte te beweeg. Soos met transversale golwe, word periode met die simbool $T$ voorgestel en word periode gewoonlik in sekondes (s) gemeet.\par 
        \label{m38782*id292542} Die \textsl{frekwensie} ($f$) is die aantal golflengtes per sekonde. Deur die bogenoemde definisies te gebruik asook die feit dat periode die tyd is wat geneem word vir 1 golflengte, kan ons aflei dat:\par 
        \label{m38782*id291687}\nopagebreak\noindent{}
          
    \begin{equation*}
    f=\frac{1}{T}
      \end{equation*}
        \label{m38782*id291706}asook dat,\par 
        \label{m38782*id292764}\nopagebreak\noindent{}
    \begin{equation*}
    T=\frac{1}{f}
      \end{equation*}
      \label{m38782*uid11}
            \section{Spoed van 'n Longitudinale Golf}
            \nopagebreak
            \label{m38782*id292794}Die spoed van 'n longitudinale golf het dieselfde definisie as die spoed van 'n transversale golf:\par
%         \label{m38782*id292798}\nopagebreak\noindent{}
%           
%     \begin{equation*}
%     v=f\ensuremath{\cdot}\lambda 
%       \end{equation*}
%         \label{m38782*id292818}where
% \label{m38782*eip-id1170811315120}\begin{itemize}[noitemsep]
%             \item $v=\text{speed\; in\; m}\ensuremath{\cdot}\text{s}{}^{-1}$\item $f=\text{frequency\; in\; Hz}$\item $\lambda =\text{wavelength\; in\; m}$\end{itemize}
%         \par 
% \par

\Definition{Golfspoed}{Golfspoed is die afstand per tydseenheid waarmee die golf deur die medium beweeg.\\
Hoeveelheid: Golfspoed ($v$) \hspace{1cm} Eenheid naam: meter per sekond \hspace{1cm} Eenheid: $\text{m}\cdot \text{s}^{-1}$} 
   
        \label{m38806*id319706}Die afstand tussen twee opeenvolgende verdigtings is 1 golflengte ($\lambda$). Dus, in die tyd van 1 periode, beweeg die golf 'n afstand gelyk aan 1 golflengte. Dus is die golfspoed ($v$):\par 
        \label{m38806*id319732}\nopagebreak\noindent{}
    \begin{equation*}
    v=\frac{\text{afstand}\phantom{\rule{4.pt}{0ex}}\text{beweeg}}{\text{tyd}\phantom{\rule{4.pt}{0ex}}\text{geneem}}=\frac{\lambda }{T}
      \end{equation*}
        \label{m38806*id319776}maar $f=\frac{1}{T}$. Dus:\par 
        \label{m38806*id319802}\nopagebreak\noindent{}
          
    \begin{equation*}
    \begin{array}{ccc}\hfill v& =& \frac{\lambda }{T}\hfill \\ & =& \lambda \ensuremath{\cdot}\frac{1}{T}\hfill \\ & =& \lambda \ensuremath{\cdot}f\hfill \end{array}
      \end{equation*}
        \label{m38806*id319870}Hierdie is die \textsl{golfvergelyking}. Ons het dus dat $v=\lambda \ensuremath{\cdot}f$ waar\par 
        \label{m38806*id319901}\begin{itemize}[noitemsep]
            \label{m38806*uid22}\item $v=$ golfspoed in $\text{m}\ensuremath{\cdot}\text{s}{}^{-1}$\label{m38806*uid23}\item $\lambda =$ golflengte in $\text{m}$
\item $f=$ frekwensie in $\text{Hz}$
\end{itemize}
\par

\begin{wex}
{Spoed van Longitudinale Golwe}{Die musieknoot ``A`` is 'n klankgolf. Die noot het 'n frekwensie van $440$ Hz en 'n golflengte van $0,784$~m. Bereken die golfspoed van die noot.}{
\westep{Stel vas wat is gegee en wat is gevra:}
Gebruik:
\begin{eqnarray*}
f &=& 440 \ \text{Hz} \\
\lambda &=& 0,784\ \text{m}
\end{eqnarray*}
Ons moet die spoed van die noot ``A'' vasstel.

\westep{Beraam 'n plan van aksie gebaseer op wat gegee is}
Ons is die frekwensie en die golflengte van die noot gegee. Ons kan dus die volgende vergelyking gebruik:
\begin{equation*}
v=f\cdot \lambda 
\end{equation*}

\westep{Bereken die golfspoed}
\begin{eqnarray*}
v&=&f\cdot \lambda\\
&=&(440\;\text{Hz})(0,784~\text{m})\\
&=&345~\text{m}\cdot\text{s}^{-1}
\end{eqnarray*}

\westep{Skryf die finale antwoord neer}
Die ``A`` musieknoot het 'n golfspoed van $345$~\ms.
}
\end{wex}

\begin{wex}
{Spoed van longitudinale golwe}{\begin{minipage}{.85\textwidth} 'n Longitudinale golf beweeg vanaf een medium na 'n ander. Tydens die oorgang vermeerder die golfspoed. Hoe word die golf se  
\begin{enumerate}[noitemsep, label=\textbf{\arabic*}. ]
\item periode
\item golflengte
\end{enumerate}
geaffekteer (skryf slegs \emph{vermeerder, verminder, bly dieselfde})?
 \end{minipage}}{
\westep{Bepaal wat word gevra:}
Ons moet vasstel hoe die periode en golflengte van 'n longitudinale golf verander wanneer die golfspoed vermeerder.

\westep{Beraam 'n plan van aksie gebaseer op wat gegee is:}
Ons moet vasstel wat die verhouding tussen periode, golflengte en golfspoed is.

\westep{Bespreek hoe die periode verander:}
Ons weet dat die frekwensie van 'n longitudinal golf bepaal word deur die frekwensie van die vibrasies wat tot die golf oorsprong gee. Dus bly die frekwensie altyd dieselfde, ongeag van enige veranderinge in golfspoed. Siende dat die periode die resiprook van die frekwensie is, bly die periode ook dieselfde.

\westep{Bespreek hoe die golflengte verander:}
Die frekwensie bly dieselfde. Volgens die golfvergelyking
\begin{equation*}
v = f\cdot\lambda
\end{equation*}
as $f$ onveranderd bly en $v$ vermeerder, dan moet die golflengte ($\lambda$) ook vermeerder.
}
\end{wex}

    \noindent
  \label{m38782**end}
                   \label{m38783*cid8}


\summary{VPduk}            
\nopagebreak
      \label{m38783*id293550}\begin{itemize}[noitemsep] 
            \label{m38783*uid20}\item 'n Longitudinale golf is 'n golf waar die deeltjies van die medium parallel aan die bewegingsrigting van die golf beweeg.
\label{m38783*uid21}\item Meeste longitudinale golwe bestaan uit hoogdruk dele, waar die deeltjies van die medium naaste aan mekaar is (verdigtings), en laagdruk dele, waar die deeltjies van die medium verder uitmekaar is (verdunnings).
\label{m38783*uid22}\item Die golflengte van 'n longitudinale golf is die afstand tussen twee opeenvolgende verdigtings of twee opeenvolgende verdunnings.
\label{m38783*uid23}\item Die verhouding tussen periode ($T$) en frekwensie ($f$) word gegee deur:
\label{m38783*id293619}\nopagebreak\noindent{}
    \begin{equation*}\nonumber
    T=\frac{1}{f}\phantom{\rule{3pt}{0ex}}\text{or}\phantom{\rule{3pt}{0ex}}f=\frac{1}{T}
      \end{equation*}
    \label{m38783*uid24}\item Die verhouding tussen die golfspoed ($v$), frekwensie ($f$) en golflengte ($\lambda$) word gegee deur die golfvergelyking:
\label{m38783*id293694}\nopagebreak\noindent{}
    \begin{equation*}\nonumber
    v=f\lambda
      \end{equation*}
    \label{m38783*uid25}\item Grafieke van posisie vs. tyd, spoed vs. tyd en versnelling vs. tyd kan geteken word en word in figuur \label{m38783*uid26} opgesom.
\item Klankgolwe is voorbeelde van longitudinale golwe. Die spoed van klank hang van die soort medium, temperatuur en druk af. Klankgolwe beweeg vinniger in vastestowwe as in vloeistowwe en vinniger in vloeistowwe as in gasse. Klankgolwe beweeg ook vinniger teen ho\"er temperature en ho\"er druk.
\end{itemize}
\begin{table}[H]
\begin{center}
\begin{tabular}{|l|c|c|}\hline \hline 
\multicolumn{3}{|c|}{\textbf{Fisiese Hoeveelhede}}\\ \hline \hline
\multicolumn{1}{|c|}{\textbf{Hoeveelheid}} & \textbf{Eenheid naam} & \textbf{Eenheid simbool}\\ \hline
Amplitude ($A$) & meter & m \\ \hline
Golflengte ($\lambda$) & meter & m \\ \hline
Periode ($T$) & sekond & s \\ \hline
Frekwensie ($f$) & hertz & Hz \ \ ($s^{-1}$) \\ \hline
Golfspoed ($v$)& meter per sekond & $\text{m} \cdot \text{s}^{-1}$ \\ \hline
\end{tabular}
\end{center}
\caption{\textbf{Longitudinale golwe} en hul eenhede}
\label{table:electrostatics::units}
\end{table}
% \begin{table}[H]
% \begin{center}
% \begin{tabular}{|l|c|c|c|}\hline \hline 
% \multicolumn{4}{|c|}{\textbf{Eenhede}}\\ \hline \hline
% \textbf{Hoeveelheid} & \textbf{Simbool} & \textbf{Eenheid} & \textbf{S.I. Eenheid}\\ \hline
% Amplitude & $A$ & \multicolumn{2}{c|}{m} \\ \hline
% Golflengte & $\lambda$ & \multicolumn{2}{c|}{m}  \\ \hline
% Periode & $T$ & \multicolumn{2}{c|}{s}  \\ \hline
% Frekwensie & $f$ & \multicolumn{2}{c|}{$\text{s}^{-1}$}  \\ \hline
% Golfspoed & $v$ & \multicolumn{2}{c|}{$\text{m} \cdot \text{s}^{-1}$} \\ \hline
% \end{tabular}
% \end{center}
% \caption{\textbf{Longitudinale golwe} en hul eenhede}
% \label{table:electricity::units}
% \end{table}
   \label{m38783*cid9}


\begin{eocexercises}{Longitudinale Golwe}
            \nopagebreak
\label{m38783*id293753}\begin{enumerate}[noitemsep, label=\textbf{\arabic*}. ] 
\item Watter van die volgende is nie longitudinale golwe nie?
\begin{enumerate}[noitemsep, label=\textbf{\alph*}. ] 
    \item lig
    \item klank
    \item ultraklank
\end{enumerate}
\item Deur watter van die volgende media kan longitudinale golwe (soos klankgolwe) nie beweeg nie?
\begin{enumerate}[noitemsep, label=\textbf{\alph*}. ] 
    \item vastestof
    \item vloeistof
    \item gas
    \item vakuum
\end{enumerate}

\par
\item 'n Longitudinale golf het 'n kompressie-tot-verdigting afstand van 10~m. Dit neem 5 s om verby dieselfde punt te beweeg.
\begin{enumerate}[noitemsep, label=\textbf{\alph*}. ] 
    \item Wat is die golflengte van die longitudinale golf?
    \item Wat is die spoed van die golf?
\end{enumerate}

\item 'n Fluit produseer 'n klank wat teen $320\phantom{\rule{2pt}{0ex}}\text{m}\ensuremath{\cdot}\text{s}{}^{-1}$ beweeg. Die klank het 'n frekwensie van 256 Hz. Bereken:
\begin{enumerate}[noitemsep, label=\textbf{\alph*}. ] 
    \item die periode van die noot.
    \item die golflengte van die noot.
\end{enumerate}
\end{enumerate}

% Automatically inserted shortcodes - number to insert 4
\par \practiceinfo
\par \begin{tabular}[h]{cccccc}
% Question 1
(1.)	029d	&
% Question 2
(2.)	029e	&
% Question 3
(3.)	029f	&
% Question 4
(4.)	029g	&
\end{tabular}
% Automatically inserted shortcodes - number inserted 4

\end{eocexercises}
