         \chapter{Kwantitatiewe aspekte van chemiese verandering }\fancyfoot[LO,RE]{Chemie: Chemiese verandering}\label{chap:quanchem}
    \setcounter{figure}{1}
    \setcounter{subfigure}{1}
    \label{0044f0dab6cfd2ca2bac282dc4009886}
         \section{Atoommassa en die mol}
    \nopagebreak

 'n Vergelyking van 'n chemiese reaksie kan  'n groot hoeveelheid baie nuttige inligting oordra. Die vergelyking vertel ons watter reaktante en produkte aan die reaksie deelneem en ook in watter verhouding die reaktante verbind om die produkte te vorm. Kyk na die vergelyking hieronder:\\

        $\text{Fe}+\text{S}\to \text{FeS}$\\
In hierdie reaksie sal elke atoom van yster ($\text{Fe}$) reageer met 'n enkele atoom van swael ($\text{S}$) om ystersulfied ($\text{FeS}$) te vorm. Wat die vergelyking ons nie vertel nie, is die \textbf{hoeveelhede} van elke stof wat betrokke is by die reaksie. Jy kan byvoorbeeld 'n klein hoeveelheid yster gegee word vir die reaksie. Hoe sal jy weet hoeveel atome yster is daar in hierdie monster? Hoe gaan jy bepaal hoeveel atome swael benodig word om met die yster waaroor jy beskik, te reageer? Is daar 'n manier om uit te vind watter massa ystersulfied in die reaksie gevorm sal word? Hierdie vrae is almal belangrik, veral wanneer die reaksie in die industrie gebruik word waar dit belangrik is om die hoeveelhede van die reaktante en die produkte wat gevorm word vooraf te weet. In hierdie  hoofstuk gaan ons kyk hoe om die veranderings wat plaasvind tydens chemiese reaksies te kwantifiseer.
            \subsection*{Die Mol}
            \nopagebreak
Soms is dit belangrik om te weet presies hoeveel deeltjies (bv. atome of molekules) aanwesig is in 'n monster van 'n stof of watter hoeveelheid van 'n stof benodig word sodat  'n reaksie kan plaasvind.\\
Die hoeveelheid van  'n stof is so belangrik in chemie dat dit sy eie naam het, naamlik die mol.\\
\Definition{Mol} {Die mol (afkorting 'n') is die SI (Standaard Internasionale) eenheid vir 'hoeveelheid van 'n stof'. } 
Die mol is 'n tel-eenheid net soos ure of dae. Ons kan maklik 1 sekonde of 1 minuut of 1 uur aftel. As ons in groter eenhede van tyd wil meet, verwys ons na dae, maande en jare. Selfs langer tydperke is moontlik soos eeue en millennia. Die mol is selfs groter as hierdie getalle. Die mol is $602 ~204 ~500 ~000 ~000 ~000 ~000 ~000$ of $6,022 \times 10^{23}$. Dit is 'n baie groot getal! Ons noem hierdie hoeveelheid Avogadro se getal. 
\Definition{Avogadro se getal} {Die aantal deeltjies in 'n mol en dis gelyk aan $6,022\ensuremath{\times}{10}^{23}$.} 
As ons hierdie aantal koeldrankblikkies het, kan ons op die oppervlak van die aarde bedek met blikkies tot 'n diepte van meer as $300\phantom{\rule{2pt}{0ex}}\text{km}$! As jy atome kon tel teen 'n tempo van 10 miljoen per sekonde, dan sou dit jou 2 miljard jaar neem om die atome in een mol te tel! \\

\IFact{Die oor\-spronk\-li\-ke hi\-po\-te\-se wat deur Amadeo Avogadro voorgestel is, is dat "gelyke volumes van gasse, by die\-self\-de temp\-e\-ra\-tuur en druk, dieselfde aantal molekules bevat”. Sy idees is nie deur die we\-ten\-skap\-lik\-e ge\-meen\-skap aanvaar nie en dit was eers vier jaar ná sy dood dat sy oor\-spronk\-li\-ke hipotese aanvaar is en dat dit bekend geword het as 'Avogadro se wet’. Ter ere van sy bydrae tot wetenskap is die aantal deeltjies in 'n mol bekend as Avogadro se getal. Ons gebruik Avogadro se getal en die mol in chemie om ons te help kwantifiseer wat gebeur in  'n chemiese reaksie. Die mol is 'n baie spesiale getal.}
As ons $12,0 ~\text{g}$ koolstof afmeet, het ons een mol of $6,022 \times 10^{23}$ koolstofatome. $63,5 ~\text{g}$ koper is een mol van die koper of $6,022 \times 10^{23}$ koperatome. Om die waarheid te s\^{e}, as ons die relatiewe atoommassa van enige element op die periodieke tabel afmeet, het ons een mol van die element.

            \begin{exercises}{Moles and mass }
            \nopagebreak
      \label{m38717*id276067}\begin{enumerate}[noitemsep, label=\textbf{\arabic*}. ] 
\item Voltooi die volgende tabel:
    % \textbf{m38717*id276082}\par
          \begin{table}[H]
    % \begin{table}[H]
    % \\ 'id2966513' '1'
        \begin{center}
      \label{m38717*id276082}
    \noindent
      \begin{tabular}{|l|l|l|l|}\hline
\textbf{Element} & \textbf{Relatiewe atoommassa (u)} & \textbf{Monster se massa (g)} & \textbf{Aantal mol in die monster} \\ \hline
        Waterstof & 1.01 & 1.01 & \\ \hline
        Magnesium & 24.31 & 24.31 & \\ \hline
        Koolstof & 12.01 & 24.02 & \\ \hline
        Chloor & 35.45 & 70.9 & \\ \hline
        Stikstof & 14.01 & 42.03 & \\ \hline
    \end{tabular}
      \end{center}
\end{table}
    \par
          \label{m38717*uid3}\item 
Hoeveel atome is daar in:
\label{m38717*id276311}\begin{enumerate}[noitemsep, label=\textbf{\alph*}. ] 
            \label{m38717*uid4}\item 1 mol van 'n stof
\label{m38717*uid5}\item 2 mol kalsium
\label{m38717*uid6}\item 5 mol fosfor
\label{m38717*uid7}\item $24,31\phantom{\rule{2pt}{0ex}}\text{g}$ magnesium
\label{m38717*uid8}\item $24,02\phantom{\rule{2pt}{0ex}}\text{g}$ koolstof
\end{enumerate}
                \end{enumerate}
    \label{m38717*cid3}
\practiceinfo
 \par \begin{tabular}[h]{cccccc}
 (1.) 02bs  &  (2.) 02bt  & \end{tabular}
\end{exercises}
            \subsection*{Mol\^{e}re massa}
            \nopagebreak
\Definition{Mol\^{e}re massa } { Mol\^{e}re massa (M) is die massa van 1 mol van 'n chemiese stof. Die eenheid vir die mol\^{e}re massa is gram per mol of $\text{g}\ensuremath{\cdot}\text{ mol}{}^{-1}$. } 
Jy sal onthou dat wanneer die massa, in gram, van 'n element gelyk is aan sy relatiewe atoommassa, jy te doene het met  'n monster van die element wat presies een mol van daardie element bevat. Hierdie massa word die \textbf{mol\^{e}re massa} van daardie element genoem.\\
\Note{Jy sal sien dat die mol\^{e}re massa soms geskryf word as ${M}_{m}$. Ons sal $M$ in hierdie boek gebruik, maar jy moet bewus wees van die alternatiewe notasie. }
\label{m38717*id276445}Dit is die moeite werd om die volgende te onthou: Die relatiewe atoommassa wat op die periodieke tabel verskyn, kan op twee maniere geïnterpreteer word.
\begin{enumerate}[noitemsep, label=\textbf{\arabic*}. ] 
\item Die massa van 'n \textsl{enkele, gemiddelde} atoom van daardie element in vergelyking met die massa van 'n atoom koolstof.
\item Die massa van een mol van die element. Die tweede gebruik is die mol\^{e}re massa van die element.
\end{enumerate}
    % \textbf{m38717*uid11}\par
          \begin{table}[H]
    % \begin{table}[H]
    % \\ '' '0'
        \begin{center}
      \label{m38717*uid11}
    \noindent
      \begin{tabular}{|l|l|l|p{3cm}|}\hline
                \textbf{Element}
               &
                \textbf{Relatiewe atoommassa (u)}
               &
                \textbf{Mol\^{e}re massa ($\text{g}\ensuremath{\cdot}\text{ mol}{}^{-1}$)}
               &
                \textbf{Massa van een mol van die element (g)} \\ \hline
        Magnesium &
        24,31 &
        24,31 &
        24,31 \\ \hline
        Litium &
        6,94 &
        6,94 &
        6,94 \\ \hline
        Suurstof &
        16 &
        16 &
        16 \\ \hline
        Stikstof &
        14,01 &
        14,01 &
        14,01  \\ \hline
        Yster &
        55,85 &
        55,85 &
        55,85 \\ \hline
    \end{tabular}
      \end{center}
    \caption{Die verhouding tussen die relatiewe atoommassa, mol\^{e}re massa en die massa van een mol vir 'n aantal elemente.}
\end{table}


      \begin{wex}{Bereken die aantal mol as massa gegee is}{
       %problem
      \label{m38717*id276776} Bereken die aantal mol van yster $\text{(Fe)}$ in 'n $11,7 ~\text{g}$ monster.  }
{
%solution
\westep{Vind die mol\^{e}re massa van yster}
As ons kyk na die periodieke tabel, sien ons dat die molêre massa van yster $55,8 \text{ g} \cdot \text{ mol}^{-1}$ is. Dit beteken $1$ mol yster het 'n massa van $55,8 ~\text{g}$.
      \westep{Bepaal die massa van yster} 
      \label{m38717*id276848}As $1$ mol van yster 'n massa van $55,8 \text{ g}$, het, dan volg dat die aantal mol yster in $111,7 ~\text{g}$  die volgende moet wees: 
\begin{eqnarray*}
n & = & \frac{111,7 ~\text{g}}{55,8 ~\text{g} \cdot{\text{ mol}}^{-1}} \\
 & = & \frac{111,7 \text{ g} \cdot \text{ mol}}{55,8 \text{ g}} \\
 & = & 2 ~\text{ mol}
\end{eqnarray*}
Daar is $2$ mol yster in die monster.
}
    \end{wex}

      \begin{wex}{Bereken die massa as aantal mol gegee is }{
%problem
\begin{minipage}{.85\textwidth}
      \label{m38717*id276928}Jy het 'n monster met $5$ mol sink.
      \label{m38717*id276934}\begin{enumerate}[noitemsep, label=\textbf{\alph*}. ] 
            \label{m38717*uid12}\item Wat is die massa van die sink in die monster?
\label{m38717*uid13}\item Hoeveel atome sink is in die monster?
\end{enumerate}
\end{minipage}
}
{
%solution
\westep{Vind die mol\^{e}re massa van sink}
      \label{m38717*id276984} Mol\^{e}re massa van sink is $65,4 ~\text{g} \cdot \text{ mol}^{-1}$, wat beteken dat $1$ mol sink 'n massa van $65,4 ~\text{g}$ het.
      \westep{Bepaal die massa}  
      \label{m38717*id277021}As $1$ mol sink 'n massa het van $65,4 ~\text{g}$, dan sal $5$ mol sink se massa die volgende wees: $65,4 ~\text{g} \times 5 ~\text{ mol}=327 ~\text{g}$ (antwoord vir a) 
      \westep{Bepaal die aantal atome} 
$5 \text{ mol} \times 6,022 \times {10}^{23} \text{ atome} \cdot \text{ mol}^{-1} = 3,011 \times {10}^{23} \text{ atome}$
      \label{m38717*id277263}(antwoord vir b)
}
    \end{wex}
            

\begin{exercises}{Moles en mol\^{e}re massa}
\label{m38717*id277281}\begin{enumerate}[noitemsep, label=\textbf{\arabic*}. ] 
%Q1
\label{m38717*uid14}\item Gee die mol\^{e}re massa van elk van die volgende elemente:
\label{m38717*id277295}\begin{enumerate}[noitemsep, label=\textbf{\alph*}. ] 
\label{m38717*uid15}\item waterstofgas
\label{m38717*uid16}\item stikstofgas
\label{m38717*uid17}\item broomgas
\end{enumerate}
%Q2
                \label{m38717*uid18}\item Bereken die aantal mol in elk van die volgende monsters:
\label{m38717*id277346}\begin{enumerate}[noitemsep, label=\textbf{\alph*}. ] 
            \label{m38717*uid19}\item $21,6 \text{ g}$ boor ($\text{B}$)
\label{m38717*uid20}\item $54,9 \text{ g}$ mangaan ($\text{Mn}$)
\label{m38717*uid21}\item $100,3 \text{ g}$ kwik ($\text{Hg}$)
\label{m38717*uid22}\item $50 \text{ g}$ barium ($\text{Ba}$)
\label{m38717*uid23}\item $40 \text{ g}$ lood ($\text{Pb}$)
\end{enumerate}
                \end{enumerate}
\practiceinfo
\par 
 \par \begin{tabular}[h]{cccccc}
 (1.) 02ue  &  (2.) 02uf  & \end{tabular}
\end{exercises}
            \subsection*{'n Vergelyking om mol en massa te bereken}
            \nopagebreak
      \label{m38717*id277432}Die eenheid vir mol\^{e}re massa is
$\text{mol\^{e}re ~massa ~(M)} = \dfrac{\text{massa ~(g)}}{\text{mol ~(mol)}}$ \\
Ons kan dit herrangskik om die getal mol te gee:
      \label{m38717*id277436}\nopagebreak\noindent{}
    \begin{equation*}
    \mathbf{n} = \dfrac{\mathbf{m}}{\mathbf{M}}
      \end{equation*}

      \label{m38717*id277605}
Die volgende diagram kan help om die verhouding tussen hierdie drie veranderlikes te onthou. Jy moet die horisontale lyn sien as  'n deelteken en die vertikale lyn as  'n ver\-me\-nig\-vul\-dig\-ings\-te\-ken. So as jy byvoorbeeld 'M' wil bereken, dan is die oorblywende twee letters in  die driehoek 'n' (onder) en 'm' (bo) met 'n deelteken tussen hulle. Jou berekening sal dan wees:
 $\text{M}=\dfrac{\text{m}}{\text{n}}$
\Tip{Onthou dat wanneer jy die vergelyking  $\text{n}=\frac{\text{m}}{\text{M}}$ gebruik, die massa altyd in gram ($\text{g}$) en mol\^{e}re massa in gram per mol moet wees ($\text{g} \cdot \text{ mol}{}^{-1}$).}
      \label{m38717*id277613}
    \setcounter{subfigure}{0}
	\begin{figure}[H] % horizontal\label{m38717*id277616}
\begin{center}
\scalebox{.8}{
\begin{pspicture}(-3,-3)(3,3)
%\psgrid[gridcolor=lightgray]
\psline(-3,-2)(0,2)
\psline(3,-2)(0,2)
\psline(-3,-2)(3,-2)
\psline(-1.6,-0.2)(1.6,-0.2)
\psline(0,-0.2)(0,-2)
\rput(0,0.8){\textbf{m}}
\rput(-0.8,-1){\textbf{n}}
\rput(0.8,-1){\textbf{M}}
\end{pspicture}
}
\end{center}
 \end{figure}       
      \par 
\label{m38717*secfhsst!!!underscore!!!id409}
      \noindent
      \begin{wex}{Bereken mol as massa gegee is}{
%problem      
\label{m38717*probfhsst!!!underscore!!!id410}
      \label{m38717*id277635}Bereken die aantal mol koper in 'n monster met 'n massa van $127 \text{ g}$. 
      }
{
%solution
 \westep{Skryf die vergelyking neer}
      \label{m38717*id277680}\nopagebreak\noindent{}
        
    \begin{equation*}
    \text{n}=\frac{\text{m}}{\text{M}}
      \end{equation*}
      \label{m38717*id277705}\nopagebreak\noindent{}
 \westep{Vind die aantal mol}        
    \begin{equation*}
    n=\frac{127 \text{ g}}{63,5 \text{g} \cdot \text{ mol}^{-1}}=2 \text{ mol}
      \end{equation*}
      \label{m38717*id277735}Daar is 2 mol koper in die monster.
}
    \end{wex}
    
    \noindent
\label{m38717*secfhsst!!!underscore!!!id494}
      \noindent
      \begin{wex}{Berekening van atome as massa gegee word}{
%problem
Bereken die aantal atome in 'n monster van aluminium met 'n massa van $81 \text{ g}$.
      }
{
%solution
\westep{Bepaal die aantal mol}
      \label{m38717*id277959}\nopagebreak\noindent{}
        
    \begin{equation*}
    n=\frac{m}{M}=\frac{81 \text{ g}}{27,0 \text{ g} \cdot \text{ mol}^{-1}} = 3 \text{ mol}
      \end{equation*}
      \westep{Bepaal die aantal atome}
      \label{m38717*id278019}Aantal atome in 3 mol aluminium $=3 \times 6,022 \times 10^{23}$ \\
      \label{m38717*id278053}Daar is $1,8069 \times 10^{24}$ aluminium atome in 'n monster van $81 \text{ g}$.
}
    \end{wex}
    \noindent
\label{m38717*secfhsst!!!underscore!!!id539}
            \begin{exercises} {Some simple calculations} \vspace{-1cm}
      \label{m38717*id278090}\begin{enumerate}[noitemsep, label=\textbf{\arabic*}. ] 
%Q1
            \label{m38717*uid24}\item Bereken die aantal mol in elk van die volgende monsters:
\label{m38717*id278106}\begin{enumerate}[noitemsep, label=\textbf{\alph*}. ] 
            \label{m38717*uid25}\item $5,6 \text{ g}$ kalsium
\label{m38717*uid26}\item $0,02 \text{ g}$ mangaan
\label{m38717*uid27}\item $40\text{ g}$ aluminium
\end{enumerate}
%Q2 
               \label{m38717*uid28}\item 'n Loodsinker het 'n massa van $5 \text{ g}$.
\label{m38717*id278159}\begin{enumerate}[noitemsep, label=\textbf{\alph*}. ] 
            \label{m38717*uid29}\item Bereken die aantal mol lood wat die sinker bevat.
\label{m38717*uid30}\item Hoeveel lood atome is in die sinker?
\end{enumerate}
%Q3
                \label{m38717*uid31}\item Bereken die massa van elk van die volgende monsters:
\label{m38717*id278201}\begin{enumerate}[noitemsep, label=\textbf{\alph*}. ] 
            \label{m38717*uid32}\item $2,5\text{ mol}$ magnesium
\label{m38717*uid33}\item $12 \text{ mol}$ litium
\label{m38717*uid34}\item $4,5 \times 10^{25}$ atome silikon
\end{enumerate}
                \end{enumerate}
\practiceinfo

 \par \begin{tabular}[h]{cccccc}
 (1.) 02ug  &  (2.) 02uh  &  (3.) 02ui  & \end{tabular}
\end{exercises}
            \subsection*{Verbindings}
            \nopagebreak
Tot dusver het ons slegs die mol, massa en mol\^{e}re massa in verhouding tot \textsl{elemente} bespreek. Wat gebeur as ons te doene het met 'n verbinding? Is dieselfde konsepte en re\"{e}ls dan van toepassing? Die antwoord is \textsl{ja}. Maar jy moet onthou dat al jou berekeninge van toepassing is op die hele verbinding. Wanneer jy die mol\^{e}re massa van 'n kovalente verbinding wil bepaal, moet jy die mol\^{e}re massa van elke atoom in daardie verbinding in berekening bring. Die aantal mol sal ook van toepassing wees op die hele molekule. As jy byvoorbeeld een mol van salpetersuur ($\text{HNO}_{3}$) se molêre massa bereken, is dit $63,01~\text{g}\cdot{\text{ mol}}^{-1}$ en daar is $6,022 \times 10^{23}$ molekules van salpetersuur. Vir rooster- strukture moet ons formule-massa gebruik. Formule-massa is die massa van al die atome in een formule-eenheid van die verbinding. Byvoorbeeld, een mol natriumchloried ($\text{NaCl}$) het 'n formule-massa van  $63,01~\text{g}\cdot{\text{ mol}}^{-1}$ en bestaan uit $6,022 \times 10^{23}$ molekules van natriumchloried.\\
In 'n gebalanseerde chemiese vergelyking toon die getal wat voor die element of verbinding geskryf staan die \textbf{molverhouding} aan  waarin die reaktante verbind om 'n produk te vorm. As daar geen getal aan die voorkant van die element se simbool verskyn nie, beteken dit dat die getal '1' is.\\
\mindsetvid{Khan academy video on the mole}{VPezc}
      \label{m38717*id278442}bv. ${\text{N}}_{2}+3{\text{H}}_{2}\to 2\text{N}{\text{H}}_{3}$\par 
      \label{m38717*id278488}In hierdie reaksie reageer, $1$ mol stikstof molekules met $3$ mol waterstof molekules om $2$ mol  ammoniak molekules te vorm.
\label{m38717*secfhsst!!!underscore!!!id566}
      \begin{wex}{Berekening van mol\^{e}re massa}{
%problem
      \label{m38717*probfhsst!!!underscore!!!id567}
      \label{m38717*id278505}Bereken die molêre massa van $\text{H}_{2}\text{SO}_{4}$.
      }
{
%solution
\westep{Gee die mol\^{e}re massa van elke element}
Waterstof $=1,01 \text{ g} \cdot \text{ mol}^{-1}$ \\ 
Swawel $=32,1\text{ g} \cdot \text{ mol}^{-1}$ \\
Oxygen $=16,0 \text{ g} \cdot \text{ mol}^{-1}$ 
      \westep{Bereken die mol\^{e}re massa van die verbinding}  
      \label{m38717*id278632}\nopagebreak\noindent{}
    \begin{equation*}
    {M}_{(\text{H}_{2}\text{SO}_{4})}=(2 \times 1,01 \text{ g} \cdot \text{ mol}^{-1}) + (32,1 \text{ g} \cdot \text{ mol}^{-1}) + (4 \times 16,0 \text{ g} \cdot \text{ mol}^{-1} ) = 98,12 \text{ g} \cdot \text{ mol}^{-1}
      \end{equation*}
}
    \end{wex}
    \noindent

            \label{m38717*secfhsst!!!underscore!!!id641} 
      \begin{wex}{Berekening van mol as massa gegee is}
{
%problem
Bereken die aantal mol in $1\text{ kg}$ $\text{MgCl}_{2}$.
     }
{
%solution
      \westep{Sit massa om in gram}  
\label{m38717*id278854}\nopagebreak\noindent{}
    \begin{equation*}
    m = 1~\text{ kg} \times 1 ~000=1 ~000\text{ g}
      \end{equation*}
\westep{Bereken die mol\^{e}re massa}
\label{m38717*id278912}\nopagebreak\noindent{}
    \begin{equation*}
    {M}_{(\text{MgCl}_{2})} = 24,3 \text{ g} \cdot \text{ mol}^{-1} + (2 \times 35,45 \text{ g} \cdot \text{ mol}^{-1}) = 95,2 \text{g} \cdot \text{ mol}^{-1}
      \end{equation*}    
      \westep{Bereken die aantal mol}
      \label{m38717*id279005}\nopagebreak\noindent{}
    \begin{equation*}
    n=\frac{1~000 \text{ g}}{95,2 \text{ g} \cdot \text{ mol}^{-1}} = 10,5 \text{ mol}
      \end{equation*}
      \label{m38717*id279046}Daar is $10,5 \text{ mol}$ magnesiumchloried in 'n $1 \text{ kg}$ monster.
}
    \end{wex}
\label{m38717*secfhsst!!!underscore!!!id832}
            \begin{groupdiscussion}{Verstaan van begrippe soos die mol, molekules en Avogadro se getal
      }
            \nopagebreak
      \label{m38717*id279596}Verdeel in drie groepe en gebruik ongeveer 20 minute om die volgende vrae te beantwoord:
      \label{m38717*id279603}\begin{enumerate}[noitemsep, label=\textbf{\arabic*}. ] 
            \label{m38717*uid39}\item Wat is die eenhede van die mol? Wenk: Gaan na die definisie van die mol.
\label{m38717*uid40}\item Jy het 'n $46 \text{ g}$ monster stikstofdioksied ($\text{NO}_{2}$)
\label{m38717*id279631}\begin{enumerate}[noitemsep, label=\textbf{\alph*}. ] 
\item Hoeveel \textbf{mol} $\text{NO}_{2}$ is daar in die monster?
\item Hoeveel mol stikstofatome is daar in die monster?
\item Hoeveel mol suurstofatome is daar in die monster?
\item Hoeveel \textbf{molekules} of $\text{NO}_{2}$ is daar in die monster?
\item Wat is die verskil tussen 'n mol en 'n molekule?
\end{enumerate}
        \label{m38717*uid44}\item Dis soms moeilik om  'n idee te vorm oor die presiese grootte van \textbf{Avogadro se getal}.
\label{m38717*id279703}\begin{enumerate}[noitemsep, label=\textbf{\alph*}. ] 
            \label{m38717*uid45}\item Skryf Avogadro se getal neer sonder om wetenskaplike notasie te gebruik.
\label{m38717*uid46}\item Hoe lank sal dit neem om te tel tot by Avogadro se getal? Jy kan aanneem
              dat jy twee syfers per sekonde tel.
\end{enumerate}
        \end{enumerate}
\end{groupdiscussion}

% \label{m38717*eip-945}
%     \setcounter{subfigure}{0}
% 	\begin{figure}[H] % horizontal\label{m38717*moles-1}
%     \textnormal{Khan academy video on the mole - 1} \nopagebreak
%   \label{m38717*yt-media2}\label{m38717*yt-video2}
%             \raisebox{-5 pt}{ \includegraphics[width=0.5cm]{col11305.imgs/summary_www.png}} { (Video:  P10085 )}
%  \end{figure}       \par \label{m38717*secfhsst!!!underscore!!!id850}

            \begin{exercises}{  More advanced calculations }
            \nopagebreak \noindent
      \label{m38717*id279756}\begin{enumerate}[noitemsep, label=\textbf{\arabic*}. ] 
%Q1
            \label{m38717*uid47}\item Bereken die mol\^{e}re massa van die volgende chemiese verbindings:
\label{m38717*id279772}\begin{enumerate}[noitemsep, label=\textbf{\alph*}. ] 
            \label{m38717*uid48}\item $\text{KOH}$
\label{m38717*uid49}\item $\text{FeCl}{}_{3}$
\label{m38717*uid50}\item ${\text{Mg(OH)}}_{2}$
\end{enumerate}
%Q2
                \label{m38717*uid51}\item Hoeveel mol is teenwoordig in:
\label{m38717*id279848}\begin{enumerate}[noitemsep, label=\textbf{\alph*}. ] 
            \label{m38717*uid52}\item $10 \text{ g}$ $\text{Na}_{2}\text{SO}{}_{4}$
\label{m38717*uid53}\item $34 \text{ g}$ $\text{Ca(OH)}{}_{2}$
\label{m38717*uid54}\item $2,45 \times 10{}^{23}$ molekules $\text{CH}{}_{4}$?
\end{enumerate}
%Q3
                \label{m38717*uid55}\item Vir 'n monster van $0,2 \text{ mol}$ magnesiumbromied ($\text{MgBr}_{2}$), bereken:
\label{m38717*id279964}\begin{enumerate}[noitemsep, label=\textbf{\alph*}. ] 
            \label{m38717*uid56}\item die aantal mol ${\text{Mg}}^{+}$ ione
\label{m38717*uid57}\item die aantal mol ${\text{Br}}^{-}$ ione
\end{enumerate}
%Q4
                \label{m38717*uid58}\item Jy het 'n monster met $3 \text{ mol}$ kalsiumchloried.
\label{m38717*id280031}\begin{enumerate}[noitemsep, label=\textbf{\alph*}. ] 
            \label{m38717*uid59}\item Wat is die chemiese formule van kalsiumchloried?
\label{m38717*uid60}\item Hoeveel atome kalsium is in die monster aanwesig?
\end{enumerate}
%Q5 
               \label{m38717*uid61}\item Bereken die massa van::
\label{m38717*id280072}\begin{enumerate}[noitemsep, label=\textbf{\alph*}. ] 
            \label{m38717*uid62}\item $3\text{ mol}$ $\text{NH}{}_{4}\text{OH}$
\label{m38717*uid63}\item $4,2 \text{ mol}$ $\text{Ca}\left(\text{NO}{}_{3}\right){}_{2}$\end{enumerate}

\end{enumerate}
\practiceinfo
\par 
 \par \begin{tabular}[h]{cccccc}
 (1.) 02uj  &  (2.) 02uk  &  (3.) 02um  &  (4.) 02un  &  (5.) 02up  &    &   & \end{tabular}
\end{exercises}
% \subsection*{Empirical formula and molecular formula}
%       \label{m38712*id280317}The \textbf{empirical formula} of a chemical compound is a simple expression of the relative number of each type of atoom in that compound. In contrast, the \textbf{molecular formula} of a chemical compound gives the actual number of atoms of each element found in a molecule of that compound.\par 
% \label{m38712*fhsst!!!underscore!!!id885}
% \Definition{   \label{id2501853}Empirical formula } { \label{m38712*meaningfhsst!!!underscore!!!id885}
%       \label{m38712*id280341}The empirical formula of a chemical compound gives the relative number of each type of atoom in that compound. \par 
%        } 
% \label{m38712*fhsst!!!underscore!!!id888}
% \Definition{   \label{id2501878} Molecular formula } { \label{m38712*meaningfhsst!!!underscore!!!id888}
%       \label{m38712*id280360}The molecular formula of a chemical compound gives the exact number of atoms of each element in one molecule of that compound. \par 
%        } 
%       \label{m38712*id280372}The compound \textsl{ethanoic acid} for example, has the molecular formula $\text{CH}{}_{3}\text{COOH}$ or simply $\text{C}{}_{2}\text{H}{}_{4}\text{O}{}_{2}$. In one molecule of this acid, there are two carbon atoms, four hydrogen atoms and two oxygen atoms. The ratio of atoms in the compound is 2:4:2, which can be simplified to 1:2:1. Therefore, the empirical formula for this compound is $\text{CH}{}_{2}\text{O}$. The empirical formula contains the smallest whole number ratio of the elements that make up a compound.\par 
         \section{Samestelling}
    \nopagebreak
%            \label{m38712} $ \hspace{-5pt}\begin{array}{cccccccccccc}   \includegraphics[width=0.75cm]{col11305.imgs/summary_fullmarks.png} &   \includegraphics[width=0.75cm]{col11305.imgs/summary_video.png} &   \includegraphics[width=0.75cm]{col11305.imgs/summary_presentation.png} &   \end{array} $ \hspace{2 pt}\raisebox{-5 pt}{} {(section shortcode: P10086 )} \par 
      \label{m38712*id280450}Kennis van die empiriese of molekul\^{e}re formule van 'n verbinding kan help om die samestelling van die verbinding in meer detail te bepaal. Die teenoorgestelde is ook waar. Kennis van die \textsl{samestelling} van 'n stof kan jou help om sy formule te bepaal. Daar is vier verskillende wyses waarop vrae rondom die samestelling van verbindings gevra kan word:
      \label{m38712*id280463}\begin{enumerate}[noitemsep, label=\textbf{\arabic*}. ] 
\item Vrae waar die formule van die \textbf{verbinding} aan jou gegee word en jy gevra word om die \textbf{persentasie} wat elke element tot die massa van die verbinding bydrae (persentasie-samestelling), te \textbf{bereken}.
\item Probleme waar jy die \textbf{persentasie-samestelling} gegee word en gevra word om die \textbf{formule te bereken}.
\item Vrae waar die \textbf{produkte} van  'n chemiese reaksie aan jou gegee word en jy die formule van een van die reaktante moet \textbf{bereken}. Daar word dikwels in hierdie verband verwys na verbrandingsanalise-probleme.
\item Probleme waar jy gevra sal word om die aantal mol van \textbf{kristalwater} te vind.
\end{enumerate}
Die volgende uitgewerkte voorbeelde sal demonstreer hoe om  elkeen van hierdie soorte problem aan te pak.
            \label{m38712*secfhsst!!!underscore!!!id901}
      \noindent
\mindsetvid{Percentage composition and empirical formula}{VPbue}
\clearpage
\begin{wex}{Berekening van die persentasie-samestelling volgens die  massa van elemente in 'n verbinding}
{
%problem
Bereken die persentasie wat elke element tot die totale massa van swaelsuur (${\text{H}}_{2}{\text{SO}}_{4}$) bydrae.
}
{
%solution
\westep{Bereken die mol\^{e}re massa}
$\text{Waterstof}=2 \times 1,01 = 2,02 \text{ g} \cdot \text{ mol}^{-1}$ \\ 
$\text{Swawel}=32,1 \text{ g} \cdot \text{ mol}^{-1}$ \\
$\text{Suurstof}=4 \times 16,0 = 64,0 \text{ g} \cdot \text{ mol}^{-1}$
      \westep{Gebruik die inligting uit die vorige stap om die mol\^ere massa van swaelsuur te bereken.}
$\text{Mass}=2,02 \text{ g} \cdot \text{ mol}^{-1} + 32,1 \text{ g} \cdot \text{ mol}^{-1} + 64,0 \text{ g} \cdot \text{ mol}^{-1} = 98,12 \text{ g} \cdot \text{ mol}^{-1}$
      \westep{Gebruik die vergelyking}
      \label{m38712*id280688}$\text{Persentasie volgens massa}=\dfrac{\text{atoommassa}\times \text{formule massa van element}}{\text{mol\^{e}re massa van H}{}_{2}\text{SO}{}_{4}} \times 100\%$ \\

        \textsl{Waterstof}      
      \label{m38712*id280735}\nopagebreak\noindent{}        
    \begin{equation*}
    \frac{2,02 \text{ g} \cdot \text{ mol}^{-1}}{98,12 \text{ g} \cdot \text{ mol}^{-1}}\ensuremath{\times}100\%=2,0587\%
      \end{equation*}

        \textsl{Swael}      
      \label{m38712*id280786}\nopagebreak\noindent{}        
    \begin{equation*}
    \frac{32,1 \text{ g} \cdot \text{ mol}^{-1}}{98,12 \text{ g} \cdot \text{ mol}^{-1}}\ensuremath{\times}100\%=32,7150\%
      \end{equation*}

        \textsl{Suurstof}     
      \label{m38712*id280837}\nopagebreak\noindent{}
    \begin{equation*}
    \frac{64,0 \text{ g} \cdot \text{ mol}^{-1}}{98,12 \text{ g} \cdot \text{ mol}^{-1}}\ensuremath{\times}100\%=65,2263\%
      \end{equation*}
      \label{m38712*id280876}(Kontroleer aan die einde of jou totale persentasies gelyk is aan $100\%$!) \\
      \label{m38712*id280880}Met ander woorde, in een molekule van swaelsuur dra waterstof $2,06\%$ by tot die massa van die verbinding, swael $32,71\%$ en suurstof $65,23\%$. 
}
    \end{wex}
    \noindent
\label{m38712*secfhsst!!!underscore!!!id1029}
      \noindent 
      \begin{wex}{Bepaling van die empiriese formule van 'n verbinding}{
 %problem
 'n Verbinding bevat $52,2\%$ koolstof ($\text{C}$), $13,0\%$ waterstof ($\text{H}$) en $34,8\%$ suurstof ($\text{O}$). Bepaal die empiriese formule.      
}
{ %solution
\westep{Gee die massas}
      \label{m38712*id280928}Koolstof $=52,2 \text{ g}$, waterstof $=13,0 \text{ g}$ en suurstof $=34,8 \text{ g}$ 
      \westep{Bereken die aantal mol} 
      \label{m38712*id280954}\nopagebreak\noindent{}
        
    \begin{equation*}
    \text{n}=\frac{\text{m}}{\text{M}}
      \end{equation*}
      \label{m38712*id280975}Daarom: 
      \label{m38712*id280978}\nopagebreak\noindent{}
        
    \begin{equation*}
    \text{n}\left(\text{Koolstof}\right)=\frac{52,2 \text{ g}}{12,0 \text{ g} \cdot \text{ mol}^{-1}}=4,35\text{ mol}
      \end{equation*}
      \label{m38712*id281042}\nopagebreak\noindent{}
        
    \begin{equation*}
    \text{n}\left(\text{Waterstof}\right)=\frac{13,0 \text{ g}}{1,01 \text{ g} \cdot \text{ mol}^{-1}}=12,871\text{ mol}
      \end{equation*}
      \label{m38712*id281111}\nopagebreak\noindent{}
        
    \begin{equation*}
    \text{n}\left(\text{Suurstof}\right)=\frac{34,8 \text{g}}{16,0 \text{g} \cdot \text{ mol}^{-1}}=2,175\text{ mol}
      \end{equation*}
      \westep{Vind die kleinste aantal mol}
Gebruik die verhoudings van die molgetalle hierbo bereken om die empiriese formule te bepaal.\\
$\text{eenhede in  empiriese formule} = \dfrac{\text{aantal mol van hierdie element}}{\text{kleinste getal mol (bereken)}}$\newline \\
In hierdie geval is die kleinste aantal mol $2,175$. Dus:\\ 
      \label{m38712*id281179}
        \textsl{Koolstof}
 
      \label{m38712*id281185}\nopagebreak\noindent{}
        
    \begin{equation*}
    \frac{4,35}{2,175}=2
      \end{equation*}
      \label{m38712*id281217}
        \textsl{Waterstof}
     
      \label{m38712*id281223}\nopagebreak\noindent{}
        
    \begin{equation*}
    \frac{12,871}{2,175}=6
      \end{equation*}
      \label{m38712*id281254}
        \textsl{Suurstof}
      
      \label{m38712*id281261}\nopagebreak\noindent{}
        
    \begin{equation*}
    \frac{2,175}{2,175}=1
      \end{equation*}
      \label{m38712*id281292}Daarom is die empiriese formule van hierdie stof: ${\text{C}}_{2}{\text{H}}_{6}\text{O}$.
}
    \end{wex}
    \noindent
\label{m38712*secfhsst!!!underscore!!!id1235}
      \noindent
      \begin{wex}{Bepaling van die formule van 'n verbinding}{
%problem
$207 \text{ g}$ lood verbind met suurstof om $239 \text{ g}$ loodoksied te vorm. Gebruik hierdie inligting om
    die formule van loodoksied uit te werk (relatiewe atoommassas: $\text{Pb}=207,2 \text{ u}$ en $\text{O} = 16,0 \text{ u}$).
}
{
%solution
\westep{Vind die massa van suurstof}
      \label{m38712*id281379}\nopagebreak\noindent{}
    \begin{equation*}
    239 \text{ g}-207 \text{ g}=32 \text{ g}
      \end{equation*}
      \westep{Bepaal die aantal mol suurstof}
      \label{m38712*id281407}\nopagebreak\noindent{}
        
    \begin{equation*}
    \text{n}=\frac{\text{m}}{\text{M}}
      \end{equation*}
      \label{m38712*id281427}
        \textsl{Lood}
       
      \label{m38712*id281433}\nopagebreak\noindent{}
        
    \begin{equation*}
    n = \frac{207 \text{ g}}{207,2 \text{ g} \cdot \text{ mol}^{-1}}=1 \text{ mol}
      \end{equation*}
      \label{m38712*id281460}
        \textsl{Suurstof}
    
      \label{m38712*id281467}\nopagebreak\noindent{}
        
    \begin{equation*}
    n= \frac{32 \text{ g}}{16,0 \text{ g} \cdot \text{ mol}^{-1}}=2 \text{ mol}
      \end{equation*}
      \westep{Vind die molverhouding}
      \label{m38712*id281498}Die molverhouding van $\text{Pb}:\text{O}$ in die produk is $1:2$, wat beteken dat daar vir elke atoom lood twee atome suurstof sal wees. Die formule van die verbinding is $\text{PbO}{}_{2}$. 
}
    \end{wex}
    \noindent
\label{m38712*secfhsst!!!underscore!!!id1308} 
      \noindent 
      \begin{wex}{Empiriese formule en molekul\^{e}re formule
      }
 {
%problem
\begin{minipage}{.85\textwidth}
Asyn, wat in ons huise gebruik word, is 'n verdunde vorm van asynsuur. 'n Monster van
      asynsuur het die volgende persentasiesamestelling: $39,9\%$ koolstof, $6,7\%$ waterstof en $53,4\%$ suurstof.  
\begin{enumerate}[noitemsep, label=\textbf{\arabic*}. ] 
\item Bepaal die empiriese formule van asynsuur.
\item Bepaal die molekulêre formule van asynsuur indien die mol\^{e}re massa van asynsuur $60,06 \text{ g} \cdot \text{ mol}{}^{-1}$.
\end{enumerate} \end{minipage}
     }
{
%solution
\westep{Bepaal die massa}
      \label{m38712*id281607}In $100 \text{ g}$ asynsuur is daar $39,9 \text{ g C}$, $6,7 \text{ g H}$ en $53,4\text{ g O}$ 
      \westep{Bereken die aantal mol}

        $\text{n}=\dfrac{\text{m}}{\text{M}}$
      
      \label{m38712*id281653}\nopagebreak\noindent{}
        
    \begin{eqnarray*}
{\text{n}}_{\text{C}} & = & \frac{39,9 \text{ g}}{12,0 \text{ g} \cdot \text{ mol}^{-1}} = 3,325 \text{ mol} \\
{\text{n}}_{\text{H}} & = & \frac{6,7 \text{ g}}{1,01 \text{ g} \cdot \text{ mol}^{-1}} = 6,6337 \text{ mol} \\
{\text{n}}_{\text{O}} & = & \frac{53,4 \text{ g}}{16,0 \text{ g} \cdot \text{ mol}^{-1}} = 3,3375 \text{ mol}
      \end{eqnarray*}
      \westep{Bepaal die empiriese formule} 
\begin{tabular}{c@{:}c@{:}c}
$\text{C}$ & $\text{H}$ & $\text{O}$\\
$3,325~$ & $~6,6337~$ & $~3,3375$ \\
$1$ & $2$ & $1$\\
\end{tabular}\\
Empiriese formule is $\text{CH}{}_{2}\text{O}$ 
      \westep{Bepaal nou die molekul\^{e}re formule} 
      \label{m38712*id281834}Die mol\^{e}re massa van asynsuur, as die empiriese formule gebruik word, is $30,02 \text{ g} \cdot \text{ mol}{}^{-1}$. Die vraag gee egter die mol\^{e}re massa as $60,06~\text{g}\cdot \text{ mol}^{-1}$. Die werklike aantal mol van elke element moet dus dubbeld soveel wees as in die empiriese formule. Die molekul\^{e}re formule is dus  ($\frac{60,06}{30,02}=2$).
      \label{m38712*id281854}Die molekul\^ere formule is dus $\text{C}{}_{2}\text{H}{}_{4}\text{O}{}_{2}$ of $\text{CH}{}_{3}\text{COOH}$
}
    \end{wex}
    \noindent
\par
            \label{m38712*eid672431}
      \noindent
      \begin{wex}{Kristalwater}{
 %problem
\label{m38712*pid47982}
\label{m38712*id64827}Aluminiumtrichloried (${\text{AlCl}}_{3}$) is 'n ioniese stof wat kristalle vorm in die vaste fase. Watermolekules mag vasgevang wees binne-in die kristalrooster. Ons skryf dit as volg neer: ${\text{AlCl}}_{3} \cdot n{\text{H}}_{2}\text{O}$. Carine verhit  aluminiumtrichloried-kristalle totdat al die water verdamp het en vind dat die massa na verhitting $2,8 \text{ g}$. Die massa voor verhitting was $5 \text{ g}$. Wat is die aantal mol watermolekules in die aluminiumtrichloried?
}
{
%solution
\westep{Bepaal die aantal watermolekules}Ons moet eers n, die aantal watermolekules wat teenwoordig is in die kristal, vasstel. Om dit te doen moet ons eers besef dat die massa water wat verloor is, gelyk is aan $5 \text{ g} - 2,8 \text{ g} = 2,2 \text{ g}$.
 \westep{Bepaal die massa verhouding} \label{m38712*id3892}Die massa verhouding is:\\
\begin{tabular}{r@{:}l}
 $\text{AlCl}_3~$ & $~\text{H}_{2}\text{O}$ \\
   $2,8~$ & $~2,2$ \\
\end{tabular}
\westep{Bereken die molverhouding}
Om die molverhouding vas te stel, verdeel ons die massa-verhouding deur die mol\^{e}re massa van elke element:\\
\begin{tabular}{r@{:}l}
 $\text{AlCl}_3~$ & $~\text{H}_{2}\text{O}$ \\
    $\dfrac{2,8 \text{ g}}{133,35 \text{ g} \cdot \text{ mol}^{-1}}~$ & $~\dfrac{2,2 \text{ g}}{18,02 \text{ g} \cdot \text{ mol}^{-1}}$ \\
$0,02099...~$ & $~0,12...$  \\
\end{tabular}\\
Nou doen ons die volgende: \\
\begin{tabular}{r@{:}l}
 $\text{AlCl}_3~$ & $~\text{H}_{2}\text{O}$ \\
$0,020997375~$ & $~0,12208657$ \\
% $\frac{0,020997375}{0,020997375}~$ & $~\frac{0,12208657}{0,020997375}$ \\
$\dfrac{0,021}{0,021}~$ & $~\dfrac{0,122}{0,021}$ \\
$1~$ & $~6$ \\
\end{tabular}\\
So die molverhouding van aluminiumtrichloried tot water is: $1:6$
\westep{Skryf die finale antwoord neer}
Ons weet nou dat daar $6$ mol watermolekules in die kristal aanwesig is. Die formule is $\text{AlCl}_{3} \cdot 6\text{H}_{2}\text{O}$.
}
    \end{wex}
Ons kan eksperimente uitvoer om die samestelling van stowwe te bepaal. Blou kopersulfaat ($\text{CuSO}_{4}$) kristalle bevat byvoorbeeld water. Tydens verhitting verdamp die kristalwater en die blou kristalle word wit. Deur die begin- en eind- produkte te weeg, kan ons die hoeveelheid water wat in kopersulfaat aanwesig is, bepaal. Nog 'n voorbeeld is die verhitting van magnesiumlint. As ons 'n bekende hoeveelheid magnesiumlint verhit en dan die massa na verhitting bepaal, kan ons die inligting gebruik om die samestelling van die stof te bepaal. Nog 'n voorbeeld is die redusering van koperoksied na koper.

    \noindent

% \label{m38712*eip-762}
%     \setcounter{subfigure}{0}
% 	\begin{figure}[H] % horizontal\label{m38712*formulae-1}
%     \textnormal{Khan academy video on molecular and empirical formulae - 1} \nopagebreak
%   \label{m38712*yt-media1}\label{m38712*yt-video1}
%             \raisebox{-5 pt}{ \includegraphics[width=0.5cm]{col11305.imgs/sum\mary_www.png}} { (Video:  P10087 )}
%  \end{figure}       \par \label{m38712*eip-306}
%     \setcounter{subfigure}{0}
% 	\begin{figure}[H] % horizontal\label{m38712*masscompostion-1}
%     \textnormal{Khan academy video on mass composition - 1}\nopagebreak
%   \label{m38712*yt-media3}\label{m38712*yt-video3}
%             \raisebox{-5 pt}{ \includegraphics[width=0.5cm]{col11305.imgs/summary_www.png}} { (Video:  P10088 )}
%  \end{figure}       \par \label{m38712*secfhsst!!!underscore!!!id1437}
            \begin{exercises}{Moles and empirical formulae}
      \label{m38712*id281924}\begin{enumerate}[noitemsep, label=\textbf{\arabic*}. ] 
%Q1
            \label{m38712*uid73}\item Kalsiumchloried word geproduseer as die produk van 'n chemiese reaksie.
\label{m38712*id281940}\begin{enumerate}[noitemsep, label=\textbf{\alph*}. ] 
            \label{m38712*uid74}\item Wat is die formule van kalsiumchloried?
\label{m38712*uid75}\item Watter persentasie dra elk van die elemente by tot die massa van 'n molekule van kalsiumchloried?
\label{m38712*uid76}\item Indien die monster $5 \text{ g}$ kalsiumchloried bevat, wat is die massa van kalsium in die monster?
\label{m38712*uid77}\item Hoeveel mol kalsiumchloried is in die monster?
\end{enumerate}
%Q2
                \label{m38712*uid78}\item $13\phantom{\rule{2pt}{0ex}}\text{g}$ sink verbind met $6,4\phantom{\rule{2pt}{0ex}}\text{g}$ swawel.
\label{m38712*id282007}\begin{enumerate}[noitemsep, label=\textbf{\alph*}. ] 
\item Watter massa van sink sulfied sal geproduseer word?
\item Watter persentasie dra elk van die elemente in sinksulfied by tot sy massa?
\item Die mol\^ere massa van sinksulfied is $97,44 \text{ g} \cdot \text{ mol}^{−1}$ . Verkry die molekul\^ere formule vir sinksulfied.
\end{enumerate}
%Q3
                \label{m38712*uid82}\item 'n Kalsium mineraal bestaan ​​uit $29,4\%$ kalsium, $23,5\%$ swawel en $47,1\%$ suurstof volgens massa. Bereken die empiriese formule van die mineraal.
%Q4
\label{m38712*uid83}\item 'n Gechlorineerde koolwaterstof-verbinding is ontleed en daar is gevind dat dit bestaan ​​uit $24,24\%$ koolstof, $4,04\%$ waterstof en $71,72\%$ chloor. Vanuit  'n ander eksperiment volg dat die mol\^ere massa van die verbinding $99\text{ g} \cdot \text{ mol}{}^{-1}$. Lei die empiriese en molekulêre formules vanuit die gegewe inligting af.
%Q5
\item Magnesiumsulfaat het die formule $\text{MgSO}_{4} \cdot \text{n H}_{2}\text{O}$. 'n Monster wat $5,0 ~\text{g}$ magnesiumsulfaat bevat word verhit totdat al die water verdamp het. Die finale massa is $2,6~\text{g}$. Hoeveel watermolekules was in die oorspronklike monster? 
\end{enumerate}
\practiceinfo
\par 
 \par \begin{tabular}[h]{cccccc}
 (1.) 02uq  &  (2.) 02ur  &  (3.) 02us  &  (4.) 02ut  & (5.) 02uu \end{tabular}
\end{exercises}
\section{Hoeveelheid van stowwe}
            \subsection*{Mol\^{e}re Volumes van Gasse}
            \nopagebreak
            \par
            \label{m38712*eip-id1168064596799}
 \Definition{Mol\^{e}re volume van gasse} {Een mol gas beslaan $22,4{\text{ dm}}^{3}$ by standaard temperatuur en druk. } 
Dit geld vir enige gas wat by standaard temperatuur en druk is. In graad 11 sal jy meer hieroor leer asook kennis maak met die gaswette.\\
Byvoorbeeld, $2~\text{ mol}$ $\text{H}_2$ gas sal 'n volume van $44,8{\text{ dm}}^{3}$ beslaan by standaard temperatuur en druk (S.T.D.). en $67,2{\text{ dm}}^{3}$ ammoniak gas ($\text{NH}_3$) bevat $3~\text{ mol}$ ammoniak.
\Note{Standaard temperatuur en druk (S.T.D.) word gedefinieer as 'n temperatuur van 273,15 K en 'n druk van 0.986 atmosfeer.}
    \label{m38712*cid8}
            \subsection*{Mol\^{e}re konsentrasies van vloeistowwe}
            \nopagebreak
n Tipiese oplossing word gemaak deur vaste stof op te los in 'n vloeistof. Die hoeveelheid van die vaste stof wat opgelos word in  'n gegewe volume water, staan bekend as die \textbf{konsentrasie} van die vloeistof. Wiskundig word konsentrasie (C) gedefinieer as die mol opgeloste stof (n) per eenheid volume (V) van  'n oplossing.\\
\mindsetvid{solutions and moles}{VPbur}
      \label{m38712*id282860}\nopagebreak\noindent{}      
    \begin{equation*}
    \text{C}=\frac{\text{n}}{\text{V}}
      \end{equation*}
	\begin{figure}[H] % horizontal\label{m38717*id277616}
\begin{center}
\scalebox{0.65}{
\begin{pspicture}(-3,-3)(3,3)
%\psgrid[gridcolor=lightgray]
\psline(-3,-2)(0,2)
\psline(3,-2)(0,2)
\psline(-3,-2)(3,-2)
\psline(-1.6,-0.2)(1.6,-0.2)
\psline(0,-0.2)(0,-2)
\rput(0,0.8){\Large{\textbf{n}}}
\rput(-0.8,-1){\Large{\textbf{C}}}
\rput(0.8,-1){\Large{\textbf{V}}}
\end{pspicture}
}
\end{center}
 \end{figure}
Vir hierdie vergelyking is die eenheid waarin volume gemeet word $\text{dm}{}^{3}$ (wat gelyk is aan 1 liter). Daarom is die eenheid van  konsentrasie  $\text{ mol} \cdot {\text{dm}}^{-3}$. 
\label{m38712*fhsst!!!underscore!!!id1650}
\Definition{ Konsentrasie } {Konsentrasie is 'n maatstaf van die hoeveelheid opgeloste stof wat in 'n gegewe volume vloeistof opgelos is. Dit word gemeet in $\text{ mol} \cdot {\text{dm}}^{-3}$.} 
 
      \noindent
      \begin{wex}{Berekening van Konsentrasie I}
{
%problem
      \label{m38712*probfhsst!!!underscore!!!id1654}
      \label{m38712*id283003}As $3,5\text{ g}$ natriumhidroksied ($\text{NaOH}$) oplos in $2,5 {\text{ dm}}^{3}$ water, wat is die konsentrasie van die oplossing in $\text{ mol}\ensuremath{\cdot}{\text{dm}}^{-3}$? }
{
%solution
\westep{Bepaal die aantal mol natriumhidroksied} 
      \label{m38712*id283067}\nopagebreak\noindent{}
    \begin{equation*}
    \text{n}=\frac{\text{m}}{\text{M}}=\frac{3,5 \text{ g}}{40,01 \text{ g} \cdot \text{ mol}^{-1}} = 0,0875 \text{ mol}
      \end{equation*}
      \westep{Bereken die konsentrasie} 
\begin{equation*}
\text{C}=\frac{\text{n}}{\text{V}}=\frac{0,0875 \text{ mol}}{2,5 \text{ dm}^{3} }=0,035 \text{ mol} \cdot \text{dm}^{-3}
\end{equation*}
Die konsentrasie van die oplossing is $0,035 \text{ mol} \cdot {\text{dm}}^{-3}$.
}
    \end{wex}

    \noindent
\par


      \begin{wex}{Berekening van Konsentrasie II }
{
%problem
Jy het 'n $1 {\text{ dm}}^{3}$ houer waarin 'n oplossing van kaliumpermanganaat ($\text{KMnO}{}_{4}$) gemaak moet word. Watter massa $\text{KMnO}{}_{4}$ is nodig is om 'n oplossing met 'n konsentrasie van $0,2 \text{ mol}\cdot \text{dm}^{-3}$ voor te berei? 
     }
{
%solution
\westep{Bereken die aantal mol}
$\text{C}=\dfrac{\text{n}}{\text{V}}~$ dus:
      \label{m38712*id283321}\nopagebreak\noindent{}        
    \begin{equation*}
    \text{n}=\text{C}\ensuremath{\times}\text{V}=0,2 \text{ mol} \cdot \text{dm}^{-3} \times 1 \text{dm}^{-3} = 0,2 \text{ mol}
      \end{equation*}
      \westep{Bepaal die massa}  
$\text{m}=\text{n} \times \text{M} = 0,2 \text{ mol} \times 158 \text{ g} \cdot \text{ mol}^{-1} = 31,6 \text{ g}$\\
Die massa $\text{KMnO}{}_{4}$ wat benodig word, is $31,6 \text{ g}$.
 
}
    \end{wex}
    \noindent
\label{m38712*secfhsst!!!underscore!!!id1795} 

      \begin{wex}{Berekening van Konsentrasie III }
{
%problem      
\label{m38712*id283476}Hoeveel natriumchloried (in g) sal nodig wees om 'n $500 {\text{ cm}}^{3}$ oplossing voor te berei met 'n konsentrasie van $0,01 \text{ mol} \cdot \text{dm}^{-3}$?
   }
{
%solution
\westep{Bepaal die volume}
$\text{V}= 500 \text{cm}^{3} \dfrac{1 \text{ dm}^{3}}{1 000 \text{ cm}^{3}}=0,5 {\text{ dm}}^{3}$
      \westep{Bepaal die aantal mol} 
$\text{n}=\text{C} \times \text{V}= 0,01 \text{ mol} \cdot \text{dm}^{-3} \times 0,5 \text{ dm}^{-3} = 0,005 \text{ mol}$
      \westep{Bereken die massa}
$\text{m}=\text{n} \times \text{M}= 0,005 \text{ mol} \times 58,45 \text{ g} \cdot \text{ mol}^{-1} = 0,29 \text{ g}$\\
Die massa natriumchloried wat benodig word, is $0,29 \text{ g}$ 
}
    \end{wex}
    \noindent \vspace{-1cm}
\label{m38712*secfhsst!!!underscore!!!id1879}
            \begin{exercises}{ Concentration of solutions
      }
            \nopagebreak \noindent \vspace{-1cm}
      \label{m38712*id283713}\begin{enumerate}[noitemsep, label=\textbf{\arabic*}. ] 
%Q1
\item $5,95 \text{ g}$ kaliumbromied is in $400 {\text{ cm}}^{3}$ water opgelos. Bereken die konsentrasie van die oplossing.
%Q2
\item $100 \text{ g}$ natriumchloried ($\text{NaCl}$) is opgelos in $450 {\text{ cm}}^{3}$ water.
  \begin{enumerate}[noitemsep, label=\textbf{\alph*}. ] 
    \item Hoeveel mol NaCl is teenwoordig in die oplossing?
    \item Wat is die volume van die water (in ${\text{dm}}^{3}$)?
    \item Bereken die konsentrasie van die oplossing.
\end{enumerate}
%Q3
\item Wat is die molariteit van die oplossing wat gevorm word deur $80 \text{ g}$ natriumhidroksied ($\text{NaOH}$) by $500 {\text{ cm}}^{3}$ te voeg? 
%Q4
\item Watter massa (g) waterstofchloried ($\text{HCl}$) word benodig om $1000 {\text{ cm}}^{3}$ van 'n oplossing met konsentrasie $1 \text{ mol} \cdot {\text{dm}}^{-3}$?
%Q5
\item Hoeveel mol $\text{H}{}_{2}\text{SO}{}_{4}$ is daar in $250 {\text{ cm}}^{3}$ van 'n $0,8 \text{ mol} \cdot \text{dm}^{-3}$ swaelsuur oplossing? Wat is die massa van die suur in hierdie oplossing?
\end{enumerate}
\practiceinfo
 \begin{tabular}[h]{cccccc}
 (1.) 02uv  &  (2.) 02uw  &  (3.) 02ux  &  (4.) 02uy  &  (5.) 02uz  & \end{tabular}
\end{exercises}
            \section{Stoichiometriese berekeninge}
            \nopagebreak
      \label{m38712*id283990}Stoichiometrie is die berekening van die hoeveelhede van die reaktante en produkte in chemiese reaksies. Dit is belangrik om te weet hoeveel van  'n produk sal in 'n chemiese reaksie gevorm word, of hoeveel van 'n reagens benodig word om 'n spesifieke produk te maak.\par 
Die volgende diagram toon hoe die konsepte waarmee ons kennis gemaak het in hierdie hoofstuk, aansluit by mekaar sowel as by die aspek van balansering van chemiese vergelykings:\\
\begin{figure}[H]
 \begin{center}
\scalebox{1} % Change this value to rescale the drawing.
{
\begin{pspicture}(0,-1.10125)(10.12625,1.10125)
\psframe[linewidth=0.04,dimen=outer](6.4678125,0.955625)(3.4278126,-0.124375)
\psline[linewidth=0.06cm,arrowsize=0.05291667cm 2.0,arrowlength=1.4,arrowinset=0.0]{<->}(2.3278124,0.395625)(3.2878125,0.395625)
\psline[linewidth=0.06cm,arrowsize=0.05291667cm 2.0,arrowlength=1.4,arrowinset=0.0]{<->}(6.6078124,0.375625)(7.5678124,0.375625)
\usefont{T1}{ptm}{m}{n}
\rput(2.0478125,0.475625){\Huge{$\rbrace$}}
\usefont{T1}{ptm}{m}{n}
\rput(7.7076564,0.395625){\Huge{$\lbrace$}}
\usefont{T1}{ptm}{m}{n}
\rput(8.5,1.005625){mass}
\usefont{T1}{ptm}{m}{n}
\rput(8.96,0.605625){mol\^{e}re massa}
\usefont{T1}{ptm}{m}{n}
\rput(8.671094,0.205625){volume}
\usefont{T1}{ptm}{m}{n}
\rput(9.110937,-0.194375){concentration}
\usefont{T1}{ptm}{m}{n}
\rput(1.48,1.005625){mass}
\usefont{T1}{ptm}{m}{n}
\rput(1.08,0.605625){mol\^{e}re massa}
\usefont{T1}{ptm}{m}{n}
\rput(1.3910937,0.205625){volume}
\usefont{T1}{ptm}{m}{n}
\rput(0.9509375,-0.194375){konsentrasie}
\usefont{T1}{ptm}{m}{n}
\rput(4.9453125,0.405625){AANTAL MOL}
\usefont{T1}{ptm}{b}{n}
\rput(4.978906,-0.874375){Gebalanseerde vergelyking}
\usefont{T1}{ptm}{b}{n}
\rput(1.018125,-0.834375){Reaktanse}
\usefont{T1}{ptm}{b}{n}
\rput(8.826875,-0.894375){Produkte}
\end{pspicture} 
}
 \end{center}

\end{figure}
\mindsetvid{moles and chemical reactions}{VPbxk}  
\label{m38712*secfhsst!!!underscore!!!id1903} 
      \begin{wex}{Stoigiometriese berekening I }
{
%problem
Watter volume suurstof by S.T.D. word benodig vir die volledige verbranding van $2 {\text{ dm}}^{3}$ propaan ($\text{C}{}_{3}\text{H}{}_{8}$)? (Wenk: $\text{CO}{}_{2}$ en $\text{H}{}_{2}\text{O}$ is die produkte in hierdie reaksie (en ook in alle verbrandingreaksies))
      }
{
%solution
\westep{Skryf die gebalanseerde vergelyking neer} 
${\text{C}}_{3}{\text{H}}_{8} \text{ (g)} + 5{\text{O}}_{2} \text{ (g)} \to 3\text{C}{\text{O}}_{2} \text{ (g)} + 4{\text{H}}_{2}\text{O} \text{ (g)}$
       
      \westep{Bepaal die verhouding} 
Omdat al die reaktante gasse is, kan ons die molverhouding gebruik om 'n vergelyking te tref. Uit die gebalanseerde vergelyking is die verhouding van suurstof tot propaan in die reaktante  $5:1$.
      \westep{Bepaal die antwoord}  
      \label{m38712*id284304}Een volume propaan benodig vyf volumes suurstof, dus $2 {\text{ dm}}^{3}$ propaan benodig $10 {\text{ dm}}^{3}$ suurstof vir die reaksie om volledig plaas te vind.
}
    \end{wex}
    \noindent
\label{m38712*secfhsst!!!underscore!!!id1972} 
      \begin{wex}{Stoigiometriese berekening 2 }
{
%problem
      \label{m38712*probfhsst!!!underscore!!!id1973}
      \label{m38712*id284347}Bereken die massa yster(II)sulfied wat gevorm word wanneer $5,6 \text{ g}$ yster volledig met swael reageer.
      }
{
%solution
\westep{Skryf die gebalanseerde vergelyking neer}
      \label{m38712*id284378}$\text{Fe} \text{ (s)} + \text{S} \text{ (s)} \to \text{FeS} \text{ (s)}$
       
      \westep{Bereken die aantal mol}  Ons vind die aantal mol van die gegewe stof:
      \label{m38712*id284430}\nopagebreak\noindent{}
        
    \begin{equation*}
    \text{n}=\frac{\text{m}}{\text{M}}=\frac{5,6 \text{ g}}{55,8 \text{ g} \cdot \text{ mol}^{-1}} = 0,1\text{ mol}
      \end{equation*}
      \westep{Vind die molverhouding} Ons vind die molverhouding tussen wat gegee is en wat gelewer word deur die reaksie. Vanuit die vergelyking kan ons aflei dat $1 \text{ mol}$ $\text{Fe}$  'n opbrengs van  $1 \text{ mol}$ $\text{FeS}$. Gevolglik sal $0,1\text{ mol}$ yster in die reaktante $0,1 \text{ mol}$ ystersulfied in die produk oplewer. 
      \westep{Bepaal die massa van ystersulfied}
      \label{m38712*id284499}\nopagebreak\noindent{}
    \begin{equation*}
    m=n \times M = 0,1 \text{ mol} \times 87,9 \text{ g} \cdot \text{ mol}^{-1} = 8,79 \text{ g}
      \end{equation*}
      \label{m38712*id284548}Die massa yster(II)sulfied wat tydens hierdie reaksie gevorm word, is $8,79 \text{ g}$. 
}
    \end{wex}
    \noindent
\subsection*{Teoretiese opbrengs}
Wanneer ons 'n bekende massa van 'n reaktans het en gevra word om uit te werk hoeveel van die produk gevorm word, is ons besig om die teoretiese opbrengs van die reaksie te bepaal. In die laboratorium kry chemici byna nooit hierdie presiese hoeveelheid van die produk nie. By elke stap van 'n reaksie gaan ‘ n klein hoeveelheid van die produk en reaktante 'verlore ‘ as gevolg van die feit dat  'n reaktans nie volledig reageer nie of omdat sekere ongewenste produkte gevorm word. Die werklike hoeveelheid van die produk wat gevorm word, staan bekend as die ware opbrengs. Jy kan die persentasie opbrengs bereken met die volgende formule:
\begin{equation*}
 \text{\% ~opbrengs} = \frac{\text{werklike ~opbrengs}}{\text{teoretiese ~opbrengs}} \times 100
\end{equation*}

 \label{m38712*secfhsst!!!underscore!!!id2067}
      \noindent 
      \begin{wex}{Reaksie om kunsmis te produseer (industrieel) }
{
%problem     
\label{m38712*probfhsst!!!underscore!!!id2068}
      \label{m38712*id284606}Swaelsuur ($\text{H}{}_{2}\text{SO}{}_{4}$) reageer met ammoniak ($\text{NH}{}_{3}$) om die kunsmis ammoniumsulfaat (($\text{NH}{}_{4}$)${}_{2}\text{SO}{}_{4}$) te produseer. Wat is die teoretiese opbrengs van ammoniumsulfaat wat verkry kan word vanuit $2,0 \text{ kg}$ swaelsuur? Daar is vasgestel dat $2,2 \text{ kg}$ kunsmis gevorm word. Berekén die $\%$ opbrengs. }
{
%solution
\westep{Skryf die gebalanseerde vergelyking neer}
      \label{m38712*id284813}\nopagebreak\noindent{}
\label{m38712*id284690}${\text{H}}_{2}{\text{SO}}_{4} \text{ (aq)} + 2{\text{NH}}_{3}\text{ (g)} \to {({\text{NH}}_{4})}_{2}{\text{SO}}_{4}  \text{ (aq)}$
\westep{Bereken die getal mol van die gegewe stof}
    \begin{equation*}
    \text{n} ({\text{H}}_{2}{\text{SO}}_{4}) = \frac{\text{m}}{\text{M}} = \frac{2~000 \text{ g}}{98,12 \text{ g} \cdot {\text{ mol}}^{-1}} = 20,38320424\text{ mol}
      \end{equation*}
      \westep{Vind die molverhouding}  
      \label{m38712*id285156}Uit die gebalanseerde vergelyking is die molverhouding van $\text{H}{}_{2}\text{SO}{}_{4}$ in die reaktante tot $(\text{NH}{}_{4}){}_{2}\text{SO}{}_{4}$ in die produk $1:1$. Daarom sal $20,383 \text{ mol}$ $\text{H}{}_{2}\text{SO}{}_{4}$ 'n opbrengs van $20,383 \text{ mol}$ van $(\text{NH}{}_{4}){}_{2}\text{SO}{}_{4}$ oplewer. 
\westep{Skryf nou die antwoord neer}
      \label{m38712*id285290}Die maksimum massa ammoniumsulfaat wat geproduseer kan word, word bereken as volg:
      \label{m38712*id285296}\nopagebreak\noindent{}
    \begin{equation*}
    \text{m}=\text{n} \times \text{M} = 20,383 \text{ mol} \times 114,04 \text{ g} \cdot {\text{ mol}}^{-1} = 2~324,477 \text{ g}
      \end{equation*}
      
      \label{m38712*id285362}Die maksimum hoeveelheid ammoniumsulfaat wat gevorm kan word is dus $2,324 \text{ kg}$.
 \westep{Bereken die \% opbrengs}
\begin{equation*}
\text{\% opbrengs} = \frac{\text{werklike opbrengs}}{\text{teoretiese opbrengs}} \times 100 = \frac{2,2}{2,324} \times 100 = 94,64 \%\end{equation*}
}
    \end{wex}
\label{m38717*secfhsst!!!underscore!!!id695}


      \begin{wex}{ Berekening van die massa van reaktante en produkte }
{
%problem
\begin{minipage}{\textwidth}
Bariumchloried en swaelsuur reageer volgens die volgende vergelyking om bariumsulfaat en soutsuur te vorm.\\
${\text{BaCl}}_{2}+{\text{H}}_{2}{\text{SO}}_{4}\to {\text{BaSO}}_{4}+2\text{HCl}$
\\
\label{m38717*id279141}As jy $2 \text{ g}$ $\text{BaCl}{}_{2}$ het:
\label{m38717*id279158}\begin{enumerate}[noitemsep, label=\textbf{\arabic*}. ] 
\item Watter hoeveelheid (in g) van $\text{H}{}_{2}\text{SO}{}_{4}$ benodig jy sodat al die bariumchloried opgebruik word in die reaksie?
\item Watter massa $\text{HCl}$ word tydens die reaksie geproduseer?
\end{enumerate}     
\end{minipage}
}
{
%solution
\westep{Bepaal die aantal mol swaelsuur}
        
    \begin{equation*}
    \text{n}=\frac{\text{m}}{\text{M}}=\frac{2 \text{ g}}{208,2 \text{ g} \cdot \text{ mol}^{-1}}=0,0096 \text{ mol}
      \end{equation*}
      \westep{Bepaal die aantal mol} 
      \label{m38717*id279344}Volgens die gebalanseerde vergelyking, sal $1$ mol $\text{BaCl}{}_{2}$ reageer met 1 mol $\text{H}{}_{2}\text{SO}{}_{4}$. Dus, sal $0,0096 \text{ mol}$ $\text{BaCl}{}_{2}$ reageer met dieselfde aantal $\text{H}{}_{2}\text{SO}{}_{4}$ omdat hulle molverhouding $1:1$ is.
      \westep{Bepaal die massa swawelsuur}  
$ \text{m}=\text{n} \times \text{M} = 0,0096 \text{ mol} \times 98,12 \text{ g} \cdot \text{ mol}^{-1} = 0,94 \text{ g}$ \\(antwoord van vraag 1) 
      \westep{Bepaal die aantal mol soutsuur} 
      \label{m38717*id279513}Volgens die gebalanseerde vergelyking word $2$ mol $\text{HCl}$ gevorm vir elke $1$ mol van die twee reaktante. Daarom is die aantal mol $\text{HCl}$ wat geproduseer is ($2 \times 0,0096 \text{ mol}$), en dis gelyk aan $0,0192 \text{ mol}$.
      \westep{Bepaal die massa van soutsuur}
   $\text{m}=\text{n} \times \text{M} = 0,0192 \text{ mol} \times 36,46 \text{ g} \cdot \text{ mol} = 0,7 \text{ g}$\\
(antwoord van vraag 2) 
}
\end{wex}

\begin{exercises}{Stoichiometry}
            \nopagebreak \noindent
      \label{m38712*id285393}\begin{enumerate}[noitemsep, label=\textbf{\arabic*}. ] 
%Q1
            \label{m38712*uid101}\item Diboraan, $\text{B}{}_{2}\text{H}{}_{6}$, is voorheen oorweeg vir gebruik as 'n vuurpyl-brandstof. Die verbrandingsreaksie vir diboraan is:\\
${\text{B}}_{2}{\text{H}}_{6} \text{ (g)} + 3{\text{O}}_{2} \text{ (g)} \to 2\text{H}\text{B}{\text{O}}_{2} \text{ (g)} + 2{\text{H}}_{2}\text{O} ~\left( \ell \right)$\\
As ons $2,37 \text{ g}$ diboraan laat reageer, hoeveel gram water sal uit die reaksie gevorm word?
%Q2
\item Natriumasied is 'n verbinding wat algemeen gebruik word in lugsakke. Wanneer dit geaktiveer word, vind die volgende reaksie plaas: \\
$2{\text{NaN}}_{3} \text{ (s)} \to 2\text{Na} \text{ (s)} + 3{\text{N}}_{2} \text{ (g)}$\\
Indien $23,4 \text{ g}$ natriumasied gebruik word, hoeveel mol stikstofgas kan geproduseer word? Watter volume sou hierdie stikstofgas beslaan by STD?
%Q3
\label{m38712*uid103}\item Fotosintese is 'n chemiese reaksie wat noodsaaklik is vir die bestaan ​​van lewe op
         Aarde. Tydens fotosintese word koolstofdioksiedgas, vloeibare water en ligenergie omgesit na glukose ($\text{C}{}_{6}\text{H}{}_{12}\text{O}{}_{6}$) en suurstofgas.
\label{m38712*id285674}\begin{enumerate}[noitemsep, label=\textbf{\alph*}. ] 
            \label{m38712*uid104}\item Skryf die vergelyking vir die fotosintese- reaksie neer.
\label{m38712*uid105}\item Balanseer die vergelyking.
\label{m38712*uid106}\item As $3 \text{ mol}$ koolstofdioksied in die fotosintese- reaksie gebruik word, watter massa glukose sal geproduseer word?
\end{enumerate}
                \end{enumerate}
\practiceinfo
\par 
 \par \begin{tabular}[h]{cccccc}
 (1.) 02v0  &  (2.) 02v1  &  (3.) 02v2  & \end{tabular}
\end{exercises}

%     \label{m38712*eip-269} 
%     \setcounter{subfigure}{0}
% 	\begin{figure}[H] % horizontal\label{m38712*slidesharefigure}
%     \label{m38712*slidesharemedia}\label{m38712*slideshareflash}
%             \raisebox{-5 pt}{ \includegraphics[width=0.5cm]{col11305.imgs/summary_www.png}} { (Presentation:  P10090 )}
%  \end{figure}       \par
\summary{VPeyf}
            \nopagebreak
      \label{m38712*id285735}\begin{itemize}[noitemsep]
\item Dit is belangrik om die veranderinge wat plaasvind tydens 'n chemiese reaksie te kan kwantifiseer.
\item Die \textbf{mol (n)} (afkorting mol) is 'n SI-eenheid wat gebruik word om 'n hoeveelheid van  'n stof te beskryf wat dieselfde aantal partikels bevat as wat daar atome in 12 g van koolstof aanwesig is.
\item Die aantal deeltjies in 'n mol word die \textbf{Avogadro konstante} genoem  en die waarde daarvan is $6,022 \times {10}^{23}$. Hierdie deeltjies kan atome, molekule of ander partikel-eenhede wees, afhangend van die spesifieke stof.
\item Die \textbf{mol\^{e}re massa (M)} is die massa van een mol van 'n stof en word gemeet in gram per mol of $\text{g} \cdot \text{ mol}{}^{-1}$. Die numeriese waarde van 'n element se molêre massa is dieselfde as sy relatiewe atoommassa. Vir 'n kovalente verbinding het die molêre massa dieselfde numeriese waarde as die molekulêre massa van die verbinding. Vir 'n ioniese stof het die molêre massa dieselfde numeriese waarde as die formule massa van die stof.
\item Die verhouding tussen die mol (n), massa in gram (m) en molêre massa (M) word gedefinieer deur die volgende vergelyking:
\label{m38712*id285862}\nopagebreak\noindent{}
    \begin{equation*}
    \text{n}=\frac{\text{m}}{\text{M}}
      \end{equation*}
\item In 'n gebalanseerde chemiese vergelyking beskryf die syfer aan die voorkant van die chemiese simbole die \textbf{molverhouding} van die reaktante en produkte.
\item Die \textbf{empiriese formule} van 'n verbinding is 'n uitdrukking van die relatiewe aantal van elke soort atoom in die verbinding.
\item Die \textbf{molekul\^{e}re formule} van 'n verbinding beskryf die werklike aantal atome van elke element in 'n molekule van die verbinding.
\item Die formule van 'n stof kan gebruik word om die \textbf{persentasie-massa} te bereken wat elke element bydra tot die verbinding.
\item Die \textbf{persentasie-samestelling} van 'n stof kan gebruik word om die chemiese formule af te lei.
\item Ons kan die produkte van 'n reaksie gebruik om die formule van een van die reaktante te bepaal.
\item Ons kan die aantal mol bepaal van kristalwater.
\item Een mol van 'n gas beslaan 'n volume van $22,4 {\text{ dm}}^{3}$ at S.T.D.
\item Die \textbf{konsentrasie} van 'n oplossing kan bereken word met behulp van die volgende vergelyking,
\label{m38712*id286019}\nopagebreak\noindent{}
    \begin{equation*}
    \text{C}=\frac{\text{n}}{\text{V}}
      \end{equation*}
waar C die konsentrasie (in $\text{ mol} \cdot {\text{dm}}^{-3}$), n die aantal mol van opgeloste stof in die oplossing en V die volume van die oplossing (in ${\text{dm}}^{-3}$). Die konsentrasie is 'n meting van die hoeveelheid opgeloste stof in 'n gegewe hoeveelheid (volume) vloeistof.
\item Die konsentrasie van 'n oplossing word gemeet in $\text{ mol} \cdot {\text{dm}}^{-3}$.
\item \textbf{Stoichiometrie} is die berekening van die hoeveelhede van die reaktante en produkte in  chemiese reaksies. Dit is ook die numeriese verhouding tussen reaktante en produkte.
\item Die teoretiese opbrengs van 'n reaksie is die maksimum hoeveelheid van die produk wat ons verwag om uit 'n reaksie te kry.\end{itemize}
\label{m38712*secfhsst!!!underscore!!!id2334}
            \begin{eocexercises}{Quantitative aspects of chemical change}
            \nopagebreak \noindent
\begin{enumerate}[noitemsep, label=\textbf{\arabic*}. ] 
%Q1
\item Skryf slegs die woord / term neer vir elk van die volgende beskrywings:
 \begin{enumerate}[noitemsep, label=\textbf{\alph*}. ] 
 \item die massa van een mol van 'n stof
 \item die aantal deeltjies in een mol van 'n stof
\end{enumerate}
%Q2 
 \item $5\phantom{\rule{2pt}{0ex}}\text{g}$ magnesiumchloried word gevorm as die produk van 'n chemiese reaksie.  Kies die korrekte weergawe uit die moontlikhede hieronder:
  \begin{enumerate}[noitemsep, label=\textbf{\alph*}. ] 
  \item $0,08$ mol magnesiumchloried word gevorm in die reaksie
  \item die aantal atome $\text{Cl}$ in die produk is $0,6022\ensuremath{\times}{10}^{23}$
  \item die aantal atome $\text{Mg}$ is $0,05$
  \item die verhouding van $\text{Mg}$ atome tenoor $\text{Cl}$ atome in die produk is $1:1$
  \end{enumerate}
%3
 \item 2 mol suurstofgas reageer met waterstof. Wat is die massa van die suurstof in die reaktante?
  \begin{enumerate}[noitemsep, label=\textbf{\alph*}. ]
  \item $32 \text{ g}$
  \item $0,125 \text{ g}$
  \item $64 \text{ g}$
  \item $0,063 \text{ g}$
  \end{enumerate}
%Q4
 \item In die verbinding kaliumsulfaat ($\text{K}{}_{2}\text{SO}{}_{4}$), maak suurstof  $x\%$ van die massa van die verbinding uit. x =?
  \begin{enumerate}[noitemsep, label=\textbf{\alph*}. ]
  \item $36,8$
  \item $9,2$
  \item $4$
  \item $18,3$
  \end{enumerate}
%Q5
 \item Die konsentrasie van 'n $150 {\text{ cm}}^{3}$ oplossing wat $5 \text{ g}$ $\text{NaCl}$ bevat is:
  \begin{enumerate}[noitemsep, label=\textbf{\alph*}. ]
  \item $0,09 \text{ mol} \cdot \text{dm}^{-3}$
  \item $5,7 \times 10^{-4} \text{ mol} \cdot \text{dm}^{-3}$
  \item $0,57 \text{ mol} \cdot \text{dm}^{-3}$
  \item $0,03 \text{ mol} \cdot \text{dm}^{-3}$
  \end{enumerate}

%Q6
\item Bereken die aantal mol in:
 \begin{enumerate}[noitemsep, label=\textbf{\alph*}. ] 
 \item $5 \text{ g}$ metaan (${\text{CH}}_{4}$)
 \item $3,4 \text{ g}$ soutsuur
\item $6,2 \text{ g}$ kaliumpermanganaat (${\text{KMnO}}_{4}$)
 \item $4 \text{ g}$ neon
 \item $9,6 \text{ kg}$ titaniumtetrachloried (${\text{TiCl}}_{4}$)
 \end{enumerate}
%Q7 
\item Bereken die massa van:
 \begin{enumerate}[noitemsep, label=\textbf{\alph*}. ] 
 \item $0,2 \text{ mol}$ kaliumhidroksied ($\text{KOH}$)
 \item $0,47 \text{ mol}$ stikstofdioksied
 \item $5,2 \text{ mol}$ helium
 \item $0,05 \text{ mol}$ koper (II) chloried (${\text{CuCl}}_{2}$)
 \item $31,31 \times {10}^{23}$ molekules koolstofmonoksied ($\text{CO}$)\end{enumerate}
%Q8
\item Bereken die persentasie wat elke element bydra tot die totale massa van:
 \begin{enumerate}[noitemsep, label=\textbf{\alph*}. ] 
 \item Chlorobenseen (${\text{C}}_{6}{\text{H}}_{5}\text{Cl}$)
 \item Litiumhidroksied ($\text{LiOH}$)
 \end{enumerate}
%Q9
\item CFC's (chloorfluoro-koolstowwe) is een van die gasse wat bydra tot die vernietiging van die osoonlaag. Natalia het  'n CFC ontleed en gevind dat dit $58,64\%$ chloor, $31,43\%$ fluoor en $9,93\%$ koolstof bevat. Wat is die empiriese formule van hierdie verbinding?
%Q10
\item $14 \text{ g}$ stikstof verbind met suurstof om $46 \text{ g}$ stikstofoksied te vorm. Gebruik hierdie inligting om die formule van die oksied uit te werk.
%Q11
\item Jodium kom voor as een van drie oksiede (${\text{I}}_{2}{\text{O}}_{4}$ ; ${\text{I}}_{2}{\text{O}}_{5}$ ; ${\text{I}}_{4}{\text{O}}_{9}$). Neels het een van hierdie oksiede geproduseer en wil vasstel watter een dit is. As hy aan die begin $508 \text{ g}$ jodium gehad het en uiteindelik $652 \text{ g}$ van die oksied geproduseer het, watter oksied is hier ter sprake?
%Q12
\item  'n Gefluoreerde koolwaterstof ('n koolwaterstof is 'n chemiese verbinding wat waterstof en koolstof bevat) is ontleed en daar is gevind dat dit $8,57\%$ H, $51,05\%$ C en $40,38\%$ F.
 \begin{enumerate}[noitemsep, label=\textbf{\alph*}. ] 
 \item Wat is die empiriese formule?
 \item Wat is die molekulêre formule indien die mol\^{e}re massa $94,1\text{ g}\cdot {\text{ mol}}^{-1}$ is?
 \end{enumerate}
%Q13
\item Kopersulfaatkristalle sluit dikwels water in. Mphilonhle probeer om vas te stel hoe\-veel mol water in die kopersulfaatkristalle aanwesig is. Sy weeg $3 \text{ g}$ kopersulfaat af en verhit dit. Na verhitting vind sy dat die massa $1,9 \text{ g}$. Bereken die aantal mol water in die kristalle? (Kopersulfaat word verteenwoordig deur ${\text{CuSO}}_{4}\cdot x{\text{H}}_{2}\text{O}$).        
%Q14
\item $300\phantom{\rule{2pt}{0ex}}{\text{cm}}^{3}$ van 'n $0,1\phantom{\rule{2pt}{0ex}}\text{ mol}\ensuremath{\cdot}{\text{dm}}^{-3}$ swawelsuur-oplossing is bygevoeg by $200\phantom{\rule{2pt}{0ex}}{\text{cm}}^{3}$ van 'n $0,5\phantom{\rule{2pt}{0ex}}\text{ mol}\ensuremath{\cdot}{\text{dm}}^{-3}$ oplossing van natriumhidroksied.
 \begin{enumerate}[noitemsep, label=\textbf{\alph*}. ] 
 \item Skryf 'n gebalanseerde vergelyking neer vir die reaksie wat plaasvind as hierdie twee oplossings gemeng word.
 \item Bereken die aantal mol swaelsuur wat by die natriumhidroksiedoplossing gevoeg is.
 \item Is die aantal mol swaelsuur genoeg om die natriumhidroksiedoplossing ten volle te neutraliseer? Motiveer jou antwoord deur te verwys na alle relevante berekeninge.
% (IEB Paper 2 2004)
 \end{enumerate}
%Q15
\item 'n Leerder is gevra om $200 {\text{ cm}}^{3}$ van  'n natriumhidroksied ($\text{NaOH}$) oplossing met  'n konsentrasie van $0,5 \text{ mol} \cdot {\text{dm}}^{-3}$ te maak.
 \begin{enumerate}[noitemsep, label=\textbf{\alph*}. ] 
 \item Bepaal die massa natriumhidroksiedkorrels wat hy nodig het om dit te doen.
 \item Met behulp van 'n akkurate skaal kry die leerder die korrekte massa afgemeet. Hy voeg presies $\text{NaOH}$ suiwer water by die korrels. Sal die konsentrasie van die oplossing korrek wees? Verduidelik jou antwoord. 
 \item Die leerder kry dan $300\phantom{\rule{2pt}{0ex}}{\text{cm}}^{3}$ van 'n $0,1 \text{ mol} \cdot {\text{dm}}^{-3}$ oplossing van swawelsuur ($\text{H}{}_{2}\text{SO}{}_{4}$) en voeg dit by $200 {\text{ cm}}^{3}$ van 'n $0,5 \text{ mol} \cdot {\text{dm}}^{-3}$ oplossing van $\text{NaOH}$ at $25{}^{0}\text{C}$.
 \item Skryf 'n gebalanseerde vergelyking neer vir die reaksie wat plaasvind as hierdie twee oplossings gemeng word.
 \item Bereken die aantal mol $\text{H}{}_{2}\text{SO}{}_{4}$ wat by die NaOH oplossing gevoeg is.
% \label{m38712*uid160}\item Is the number of moles of $\text{H}{}_{2}\text{SO}{}_{4}$ calculated in the previous question enough to fully neutralise the $\text{NaOH}$ solution? Support your answer by showing all the relevant calculations.
% (IEB Paper 2, 2004)
\end{enumerate}
%Q16
\item $96,2 \text{ g}$ swael reageer met 'n onbekende hoeveelheid sink volgens die volgende vergelyking: 
$\text{Zn}+\text{S}\to \text{ZnS}$
 \begin{enumerate}[noitemsep, label=\textbf{\alph*}. ] 
 \item Watter massa sink sal jy moet gebruik as jy al die swael wil gebruik in die reaksie?
 \item Bereken die teoretiese opbrengs vir hierdie reaksie.
 \item Dit is bevind dat $275~\text{g}$ sinksulfied geproduseer is. Berekén die \% opbrengs.
 \end{enumerate}
%Q17
\item Kalsiumchloried reageer met koolsuur om kalsiumkarbonaat en soutsuur te vorm volgens die volgende vergelyking:\\
${\text{CaCl}}_{2}+{\text{H}}_{2}{\text{CO}}_{3}\to {\text{CaCO}}_{3}+2\text{HCl}$\\
As jy $10 \text{ g}$ kalsiumkarbonaat wil produseer deur middel van hierdie chemiese reaksie, watter hoeveelheid (in g) kalsiumchloried moet jy aan die begin van die reaksie gebruik?
                \end{enumerate}
\practiceinfo
\par 
 \par \begin{tabular}[h]{cccccc}
 (1.) 02v3  &  (2.) 02v4  &  (3.) 02v5  &  (4.) 02v6  &  (5.) 02v7  &  (6.) 02v8  &  (7.) 02v9  &  (8.) 02va  &  (9.) 02vb  &  (10.) 02vc  &  (11.) 02vd  &  (12.) 02ve  &  (13.) 02vf  &  (14.) 02vg  &  (15.) 02vh  & (16.) 02vi & (17.) 02vj \end{tabular}
\end{eocexercises}
