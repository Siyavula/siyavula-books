         \chapter{Die periodieke tabel}\fancyfoot[LO,RE]{Chemie: Matter and materials}
    %\setcounter{figure}{1}
    %\setcounter{subfigure}{1}
 \label{m38760*cid9}
            \section{Die rangskikking van elemente in die periodieke tabel}
            \nopagebreak

      \label{m38760*id261491}Die \textbf{periodieke tabel van die elemente} is 'n metode waarvolgens die chemiese elemente in 'n tabel gerangskik is,
in volgorde van toenemende atoomgetal. Die meeste van die werk wat gedoen is om by die periodieke tabel te kon uitkom, kan toegeskryf word aan 'n man met die naam \textbf{Dmitri Mendeleev} in 1869. Mendeleev was 'n Russiese chemikus wat die tabel op so 'n wyse ontwerp het dat herhalende ("periodieke") tendense of patrone in die eienskappe van die elemente getoon kon word. Deur gebruik te maak van die tendense wat hy waargeneem het, het hy selfs gapings gelaat vir daardie elemente wat volgens hom ‘ontbreek’ het. Hy het ook die eienskappe, wat die ontbrekende elemente sou h\^{e} wanneer hulle ontdek word, voorspel. Baie van hierdie elemente is inderdaad ontdek; dit bewys dat Mendeleev se voorspellings korrek was.\par 
      \label{m38760*id261511}Om die herhalende eienskappe wat hy waargeneem het te wys, het Mendeleev nuwe rye in sy tabel begin, 
sodat elemente met soortgelyke eienskappe in dieselfde vertikale kolomme, genoem \textbf{groepe} kon wees. Na elke ry word verwys as  'n \textbf{periode}. Figuur~\ref{fig:atom:periodic} toon 'n vereenvoudigde weergawe van die periodieke tabel. Die
volle periodieke tabel is opgeneem aan die voorkant van hierdie boek. Jy kan 'n aanlyn periodieke tabel sien by \textsl{http://periodictable.com/}. \par 
    \setcounter{subfigure}{0}
	\begin{figure}[H] % horizontal\label{m38760*uid133}
 \begin{center}
\begin{pspicture}(-7.8,-1)(6.8,4)
%\psgrid[gridcolor=gray]
\psset{unit=0.75}
\pspolygon[fillstyle=solid,fillcolor=green!50!blue](-9,3)(-9,4)(-8,4)(-8,3)(-9,3)
\pspolygon[fillstyle=solid,fillcolor=lightgray](-9,0)(-9,3)(-8,3)(-7,3)(-7,1)(3,1)(3,2)(4,2)(4,0)(-9,0)
\pspolygon[fillstyle=solid,fillcolor=cyan](3,2)(3,3)(4,3)(4,2)(3,2)
\pspolygon[fillstyle=solid,fillcolor=cyan](4,0)(4,2)(5,2)(5,1)(6,1)(6,0)(4,0)
\pspolygon[fillstyle=solid,fillcolor=green!50!blue](5,2)(4,2)(4,3)(8,3)(8,4)(9,4)(9,0)(6,0)(6,1)(5,1)(5,2)
\psline(-8,3)(-8,0)
\psline(-7,1)(-7,0)
\psline(-6,1)(-6,0)
\psline(-5,1)(-5,0)
\psline(-4,1)(-4,0)
\psline(-3,1)(-3,0)
\psline(-2,1)(-2,0)
\psline(-1,1)(-1,0)
\psline(0,1)(0,0)
\psline(1,1)(1,0)
\psline(2,1)(2,0)
\psline(3,1)(3,0)
\psline(5,3)(5,2)
\psline(5,1)(5,0)
\psline(6,3)(6,1)
\psline(7,3)(7,0)
\psline(8,3)(8,0)
\psline(-9,3)(-8,3)
\psline(-9,2)(-7,2)
\psline(-9,1)(9,1)
\psline(5,2)(9,2)
\psline(8,3)(9,3)
\rput(-8.5,3.5){\textbf{H}}
\rput(-8.5,2.5){\textbf{Li}}
\rput(-8.5,1.5){\textbf{Na}}
\rput(-8.5,0.5){\textbf{K}}
\rput(-7.5,2.5){\textbf{Be}}
\rput(-7.5,1.5){\textbf{Mg}}
\rput(-7.5,0.5){\textbf{Ca}}
\rput(-6.5,0.5){\textbf{Sc}}
\rput(-5.5,0.5){\textbf{Ti}}
\rput(-4.5,0.5){\textbf{V}}
\rput(-3.5,0.5){\textbf{Cr}}
\rput(-2.5,0.5){\textbf{Mn}}
\rput(-1.5,0.5){\textbf{Fe}}
\rput(-0.5,0.5){\textbf{Co}}
\rput(0.5,0.5){\textbf{Ni}}
\rput(1.5,0.5){\textbf{Cu}}
\rput(2.5,0.5){\textbf{Zn}}
\rput(3.5,2.5){\textbf{B}}
\rput(3.5,1.5){\textbf{Al}}
\rput(3.5,0.5){\textbf{Ga}}
\rput(4.5,2.5){\textbf{C}}
\rput(4.5,1.5){\textbf{Si}}
\rput(4.5,0.5){\textbf{Ge}}
\rput(5.5,2.5){\textbf{N}}
\rput(5.5,1.5){\textbf{P}}
\rput(5.5,0.5){\textbf{As}}
\rput(6.5,2.5){\textbf{O}}
\rput(6.5,1.5){\textbf{S}}
\rput(6.5,0.5){\textbf{Se}}
\rput(7.5,2.5){\textbf{F}}
\rput(7.5,1.5){\textbf{Cl}}
\rput(7.5,0.5){\textbf{Br}}
\rput(8.5,3.5){\textbf{He}}
\rput(8.5,2.5){\textbf{Ne}}
\rput(8.5,1.5){\textbf{Ar}}
\rput(8.5,0.5){\textbf{Kr}}
\psline[linewidth=0.1,arrows=<->](-9.5,4)(-9.5,0)
\pcline[linestyle=none](-9.5,0)(-9.5,4)
\aput{:U}{Periode}
\rput(-8.5,5){groep getal}
\rput(-8.5,4.6){1}
\rput(-7.5,3.5){2}
\rput(3.5,3.5){13}
\rput(4.5,3.5){14}
\rput(5.5,3.5){15}
\rput(6.5,3.5){16}
\rput(7.5,3.5){17}
\rput(8.5,4.5){18}
\psline[linewidth=0.1,arrows=->](-7,-0.5)(5,-0.5)
\pcline[linestyle=none](-7,-0.5)(5,-0.5)
\bput{:U}{Groep}
\end{pspicture}
\end{center}
\caption{ 'n Vereenvoudigde diagram van  'n gedeelte van die periodieke pabel. Metale is in gray, metallo\"ide in ligblou en nie-metale in donker blou.}
\label{fig:atom:periodic}
 \end{figure}       
            
\subsection*{Definisies en belangrike begrippe}
Voordat ons kan praat oor die tendense in die periodieke tabel, moet ons eers 'n paar terme definieer wat gebruik word:
\begin{itemize}[noitemsep]
\item \textbf{Atoomradius} \\
Die atoomradius is 'n maatstaf (meting) van die grootte van 'n atoom. 
\item \textbf{Ionisasie-energie}\\
Die eerste ionisasie-energie is die energie wat nodig is om 'n elektron uit 'n atoom in die gasfase te verwyder. Die ionisasie-energie is verskillend vir verslillende elemente. Ons definieer ook tweede, derde, vierde, ens. ionisasie-energie\"{e}. Dit is die energie wat nodig is om die tweede, derde, of vierde elektron onderskeidelik te verwyder. 
\item \textbf{Elektronaffiniteit}\\
Elektronaffiniteit kan gesien word as die mate waartoe 'n element elektrone wil bekom.
\item \textbf{Elektronegatiwiteit}\\
Elektronegatiwiteit dui die neiging van atome aan om elektrone aan te trek. Die elektronegatiwiteit van die elemente begin by ongeveer 0,7 (frankium ($\text{Fr}$)) en gaan tot 4 (Fluoor ($\text{F}$)).
\item 'n \textbf{Groep} is 'n vertikale kolom in die periodieke tabel en word as die belangrikste manier van klassifikasie van die elemente beskou. As jy na 'n periodieke tabel kyk, sal jy sien die groepe is aan die bokant van elke kolom genommer. Die groepe is van links na regs genommer, beginnende met 1 en eindigende met 18. Dit is die konvensie (gebruik) wat ons in hierdie boek sal gebruik. In sommige periodieke tabelle sal jy sien dat die groepe as volg van links na regs genommer is: 1, 2, dan is 'n oop ruimte met die \textbf{oorgangelemente} gevolg deur groepe 3 tot 8. Nog 'n manier om die groepe te nommer  is om Romeinse syfers te gebruik.
\item 'n \textbf{Periode} is 'n horisontale ry in die periodieke tabel van die elemente. Die periodes is genommer van bo na onder; begin met 1 en eindig met 7.
\end{itemize}
Elke element op die periodieke tabel het  'n periode- en  'n groepgetal. Byvoorbeeld, $\text{B}$ (boor) is in periode 2 en groep 13. Ons kan ook die elektronstruktuur van 'n element, vanaf sy posisie op die periodieke tabel, bepaal. In hoofstuk~\ref{chap:atom} het jy met die elektronkonfigurasie (elektronverspreiding) van verskillende elemente gewerk. Deur van die periodieke tabel gebruik te maak kan ons maklik die elektronkonfigurasies van enige element gee. Om te sien hoe dit werk kyk na die volgende:\\
\begin{figure}[H]
 \begin{center}
\scalebox{0.6}{
  \begin{pspicture}(0,0)(10,10)
{
   \pspolygon[fillstyle=solid,fillcolor=cyan](0.0)(0,7)(1,7)(1,6)(2,6)(2,0)(0,0)
   \pspolygon[fillstyle=solid,fillcolor=white](2,0)(2,4)(12,4)(12,0)(2,0)
   \pspolygon[fillstyle=solid,fillcolor=teal](12,0)(12,6)(17,6)(17,7)(18,7)(18,0)(12,0)
   \rput(1,3.5){\Large{\textbf{s-blok}}}
   \rput(5.5,2){\Large{\textbf{d-blok}}}
   \rput(13.5,3.5){\Large{\textbf{p-blok}}}
\rput(0,8){\Large{groep getal}}
\rput(0.5,7.5){\Large{1}}
\rput(1.5,6.5){\Large{2}}
\rput(12.5,6.5){\Large{13}}
\rput(13.5,6.5){\Large{14}}
\rput(14.5,6.5){\Large{15}}
\rput(15.5,6.5){\Large{16}}
\rput(16.5,6.5){\Large{17}}
\rput(17.5,7.5){\Large{18}}
}
  \end{pspicture}
}
 \end{center}
\end{figure}
Ons kom ook agter dat die periodegetal die aantal energievlakke wat opgevul word, aandui. Byvoorbeeld, fosfor ($\text{P}$) is in die derde periode en 'n groep 15. As ons kyk na die bostaande figuur, sien ons dat die p-orbitaal opgevul word. So ook word die derde energievlak opgevul. Dus is die elektronkonfigurasie: $\text{[Ne]}3s^{2}3p^{3}$. (Fosfor is in die derde groep in die p-blok, dus moet dit 3 elektrone in die p-orbitaal h\^{e}.)
      \label{m38760*uid146}
            \subsection*{Periode in die periodieke tabel}
            \nopagebreak

            \label{m38760*id261855} Die volgende diagram illustreer sommige van die belangrikste tendense in die periodes: \\
\begin{figure}[H]

\begin{center}
\scalebox{0.7}{
\begin{pspicture}(0,0)(10,10)
%\psgrid[gridcolor=gray]
\rput(5,5){
\psset{unit=0.75}
\pspolygon(0,0)(0,7)(1,7)(1,6)(2,6)(2,4)(12,4)(12,6)(17,6)(17,7)(18,7)(18,0)(0,0)
\rput(1,8){\Large{periode-getal}}
\rput(-1,6.5){\Large{1}}
\rput(-1,5.5){\Large{2}}
\rput(-1,4.5){\Large{3}}
\rput(-1,3.5){\Large{4}}
\rput(-1,2.5){\Large{5}}
\rput(-1,1.5){\Large{6}}
\rput(-1,0.5){\Large{7}}
\psline[linewidth=0.08,linecolor=red,arrowinset=0]{<-}(0.2,0.5)(17.8,0.5)
\rput(8.8,1.1){\Large{Atoomradius}}
\psline[linewidth=0.08,linecolor=blue,arrowinset=0]{->}(0.2,1.8)(17.8,1.8)
\rput(8.8,2.3){\Large{Ionisasie-energie}}
\psline[linewidth=0.08,linecolor=blue,arrowinset=0]{->}(0.2,3.1)(17.8,3.1)
\rput(8.8,3.5){\Large{Elektronegatiwiteit}}
}
\end{pspicture}
}
\end{center}

\caption{Tendense in die periodieke tabel.}
\label{fig:atom:periodic2}
 \end{figure} 

\mindsetvid{Ionization energy}{VParr}

 Tabel~\ref{tab:period3trends} gee 'n opsomming van die patrone of tendense van die eienskappe van die elemente in periode 3. Soortgelyke tendense word waargeneem in die ander periodes van die periodieke tabel. \\
\begin{table}[H]
 \begin{center}
  \begin{tabular}{|p{2cm}|p{1cm}|p{1cm}|p{1.4cm}|p{1.4cm}|p{1.4cm}|p{1.4cm}|p{1.4cm}|} \hline
\textbf{Element} & $^{23}_{11}\text{Na}$ & $^{24}_{12}\text{Mg}$ &  $^{27}_{13}\text{Al}$ & $^{28}_{14}\text{Si}$ &  $^{31}_{15}\text{P}$ & $^{32}_{16}\text{S}$ & $^{35}_{17}\text{Cl}$ \\ \hline
   \textbf{Verbindings met chloor} & $\text{NaCl}$ & $\text{MgCl}_2$ & $\text{AlCl}_{3}$ & $\text{SiCl}_{4}$ & $\text{PCl}_{5}$ or $\text{PCl}_{3}$ & $\text{S}_{2}\text{Cl}_{2}$ & geen verbindings  \\ \hline
\textbf{Verbindings met suurstof} & $\text{Na}_{2}\text{O}$ & $\text{MgO}$ & $\text{Al}_{2}\text{O}_{3}$ & $\text{SiO}_{2}$ & $\text{P}_{4}\text{O}_{6}$ or $\text{P}_{4}\text{O}_{10}$ & $\text{SO}_{3}$ or $\text{SO}_{4}$ & $\text{Cl}_{2}\text{O}_{7}$ or $\text{Cl}_{2}\text{O}$  \\ \hline
\textbf{Elektron-konfigurasie} & $\textsf{[Ne]}3s^{1}$ & $\textsf{[Ne]}3s^{2}$ & $\textsf{[Ne]}3s^{2}3p^{1}$ & $\textsf{[Ne]}3s^{2}3p^{2}$ & $\textsf{[Ne]}3s^{2}3p^{3}$ & $\textsf{[Ne]}3s^{2}3p^{4}$ & $\textsf{[Ne]}3s^{2}3p^{5}$ \\ \hline
\textbf{Atoomradius} & \multicolumn{7}{p{8cm}|}{Neem af oor die periode} \\ \hline
\textbf{Eerste ionisasie-energie} & \multicolumn{7}{p{8cm}|}{Die algemene tendens ia  'n afname van links na regs in  'n periode} \\ \hline
\textbf{Elektro-negatiwiteit} & \multicolumn{7}{p{8cm}|}{Neem toe van links na regs in  'n periode} \\ \hline
\textbf{Smelt- en kookpunte} & \multicolumn{7}{p{8cm}|}{Neem toe tot by silikon en dan af tot by argon} \\ \hline
\textbf{Elektriese geleidingsvermo\"{e}} & \multicolumn{7}{p{10cm}|}{Neem toe van natrium tot aluminium. Silikon is  'n semi-geleier (halfgeleier). Die res is isolators. } \\ \hline
  \end{tabular}
\caption{Opsomming van die tendense in periode 3}
\label{tab:period3trends}
 \end{center}

\end{table}
Let daarop dat argon ($^{40}_{18}\text{Ar}$) uitgelaat is. Argon is 'n edelgas met elektronkonfigurasie: $\text{[Ne]}3s^{2}3p^{6}$. Argon vorm nie verbindings met suurstof of chloor nie. 
\begin{exercises}{Periods on the periodic table}
{
\begin{enumerate}[noitemsep, label=\textbf{\arabic*}. ]
\item Gebruik Tabel~\ref{tab:period3trends} en Figuur~\ref{fig:atom:periodic2} om jou te help om 'n soortgelyke tabel saam te stel vir die elemente in periode 2.
\item Verwys na die onderstaande tabel wat inligting verskaf oor die ionisasie-energie (in $\text{kJ} \cdot \text{mol}^{-1}$) en atoomgetal (Z) vir 'n aantal elemente in die periodieke tabel:\\
\begin{table}[H]
\begin{center}
\begin{tabular}{|l|l|c|l|l|c|}\hline
\textbf{Z} & Naam van element & Ionisasie-energie & \textbf{Z} & Naam van element & Ionisasie-energie \\\hline
1 &   & 1310 & 10 &        & 2072 \\\hline
2 &     & 2360 & 11 &      & 494  \\\hline
3 &    & 517  & 12 &   & 734  \\\hline
4 &  & 895  & 13 &   & 575  \\\hline
5 &      & 797  & 14 &     & 783  \\\hline
6 &     & 1087 & 15 &  & 1051 \\\hline
7 &   & 1397 & 16 &     & 994  \\\hline
8 &     & 1307 & 17 &    & 1250 \\\hline
9 &   & 1673 & 18 &       & 1540 \\\hline
\end{tabular}
\end{center}
\end{table}
\begin{enumerate}[noitemsep, label=\textbf{\alph*}. ]
 \item Vul die name van die elemente in.
\item Teken 'n lyngrafiek om die verhouding (verwantskap) tussen die atoomgetal (op die x-as) en ionisasie-energie (y-as) te toon..
\item Beskryf enige neigings wat jy waarneem.
\item Verduidelik waarom \ldots
	\begin{enumerate}[noitemsep, label=\textbf{\roman*}. ]
	\item die ionisasie-energie vir $Z=2$ ho\"{e}r is as vir $Z=1$
	\item die ionisasie-energie vir $Z=3$ laer is as vir $Z=2$
	\item die ionisasie-energie toeneem tussen $Z=5$ en $Z=7$
	\end{enumerate}

\end{enumerate}
\end{enumerate}

% Automatically inserted shortcodes - number to insert 2
\par \practiceinfo
\par \begin{tabular}[h]{cccccc}
% Question 1
(1.)	02b1	&
% Question 2
(2.)	02b2	&
\end{tabular}
% Automatically inserted shortcodes - number inserted 2
}
\end{exercises}


\section{Chemiese eienskappe van die groepe}
 \label{m38760*secfhsst!!!underscore!!!id1062}
            \nopagebreak
            \label{m38760*id261554} In sommige groepe, vertoon die elemente baie soortgelyke chemiese eienskappe. Aan 'n paar van die groepe is selfs spesiale name gegee om hulle te identifiseer. Die eienskappe van elke groep word meestal bepaal deur die elektronkonfigurasie van die atome van die elemente in die groep. Die name van die groepe is saamgevat in figuur~\ref{fig:atom:periodic}\par
\begin{figure}[H]

\begin{center}
\scalebox{0.7}{
\begin{pspicture}(-7.8,-1)(6.8,4)
%\psgrid[gridcolor=gray]
\psset{unit=0.75}
%alkali metals
\psline(-9,0)(-9,7)(-8,7)(-8,0)(-9,0)
\uput[90]{90}(-8.5,1){\Large{Alkalimetale}}
%alkali earth metals
\psline(-8,0)(-8,6)(-7,6)(-7,0)(-8,0)
\uput[90]{90}(-7.5,0.1){\Large{Alkali-aardmetale}}
%transition metals
\psline(-7,0)(-7,4)(3,4)(3,0)(-7,0)
\uput[0](-5,2){\Large{Oorgangsmetale}}
%group 13
\psline(3,0)(3,6)(4,6)(4,0)(3,0)
\uput[90]{90}(3.5,1){\Large{Groep 13}}
%group 14
\psline(4,0)(4,6)(5,6)(5,0)(4,0)
\uput[90]{90}(4.5,1){\Large{Groep 14}}
%pnictogens (group 15)
\psline(5,0)(5,6)(6,6)(6,0)(5,0)
\uput[90]{90}(5.5,1){\Large{Groep 15}}
%chalcogens (group 16)
\psline(6,0)(6,6)(7,6)(7,0)(6,0)
\uput[90]{90}(6.5,1){\Large{Groep 16}}
%halogens
\psline(7,0)(7,6)(8,6)(8,0)(7,0)
\uput[90]{90}(7.5,1){\Large{Halogene}}
%noble gases
\psline(8,0)(8,7)(9,7)(9,0)(8,0)
\uput[90]{90}(8.5,1){\Large{Edelgasse}}
\end{pspicture}
}
\end{center}

\caption{Groepe in die periodieke tabel}
\label{fig:atom:periodic}
\end{figure}  
 'n Paar punte om van kennis te neem oor die groepe is:
        \label{m38757*id261581}\begin{itemize}[noitemsep]
            \label{m38757*uid135}\item Hoewel waterstof in groep 1 verskyn is, is dit nie 'n alkalimetaal nie.
\item Groep 15 elemente word soms die ‘pniktogene’ genoem.
\label{m38757*id6232}\item Groep 16 elemente is somstyds bekend as die ‘chalkogene’.
\label{m38757*uid142}\item Die \textbf{halogene} en die \textbf{alkali-aardmetale} is baie reaktiewe groepe.
\label{m38757*uid143}\item Die \textbf{edelgasse} is \textsl{inert} (onaktief).   
\end{itemize}            

\mindsetvid{Groups in the periodic table}{VPasj}

        \label{m38760*id261833}Die volgende diagram illustreer sommige van die belangrikste tendense in die groepe van die periodieke tabel: \\
\begin{figure}[H]

\begin{center}
\scalebox{0.7}{
\begin{pspicture}(0,0)(20,20)
%\psgrid[gridcolor=gray]
\rput(5,5){
\psset{unit=0.75}
\pspolygon(0,0)(0,7)(1,7)(1,6)(2,6)(2,4)(12,4)(12,6)(17,6)(17,7)(18,7)(18,0)(0,0)
\rput(0,8){\Large{groep getal}}
\rput(0.5,7.5){\Large{1}}
\rput(1.5,6.5){\Large{2}}
\rput(12.5,6.5){\Large{13}}
\rput(13.5,6.5){\Large{14}}
\rput(14.5,6.5){\Large{15}}
\rput(15.5,6.5){\Large{16}}
\rput(16.5,6.5){\Large{17}}
\rput(17.5,7.5){\Large{18}}
\psline[linewidth=0.1,arrowinset=0,linecolor=red]{->}(0.8,6.8)(0.8,0.2)
\uput[0](0.8,3){\Large{Atoom}}
\uput[0](0.8,2.3){\Large{radius}}
\psline[linewidth=0.1,arrowinset=0,linecolor=blue]{<-}(4.2,6.8)(4.2,0.2)
\uput[0](4.2,3){\Large{Ionisasie}}
\uput[0](4.2,2.3){\Large{energie}}
\psline[linewidth=0.1,arrowinset=0,linecolor=blue]{<-}(7.5,6.8)(7.5,0.2)
\uput[0](7.5,3){\Large{Elektro-}}
\uput[0](7.5,2.3){\Large{negatiwiteit}}
\psline[linewidth=0.1,arrowinset=0,linecolor=blue]{<-}(10.9,6.8)(10.9,0.2)
\uput[0](10.9,3){\Large{smelt- en}}
\uput[0](10.9,2.3){\Large{kook punt}}
\psline[linewidth=0.1,arrowinset=0,linecolor=red]{->}(15,6.8)(15,0.2)
\uput[0](15,3){\Large{Digtheid}}
}
\end{pspicture}
}
\end{center}
\caption{Tendense in die groep in die periodieke tabel.}
\label{fig:atom:periodic1}
 \end{figure} 
Tabel~\ref{tab:group1trends} gee 'n opsomming van die patrone of tendense in die eienskappe van die elemente in Groep 1. Soortgelyke tendense word waargeneem vir die elemente in die ander groepe van die periodieke tabel. \\
\begin{table}[H]
 \begin{center}
  \begin{tabular}{|l|p{1cm}|p{1cm}|p{1cm}|p{1cm}|p{1cm}|} \hline
\textbf{Element} & $^{7}_{3}\text{Li}$ & $^{7}_{3}\text{Na}$ &  $^{7}_{3}\text{K}$ & $^{7}_{3}\text{Rb}$ &  $^{7}_{3}\text{Cs}$ \\ \hline
\textbf{Elektronstruktuur} & $\textsf{[He]}2s^{1}$ & $\textsf{[Ne]}3s^{1}$ & $\textsf{[Ar]}4s^{1}$ & $\textsf{[Kr]}4s^{1}$ & $\textsf{[Xe]}5s^{1}$ \\ \hline
   \multirow{2}{*}{\textbf{Groep 1 chloriedes}} & $\text{LiCl}$ & $\text{NaCl}$ & $\text{KCl}$ & $\text{RbCl}$ & $\text{CsCl}$ \\ \cline{2-6}
   & \multicolumn{5}{|p{6cm}|}{Groep 1 elemente vorm almal halogeenverbindings in  'n 1:1 verhouding} \\ \hline
\multirow{2}{*}{\textbf{Groep 1 oksiedes}} & $\text{Li}_{2}\text{O}$ & $\text{Na}_{2}\text{O}$ & $\text{K}_{2}\text{O}$ & $\text{Rb}_{2}\text{O}$ & $\text{Cs}_{2}\text{O}$\\ \cline{2-6}
   & \multicolumn{5}{|l|}{Groep 1 elemente vorm almal oksiedverbindings in  'n 2:1 verhouding} \\ \hline
\textbf{Atoomradius} & \multicolumn{5}{p{6cm}|}{Neem toe van bo na onder in  'n groep.} \\ \hline
\textbf{Eerste ionisasie-energie} & \multicolumn{5}{p{6cm}|}{Neem af van bo na onder in  'n groep.} \\ \hline
\textbf{Elektronegatiwiteit} & \multicolumn{5}{p{6cm}|}{Neem af van bo na onder in  'n groep.} \\ \hline
\textbf{Smelt-en kookpunt} & \multicolumn{5}{p{6cm}|}{Neem af van bo na onder in  'n groep.} \\ \hline
\textbf{Digtheid} & \multicolumn{5}{p{6cm}|}{Neem toe van bo na onder in  'n groep.} \\ \hline
  \end{tabular}
\caption{Opsomming van die tendense in groep 1}
\label{tab:group1trends}
 \end{center}

\end{table}
\par 
Ons kan die bogenoemde inligting gebruik om die chemiese eienskappe van onbekende elemente te voorspel. Neem as voorbeeld die element frankium ($\text{Fr}$). Ons kan sê dat sy elektronstruktuur $\textsf{[Rn]}7s^{1}$ sal wees, dit sal 'n laer eerste ionisasie-energie as sesium ($\text{Cs}$) h\^{e} en sy smelt-en kookpunt sal ook laer wees as dié van sesium.\\
Jy behoort van hoofstuk~\ref{chap:classification} te onthou dat die metale aan die linkerkant van die periodieke tabel gevind word, 
die nie-metale is aan die regterkant en metallo\"ide op die sigsaglyn (zigzag) beginnende by boor.
\begin{exercises}{Groups in the periodic table}
{
            \nopagebreak \noindent
\begin{enumerate}[noitemsep, label=\textbf{\arabic*}. ]
\item Gebruik tabel~\ref{tab:group1trends} en figuur~\ref{fig:atom:periodic1} om jou te help om soortgelyke tabelle vir groep 2 en groep 17 saam te stel. 
 \item Die volgende twee atome is gegee. Vergelyk hierdie elemente in terme van die volgende eienskappe. Verduidelik die verskille in elke geval.
$^{24}_{12}\text{Mg}$ en $^{40}_{20}\text{Ca}$. 
\begin{enumerate}[noitemsep, label=\textbf{\alph*}. ]
 \item Grootte van die atoom (atoomradius)
\item Elektronegatiwiteit
\item Eerste ionisasie-energie
\item Kookpunt
\end{enumerate}
 \item Bestudeer die volgende grafiek en verduidelik die tendens in elektronegatiwiteit van die groep 2 elemente.\\
\begin{pspicture}(-2.25,-2)(29,12)
  \psaxes[axesstyle=axes,Dx=1,Dy=0.5,labels=y,ticks=y]{-}(5,1.5)
  \listplot[plotstyle=bar,barwidth=0.5cm]{0.5 1.5
1.5 1.3
2.5 1
3.5 1
4.5 0.9}
\rput{0}(0.5,-0.2){$\text{Be}$}
\rput{0}(1.5,-0.2){$\text{Mg}$}
\rput{0}(2.5,-0.2){$\text{Ca}$}
\rput{0}(3.5,-0.2){$\text{Sr}$}
\rput{0}(4.5,-0.2){$\text{Ba}$}
\end{pspicture}
 
\item            \label{m38760*id262476}Verwys na die lys van elemente hieronder: \label{m38760*id7632}\begin{itemize}[noitemsep]
            \item Litium ($\mathrm{Li}$)\item Chloor ($\mathrm{Cl}$)\item Magnesium ($\mathrm{Mg}$)\item Neon ($\mathrm{Ne}$)\item Suurstof ($\mathrm{O}$)\item Kalsium ($\mathrm{Ca}$)\item Koolstof ($\mathrm{C}$)\end{itemize}
        Watter van die elemente hierbo genoem:
        \label{m38760*id262499}\begin{enumerate}[noitemsep, label=\textbf{\alph*}. ] 
            \label{m38760*uid158}\item behoort tot Groep 1
\label{m38760*uid159}\item is 'n halogeen
\label{m38760*uid160}\item is 'n edelgas
\label{m38760*uid161}\item is 'n alkalimetaal
\label{m38760*uid162}\item het 'n atoomgetal van 12
\label{m38760*uid163}\item het 4 neutrone in sy atoomkern
\label{m38760*uid164}\item bevat elektrone in die 4de energievlak
\label{m38760*uid166}\item het al sy energie-orbitale gevul
\label{m38760*uid167}\item het chemiese eienskappe wat die naaste aan mekaar is
\end{enumerate}
\end{enumerate}

\practiceinfo
 \par \begin{tabular}[h]{cccccc}
 (1.) 02b3  & (2.) 02b4 & (3.) 02b5 & (4.) 02b6 \end{tabular}
}
\end{exercises}
%         \par 
%         \label{m38757*eip-6}
%     \setcounter{subfigure}{0}
% 	\begin{figure}[H] % horizontal\label{m38757*periodictable-3}
%     \textnormal{Khan academy video on periodic table - 2}\vspace{.1in} \nopagebreak
%   \label{m38757*yt-media3}\label{m38757*yt-video3}
%             \raisebox{-5 pt}{ \includegraphics[width=0.5cm]{col11305.imgs/summary_www.png}} { (Video:  P10028 )}
%  \end{figure}       \par 
% \label{m38760*eip-148}
%     \setcounter{subfigure}{0}
% 	\begin{figure}[H] % horizontal\label{m38760*periodictable-1}
%     \textnormal{Khan academy video on the periodic table - 1}\vspace{.1in} \nopagebreak
%   \label{m38760*yt-media1}\label{m38760*yt-video1}
%             \raisebox{-5 pt}{ \includegraphics[width=0.5cm]{col11305.imgs/summary_www.png}} { (Video:  P10026 )}
%  \end{figure}       \par 
% Die volgende aanbieding bied 'n opsomming van die periodieke tabel
%     \setcounter{subfigure}{0}
% 	\begin{figure}[H] % horizontal\label{m38757*slidesharefigure}
%     \label{m38757*slidesharemedia}\label{m38757*slideshareflash}\raisebox{-5 pt}{ \includegraphics[width=0.5cm]{col11305.imgs/summary_www.png}} { (Presentation:  P10029 )}
%  \end{figure}       \par 
    \label{m38757*eip-572}

\begin{activity}{Ontwerp jou eie periodieke tabel}
            \nopagebreak
            \label{m38760*eip-603}
% \begin{minipage}{.5\textwidth}
Jy is die amptelike chemikus vir die planeet Zog. Jy het al die elemente ontdek wat ons het hier op aarde het, maar jy het nie 'n periodieke tabel nie. Die burgers van Zog wil weet hoe al hierdie elemente met mekaar verband hou. Hoe sal jy die periodieke tabel ontwerp? Dink na oor hoe jy die data sal organiseer wat jy het en watter eienskappe jy sal insluit. Moenie net Mendeleev se idees gebruik nie, wees kreatief en kom met 'n paar van jou eie idees. Doen navorsing oor ander vorme van die periodieke tabel en stel een saam wat sin maak vir jou. Bied jou idees vir die klas aan.
% \end{minipage}
% \begin{minipage}{.5\textwidth}
\begin{center}
\includegraphics[width=.8\textwidth]{photos/Circular_periodic_table.png}\\
\textsl{Picture from Wikimedia commons}
\end{center}
% \end{minipage}

\end{activity}
\summary{VPddg}
            \nopagebreak
            \label{m38757*uid0123}\begin{itemize}[noitemsep]
            \label{m38757*id79342}\item Elemente is in periodes en groepe gerangskik in die periodieke tabel. Die elemente is volgens toenemende atoomgetal gerangskik.
\label{m38757*id97342}\item 'n \textbf{Groep} is 'n kolom met elemente in die periodieke tabel wat soortgelyke eienskappe het. 'n \textbf{Periode} is 'n ry in die periodieke tabel.
\item Die groepe in die periodieke tabel is genommer vanaf 1 tot 18. Groep 1 is bekend as die alkalimetale, Groep 2 is bekend as die alkali-aardmetale, groep 17 is bekend as die halogene en groep 18 is bekend as die edelgasse. Die elemente in 'n groep het soortgelyke eienskappe.
\item Verskeie tendense soos ionisasie-energie en atoomdeursnee kan in die periodes van die periodieke tabel waargeneem word. \label{m38757*uid184}
\label{m38757*uid186}\item 'n Element se \textbf{ionisasie-energie} is die energie wat nodig is om 'n elektron uit 'n atoom in die gastoestand te verwyder.
\label{m38757*uid187}\item Ionisasie-energie neem toe van links na regs in 'n \textbf{periode} van die periodieke tabel.
\label{m38757*uid188}\item Ionisasie-energie neem af van bo na onder in 'n \textbf{groep} van die periodieke tabel.
\end{itemize}
        \label{m38757*eip-219}
            



\begin{eocexercises}{ Periodieke Tabel}
            \nopagebreak \noindent
\label{m38757*uid091221}\begin{enumerate}[noitemsep, label=\textbf{\arabic*}. ] 
\item Vermeld (s\^{e}) of die volgende stellings waar of onwaar is. Verbeter die stelling indien dit onwaar is:
  \label{m38757*id073324}\begin{enumerate}[noitemsep, label=\textbf{\alph*}. ] 
  \item Groep 1 elemente is soms bekend as die alkalie-aardmetale.
  \item Groep 8 elemente is bekend as die edelgasse.
  \item Groep 7 elemente is baie onreaktief.
  \item Die oorgangselemente word tussen groepe 3 en 4 gevind.\end{enumerate}
\item Gee een woord of term vir elk van die volgende:
   \label{m38757*id0734}\begin{enumerate}[noitemsep, label=\textbf{\alph*}. ] 
   \item Die energie wat nodig is om 'n elektron uit 'n atoom te verwyder
   \item 'n Horisontale ry in die periodieke tabel
   \item 'n Baie reaktiewe groep elemente wat net een elektron tekort het in hul buitenste vlak.
   \end{enumerate}
\item Gegee $^{80}_{35}\text{Br}$ en $^{80}_{35}\text{Cl}$. Vergelyk hierdie elemente in terme van die volgende eienskappe. Verduidelik in elke geval die verskille.
  \begin{enumerate}[noitemsep, label=\textbf{\alph*}. ]
   \item Atoomradius
   \item Elektronegatiwiteit
   \item Eerste ionisasie-energie
   \item Kookpunt
  \end{enumerate}
\item Gegee die volgende tabel:
  \begin{table}[H]
   \begin{center}
    \begin{tabular}{|l|l|l|l|l|l|l|l|l|} \hline
     \textbf{Element} & $\textsf{Na}$ & $\textsf{Mg}$ & $\textsf{Al}$ & $\textsf{Si}$ & $\textsf{P}$ & $\textsf{S}$ & $\textsf{Cl}$ & $\textsf{Ar}$ \\ \hline
     \textbf{Atoomgetal} & 11 & 12 & 13 & 14 & 15 & 16 & 17 & 18 \\ \hline
     \textbf{Digtheid ($g \cdot cm^{-3}$)} & 0,97 & 1,74 & 2,70 & 2,33 & 1,82 & 2,08 & 3,17 & 1,78 \\ \hline
     \textbf{Smeltpunt ($^{\circ} C$)} & 370,9 & 923,0 & 933,5 & 1687 & 317,3 & 388,4 & 171,6 & 83,8 \\ \hline
     \textbf{Kookpunt ($^{\circ} C$)} & 1156 & 1363 & 2792 & 3538 & 550 & 717,8 & 239,1 & 87,3 \\ \hline
     \textbf{Elektronegatiwiteit} & 0.93 & 1.31 & 1.61 & 1.90 & 2.19 & 2.58 & 3.16 & - \\ \hline
    \end{tabular}
   \end{center}
  \end{table}
Teken grafieke om die patrone (tendense) vir die volgende fisiese eienskappe te toon:
  \begin{enumerate}[noitemsep, label=\textbf{\alph*}. ]
  \item Digtheid
  \item Kookpunt
  \item Smeltpunt
  \item Elektronegatiwiteit
  \end{enumerate}
\item 'n Grafiek wat toon hoe die eerste ionisasie-energie van die elemente in die periode 3 varieer:\\
\begin{pspicture}(-2.25,-2)(29,12)
  \psaxes[axesstyle=axes,Dx=1,Dy=.5,ticks=none,labels=none]{-}(8,2)
  \listplot[plotstyle=bar,barwidth=0.5cm]{0.5 .514
1.5 .765
2.5 .599
3.5 .816
4.5 1.049
5.5 1.036
6.5 1.297
7.5 1.576}
\rput{0}(0.5,-0.2){$\text{Na}$}
\rput{0}(1.5,-0.2){$\text{Mg}$}
\rput{0}(2.5,-0.2){$\text{Al}$}
\rput{0}(3.5,-0.2){$\text{Si}$}
\rput{0}(4.5,-0.2){$\text{P}$}
\rput{0}(5.5,-0.2){$\text{S}$}
\rput{0}(6.5,-0.2){$\text{Cl}$}
\rput{0}(7.5,-0.2){$\text{Ar}$}
\end{pspicture}
  \begin{enumerate}[noitemsep, label=\textbf{\alph*}. ]
  \item Verduidelik die patroon deur na die elektronkonfigurasie te verwys.
  \item Voorspel die patroon vir die eerste ionisasie-energie van die elemente in periode 2.
  \item Teken 'n rowwe grafiek van die patroon wat in die vorige vraag voorspel is.
  \end{enumerate}
\end{enumerate}
 
% Automatically inserted shortcodes - number to insert 5
\par \practiceinfo
\par \begin{tabular}[h]{cccccc}
% Question 1
(1.)	02b7	&
% Question 2
(2.)	02b8	&
% Question 3
(3.)	02b9	&
% Question 4
(4.)	02ba	&
% Question 5
(5.)	02bb	&
\end{tabular}
% Automatically inserted shortcodes - number inserted 5
\end{eocexercises}
